\chapter{Derived Affines}
Talk by Benedict Morrissey.

\section{3 perspectives on derived affines}

First recall the notion of affines in classical AG: $\Aff_k^{Cl} \simeq (\cC Ring)^{\op}$. We get schemes by gluing these together.
There's also the functor of points viewpoint: $X \in \Aff_k^{Cl}$ defines a sheaf by sending $\Spec R \mapsto \Hom
(\Spec R, X)$. The schemes are then precisely the sheaves in the Zariski topology.  Already in classical AG, there exist 
constructions which move us out of this category: 
both Serre's intersection theorem and Illusie's notion of the cotangent complex use derived functors.
So by introducing DAG, we will understand better these structures in classical AG.

We will talk about 3 approaches to derived affines --- all of these consist of embedding the classical category $CRing_{k}$ into a larger category in which we have a derived tensor product.  In this section we assume that we are working over a ring $k$ of characteristic zero\footnote{We note here that we neglect to mention the important generalization of rings given by E-$\infty$ ring spectra, as described in chapter 7 of \cite{Lurie_Higher_algebra}.  In the case where we are working over $k$ a $\mathbb{Q}$-algebra this infinity category is equivalent to those described in this section as shown in  \cite{Lurie_DAG_V} proposition 4.1.11.  When we remove the characteristic zero assumption the statements about the Model structure on CDGA's no longer hold.  One can still use simplicial commutative rings or E-$\infty$ algebras, though these give different $\infty$ categories.}.
\begin{enumerate}
\item \label{item:scr} Simplicial commutative rings;
\item \label{item:cdga} Commutative differential graded algebras. (CDGA's)
\item \label{item:lawvere} Lawvere theory;
\end{enumerate}

\begin{rem}
Classically gluing is easy. For example, fiber products are computed by reducing to the affine case, where it's just the
tensor product of rings. In DAG, the derived tensor product is only defined up to quasi-isomorphism, so gluing can only be
defined in a category which allows homotopy, such as an $\infty$-category. For today's talk we mostly use the model category description;
an application of Dwyer-Kan localization produces an $\infty$-category.
\end{rem}

\subsection{Simplicial Commutative Rings}

For approach \ref{item:scr}, recall that the simplicial category $\Delta$ is:
\[ 	Ob(\Delta) = \{n \in \N \cup \{0\} \}		\]
where morphisms are compositions of face maps $\delta_{i}^{n}:[n-1]\rightarrow [n]$ for $0 \leq i < n$ and degeneracy maps $s_{i}^{n}: [n+1]\rightarrow [n]$, subject to the simplicial identities as can be found in e.g. \cite{gillam2013simplicial}. 
\begin{defin}
The \textbf{category of simplicial commutative rings} is the category of contravariant functors:
\[	 SCR_k = \Hom(\Delta^{\op}, CRing_k).	\] 
\end{defin}

\begin{rem}
There's a model category structure on this: fibrations are Kan fibrations on the underlying simplicial sets, i.e. morphisms
$f: A \to B$ of simplicial commutative rings, such that all horns have fillers:
\[
\begin{tikzcd}
\Lambda^n_j\arrow[hook]{d}\arrow{r} & A\arrow{d}{f} \\
\Delta^n\arrow{r}\arrow[dashed]{ur} & B.
\end{tikzcd}
\]

Weak equivalences
are weak homotopy equivalences on the underlying simplicial sets. Cofibrations are then determined from the axioms of a
model category; note that they are \textit{not} the same as cofibrations of the underlying simplicial sets.
\end{rem}

\begin{rem}
\label{rem:transfer_model_structure}
We're using transfer to put the model structure on $SCR_k$. To explain what that means, under suitable conditions, there's 
a general procedure for defining a model structure on a category $\cB$, given a model category $\cA$ and an adjoint functor
pair:
\[
\begin{tikzcd}
 \cA \arrow[shift left]{r}{F} & \cB \arrow[shift left]{l}{G} .
\end{tikzcd}
\]
The procedure forces the adjoint functor pair to be a Quillen adjunction. In our case, we use the free-forgetful adjunction:
\[
\begin{tikzcd}
 s\cS et \arrow[shift left]{r}{F} & SCR_k \arrow[shift left]{l}{U}
\end{tikzcd}
\]
to transfer the Kan model structure to $SCR_k$. The key point which allows this to work is that all objects are fibrant. 
Cofibrations are more difficult to characterize, but the cofibrant objects are precisely the quasi-free ones. (That is,
the ones isomorphic to a free object.)
\end{rem}

\subsection{CDGA's}

Next, we introduce CDGA's and the Dold-Kan equivalence --- which shows that this category is the same as that of simplicial commutative rings under our assumption that we are working over a characteristic zero field. Recall that we have a Quillen equivalence:
\[
\begin{tikzcd}
	s\cV ect \arrow[shift left]{r} & dg-\cV ect^{\leq 0} \arrow[shift left]{l}
\end{tikzcd}
\]
between simplicial vector spaces and differential graded vector spaces, concentrated in nonpositive degrees. We want to
talk about commutative monoids in these categories, $sc\cA lg_k$ and $cdg-\cA lg^{\leq 0}$, respectively.
The model structure on $cdg-\cA lg^{\leq 0}$ can also be obtained by transfer from the free-forgetful adjunction; we obtain
that the weak equivalences are quasi-isomorphisms, and the fibrations are degree-wise surjections.

\begin{thm}[Symmetric monoidal Dold-Kan (A proof can be found in \cite{schwede2003equivalences})]
There is a Quillen equivalence:\footnote{Note that, in general, a Quillen equivalence is not an equivalence of categories.
It does, however, induce an equivalence of Dwyer-Kan localizations (and hence also of homotopy categories).}
\[
\begin{tikzcd}
	sc\cA lg_k \arrow[shift left]{r}{N} & cdg-\cA lg_k^{\leq 0} \arrow[shift left]{l}{\Gamma}.	
\end{tikzcd}
\]
Moreover, if the simplicial commutative algebra $A_*$ corresponds to the commutative dg-algebra $B_{\bullet}$, then
$\pi_i(A_*) \cong H^i(B_{\bullet})$.
\end{thm}

\begin{rem}
We describe $N$:
$A_* \in sc\cA lg_k$ maps in the first stage to $\tilde A_{\bullet}$, where $\tilde A_{-n} = A_n$, and the differential 
is the alternating sum of the face maps.
$N(A_*)$ is then the quotient of $\tilde A_{\bullet}$ by the images of the degeneracy maps.
For $C$ a CDGA we can describe $(\Gamma C)_{n}:=Hom_{ch^{-}}(N(\Delta^{n}), C)$ where we are in fact using the above definition of $N$ to give a functor from simplicial abelian groups to $ch^{-}$ --- the category of non positively graded chain complexes, and $\Delta^{n}$ is the simplicial abelian group freely generated by an $n$-simplex.  
\end{rem}

We have a similar result for the category of simplicial modules for a given simplicial ring, and the category of dg-modules for its image in CDGA's.  Note that the categories of simplicial modules and of non-positively graded modules for a given CDGA both have model structures.

\begin{thm}[\cite{schwede2003equivalences}]
If $A$ is a simplicial ring the categories of simplicial $A$-modules and of negatively graded $N(A)$-modules are Quillen equivalent.

If $A$ is a CDGA, the categories of negatively graded $A$-modules and of simplicial $\Gamma(A)$-modules are equivalent.
\end{thm}


We now define the truncation functor of a CDGA.  We can use the above Quillen equivalence to also define truncation functors on the category of Simplicial Commutative Rings.

Let $CDGA_{k}^{\leq n} \hookrightarrow CDGA_{k}$ denote the subcategory of $CDGA_{k}$ consisting of objects $A$ such that $H^{i}(A) =0$ for all $i>n$.  Note that $CDGA_{k}^{\leq 0} \cong CRing_{k}$.  The inclusion has a right adjoing $\tau^{\leq n}: CDGA_{k}\rightarrow CDGA^{\leq n}$.  For $A=(A_{n})$, \[(\tau^{\leq n}(A))_{m}= \begin{cases} 
      A_{m} & 0 > m > -n \\
      A/im(d^{m+1}) & m=n \\
      0 & m < n.
   \end{cases}
\]

%Note: Bousfield localization to put a model cateogry structure on n-homotopy type simplicial sets. Just declare the map
%from the $n$-sphere to the point to be a weak equivalence.

\subsection{Lawvere Theories}
\label{sect:lawvere}

We move on to approach \ref{item:lawvere} to derived affines, the Lawvere Theory description. 
This is important because
it's the only one of the 3 procedures which carries through in the analytic setting (see e.g. \cite{Porta_Yu_Higher_analytic_stacks_2014}).

The idea of Lawvere theory is to describe all objects with some type of algebraic structure as functors between the free objects and the category $Set$, or in the $\infty$-categorical case $\mathcal{S}$ --- the infinity category of (topological) spaces.
\begin{eg}  Let $Ab$ be the category of abelian groups, and let $FAb$ be the category of free abelian groups. There is an equivalence of categories
\[	Ab \cong \Fun^{\times}(FAb^{\op}, Set).	\]

To an abelian group $G$, we associate the functor $Hom(-,G)$.  Note that we have an equality $Hom(\Z, G) =_{Sets}G $ (considering $\Z$ as an additive group), by mapping a homomorphism $f:\Z\rightarrow G$ to $f(1)$.  Furthermore $Hom(\Z\times \Z, G)=G\times G$.  Let $f\in Hom(\Z\times \Z, G)=G\times G$ map $(1,0)$ and $(0,1)$ to $a$ and $b$ respectively.  Precomposing with the map $\mathbb{Z}\rightarrow \mathbb{Z}\times \mathbb{Z}$ given by $1\mapsto (1,1)$ gives the map $Z\rightarrow G$ which maps $1\mapsto ab$.

For a functor $F$, we can associate an abelian group as follows.  There's a map $\Z \to \Z \times \Z$, which sends $1\mapsto 1\times 1$. Since $F$ preserves products, we have a map $F(\Z) \times F(\Z)
\cong F(\Z \times \Z) \to F(\Z)$.  This endows $F(\Z)$ with the structure of an abelian (by definition of the multiplication) group.

To a morphism of groups $G\rightarrow H$ we gain a natural transformation of functors $Hom(-,G)\rightarrow Hom(-,H)$ by composing an element of $Hom(-,G)$ with this morphism.  A natural transformation of functors $F_{1}\rightarrow F_{2}$ gives a map $F_{1}(Z)\rightarrow F_{2}(Z)$ such that the diagram
\[
\begin{tikzcd}
F_{1}(\Z \times \Z) \arrow{r} \arrow{d} & F_{1}(\Z) \arrow{d} \\    
F_{2}(\Z\times \Z) \arrow{r} & F_{2}(\Z)
\end{tikzcd}
\]
commutes.  This shows that the map $F_{1}(Z)\rightarrow F_{2}(Z)$ is a group homomorphism.
\end{eg}

We denote by $T_{disc}$ the opposite category of free commutative rings over $k$.  Free commutative rings (over $k$) are the rings $k[x_1, \dots, x_n]$. Hence $T_{disc}$ is the subcategory of the category of affine schemes with objects the planes $\{\bbA^n\}$.  We denote by $CRing$ the category of commutative rings.

\begin{prop}
There is an equivalence of categories:
\[\Fun^{\times}(T_{disc}, Set) \cong CRing.\] 
\end{prop}

On objects we map a functor to it's value on the group ring $\bbA^{1}$, $F\mapsto F(\bbA^{1})$.  The inverse map is given by taking a ring $R$ to the functor $Hom_{CRing}(-,R)$.  The (commutative) addition and multiplication on $\bbA^{1}$ ($\bbA^{1}\times \bbA^{1}\rightarrow \bbA^{1}$ given on points by $(x,y)\mapsto x+y$ and $(x,y)\mapsto xy$ respectively) give $F(\bbA^{1})$ the structure of a (commutative) ring.  

Now pass to 
\[SCR_k \cong s\Fun^{\times}(T_{disc}, Set) \cong \Fun^{\times}(T_{disc}, sSet) \cong \Fun^{\times}
(T_{disc}, S),\] 
where $S$ is the infinity category of spaces. The last step is a very hard rectification theorem, proved
by Lurie-Bergner.\footnote{HTT Propositions 5.5.9.2}  

\section{Derived Affines as Ringed Spaces}

Finally, we take the viewpoint of seeing a scheme as a locally ringed space. For $A \in cdg-\cA lg_k$, we look at
the truncation $\Spec H^0(A)$, which is an affine scheme in the classical sense. We can regard $A$ 
as a sheaf of cdg-algebras on the truncation, as long as we can understand how localization works for cdg-algebras. 
We claim that it suffices to localize the commutative algebra $A_0$. Indeed, we have the multiplication map:
\[	\mu:	A_0 \times A_i \to A_i,		\]
so given a multiplicative subset $S \subset A_0$, we define the localization $S^{-1}A_i$ as $\mu(S^{-1}A_0 \times A_i)$.
If this makes sense, we get a sheaf $\mathcal{O}_A$ of cdg-algebras.

We would like to define derived affines as pairs $\big(\Spec H^0(A), \mathcal{O}_A\big)$. There is a subtlety: a priori 
this only gives a 2-category, and we need $\infty$-categories.  The key to resolving this is to define the notion of a sheaf valued in an $(\infty,1)$-category.



\section{Our favorite classes of morphisms}
\begin{defin}
Given $f: A \to B$ in $SCR_k$, we get maps:
\[	\pi_*(A) \otimes_{\pi_0(A)} \pi_0(B) \to \pi_*(B)	\]
of graded modules. We say that $f$ is \textbf{strong} if this is an isomorphism of graded modules.
\end{defin}

\begin{defin}
\label{def:fav_morphisms}
We define $f: A \to B$ to be \textbf{\'{e}tale} (resp. \textbf{smooth}, \textbf{Zariski open immersion}, \textbf{flat})
if $f$ is strong and $\pi_0(A) \to \pi_0(B)$ is \'{e}tale (resp. smooth, Zariski open immersion, flat) in the classical sense.
\end{defin}

\begin{rem}
The strength condition on $f$ is quite restrictive: for example, a strong map from a non-derived domain must have a non-derived
target.
\end{rem}

\begin{defin}
Let $X = \Spec(A)$ be a derived affine over $k$. Then the \textbf{small \'{e}tale site} of $X$ is:
\[	X_{\text{\'{e}t}} = \{\text{\'{e}tale maps }\Spec(B) \to \Spec(A)\}.	\]
\end{defin}

In order to obtain the small \'{e}tale site in the sense of classical AG, one needs to pass to the truncated version
of the \'etale maps: $\pi_0(f) : \Spec(\pi_0(B)) \to \Spec(\pi_0(A))$. Then one can prove there's an equivalence of 
$\infty$-categories between the derived and classical \'etale sites. In particular, this shows that 
$X_{\text{\'{e}t}}$ is a 1-category. This is one of the ingredients in the proof of the easy version
of Lurie representability. Moreover, the same holds for the small smooth site and the small Zariski site.

After introducing the cotangent complex $\bbL_f$ of a morphism $f$, we will see that $f$ is \'{e}tale iff $\pi_0(f)$ 
is of finite presentation and $\bbL_f \simeq 0$.

\begin{defin}
$f:A \to B$ is \textbf{of finite presentation} if the functor $\Map_A(B, - ) : sc\cR ing_k \to \cS$ commutes with
filtered colimits. 
\end{defin}

Unlike in the underived case, being of finite presentation is very strong, because it has a hidden regularity condition.
In particular, we have the proposition due to Lurie:

\begin{prop}
$f: A \to B$ is of finite presentation in the derived sense iff $\pi_0(f)$ is of finite presentation in the classical sense
(also called to order 0) and the cotangent complex $\bbL_f$ is perfect.
\end{prop}

\begin{eg}
Let $X = \bbA^3$, and $Y$ a closed subscheme of $X$ which is not a local complete intersection. 
Then the inclusion $\iota :Y \to X$ is not of finite presentation in the derived sense.
Indeed, by a conjecture of Quillen, which is now a theorem of Abramov, for maps between classical schemes, the cotangent 
complex is either concentrated in degrees 0 and -1, or it's unbounded. Since $Y$ is not lci, the first case is ruled out,
and $\bbL_{\iota}$ is unbounded.
\end{eg}