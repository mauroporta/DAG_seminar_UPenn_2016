\chapter{Perfect Complexes}
\label{chap:perfect_complexes}
Talk by Benedict Morrissey.

Half of this is about perfect complexes in classical AG, and the second half about what we do in the derived setting. 
In 2-3 weeks we will see that perfect complexes actually form a stack.

\section{Classical}
\label{sect:perf_classical}

Let $X$ be a scheme, then we look at $Ch^{\bullet}(QCoh(X))$.

\begin{defin}
$E^{\bullet} \in Ch^{\bullet}(QCoh(X))$ is \textbf{perfect} if it's Zariski locally quasi-isomorphic to an object
of $Ch^b(Vect_X)$. 
\end{defin}

\begin{rem}
This is not the same as requiring cohomology to be finitely supported.
\end{rem}

\begin{defin}
$E^{\bullet} \in Ch^{\bullet}(QCoh(X))$ has \textbf{Tor amplitude} in $[a,b]$ if for all $\mathcal{F} \in \mathcal{O}_X -Mod$,
\[	H^k(E^{\bullet} \otimes^{\bbL}_{\mathcal{O}_X} \mathcal{F}) = 0	\]
for $k \not \in [a,b]$. In particular, if $E$ is in the heart, this is just saying that the Tor with any given sheaf is bounded.
\end{defin}

\begin{rem}
For $\mathcal{F} = \mathcal{O}_X$, we just get the cohomology of $E^{\bullet}$.
\end{rem}

This is sometimes difficult to work with, so we have:

\begin{defin}
$E^{\bullet} \in Ch^{\bullet}(QCoh(X))$ is \textbf{almost perfect} if Zariski locally there is a $n$-quasi-isomorphic to
something in $Ch^b(Vect_X)$. n-quasi-isomorphism means isomorphism on cohomologies for $k\geq n+1$, and surjection
for degree $n$.\todo{Figure out cohomological convention}
\end{defin}

Perfect obviously implies almost perfect; this descends to the derived category of quasi-coherent sheaves.

\begin{thm}
$E^{\bullet}$ is perfect iff $E^{\bullet}$ is almost perfect (for some $n$) and has finite Tor amplitude.
\end{thm}
\begin{proof}
Locally free means flat, so tensoring it with anything preserves the tor amplitude. The other direction is in
TT, Higher algebraic K-theory of schemes, 2.2.12.
\end{proof}

Alternative definition: $E^{\bullet} \in D(Coh(X))$. Locally on some $U$ we have an $\mathcal{O}_X(U)$-module
$E|_{U}^{\bullet}$. We require that this is bounded above and has coherent cohomologies. Equivalently, $\tau_{\leq n}(E^{\bullet}|_n)$
is compact in $\tau_{\leq n} Mod_{\mathcal{O}_X(U)}$.

\begin{thm}
For $X$ affine, $E^{\bullet}$ is perfect if and only if it's globally quasi-isomorphic to an object of $Ch^b(Vect_X)$.
\end{thm}

\begin{thm}
If $X$ is smooth and Noetherian, then $D(Coh^b(X)) \simeq D(Perf(X))$.
\end{thm}

We'll prove this by Serre regularity.

\begin{defin}
A is \textbf{regular} if $\dim_k(m/m^2) = \dim_{Krull} A$. The \textbf{global dimension} of $A$ is:
\[	gldim(A) = \sup_{M \in A-Mod} \text{projdim}(M),	\]
where the latter is the minimum length of a projective resolution of $M$.
\end{defin}

\begin{thm}[Serre regularity]
If $A$ is a Noetherian local ring, TFAE:
\begin{enumerate}
\item $A$ is regular;
\item $gldim(A) < \infty$;
\item $gldim(A) = \dim_{Krull}A$.
\end{enumerate}
\end{thm}

Going back to $X$ smooth and Noetherian, we know that all local rings are regular. $E^{\bullet} \in Coh^b(X)$, then
$E^{\bullet}_p$ is a $\mathcal{O}_{X,p}$ module. 

\begin{proof}
We actually just do the case of $E$ in the heart, because it's easier. (For the other one, we probably resoluve to a double
complex, and take the total complex.)

Take a projective resolution of $E^{\bullet}_p$ as a $\mathcal{O}_{X,p}$ module; we know it must be finite:
\[	0 \to P_n \to \dots \to P_2 \to P_1 \to E^{\bullet}_p \to 0.	\]
We can do this in an open set around $p$:\todo{draw this from paper}

This takes care of one direction. We then use Tor dimension to show that, if $E$ is perfect, it's in $D(Coh^b(X))$.
\end{proof}

For the non-smooth case, we look at the ind-completion $Ind(Perf(X)) \simeq QCoh(X)$. (Always true for $X$ quasi-compact,
quasi-separated.) On the other side, $Ind(Coh^b(X)) = IndCoh(X)$. The quotient $IndCoh(X)/QCoh(X) = D_{sing}(X)$, which
really sees the singularities of $X$.


\section{derived}
\label{sect:perf_derived}

Let $A \in SCR_k$; recall that $\cM od_A$ is a stable $\infty$-category.

\begin{defin}
$\cM od^{\text{perf}}_A \subset \cM od_A$ is the smallest stable subcategory containing $A$ and closed under retracts.
Recall that $A$ is a retract of $B$ if there exist maps $i:A \to B, r:B \to A$ such that $r\circ i = \id_A$.
\end{defin}

\begin{defin}
$N \in \cC$ is \textbf{compact} if $\cH om_{\cC}(N, -)$ commutes with filtered colimits. The latter means that
the index category is nonempty, and for all $i,j \in I$, there exists $k$ such that $i \to k \leftarrow j$, and
coequalizers exist.
\end{defin}

\begin{eg}
Consider $(\cM od_A)^{\heartsuit}$. The compact objects are the finitely presented ones. We have a map $A^n \to M$,
so:
\[	\Hom(M, \varinjlim_I B_i) \simeq \varinjlim_I \Hom(M,B_i),	\]
because if we have a map $M \to B$, we can fully describe it by the composition $A^n \to B$. Each of the $n$ generators goes
to some $B_{i_k}$, so by the definition of filtered index category, there exists some $B_j$ such that $A^n \to B_j$. 

Conversely, starting with compact $M$, we look at finitely generated submodules $M_i$, and we have:
\[	\Hom(M, \varinjlim_I M_i) \simeq \varinjlim_I(M,M_i).	\]
In particular, the identity map $M \to M$ factors through some $M_j$, so $M = M_j$.
\end{eg}

\begin{thm}
$M \in \cM od_A$ is perfect iff it's compact.
\end{thm}
\begin{proof}
$\cM od_A^{\text{perf}} \subset \cM od_A^{\text{cpct}}$. \todo{add diagram from paper}
\[
\begin{tikzcd}
\;
\end{tikzcd}
\]

Since DK is an equivalence, we only need to argue that truncation and the forgetful functor preserve filtered colimits.
For the first one: filtered colimits are t-exact. For the second one: it does.

For the other direction, we have the inclusion $\cM od_A^{\text{perf}} \to \cM od_A$, we factor this though Ind, which
is just the completion with respect to filtered colimits.
\[
\begin{tikzcd}
\cM od_A^{\text{perf}} \arrow{r} & \cM od_A \\
 & Ind(\cM od_A^{\text{perf}}) \arrow{u}{\phi}
\end{tikzcd}
\]
$f$ is obviously fully faithful, because $Ind(\cM od_A^{\text{perf}})^{\omega} = \Mod_A^{perf}$.\footnote{We'll talk more about
this equality later, it follows because Perf is idempotent complete.} Mapping spaces
in $Ind$ are computed by:
\[ \cM ap_{Ind(\cC)}(\colim_{i \in I} \mathcal{F}_i, \colim_{j \in J} \mathcal{G}_j)	
= \lim_{i \in I} \colim_{j \in J}	\cM ap_{\cC}(\mathcal{F}_i, \mathcal{G}_j) .\]
This means we're computing mapping spaces by the same formula, so $\phi$ is fully faitfhul.
\end{proof}

The following is 7.2.4.5 in \cite{Lurie_Higher_algebra}:
\begin{thm}
$M \in \Mod_A^{perf}$, we have:
\begin{enumerate}
\item $\pi_n M = 0$ for $n >>0$;
\item If $\pi_m M \cong 0$ for all $m>k$, then $\pi_k M$ is finitely presented as a $\pi_0(M)$-module.
\end{enumerate}
\end{thm}

\begin{proof}
For $M$ perfect, we use compactness to get $M \simeq \lim_{n\to \infty} (\tau_{\leq n} M)$. In fact, the map must
factor through one of the terms in the limit, so $M \simeq \tau_{\leq n}M$ for some $n$.

Next, we have the adjunction:
\[
\begin{tikzcd}
\cM od_A^{\heartsuit} \arrow{r} & \cM od_A^{connective} \arrow{r}{[k]} & \cM od_A^{support \leq k}
\end{tikzcd}
\]
The adjoint truncates $\geq k$ and then shifts. Both of these preserve  \todo{finish}
\end{proof}

\begin{eg}
Think about $\Sym(k[2])$ as a module over itself. It is perfect by definition, but it's not bounded below.
\end{eg}




Recall from the last talk that we have $A \in SCR_k$, and a stable$\infty$-category $\cM od_A$. We defined the
stable $\infty$-subcategory $\cM od_A^{\text{perf}}$. We proved that $\cM od_A^{\text{perf}}\simeq \cM od_A^{\text{cpt}}$.


\begin{defin}
$M$ is almost perfect if it's almost compact, i.e. $M$ is bounded above and $\forall n\leq 0$, $\tau_{\geq n}M$ is 
compact in $\cM od_A^{\geq n}$.
\end{defin}

\begin{rem}
In the classical setting, due to Tor amplitude, perfect complexes need to be bounded below. This is no longer the case.
\end{rem}

\begin{thm}[7.2.4.11 in \cite{Lurie_Higher_algebra}]
\begin{enumerate}
\item $\cM od_A^{aperf} \subset \cM od_A$ closed under translation, finite colimits, so it's a stable subcategory of
$\cM od_A$;
\item $\cM od_A^{aperf}$ is closed under retracts;
\item $\cM od_A^{perf} \subset \cM od_A^{aperf}$;
\item $(\cM od_A^{aperf})^{\leq 0}$ closed under geometric realizations;
\item Every $M \in \cM od_A^{aperf}$ is $M = |P_{\bullet}|$, a geometric realization of a simplicial $A$-module. Each 
$P_i$ finite rank and free.\footnote{In fact, the almost perfect ones are precisely these geometric realizations - we think.
Write about this in more detail.}
\end{enumerate}
\end{thm}

\begin{proof}
Look at 2.4.1
\end{proof}

A note about geometric realizations, which are colimits of simplicial objects. Simplicial resolutions are the classical
description of the cotangent complex. One starts with a simplicial resolution in the category of $A$-modules, and the
realization is the cotangent complex.

Now assume that $X \in \Aff^{classical}$ and that $E_{\bullet}$ is perfect. We want to show that $E_{\bullet}$
is equivalent to a finite complex of vector bundles, globally. This follows from the proof before, but we show that
$D_i$ are actually vector bundles, using the finite Tor amplitude.

\begin{thm}[7.2.4.17 in \cite{Lurie_Higher_algebra}]
Say $A$ is left coherent, i.e. $A$ is connective, $\pi_nA$ is a finitely presented $\pi_0(A)$-module, and that every
finitely generated left ideal of $A$ is a finitely presented left $A$-module. Then $M \in \cM od_A$ is almost perfect
if and only if $\exists m >>0$ such that $\pi_k M = 0$ for all $k\geq m$, and $\pi_k M$ is a finitely presented
$\pi_0(A)$-module. \footnote{This formulation is using homological convention, I think. Figure out the signs.}
\end{thm}

\begin{rem}
In particular, from this it's obvious that not all almost perfect modules are perfect. Think about the discrete case,
this allows almost perfect to be unbounded below, whereas perfect have to be bounded below, due to finite Tor amplitude.
\end{rem}

\begin{defin}
The tor amplitude $Toramp(M) \leq n$ if for all discrete $A$-modules $N$, $\pi_i(M \otimes_A N) = 0$ for all $i\leq -n$.
\end{defin}

\begin{thm}[7.2.4.23 in \cite{Lurie_Higher_algebra}]
$A \in SCR_k$:
\begin{enumerate}
\item $M \in \cM od_A$, tor amplitude $\leq n$ $M[k]$ has tor amplitude $\leq n+k$;
\item $M' \to M \to M''$ a fiber sequence, $M'$ and $M''$ have tor amplitude $\leq n$, then so does $M$;
\item $M$ has tor amplitude $\leq n$, so does any retract;
\item $M$ is almost perfect, then $M$ is perfect iff it has finite Tor amplitude;
\item $M$ has $Toramp \leq n$, then $\forall N \in (\cM od(A)_{supp \leq 0})$, $\pi_i(N\otimes_A M) = 0$ for
	$i \leq -n$.
\end{enumerate}
\end{thm}

\begin{proof}
For 4, the inductive hypothesis uses 2, for the base case we want that almost perfect and flat implies perfect. The latter
follows from 7.2.4.20.
\end{proof}

