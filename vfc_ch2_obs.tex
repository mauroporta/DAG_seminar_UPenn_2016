\chapter{Obstruction Theories and Virtual Fundamental Classes}
\label{ch2:obs}
Talk by Benedict Morrissey.

Given a stack $X$, our objective is to construct a virtual fundamental class $[X]$ for it, motivated by the discussion in
\ref{ch:gw}. We will see two ways in which a derived enhancement of $X$ helps achieve this.
We would like $[X]$ to come from an algebraic cycle, i.e. an element of the Chow group.
In this case, given $f: X\to Y$ proper,
there is a well-defined pushforward  $f_*[X] \to [Y]$, which induces a pushforward
$f_*[X]^H \to [Y]^H$ on the images $[X]^H, [Y]^H$ of the VFCs in any Weil cohomology theory $H$.

However, derived Chow groups have yet to be defined, so we start with a piecemeal approach, by
defining a class in G-theory only.


\section{Construction from G-Theory}
\begin{defin}
The \textbf{G-theory} $G_0(X)$ of a classical stack $X$ is defined as the K-theory of the category of coherent sheaves on $X$:
\footnote{One can also define higher G-theory $G_i$, but we won't need this.}
\[	G_0(X) := K_0(\Coh(X)) .	\]
If $\tilde X$ is a derived stack, we set $G_0(\tilde X) = K_0((\Coh \tilde X)^{\heartsuit})$.
\end{defin}

\begin{defin}
A \textbf{derived enhancement} of a stack $X$ is a derived stack $\tilde X$ such that $t_0(\tilde X) = X$.
\end{defin}

There is a natural inclusion, left-adjoint to the truncation, which we denote $j: X \to \tilde X$.
Using the fact that pushforwards of coherent sheaves by proper maps are coherent,
\todo{check if there are other conditions, and whether $\tilde X$ derived changes anything}
we obtain $j_* : G_0(X) \to G_0(\tilde X)$.

\begin{prop}
If $X$ is quasi-compact, then $j_* : G_0(X) \to G_0(\tilde X)$ is a bijection. In this case we define:
\[	 [X]^{\vir} := j_*^{-1}[\mathcal{O}_{\tilde X}].	\]
\end{prop}

\begin{proof}
The identification actually works on the full spectrum of $G$-theory. We're using the theorem of 
the heart for $K$-theory. The identification is done as follows.
\begin{enumerate}
\item Theorem of the heart for $K$-theory. (Due to Quillen, and Batwick in the DG category setting.) If you have $\cC$ a stable
$\infty$-category, idempotent complete, with $t$-structure, and every object in the heart is bounded, then $K(\cC) = K(\cC^{\heartsuit})$.
\item $\Coh(\tilde X)^{\heartsuit} \simeq \Coh(X)^{\heartsuit}$, which follows from descent and the analogous result for
derived affines, which was proved during the first semester, in the talk on Stable $\infty$-categories.
\footnote{Throughout when we write $\Coh$ we mean $\Coh^b$.}
\end{enumerate}
\end{proof}

\begin{thm}
\label{thm:ox_bounded}
For $\tilde X$ quasi-compact,\footnote{Note that we don't need to assume that $X$ is quasi-compact.}
$\mathcal{O}_{\tilde X}$ is bounded. It follows that the following sum is finite:
\[	j_*^{-1}[\mathcal{O}_{\tilde X}] = \sum_{i=0}^{\infty} (-1)^i [H^i(\mathcal{O}_{\tilde X})],	\]
so it defines an element in $G_0(X)$.
\end{thm}

\begin{rem}
Note that the cohomology in Theorem \ref{thm:ox_bounded} is just the cohomology of the complex, NOT sheaf cohomology. 
Moreover it wouldn't make sense to
use $K$ theory instead of $G$ theory, because even if $\mathcal{O}_{\tilde X}$ is perfect, the kernels and cokernels 
of the various differentials don't need to be.
\end{rem}

\begin{proof}
We start with a vague understanding of why the theorem may be true. The counterexample is $\Spec (\Sym k[2])$, 
where the cotangent complex is unbounded. But if it's in amplitude [-1,0], it's like an exterior algebra and it works.

We work locally, $\Spec B \to \Spec A$, $\supp \bbL_{B/A} \subset[-1,0]$. $B$ is a derived lci over $A$, so the cotangent complex
is perfect, so there's a theorem that says that $B$ is homotopically of finite type over $A$. These can be constructed by attaching
finitely many cells:
\[	A = B_0 \to B_1 \to B_2 \to \dots \to B_k = B.	\]
Attaching the $i+1^{th}$ cells of $B$ looks like:
\[
\begin{tikzcd}
B_i \arrow{r} & B_{i+1} \\
\bigotimes A[\p \Delta^{i+1}]\arrow{u}\arrow{r} & A[\Delta^{i+1}]\arrow{u} .
\end{tikzcd}
\]
$B_1$ is obtained by attaching cells in degree 1. The map $B_1 \to B$ is an isomorphism on $\pi_0$. \todo{review this proof}

There's another proof by Lowrey and Sch\"urg, in \cite{derived_GRR}, which is more intuitive. Having a quasi-smooth structure 
allows one to describe
the derived space locally as the derived zero locus of a section of a vector bundle. Then the derived intersection can be computed
as a Koszul resolution, so $\mathcal{O}_{\tilde X}$ behaves like an exterior algebra, which means it's bounded. Here the quasi-compactness
is used in order to reduce to finitely many local charts, which means that the bound on $\mathcal{O}_{\tilde X}$ is uniform.
\end{proof}

\begin{rem}
The idea behind the proof of Lowrey and Sch\"urg is also that of
\textbf{Kuranishi structures}. These are essentially a machinery for working with derived stacks which remembers the 
local description as zero locus, in order to avoid using the machinery of derived geometry.
In DAG quasi-smoothness is an intrinsic property that one can check at the level of the cotangent complex, so that one doesn't
need to remember the local descriptions, which are cumbersome and don't glue well.
\end{rem}

The VFC in ordinary cohomology is defined by Konstevich to be:
\begin{equation}
\label{eq:G_definition_VFC}
	[X]^{\vir} = \Ch([X]_G^{\vir}) \Td(j^* \bbT_{\tilde X}).
\end{equation}

\begin{conj}
Definition \ref{eq:G_definition_VFC} agrees with the construction of Behrend-Fantechi, \ref{}. \todo{ref this}
\end{conj}

The conjecture has been verified for schemes (not stacks) by Ciocan-Fontanine and Kapranov, in
\cite{ciocan_fontanine_kapranov}, using the additional 
assumption (which is made in Behrend Fantechi anyway) that the cotangent complex admits a global resolution by vector bundles.




\section{Obstruction Theories}
We introduce the alternative construction of VFCs, following \cite{Behrend_Intrinsic_normal_cone_1997}. In the words of
Mauro, we want to use this as a black box which achieves:
\[	\text{Obstruction Theory} \Longrightarrow \text{VFC}.	\]

Throughout we will use $X,Y$ for underived stacks, and $\tilde X, \tilde Y$ for their derived enhancements.

\begin{defin}
An \textbf{obstruction theory} for $X$ is a morphism $\phi : E \to \bbL_X$ in $D(\Coh(X))$, such that:
\begin{align*}
&h^0(\phi) : H^0(E) \to H^0(\bbL_X) \text{ is an isomorphism,} \\
&h^{-1}(\phi) : H^{-1}(E) \to H^{-1}(\bbL_X) \text{ is surjective,} \\
&H^i(E) = 0 \text{ for } i \neq -1,0.
\end{align*} 
\end{defin}

\begin{defin}
A \textbf{perfect obstruction theory} is an obstruction theory such that $E$ is in perfect amplitude [-1,0], which means that
locally $E$ is isomorphic to a 2-term complex of vector bundles $[E^{-1} \to E^0]$.
\end{defin}

The link to derived geometry is as follows. 
\begin{prop}
\label{prop:derived_obstruction_theory}
Given a derived enhancement $j:X \to \tilde X$, with $\tilde X$ a quasi-smooth DM stack, there is a perfect obstruction theory:
\[	j^* \bbL_{\tilde X} \to \bbL_X.	\]
\end{prop}
\begin{proof}
By descent we reduce this to the case of affines, and we need only consider $A \to t_0(A)$. We have the fiber sequence:
\[	j\bbL_A \to \bbL_{\pi_0(A)} \to \bbL_{\pi_0(A)/A}.	\]
Due to the connectivity estimates, which we introduced last semester in the talk about the cotangent complex,
$\bbL_{\pi_0(A)/A}$ is 2-connective. Indeed, the fiber of $A \to \pi_0(A)$ is 1-connective,
so the cofiber, which is the shift of the fiber by 1, is 2-connective.\footnote{In order to relate the fiber and cofiber of the
morphism $A \to \pi_0(A)$, we use the fact that we are working in the stable $\infty$-category $\pi_0(A)\Mod$, and not 
in $\pi_0(A)-\Alg$.}
\end{proof}

Throughout the rest of the talk, the goal is to describe how to construct a VFC, starting with an obstruction theory. In the
smooth case, if you take the $G$-construction we did earlier, you'd get the same answer. 

We also want to describe functoriality properties for the VFC, and to that effect we introduce compatibility data
between obstruction theories. 
During the check that Kontsevich-Manin axioms are satisfied, we will need to use functoriality a lot.
The following is Definition 5.8
in \cite{Behrend_Intrinsic_normal_cone_1997}.


\begin{defin}
\label{defin:compatibility_datum}
Let $u:X' \to X$ be a morphism.
A \textbf{compatibility datum between obstruction theories} $E$ for $X$ and $ F$ for $X'$ is a choice of embeddings
$f: X \to Y$, $g: X' \to Y'$ into smooth stacks, such that the following diagrams commute:
\[
\begin{tikzcd}
X'\arrow{r}{u} \arrow[swap]{d}{g} & X\arrow{d}{f} \\ Y'\arrow{r}{v} & Y,
\end{tikzcd}
\]
\[
\begin{tikzcd}
u^* E \arrow{d}\arrow{r}{\phi} & F \arrow{r}{\psi}\arrow{d} & g^* \bbL_{Y'/Y}\arrow{d} \\
u^* \bbL_X \arrow{r} & \bbL_{X'} \arrow{r} & \bbL_{X'/X} .
\end{tikzcd}
\]
Moreover, we require the two rows to be fibration sequences in $D(\Coh(X'))$.
\end{defin}

Behrend and Fantechi prove:

\begin{prop}
Given compatibility data between obstruction theories $E$ for $X$ and $F$ for $X'$, it follows that 
$u^*[X]^{\vir, E} = [X']^{\vir, F}$.
\end{prop}

For us obstruction theories come from derived enhancements $\tilde X, \tilde X'$.
In this case, we obtain the functoriality of VFCs in a cleaner way, by giving a morphism between derived
enhancements $w: \tilde X' \to \tilde X$, fitting in the commutative diagram:
\[
\begin{tikzcd}
\; & \tilde X'\arrow{rr}{w}\arrow{ddl}{\tilde g} & & \tilde X\arrow{ddl}{\tilde f} \\
X'\arrow{rr}{u}\arrow[swap]{d}{g}\arrow{ur}{j} & & X\arrow[swap]{d}{f}\arrow{ur}{i} & \\
Y'\arrow{rr}{v} & & Y &
\end{tikzcd}
\]
Moreover, we require that the top and back square are homotopy pullbacks.

\begin{rem}
I was hoping that the top square would be enough. Unfortunately, we still need the choice of ambient 
spaces $Y, Y'$, as well as morphisms $\tilde g$, $\tilde f$, and the data for the homotopy commutativity of the back square.
However Mauro says:
\begin{enumerate}
\item In the applications we care about (stable maps), the entire back square will be there naturally.
\item Working with the derived compatibility data is still easier, in practice, than with the fibration sequences in
Definition \ref{defin:compatibility_datum}.
\end{enumerate}
\end{rem}

Let us see why the derived compatibility data implies the diagram between fibration sequences in 
Definition \ref{defin:compatibility_datum}.
The assumption is that $E = i^* \bbL_{\tilde X}$ and $F = j^*\bbL_{\tilde X'}$. We first need the map:
\[		\phi	: u^* i^* \bbL_{\tilde X} \to j^* \bbL_{\tilde X'} .	\] 
This is just given by $w$. More precisely, the commutativity of the top square gives the map on the left 
in the following diagram, and we define the top map as the composition:
\[
\begin{tikzcd}
u^*i^*\bbL_{\tilde X}\arrow{r}\arrow{d} & j^* \bbL_{\tilde X'} \\
j^*w^*\bbL_{\tilde X}\arrow{ur} &
\end{tikzcd}
\]
To get $\psi$, which must be such that the row is a fiber sequence, we make use of the maps $\tilde g, \tilde f$.
Question: how to identify $j^*\bbL_{\tilde X'/\tilde X} $ with $g^* \bbL_{Y'/Y}$? Since the back square is a pullback, we have
a canonical identification $\bbL_{\tilde X'/\tilde X}  \simeq \tilde g^* \bbL_{Y'/Y}$, and this gives:
\begin{equation}
\label{eq:get_psi}
	j^*\bbL_{\tilde X'/\tilde X} \simeq j^*	\tilde g^* \bbL_{Y'/Y} \simeq  g^* \bbL_{Y'/Y} .
\end{equation}
We take this composition to be $\psi$. Note that this chain of equivalences depends very much on the extra data of
the homotopy commutative back square.

\begin{rem}
Throughout, we want $Y', Y$ to be smooth, and $\tilde f, \tilde g$ to be quasi-smooth. Therefore, if $X, X'$ are not
smooth, we cannot expect $f, g$ to be just identity maps. In fact, the point that Behrend-Fantechi make is that $Y$ and $Y'$ 
should only be expected to exist locally.
\end{rem}

\begin{defin}
\label{def:local_embedding}
A \textbf{local embedding} $(U,M)$ of $X$ is the data of $U\to X$ an \'etale map and $U\to M$ a local immersion, 
where $M$ smooth affine $k$-scheme of finite type.
Given a local embedding, the associated \textbf{normal bundle} is $\fr N_{U|M} := \Spec_M(\Sym (I/I^2))$. Inside this 
we have the \textbf{normal cone} $\fr C_{U/M} = \Spec_M(\oplus_{n\geq 0} I^n/I^{n+1})$.
The ring homomorphism $\Sym (I/I^2) \to \oplus_{n\geq 0} I^n/I^{n+1}$ is surjective, so the map
$\fr C_{U/M}	\to \fr N_{U|M}$ is a closed embedding.
\end{defin}

The normal bundle and normal cone of Definition \ref{def:local_embedding} depend on a choice of local embedding. We would like
to have intrinsic versions, and to obtain them we have to take a limit over all local embeddings, morally speaking.
For this, we need a way of associating a cone stack to a complex. Behrend and Fantechi have a construction that achieves this,
but instead we use a slightly different version from
some unpublished notes of Marco Robalo. \todo{can we reference these?}

Let $E \in \QCoh^b(X)$. Define $\bbV(E)$ as follows. For a map $\mu:\Spec A \to X$,
\[	\Map_{/X}(\Spec A, \bbV(E)) = \Map_{QCoh(A)}(\mu^*(E), \mathcal{O}_A) \simeq \Map_{QCoh(A)}(\mathcal{O}_A,\mu^*(E^{\vee})).	\]
$E$ is perfect for the last thing to make sense. Marci: this is the vector bundle corresponding to $E$. The idea is that,
for $E = [E_0 \to E_1]$, we want $\bbV(E^{\vee}) \simeq E_1/E_0$. For example, if $E_1 = 0$, then $E_1/E_0 = BE_0$, which has for each
fiber the classifying space of the corresponding abelian group.

The first claim is that:
\[	\bbV(E) = \Spec_X(\Sym_X(E)) .	\]
This is because:
\[	\Map_{/X}(\Spec(A), \Spec_X \Sym_X(E)) = \Map_{\QCoh(A)}(\Sym_A(\mu^*(E)), \mathcal{O}_A)	= \Map_{\QCoh(A)} (\mu^*E,\mathcal{O}_A).\]

Next claim: if $E$ is of perfect amplitude $[-1,0]$ over $X$, then $\bbV(E[-1]) = h^1/h^0(E^{\vee})$.
\[	\Map_{/X}\big(\Spec A, h^1/h^0(E^{\vee})\big) = \Map_{/A}\big(\Spec A, \mu^*(h^1/h^0(E^{\vee}))\big)
= \Map_{\QCoh(A)}\big(\mu^*(E_0^{\vee}\to E_{-1}^{\vee}) , \mathcal{O}_A\big).	\]
The meaning of the functor $\bbV$ is actually more intuitive when done in more generality, see subsection \ref{subsect:picard}
for Mauro's point of view on it.

\begin{defin}
The \textbf{intrinsic normal sheaf} of $X$ is $\fr N_X := \bbV(\tau^{\geq -1} \bbL_X)$.
\end{defin}

Compare with the definition of \cite{Behrend_Intrinsic_normal_cone_1997}:
\[	\fr N_X = h^1/h_0(\bbL_X^{\vee}) = h^1/h^0(\tau^{\leq 1}\bbL_X^{\vee}) = \bbV(\tau^{\geq -1} \bbL_X) .	\]

Now consider a local immersion:
\[
\begin{tikzcd}
U\arrow{r}{f}\arrow{d}{\pi} & M \\ X & 
\end{tikzcd}
\]
We want to compare $\pi^* \fr N_X$ with $\fr N_{U/M}$; the answer is:
\[	[\fr N_{U/M} / f^* T_M] \overset{\sim}{\to} \pi^*\fr N_X.	\]
Recall that for each local immersion we have an immersed cone $\fr C_{U/M} \to \fr N_{U/M}$.
It turns out that $[\fr C_{U/M}/f^* T_M]$ glue nicely, and by descent we get an immersed cone
$\fr C_X \to \fr N_X$.

\begin{defin}
$\fr C_X$ is the \textbf{intrinsic normal cone} of $X$.
\end{defin}

Given a perfect obstruction theory $E^{\bullet} \to \bbL_X$, \cite{Behrend_Intrinsic_normal_cone_1997} show that we
obtain another immersed cone stack $\bbV(E) \to \fr N_X$. We would like to define a VFC for $X$ to be
the intersection $\bbV(E) \cap \fr C_X$ in $\fr N_X$. The problem is that $\fr N_X$ is an Artin stack, for which
the Chow group and intersection theory are not properly defined. To avoid this issue we need a choice of atlas, which
translates the problem into one
about Deligne-Mumford stacks, for which Chow groups are well-understood. \todo{reference to Vistoli would be nice}

For this we need extra data of a \textbf{global presentation} for $E^{\bullet}$: this is a 2-term complex of vector bundles
$[F^{-1} \to F^0]$ such that $F^{\bullet} \simeq E^{\bullet}$ as objects of $\Perf(X)$.
Let $F_1 := {F^{-1}}^{\vee}$, which is an atlas for the stack $\bbV{F}$. Then define $C(F)$ as the pullback:
\[
\begin{tikzcd}
C(F^{\bullet})\arrow{r}\arrow{d} & F_1\arrow{d} \\ \fr C_X \arrow{r} & \fr N_X .
\end{tikzcd}
\]

\begin{defin}
Let $0: X \to F_1$ be the zero section.
The \textbf{virtual fundamental class} of $X$ induced by the obstruction theory $E$ is the intersection of
$[C(F^{\bullet})] \in \Chow(F_1)$ with the zero section, i.e. $[X]^{\vir, E} := 0^{!}[C(F^{\bullet})]$.
\todo{Note that 0 is not flat, so $0^*$ would be undefined.}
\end{defin}

\begin{rem}
\cite{Behrend_Intrinsic_normal_cone_1997} prove that $[X]^{\vir,E}$ does not depend on the choice of global presentation.
\end{rem}




\subsection{Picard stacks}
\label{subsect:picard}

We follow and generalize \cite[Expos\'e XVIII, \S 1.4]{SGA4}.
Recalling the equivalence (up to homotopy) between groupoids and $1$-homotopy type, we can rephrase 
Definition 1.4.5 in loc.\ cit.\ as follows:

\begin{defin}
	Let $\cX$ be an $\infty$-topos.
	A \emph{Picard stack} over $X$ is a sheaf
	\[ \cF \colon \cX^{\mathrm{op}} \to \mathrm{sAb}^{\le 1} , \]
	where $\mathrm{sAb}^{\le 1}$ denotes the $\infty$-category of simplicial abelian groups whose underlying space is a $1$-homotopy type.
	We let $\mathrm{Pic}(\cX)$ denote the $\infty$-category of Picard stacks on $\cX$.
\end{defin}

The main result in loc.\ cit.\ can then be summarized as follows:

\begin{prop}[{Proposition 1.4.15 \& Corollary 1.4.17 in loc.\ cit.\ }]
	Let $\cX$ be an $\infty$-topos.
	There is an equivalence of $\infty$-categories
	\[ \mathrm{Pic}(\cX) \simeq \Sh_{\mathcal D^{[-1,0]}(\Ab)}(\cX) , \]
	where $\mathcal D^{[-1,0]}(\Ab)$ denotes the full $\infty$-subcategory of $\mathcal D(\Ab)$ (the $\infty$-derived category of abelian groups) spanned by objects in cohomological amplitude $[-1,0]$.
\end{prop}

From a modern point of view, the proof is a direct consequence of the Dold-Kan equivalence
\[ \mathrm{sAb} \simeq \mathcal D^{\le 0}(\Ab) , \]
combined with the remark that objects in cohomological degree $[-1,0]$ in $\mathcal D^{\le 0}(\Ab)$ correspond to $1$-homotopy types.
Actually, the language of higher stacks, allows us to generalize the above proposition:

\begin{prop}
	Let $\cX$ be an $\infty$-topos.
	There is an equivalence of $\infty$-categories
	\[ \Sh_{\mathrm{sAb}}(\cX) \simeq \Sh_{\mathcal D^{\le 0}(\Ab)}(\cX) . \]
\end{prop}

\begin{rem}
	From the $\infty$-categorical point of view, it is actually very unnatural to distinguish the two categories.
	In other words, the above proposition, should be perceived as \emph{tautological} (at least from the reader used to higher categorical reasoning).
	Indeed, the more natural way of seeing this question is to identify $\mathcal D(\Ab)$ with $\Sp(\mathrm{sAb})$. The Dold-Kan equivalence is then induced by the forgetful functor $\Omega^\infty \colon \Sp(\mathrm{sAb}) \to \mathrm{sAb}$.
\end{rem}

We now consider a special case of interest: namely, we will suppose that $\cX$ is the smooth-\'etale site of some derived Artin stack $X$.
In this case, we have a forgetful functor
\[ U \colon \QCoh(X)^{\le 0} \to \Sh_{\mathcal D^{\le 0}(\Ab)}(\cX) , \]
that allows to see a quasi-coherent sheaf $\cF$ on $X$ as a (higher) Picard stack on $X$.

On the other hand, suppose that $\cE \in \QCoh(X)$.
We can associate to $\cE$ a (higher) Picard stack in the following way:

\begin{defin}
	Let $\cE \in \QCoh(X)$.
	We set
	\[ \mathbb V(\cE) \coloneqq \Spec_X(\Sym_{\cO_X}(\cE)) . \]
\end{defin}

\begin{rem}
	In other words, for any $u \colon \Spec(A) \to X$, one has
	\begin{align*}
		\Map_{/X}(\Spec(A), \mathbb V(\cE)) & \simeq \Map_{\CAlg(\cO_X)}(\Sym_{\cO_X}(\cE), u_* \cO_A) \\
		& \simeq \Map_{\QCoh(X)}(\cE, u_* \cO_A) \simeq \Map_{\QCoh(A)}(u^*(\cE), \cO_A) .
	\end{align*}
\end{rem}

As $\QCoh(A)$ is naturally enriched in $\mathcal D(\Ab)$ and since for any $\cF, \cG \in \QCoh(A)$ we have
\[ \Map_{\QCoh(A)}(\cF, \cG) \simeq \tau_{\le 0} \Map_{\QCoh(A)}^{\mathcal D(\Ab)}(\cF, \cG) , \]
it is then clear that $\mathbb V(E)$ defines a (higher) Picard stack on $X$.

\begin{rem}
	Suppose that $\cE \in \QCoh(X)^{\le n}$, with $n \ge 0$.
	Then $\mathbb V(\cE)$ is $n$-truncated in the sense that $\trunc(\mathbb V(\cE))$ takes values in $n$-homotopy types.
\end{rem}

In view of the above considerations, the following question is a natural one:

\begin{question}
	Is there a reasonable full subcategory $\cC$ of $\QCoh(X)$ and a functor $F \colon \cC \to \QCoh(X)^{\le 0}$ such that the diagram
	\[ \begin{tikzcd}
		{} & & \QCoh(X)^{\le 0} \arrow{d}{U} \\
		\cC \arrow[dashed, bend left = 20]{urr}{F} \arrow[hook]{r} & \QCoh(X) \arrow{r}{\mathbb V(-)} & \Sh_{\cD^{\le 0}(\Ab)}(X)
	\end{tikzcd} \]
	commutes?
\end{question}

The answer is positive:

\begin{prop}
	Let $\cC = \Perf(X)^{\ge 0}$ be the category of perfect complexes on $X$ that are in positive cohomological amplitude.
	Let $F \coloneqq (-)^\vee \colon \Perf(X)^{\ge 0} \to \QCoh(A)^{\le 0}$ be the duality functor:
	\[ \cE^\vee \coloneqq \Hom_{\cO_X}(\cE, \cO_X) . \]
	Then the above diagram commutes.
\end{prop}

\begin{proof}
	Let $u \colon \Spec(A) \to X$ be a fixed map and let $\cE \in \Perf(X)^{\ge 0}$.
	Then, since $\cE$ is perfect, we have:
	\[ \Map_{/X}(\Spec(A), \mathbb V(\cE)) \simeq \Map_{\QCoh(A)}(u^*(\cE), \cO_A) \simeq \Map_{\QCoh(A)}(\cO_A, u^*(\cE^\vee)) . \]
	Observe now that we can identify $\Map_{\QCoh(A)}(\cO_A, u^*(\cE^\vee))$ with the underlying complex of abelian groups of $u^*(\cE^\vee)$.
	In other words, it coincides by definition with $U(\cE^\vee)(\Spec(A))$.
\end{proof}

Let now $\cF \in \Perf^{[-1,0]}(X)$ be a perfect complex in tor-amplitude $[-1,0]$.
Then $\cF[-1] \in \Perf^{\ge 0}(X)$ and therefore the above proposition supplies us with an equivalence
\[ \mathbb V(\cF[-1]) \simeq U(\cF^\vee[1]) . \]

\begin{prop}
	The stack $U(\cF^\vee [1])$ coincides with the stack $(h^1 / h^0)(\cF^\vee)$ of \cite{Behrend_Intrinsic_normal_cone_1997}.
\end{prop}

\begin{proof}
	This follows tautologically if one believes to the claim at the beginning of \cite[\S 2]{Behrend_Intrinsic_normal_cone_1997} that $(h^1 / h^0)(\cF^\vee)$ coincides with the construction $\mathrm{ch}(-)$ performed in \cite[Expos\'e XVIII, \S 1.4]{SGA4}.
	\todo{Understand why this claim is true.}
\end{proof}


\section{Stable maps}
Following \cite{bm_stacks}, let $\mathcal{M}(V,\tau,\beta)$ the moduli space of maps from a Riemann 
surface of type $\tau$ to an algebraic variety $V$,
such that $\mu^*[C] \cong \beta$. Here $\tau$ is a graph with edges labeled by $g,n$; think of it as a type of degeneracy
for Riemann surfaces, which becomes a boundary stratum in the moduli space of stable maps $\mathcal{M}_{g,n}(X)$. 
Each edge in the graph $\tau$ corresponds to an irreducible Riemann surface with genus $g$ 
and number of makred points $n$, while nodes of the graph correspond to intersections of such.

Consider the following diagram of stacks over $\mathcal{M}_{g,n}(X)$.
\[
\begin{tikzcd}
\mathcal{C}(V,\tau,\beta)\arrow{d}{\pi} \arrow{dr}{f} & \\ \mathcal{M}(V,\tau,\beta) & V .
\end{tikzcd}
\]

\begin{prop}
\label{prop:stable_maps_standard_obstruction}
$\R\pi_*(f^*T_V)^{\vee} \to \bbL_{\mathcal{M}(\tau,V,\beta)}$ is a perfect obstruction theory.
\end{prop}

\begin{rem}
Note that if we take
$V$ a point, then $\mathcal{M}(\tau,V,\beta) = \mathcal{M}(\tau)$, and we obtain $0 \to \bbL_{\mathcal{M}(\tau)}$.
But an obstruction theory is supposed to be an isomorphism in degree 0: does this hold here?
\todo{someone figure this out}
\end{rem}

\begin{proof}
We have the following sequence of maps:
\[	f^*\bbL_V \to \bbL_C \to \pi^* \bbL_{\mathcal{M}(\tau,V,\beta)}.	\]
Upon tensoring with the canonical sheaf of $C$, this becomes:
\[	f^*\bbL_V \otimes \omega_C \to	\]
\todo{someone finish this argument}
\end{proof}

Behrend proves that these satisfy a bunch of axioms, and then Behrend-Manin in \cite{bm_stacks} prove that a 
family of VFCs with said axioms
give a system of GW invariants a la Konstsevich-Manin.\todo{someone write this up}

\section{Orientations and Gromov--Witten Invariants}

\todo{Explain Thm 9.3 in Behrend Manin, so basically the idea that we can get invariants from and Orientation, Define an Orientation}

