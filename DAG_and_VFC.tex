\pdfoutput=1
%The other issue is that some packages, such as microtype, produce different output under pdflatex. By default the arXiv goes from dvi to ps to pdf, so if you need pdflatex you have to set the \pdfoutput flag in the TeX file.
\newif\ifpersonal
\newif\ifarxiv
\personaltrue % comment to remove personal notes
\arxivtrue % comment to display shortened version for journal submissions
\RequirePackage[l2tabu, orthodox]{nag} %detect whether obsolete packages are used
\documentclass[10pt,a4paper,reqno,oneside]{book} %reqno places equation numbers on the right
\linespread{1.2}
%\allowdisplaybreaks[1]
\usepackage{calligra}
\usepackage{amsmath,amsthm,amssymb,mathrsfs,mathtools,bm,eucal,tensor} % math related
\usepackage{microtype,fixltx2e} % latex technical issues
\usepackage[scaled]{beramono,berasans}
\usepackage{enumerate,comment,braket,xspace,tikz-cd} %utilities
\usepackage[all,cmtip]{xy} % utilities
\usepackage[utf8]{inputenc} % input encoding
\usepackage[T1]{fontenc} % font encoding
\usepackage{lmodern}
\definecolor{linkcolor}{HTML}{005050}
\usepackage[centering,vscale=0.7,hscale=0.7]{geometry}
\usepackage{hyperref}
\usepackage[capitalize]{cleveref}
\usepackage{graphicx}
\usepackage{xparse}
\usepackage{url}

%\makeevenhead{headings}{\thepage}{}{\leftmark}
%\setlrmarginsandblock{3cm}{3.5cm}{*}
%\setlength\marginparwidth{2.5cm}
%\checkandfixthelayout
%
%\setlength\headheight{24pt}

\usepackage{vmargin}
\setpapersize{A4}
\setmarginsrb{25mm}{10mm}{25mm}{10mm}%
{12mm}{10mm}{5mm}{10mm}

\usepackage{fancyhdr}
\pagestyle{fancy}
%%%Settings
\renewcommand{\chaptermark}[1]{\markboth{#1}{}}
\renewcommand{\sectionmark}[1]{\markright{\thesection\ #1}}
\fancyhf{}
\fancyhead[LE,RO]{\bfseries\thepage}
\fancyhead[RE]{\bfseries\footnotesize\nouppercase{\leftmark}}
\fancyhead[LO]{\bfseries\footnotesize\nouppercase{\rightmark}}

\theoremstyle{plain}
\newtheorem{thm-intro}{Theorem}
\newtheorem{thm}{Theorem}[section]
\newtheorem*{thm*}{Theorem}
\newtheorem{lem}[thm]{Lemma}
\newtheorem*{lem*}{Lemma}
\newtheorem{prop}[thm]{Proposition}
\newtheorem{conj}[thm]{Conjecture}
\newtheorem{cor}[thm]{Corollary}
\newtheorem{cor-intro}[thm-intro]{Corollary}
\newtheorem{assumption}[thm]{Assumption}
\theoremstyle{definition}
\newtheorem{defin}[thm]{Definition}
\newtheorem{question}[thm]{Question}
\newtheorem{exercise}[thm]{Exercise}
\newtheorem{defin-intro}[thm-intro]{Definition}
\newtheorem{notation}[thm]{Notation}
\theoremstyle{remark}
\newtheorem*{rem*}{Remark}
\newtheorem{eg}[thm]{Example}
\newtheorem{eg-intro}[thm-intro]{Example}
\newtheorem{rem}[thm]{Remark}
\newtheorem{rem-intro}[thm-intro]{Remark}
\numberwithin{equation}{section}
\newtheorem{construction}[thm]{Construction}

% personal remarks

\ifpersonal
\newcommand*{\personal}[1]{\textcolor[rgb]{0.6,0.6,1}{(Personal: #1)}}
\newcommand*{\todo}[1]{\textcolor{red}{(Todo: #1)}}
\else
\newcommand*{\personal}[1]{\ignorespaces}
\newcommand*{\todo}[1]{\ignorespaces}
\fi

% Fonts
\newcommand{\C}{\mathbb C}
\newcommand{\CP}{\mathbb{CP}}
\newcommand{\F}{\mathbb F}
\newcommand{\Q}{\mathbb Q}
\newcommand{\R}{\mathbb R}
\newcommand{\Z}{\mathbb Z}
\newcommand{\N}{\mathbb N}

\newcommand{\rB}{\mathrm B}
\newcommand{\rD}{\mathrm D}
\newcommand{\rH}{\mathrm H}
\newcommand{\rI}{\mathrm I}
\newcommand{\rL}{\mathrm L}
\newcommand{\rP}{\mathrm P}
\newcommand{\rQ}{\mathrm Q}
\newcommand{\rR}{\mathrm R}
\newcommand{\rb}{\mathrm b}
\newcommand{\rd}{\mathrm d}
\newcommand{\rh}{\mathrm h}
\newcommand{\rs}{\mathrm s}
\newcommand{\rt}{\mathrm t}


\newcommand{\fA}{\mathfrak A}
\newcommand{\fB}{\mathfrak B}
\newcommand{\fC}{\mathfrak C}
\newcommand{\fD}{\mathfrak D}
\newcommand{\fH}{\mathfrak H}
\newcommand{\fS}{\mathfrak S}
\newcommand{\fT}{\mathfrak T}
\newcommand{\fU}{\mathfrak U}
\newcommand{\fV}{\mathfrak V}
\newcommand{\fX}{\mathfrak X}
\newcommand{\fY}{\mathfrak Y}
\newcommand{\fZ}{\mathfrak Z}
\newcommand{\ff}{\mathfrak f}
\newcommand{\fm}{\mathfrak m}
\newcommand{\fn}{\mathfrak n}
\newcommand{\fs}{\mathfrak s}
\newcommand{\ft}{\mathfrak t}

\newcommand{\cA}{\mathcal A}
\newcommand{\cB}{\mathcal B}
\newcommand{\cC}{\mathcal C}
\newcommand{\cD}{\mathcal D}
\newcommand{\cE}{\mathcal E}
\newcommand{\cF}{\mathcal F}
\newcommand{\cH}{\mathcal H}
\newcommand{\cG}{\mathcal G}
\newcommand{\cI}{\mathcal I}
\newcommand{\cJ}{\mathcal J}
\newcommand{\cK}{\mathcal K}
\newcommand{\cL}{\mathcal L}
\newcommand{\cM}{\mathcal M}
\newcommand{\cN}{\mathcal N}
\newcommand{\cO}{\mathcal O}
\newcommand{\cP}{\mathcal P}
\newcommand{\cR}{\mathcal R}
\newcommand{\cS}{\mathcal S}
\newcommand{\cT}{\mathcal T}
\newcommand{\cU}{\mathcal U}
\newcommand{\cV}{\mathcal V}
\newcommand{\cW}{\mathcal W}
\newcommand{\cX}{\mathcal X}
\newcommand{\cY}{\mathcal Y}
\newcommand{\cZ}{\mathcal Z}
\DeclareFontFamily{U}{BOONDOX-calo}{\skewchar\font=45 }
\DeclareFontShape{U}{BOONDOX-calo}{m}{n}{<-> s*[1.05] BOONDOX-r-calo}{}
\DeclareFontShape{U}{BOONDOX-calo}{b}{n}{<-> s*[1.05] BOONDOX-b-calo}{}
\DeclareMathAlphabet{\mathcalboondox}{U}{BOONDOX-calo}{m}{n}
%\DeclareMathAlphabet{\mathcalligra}{T1}{calligra}{m}{n}
\newcommand{\cf}{\mathcalboondox f}

\newcommand{\bbA}{\mathbb A}
\newcommand{\bbD}{\mathbb D}
\newcommand{\bbG}{\mathbb G}
\newcommand{\bbL}{\mathbb L}
\newcommand{\bbP}{\mathbb P}
\newcommand{\bbT}{\mathbb T}
\newcommand{\bbV}{\mathbb V}

\newcommand{\bA}{\mathbf A}
\newcommand{\bD}{\mathbf D}
\newcommand{\bP}{\mathbf P}
\newcommand{\bQ}{\mathbf Q}
\newcommand{\bT}{\mathbf T}
\newcommand{\bX}{\mathbf X}
\newcommand{\bY}{\mathbf Y}
\newcommand{\be}{\mathbf e}
\newcommand{\br}{\mathbf r}
\newcommand{\bu}{\mathbf u}
\newcommand{\balpha}{\bm{\alpha}}
\newcommand{\bDelta}{\bm{\Delta}}
\newcommand{\brho}{\bm{\rho}}

\newcommand{\sC}{\mathscr C}
\newcommand{\sX}{\mathscr X}
\newcommand{\sD}{\mathscr D}
\newcommand{\sU}{\mathscr U}


% Decorations

% Definition of \widebar from http://tex.stackexchange.com/questions/16337/can-i-get-a-widebar-without-using-the-mathabx-package/60253#60253
\makeatletter
\let\save@mathaccent\mathaccent
\newcommand*\if@single[3]{%
	\setbox0\hbox{${\mathaccent"0362{#1}}^H$}%
	\setbox2\hbox{${\mathaccent"0362{\kern0pt#1}}^H$}%
	\ifdim\ht0=\ht2 #3\else #2\fi
}
%The bar will be moved to the right by a half of \macc@kerna, which is computed by amsmath:
\newcommand*\rel@kern[1]{\kern#1\dimexpr\macc@kerna}
%If there's a superscript following the bar, then no negative kern may follow the bar;
%an additional {} makes sure that the superscript is high enough in this case:
\newcommand*\widebar[1]{\@ifnextchar^{{\wide@bar{#1}{0}}}{\wide@bar{#1}{1}}}
%Use a separate algorithm for single symbols:
\newcommand*\wide@bar[2]{\if@single{#1}{\wide@bar@{#1}{#2}{1}}{\wide@bar@{#1}{#2}{2}}}
\newcommand*\wide@bar@[3]{%
	\begingroup
	\def\mathaccent##1##2{%
		%Enable nesting of accents:
		\let\mathaccent\save@mathaccent
		%If there's more than a single symbol, use the first character instead (see below):
		\if#32 \let\macc@nucleus\first@char \fi
		%Determine the italic correction:
		\setbox\z@\hbox{$\macc@style{\macc@nucleus}_{}$}%
		\setbox\tw@\hbox{$\macc@style{\macc@nucleus}{}_{}$}%
		\dimen@\wd\tw@
		\advance\dimen@-\wd\z@
		%Now \dimen@ is the italic correction of the symbol.
		\divide\dimen@ 3
		\@tempdima\wd\tw@
		\advance\@tempdima-\scriptspace
		%Now \@tempdima is the width of the symbol.
		\divide\@tempdima 10
		\advance\dimen@-\@tempdima
		%Now \dimen@ = (italic correction / 3) - (Breite / 10)
		\ifdim\dimen@>\z@ \dimen@0pt\fi
		%The bar will be shortened in the case \dimen@<0 !
		\rel@kern{0.6}\kern-\dimen@
		\if#31
		\overline{\rel@kern{-0.6}\kern\dimen@\macc@nucleus\rel@kern{0.4}\kern\dimen@}%
		\advance\dimen@0.4\dimexpr\macc@kerna
		%Place the combined final kern (-\dimen@) if it is >0 or if a superscript follows:
		\let\final@kern#2%
		\ifdim\dimen@<\z@ \let\final@kern1\fi
		\if\final@kern1 \kern-\dimen@\fi
		\else
		\overline{\rel@kern{-0.6}\kern\dimen@#1}%
		\fi
	}%
	\macc@depth\@ne
	\let\math@bgroup\@empty \let\math@egroup\macc@set@skewchar
	\mathsurround\z@ \frozen@everymath{\mathgroup\macc@group\relax}%
	\macc@set@skewchar\relax
	\let\mathaccentV\macc@nested@a
	%The following initialises \macc@kerna and calls \mathaccent:
	\if#31
	\macc@nested@a\relax111{#1}%
	\else
	%If the argument consists of more than one symbol, and if the first token is
	%a letter, use that letter for the computations:
	\def\gobble@till@marker##1\endmarker{}%
	\futurelet\first@char\gobble@till@marker#1\endmarker
	\ifcat\noexpand\first@char A\else
	\def\first@char{}%
	\fi
	\macc@nested@a\relax111{\first@char}%
	\fi
	\endgroup
}
\makeatother


\newcommand{\oDelta}{\widebar\Delta}
\newcommand{\oGamma}{\widebar\Gamma}
\newcommand{\oSigma}{\widebar\Sigma}
\newcommand{\oalpha}{\widebar\alpha}
\newcommand{\obeta}{\widebar\beta}
\newcommand{\otau}{\widebar\tau}
\newcommand{\oC}{\widebar C}
\newcommand{\oD}{\widebar D}
\newcommand{\oE}{\widebar E}
\newcommand{\oG}{\widebar G}
\newcommand{\oM}{\widebar M}
\newcommand{\oR}{\widebar R}
\newcommand{\oS}{\widebar S}
\newcommand{\oU}{\widebar U}
\newcommand{\oW}{\widebar W}
\newcommand{\oX}{\widebar X}
\newcommand{\oY}{\widebar Y}
\newcommand{\oPhi}{\overline{\Phi}}


\newcommand{\ok}{\widebar k}
\newcommand{\ov}{\widebar v}
\newcommand{\ox}{\widebar x}
\newcommand{\oy}{\widebar y}
\newcommand{\oz}{\widebar z}

\newcommand{\hh}{\widehat h}
\newcommand{\hf}{\widehat f}
\newcommand{\hA}{\widehat A}
\newcommand{\hB}{\widehat B}
\newcommand{\hC}{\widehat C}
\newcommand{\hE}{\widehat E}
\newcommand{\hF}{\widehat F}
\newcommand{\hI}{\widehat I}
\newcommand{\hL}{\widehat L}
\newcommand{\hU}{\widehat U}
\newcommand{\hZ}{\hat Z}
\newcommand{\hbeta}{\widehat\beta}
\newcommand{\hGamma}{\widehat\Gamma}
\newcommand{\hPhi}{\widehat{\Phi}}
\newcommand{\hPsi}{\widehat{\Psi}}

\newcommand{\hrP}{\widehat \rP}

\newcommand{\tw}{\widetilde w}
\newcommand{\tW}{\widetilde W}
\newcommand{\tk}{\tilde k}
\newcommand{\tv}{\tilde v}
\newcommand{\tB}{\widetilde B}
\newcommand{\tD}{\widetilde D}
\newcommand{\tI}{\widetilde I}
\newcommand{\tM}{\widetilde M}
\newcommand{\tN}{\widetilde N}
\newcommand{\tP}{\widetilde P}
\newcommand{\tR}{\widetilde R}
\newcommand{\tX}{\widetilde X}
\newcommand{\tfX}{\widetilde{\fX}}
\newcommand{\tfB}{\widetilde{\fB}}
\newcommand{\tsX}{\widetilde{\sX}}
\newcommand{\tH}{\widetilde H}
\newcommand{\tY}{\widetilde Y}
\newcommand{\tbeta}{\widetilde{\beta}}
\newcommand{\tphi}{\widetilde{\phi}}
\newcommand{\ttau}{\widetilde{\tau}}

% Global tropicalization
\newcommand{\Ih}{I^\mathrm{h}}
\newcommand{\Iv}{I^\mathrm{v}}
\newcommand{\IX}{I_\fX}
\newcommand{\IY}{I_\fY}
\newcommand{\SD}{S_\fD}
\newcommand{\SX}{S_\fX}
\newcommand{\SsXH}{S_{(\sX,H)}}
\newcommand{\SY}{S_\fY}
\newcommand{\CsXH}{C_{(\sX,H)}}
\newcommand{\oF}{\overline{F}}
\newcommand{\oP}{\overline{P}}
\newcommand{\oSX}{\overline{\SX}}
\newcommand{\oIX}{\overline{\IX}}


% Vanishing cycles
\newcommand{\fXe}{\fX_\eta}
\newcommand{\fXs}{\fX_s}
\newcommand{\ofX}{\widebar{\fX}}
\newcommand{\ofXs}{\widebar{\fX}_s}
\newcommand{\fYe}{\fY_\eta}
\newcommand{\fYs}{\fY_s}
\newcommand{\fXbs}{\fX_{\bar s}}
\newcommand{\fXbe}{\fX_{\bar\eta}}
\newcommand{\fDe}{\fD_\eta}
\newcommand{\LX}{\Lambda_{\fX}}
\newcommand{\LXe}{\Lambda_{\fX_\eta}}
\newcommand{\LXs}{\Lambda_{\fXbs}}
\newcommand{\QXe}{\Q_{\ell,\fX_\eta}}
\newcommand{\QXbs}{\Q_{\ell,\fXbs}}
\newcommand{\sXe}{\sX_\eta}
\newcommand{\sXs}{\sX_s}
\newcommand{\LUe}{\Lambda_{\fU_\eta}}
\newcommand{\fCbs}{\fC_{\bar s}}
\newcommand{\QUe}{\Q_{\ell,\fU_\eta}}
\newcommand{\QCe}{\Q_{\ell,\fC_\eta}}
\newcommand{\QCs}{\Q_{\ell,\fCbs}}

% stacks

\newcommand{\hcC}{\mathrm h\cC}
\newcommand{\hcD}{\mathrm h\cD}
\newcommand{\PSh}{\mathrm{PSh}}
\newcommand{\Sh}{\mathrm{Sh}}
\newcommand{\Shv}{\mathrm{Shv}}
\newcommand{\Tuupperp}{\tensor*[^\cT]{u}{^p}}
\newcommand{\Tulowerp}{\tensor*[^\cT]{u}{_p}}
\newcommand{\Tuuppers}{\tensor*[^\cT]{u}{^s}}
\newcommand{\Tulowers}{\tensor*[^\cT]{u}{_s}}
\newcommand{\pu}{\tensor*[_p]{u}{}}
\newcommand{\su}{\tensor*[_s]{u}{}}
\newcommand{\Tpu}{\tensor*[^\cT_p]{u}{}}
\newcommand{\Tsu}{\tensor*[^\cT_s]{u}{}}
\newcommand{\Dfpull}{\tensor*[^\cD]{f}{^{-1}}}
\newcommand{\Dfpush}{\tensor*[^\cD]{f}{_*}}
\newcommand{\Duuppers}{\tensor*[^\cD]{u}{^s}}
\newcommand{\Dulowers}{\tensor*[^\cD]{u}{_s}}
\newcommand{\Geom}{\mathrm{Geom}}
\newcommand{\LPr}{\mathcal{P}\mathrm{r}^\rL}
\newcommand{\RPr}{\mathcal{P}\mathrm{r}^\rR}
\newcommand{\LPromega}{\mathcal{P}\mathrm{r}^{\rL, \omega}}
\newcommand{\LPromegast}{\mathcal{P}\mathrm{r}^{\rL, \omega}_{\mathrm{Ex}}}
\newcommand{\CX}{\cC_{/X}}
\newcommand{\CY}{\cC_{/Y}}
\newcommand{\CXP}{(\cC_{/X})_{\bP}}
\newcommand{\GeomXP}{(\mathrm{Geom}_{/X})_\bP}
\newcommand{\GeomYP}{(\mathrm{Geom}_{/Y})_\bP}
\newcommand{\infcat}{$\infty$-category\xspace}
\newcommand{\infcats}{$\infty$-categories\xspace}
\newcommand{\infsite}{$\infty$-site\xspace}
\newcommand{\infsites}{$\infty$-sites\xspace}
\newcommand{\inftopos}{$\infty$-topos\xspace}
\newcommand{\inftopoi}{$\infty$-topoi\xspace}
\newcommand{\pres}{{}^{\mathrm L} \mathcal P \mathrm{res}}
\newcommand{\Grpd}{\mathrm{Grpd}}
\newcommand{\sSet}{\mathrm{sSet}}
\newcommand{\rSet}{\mathrm{Set}}
\newcommand{\Ab}{\mathrm{Ab}}
\newcommand{\DAb}{\cD(\Ab)}
\newcommand{\tauan}{\tau_\mathrm{an}}
\newcommand{\qet}{\mathrm{q\acute{e}t}}
\newcommand{\tauet}{\tau_\mathrm{\acute{e}t}}
\newcommand{\tauqet}{\tau_\mathrm{q\acute{e}t}}
\newcommand{\bPsm}{\bP_\mathrm{sm}}
\newcommand{\bPqsm}{\bP_\mathrm{qsm}}
\newcommand{\Modh}{\textrm{-}\mathrm{Mod}^\heartsuit}
\newcommand{\Mod}{\textrm{-}\mathrm{Mod}}
\newcommand{\Coh}{\mathrm{Coh}}
\newcommand{\Cohb}{\mathrm{Coh}^\mathrm{b}}
\newcommand{\Cohh}{\mathrm{Coh}^\heartsuit}
\newcommand{\QCohh}{\mathrm{QCoh}^\heartsuit}
\newcommand{\RcHom}{\rR\!\mathcal H\!\mathit{om}}
\newcommand{\kfiltered}{$\kappa$-filtered\xspace}
\newcommand{\Stn}{\mathrm{Stn}}
\newcommand{\Sch}{\mathrm{Sch}}
\newcommand{\FSch}{\mathrm{FSch}}
\newcommand{\Aff}{\mathrm{Aff}}
\newcommand{\Afflfp}{\mathrm{Aff}^{\mathrm{lfp}}}
\newcommand{\An}{\mathrm{An}}
\newcommand{\Afd}{\mathrm{Afd}}
\newcommand{\Top}{\mathcal T\mathrm{op}}
\newcommand{\bfMap}{\mathbf{Map}}



% DAnG

\newcommand{\dAnk}{\mathrm{dAn}_k}
\newcommand{\Ank}{\mathrm{An}_k}
\newcommand{\cTan}{\cT_{\mathrm{an}}}
\newcommand{\cTannc}{\cT_{\mathrm{an}}^{\mathrm{nc}}}
\newcommand{\cTank}{\cT_{\mathrm{an}}(k)}
\newcommand{\cTdisck}{\cT_{\mathrm{disc}}(k)}
\newcommand{\cTet}{\cT_{\mathrm{\acute{e}t}}}
\newcommand{\cTetnc}{\cTet^{\mathrm{nc}}}
\newcommand{\cTetk}{\cT_{\mathrm{\acute{e}t}}(k)}
\newcommand{\Strloc}{\mathrm{Str}^\mathrm{loc}}
\newcommand{\RTop}{\tensor*[^\rR]{\Top}{}}
\newcommand{\LTop}{\tensor*[^\rL]{\Top}{}}
\newcommand{\RHTop}{\tensor*[^\rR]{\mathcal{H}\Top}{}}
\newcommand{\LRT}{\mathrm{LRT}}
\newcommand{\Tor}{\mathrm{Tor}}
\newcommand{\dAfd}{\mathrm{dAfd}}
\newcommand{\dAfdk}{\mathrm{dAfd}_k}
\newcommand{\biget}{\mathrm{big,\acute{e}t}}
\newcommand{\trunc}{\mathrm{t}_0}
\newcommand{\Hyp}{\mathrm{Hyp}}
\newcommand{\HSpec}{\mathrm{HSpec}}
\newcommand{\CAlg}{\mathrm{CAlg}}
\newcommand{\trunctopoi}{\Spec^{\cG_{\mathrm{an}}^{\le 0}(k)}_{\cG_{\mathrm{an}(k)}}}

% Formal Gluing

\newcommand{\IndPro}[1]{\mathrm{Ind}(\mathrm{Pro}(#1))}
\newcommand{\GFRings}{\mathrm{GFRings}}
\newcommand{\Pro}{\mathrm{Pro}}
\newcommand{\Ind}{\mathrm{Ind}}
\newcommand{\preNbd}{\mathrm{PNbd}}
\newcommand{\Nbd}{\mathrm{Nbd}^{\circ}}
\newcommand{\cHom}{\cH \mathrm{om}}
\newcommand {\D} {\mathsf{L}}
\newcommand{\St}{\mathbf{St}}
\newcommand{\dSt}{\mathbf{dSt}}
\newcommand{\Tw}{\mathrm{Tw}}
\newcommand{\Lan}{\mathrm{Lan}}
\newcommand{\IndCoh}{\mathrm{IndCoh}}
\newcommand{\QCoh}{\mathrm{QCoh}}
\newcommand{\Perf}{\mathrm{Perf}}
\newcommand{\lex}{\mathrm{lex}}
\newcommand{\Dsing}{\rD_\mathrm{sing}}
\newcommand{\fib}{\mathrm{fib}}
\newcommand{\cofib}{\mathrm{cofib}}
\newcommand{\stMap}{\mathrm{Map}^{\mathrm{st}}}
\newcommand{\Zar}{\mathrm{Zar}}
\newcommand{\Cat}{\mathrm{Cat}}
\newcommand{\AbCat}{\mathrm{AbCat}}
\newcommand{\bfCoh}{\mathbf{Coh}}
\newcommand{\bfPerf}{\mathbf{Perf}}
\newcommand{\bfQCoh}{\mathbf{QCoh}}
\newcommand{\Catst}{\Cat_\infty^{\mathrm{Ex}}}
\newcommand{\Catstidem}{\Cat_\infty^{\mathrm{Ex}, \mathrm{idem}}}
\newcommand{\Catstlc}{\Cat_\infty^{\mathrm{Ex}, \mathrm{l.c.}}}
\newcommand{\Catstlb}{\Cat_\infty^{\mathrm{Ex}, \mathrm{l.b.}}}

\newcommand{\bfBun}{\operatorname{\mathbf{Bun}}}
\newcommand{\Bun}{\operatorname{\mathrm{Bun}}}
\newcommand{\Bunhat}{\operatorname{\mathbf{B\widehat{un}}}}

% Special symbols
\newcommand{\bcM}{\widebar{\mathcal M}}
\newcommand{\bcC}{\widebar{\mathcal C}}
\newcommand{\bcMgn}{\widebar{\mathcal M}_{g,n}}
\newcommand{\bcMol}{\widebar{\mathcal M}_{0,1}}
\newcommand{\bcMot}{\widebar{\mathcal M}_{0,3}}
\newcommand{\bcMof}{\widebar{\mathcal M}_{0,4}}
\newcommand{\bcMon}{\widebar{\mathcal M}_{0,n}}
\newcommand{\bcMgnprime}{\widebar{\mathcal M}_{g,n'}}
\newcommand{\bcMgnijprime}{\widebar{\mathcal M}_{g_{ij},n'_{ij}}}
\newcommand{\bMgnt}{\widebar{M}^\mathrm{trop}_{g,n}}
\newcommand{\Mmdisc}{M_{m\textrm{-disc}}}
\newcommand{\Gm}{\mathbb G_{\mathrm m}}
\newcommand{\Gmk}{\mathbb G_{\mathrm m/k}}
\newcommand{\Gmkprime}{\mathbb G_{\mathrm m/k'}}
\newcommand{\Gmnan}{(\Gm^n)\an}
\newcommand{\Gmknan}{(\Gmk^n)\an}
\newcommand{\Lin}{\mathit{Lin}}
\newcommand{\Simp}{\mathit{Simp}}
\newcommand{\vol}{\mathit{vol}}
\newcommand{\LanD}{\mathcal L_{an}^D}

% Categories


% Shorthands
\newcommand{\kc}{k^\circ}
\newcommand{\llb}{[\![}
\newcommand{\rrb}{]\!]}
\newcommand{\llp}{(\!(}
\newcommand{\rrp}{)\!)}
\newcommand{\an}{^\mathrm{an}}
\newcommand{\alg}{^\mathrm{alg}}
\newcommand{\loweralg}{_\mathrm{alg}}
\newcommand{\bad}{^\mathrm{bad}}
\newcommand{\ess}{^\mathrm{ess}}
\newcommand{\ness}{^\mathrm{ness}}
\newcommand{\et}{_\mathrm{\acute{e}t}}
\newcommand{\Et}{_\mathrm{\acute{E}t}}
\newcommand{\ev}{\mathrm{ev}}
%\newcommand{\eistar}{\mathbf e_i^*}
%\newcommand{\ejstar}{\mathbf e_j^*}
%\newcommand{\ekstar}{\mathbf e_k^*}
\newcommand{\mult}{\mathit{mult}}
\newcommand{\inv}{^{-1}}
\newcommand{\id}{\mathrm{id}}
\newcommand{\gn}{$n$-pointed genus $g$ }
\newcommand{\gnprime}{$n'$-pointed genus $g$ }
\newcommand{\GW}{\mathrm{GW}}
\newcommand{\GWon}{\GW_{0,n}}
\newcommand{\canal}{$\mathbb C$-analytic\xspace}
\newcommand{\nanal}{non-archimedean analytic\xspace}
\newcommand{\kanal}{$k$-analytic\xspace}
\newcommand{\ddim}{$d$-dimensional\xspace}
\newcommand{\ndim}{$n$-dimensional\xspace}
\newcommand{\narch}{non-archimedean\xspace}
\newcommand{\nminusone}{$(n\!-\!1)$}
\newcommand{\nminustwo}{$(n\!-\!2)$}
\newcommand{\red}{^\mathrm{red}}
\renewcommand{\th}{^\mathrm{\tiny th}}
\newcommand{\Wall}{\mathit{Wall}}
\newcommand{\vlb}{virtual line bundle\xspace}
\newcommand{\mvlb}{metrized \vlb}
\newcommand{\wrt}{with respect to\xspace}
\newcommand{\Zaffine}{$\mathbb Z$-affine\xspace}
\newcommand{\sw}{^\mathrm{sw}}
\newcommand{\Trop}{\mathrm{Trop}}
\newcommand{\trop}{^\mathrm{trop}}
\newcommand{\op}{^\mathrm{op}}
\newcommand{\Cech}{\check{\mathcal C}}
\newcommand{\DM}{Deligne-Mumford\xspace}
\providecommand{\abs}[1]{\lvert#1\rvert}
\providecommand{\norm}[1]{\lVert#1\rVert}
\newcommand{\fr}{\mathfrak}
\newcommand{\p}{\partial}



% Arrows
\newcommand*{\longhookrightarrow}{\ensuremath{\lhook\joinrel\relbar\joinrel\rightarrow}}
\newcommand*{\DashedArrow}[1][]{\mathbin{\tikz [baseline=-0.25ex,-latex, dashed,#1] \draw [#1] (0pt,0.5ex) -- (1.3em,0.5ex);}}

\usetikzlibrary{decorations.markings} %arrows for open immersions and closed immersions
\tikzset{
  closed/.style = {decoration = {markings, mark = at position 0.5 with { \node[transform shape, xscale = .8, yscale=.4] {/}; } }, postaction = {decorate} },
  open/.style = {decoration = {markings, mark = at position 0.5 with { \node[transform shape, scale = .7] {$\circ$}; } }, postaction = {decorate} }
}


%Operators
\DeclareMathOperator{\Alg}{Alg}
\DeclareMathOperator{\Anc}{Anc}
\DeclareMathOperator{\Area}{Area}
\DeclareMathOperator{\at}{at}
\DeclareMathOperator{\Aut}{Aut}
\DeclareMathOperator{\Bl}{Bl}
\DeclareMathOperator{\cdga}{cdga}
\DeclareMathOperator{\CH}{CH}
\DeclareMathOperator{\Ch}{Ch}
\DeclareMathOperator{\Chow}{Chow}
\DeclareMathOperator{\Coker}{Coker}
\DeclareMathOperator{\codim}{codim}
\DeclareMathOperator{\cosk}{cosk}
\DeclareMathOperator{\Der}{Der}
\DeclareMathOperator{\dgVect}{dgVect}
\DeclareMathOperator{\Div}{Div}
\DeclareMathOperator{\dist}{dist}
\DeclareMathOperator{\dMan}{dMan}
\DeclareMathOperator{\End}{End}
\DeclareMathOperator{\Ext}{Ext}
\DeclareMathOperator{\Fun}{Fun}
\DeclareMathOperator{\FunR}{Fun^R}
\DeclareMathOperator{\FunL}{Fun^L}
\DeclareMathOperator{\Gal}{Gal}
\DeclareMathOperator{\Hom}{Hom}
\DeclareMathOperator{\Image}{Im}
\DeclareMathOperator{\Int}{Int}
\DeclareMathOperator{\Isom}{Isom}
\DeclareMathOperator{\Ker}{Ker}
\DeclareMathOperator{\KurNbd}{KurNbd}
\DeclareMathOperator{\loc}{loc}
\DeclareMathOperator{\LocTopInf}{LocTopInf}
\DeclareMathOperator{\Map}{Map}
\DeclareMathOperator{\Mor}{Mor}
\DeclareMathOperator{\NE}{NE}
\DeclareMathOperator{\oStar}{\widebar{\Star}}
\DeclareMathOperator{\pt}{pt}
\DeclareMathOperator{\Pic}{Pic}
\DeclareMathOperator{\Proj}{Proj}
\DeclareMathOperator{\rank}{rank}
\DeclareMathOperator{\Res}{Res}
\DeclareMathOperator{\RHom}{RHom}
\DeclareMathOperator{\Sp}{Sp}
\DeclareMathOperator{\Spa}{Spa}
\DeclareMathOperator{\SpB}{Sp_\mathrm{B}}
\DeclareMathOperator{\Spec}{Spec}
\DeclareMathOperator{\Spf}{Spf}
\DeclareMathOperator{\Star}{Star}
\DeclareMathOperator{\supp}{supp}
\DeclareMathOperator{\Sym}{Sym}
\DeclareMathOperator{\Symp}{Symp}
\DeclareMathOperator{\Td}{Td}
\DeclareMathOperator{\Tdisc}{T_{\text{disc}}}
\DeclareMathOperator{\Tr}{Tr}
\DeclareMathOperator{\tr}{tr}
\DeclareMathOperator{\val}{val}
\DeclareMathOperator{\vdim}{vdim}
\DeclareMathOperator{\Vect}{Vect}
\DeclareMathOperator{\vir}{vir}

\DeclareMathOperator*{\hofib}{hofib}
\DeclareMathOperator*{\hocofib}{hocofib}
\DeclareMathOperator*{\colim}{colim}
\DeclareMathOperator*{\holim}{holim}
\DeclareMathOperator*{\hocolim}{hocolim}
\DeclareMathOperator*{\cotimes}{\widehat{\otimes}}


\title{Derived Algebraic Geometry Seminar: UPenn 2017}

\begin{document}

\maketitle
\tableofcontents

\chapter*{Introduction}
\addcontentsline{toc}{chapter}{Introduction} \markboth{INTRODUCTION}{}

This contains notes for the Derived Algebraic Geometry Seminar currently being held at the University of Pennsylvania 
math department in the 2016-17 academic year. Having introduced the machinery of Derived Algebraic Geometry the previous semester,
we investigate its applications to producing Virtual Fundamental Cycles. Initially we will focus on moduli spaces of stable
maps, with various boundary conditions, and how VFCs for these can be used to construct Gromov-Witten invariants and Floer-type
theories.

This is a draft and errors should be expected.




\chapter{Stable Maps and Gromov-Witten Invariants}
\label{ch:gw}

(Talk by Matei Ionita)

\section{The Counting Problem}
\label{sect:counting}
Basic idea of ennumerative geometry, as explained in \cite{invitation} 3.1: set up a moduli space M
for the objects, e.g. curves,
one wants to count: $\mathcal{M}_{g,n}(X,\beta)$, equipped with (flat) evaluation maps
$\nu_i : \mathcal{M}_{g,n}(X,\beta) \to X$, given by $\big(C, p_1, \dots p_n, \mu\big) \mapsto \mu(p_i)$. 
Each constraint $\nu_i \in \Gamma_i$, where $\Gamma_i \in H_*(X,\Z)$, gives a subscheme, of $\mathcal{M}_{g,n}(X,\beta)$.
We take the intersection of all these: 
\begin{equation}
\label{eq:intersection_count}
	\bigcap_{i=1}^m \nu_i^* \Gamma_i .
	\footnote{This pullback is an umkehr map and we need some assumptions; is properness of $\mu_i$ enough?}
\end{equation}
If the intersections are transverse and the result has dimension 0, can count the number of points. We would like to set up $\Gamma_i$
such that:
\[	\sum_{i=1}^m \codim \Gamma_i = \dim \mathcal{M}_{g,n}(X,\beta).	\]
Thus the ennumerative problem is reduced to intersection theory in M. In order to
do intersection theory successfully, M needs to be compact (proper), and we need to understand
its Chow ring, where the subschemes live.

A first modification: in order to drop the transversality assumption on $\Gamma_i$, we replace them with the Poincar\'e dual
cohomology classes $\gamma_i$, and take cup products then \ref{eq:intersection_count} is replaced by a first naive definition
of the \textbf{Gromov-Witten invariants}:
\begin{equation}
\label{eq:gw_naive_def}
	I_{g,n,\beta} := \int_{[\mathcal{M}_{g,n}(X,\beta)]} \bigwedge_i \nu_i^* \gamma_i.
\end{equation}
If $\mathcal{M}_{g,n}(X,\beta)$ is smooth and proper, then $[\mathcal{M}_{g,n}(X,\beta)]$ is the fundamental class, against
which it makes sense to evaluate cohomology classes. $I_{g,n,\beta}$ is defined to be 0 unless $\sum_i \deg \gamma_i =
\dim \mathcal{M}_{g,n}(X,\beta)$.





\section{Axiomatic Definition of GW}
\label{sect:axiom}

The axiomatic approach of Kontsevich and Manin in \cite{km_gw} is as follows. Let $\overline{\mathcal{M}}_{g,n}$ denote the
Deligne-Mumford compactification by stable curves of the moduli stack of genus $g$ curves with $n$ marked points.
We take this as a well-understood object and explain the rest.

\begin{defin}[2.2 in \cite{km_gw}]
A \textbf{system of Gromov-Witten classes for X} is a family of linear maps:
\[	I_{g,n,\beta}^X : H^*(X,\Q)^{\otimes n} \to H^*(\overline{\mathcal{M}}_{g,n},\Q)	\]
defined for $n+2g-3\geq 0$, and satisfying the following axioms.
\begin{enumerate}
\item \textbf{Effectivity}: $I_{g,n,\beta} = 0$ for $\beta$ non-effective, i.e. not in the dual of the K\"ahler cone.

\item $S_n$\textbf{-covariance}: equivariant with respect to the obvious $S_n$ action on the domain and target.

\item \textbf{Grading}: $ \deg I_{g,n,\beta} =  - 2 \int_{\beta} c_1(X) + (2-2g) \dim X$. More precisely, this means that
we set $|\gamma| = i$ for $\gamma \in H^i(X,\Q)$ and we require that:
\[	\left| I_{g,n,\beta}^X(\gamma_1, \dots, \gamma_m)\right| = \sum_{j=1}^m |\gamma_j| - 2 \int_{\beta} c_1(X) + (2g-2) \dim X. 	\]
Some comments on the grading axiom:
\begin{itemize}
\item Following the convention in \cite{km_gw}, we use the real, not complex, dimension.
\item Informally we think of $I_{g,n,\beta}^X(\gamma_1, \dots, \gamma_m)$ as obtained by pushing forward via the natural map:
\[	\mathcal{M}_{g,n}(X,\beta) \to \mathcal{M}_{g,n} .	\]
As a result, its degree is an expectation for $\dim \mathcal{M}_{g,n}- \dim \mathcal{M}_{g,n}(X,\beta)$. 
We know that $\dim \mathcal{M}_{g,n} = 2(3g-3+n)$. By deformation theory we also compute	$\vdim \mathcal{M}_{g,n}(X,\beta)$,
called the \textbf{virtual dimension}, the expected dimension whenever first-order deformations are unobstructed.

The tangent space to $\mathcal{M}_{g,n}(X, \beta)$ at a point $(C,p_1, \dots, p_n, \mu)$ is:
\[	H^1(C,T_C(-p_1-\dots-p_n)) \oplus H^0(C,\mu^*T_X).	\]
By Serre duality this is:
\[	H^0(C,\Omega^{\otimes 2}_C(p_1+\dots+p_n))^{\vee} \oplus H^0(C,\mu^*T_X).	\]
Approximating the dimensions with the Euler characteristic, we get via Riemann-Roch:
\begin{equation}
\label{eq:vdim}
	\vdim \mathcal{M}_{g,n}(X, \beta) = 2(\dim X - 3)(1-g) + 2\int_{\beta} c_1(T_X) + 2n .
\end{equation}
Substracting these we get what the grading axiom requires:
\[	\dim \mathcal{M}_{g,n}- \dim \mathcal{M}_{g,n}(X,\beta) = 2 \int_{\beta} c_1(X) - (2-2g) \dim X.	\]
\item Assume that $I_{g,n,\beta}^X(\gamma_1, \dots, \gamma_m)$ is of \textbf{codimension zero}, i.e. that:
\begin{equation}
\label{eq:codimension_zero}
\sum_{j=1}^n |\gamma_j| = 2 \int_{\beta} c_1(X) - (2-2g) \dim X.
\end{equation}
Then $\left| I_{g,n,\beta}^X(\gamma_1, \dots, \gamma_m)\right| = \dim \overline{\mathcal{M}}_{g,n}$.
We can integrate this against the fundamental class of $\overline{\mathcal{M}}_{g,n}$, which is a proper smooth
Deligne-Mumford stack. \todo{reference?} We obtain a finite number, which we take as the result of the curve count.
\end{itemize}

\item \textbf{Fundamental class.} We introduce some more terminology. Call a class \textbf{basic} if it has the smallest
$n$ which makes sense, namely:
\[	I_{0,3,\beta}^X(\gamma_1, \gamma_2, \gamma_3) \hspace{2cm} I_{1,1,\beta}^X(\gamma_1) \hspace{2cm} I_{g,0,\beta}^X
\text{ for } g\geq 2.	\]
Let $\pi: \overline{\mathcal{M}}_{g,n} \to \overline{\mathcal{M}}_{g,n-1}$ be the projection that forgets the last marked point.
Let $e_X^0 \in H^0(X,\Q)$ be the identity of the cohomology ring.
Unless the class on the LHS is basic, we require that:
\[	I^X_{g,n,\beta}(\gamma_1, \dots, \gamma_{n-1}, e_X^0) = \pi^* I^X_{g,n-1,\beta}(\gamma_1, \dots, \gamma_{n-1}).	\]
In addition, we set:
\[	I^X_{0,3,\beta}(\gamma_1, \gamma_2, e_X^0) = \left\{ \begin{array} {ll} \int_X \gamma_1 \wedge \gamma_2, & \text{if } \beta = 0, \\
0, & \text{if } \beta \neq 0. \end{array} \right.	\]

\item \textbf{Divisor.} In the case $|\gamma_n| = 2$, i.e. $\gamma_n$ is the Poincar\'e dual class of a divisor,
and if the LHS is a non-basic class, we require:
\[	\pi_{n*}I_{g,n,\beta}^X(\gamma_1, \dots, \gamma_n) = \int_{\beta}\gamma_n I_{g,n-1,\beta}^X(\gamma_1, \dots, \gamma_{n-1}).	\]

\item \textbf{Splitting.} This axiom and the next are very important: they postulate a manageable structure of the boundary
of the compactification $\overline{\mathcal{M}}_{g,n}(X,\beta)$, compatible with that of the boundary of $\overline{\mathcal{M}}_{g,n}$.
One way to get boundary maps is to let the curves have 2 irreducible components, with genera $g_1, g_2$ and marked points
$n_1 +1, n_2+1$ such that $g = g_1 + g_2$, $n = n_1+n_2$. The extra marked point on each irreducible component is where we glue them;
they become one singular point in the resulting reducible curve. For $S$ some partition of the $n$ marked points into 2 sets of
cardinality $n_1$ and $n_2$, we let $\phi_S : \overline{\mathcal{M}}_{g_1,n_1+1} \times \overline{\mathcal{M}}_{g_2,n_2+1} \to
\overline{\mathcal{M}}_{g,n}$ be the gluing map. Choose a basis $\{\Delta_a\}$ of $H^*(X,\Q)$ and define $g_{ab} = \int_V
\Delta_a \wedge \Delta_b$; let $(g^{ab}) = (g_{ab})^{-1}$ denote the entries of the inverse matrix. Then:
\[	\phi_S^* I^X_{g,n,\beta}(\gamma_1, \dots, \gamma_n) = (-1)^S \sum_{\beta_1 + \beta_2 = \beta} \sum_{a,b}
I^X_{g_1,n_1+1,\beta_1}(\otimes_{j\in S_1} \gamma_j \otimes \Delta_a) g^{ab} \otimes I^X_{g_2,n_2+1,\beta_2}(\Delta_b
\otimes \otimes_{j\in S_2} \gamma_j). 	\]
Roughly speaking, we need to introduce $\sum_{a,b}(\Delta_a \otimes \Delta_b)$ to account for the position of the extra marked points.
Integrating over these produces a factor $g_{a,b}$ that wasn't there on the LHS, so we need to multiply by $g^{ab}$ to
compensate for it.

\item \textbf{Genus reduction.} Let $\psi : \overline{\mathcal{M}}_{g-1,n+2}\to \overline{\mathcal{M}}_{g,n}$ be the map
which glues together the last 2 marked points. Then:
\[	\psi^*	I^X_{g,n,\beta}(\gamma_1, \dots, \gamma_n) = \sum_{a,b} 
I^X_{g-1,n+2,\beta}(\gamma_1, \dots, \gamma_n, \Delta_a,\Delta_b) g^{ab}.\]

The splitting and genus reduction axioms motivate the choice of stable maps compactification, see \ref{rem:boundary}.

\item \textbf{Motivic axiom.} The maps $I^X_{g,n,\beta}$ are induced by correspondences in the Chow rings:
\[	C^X_{g,n,\beta} \in C^*(X^n \times \overline{\mathcal{M}}_{g,n}).	\]
Namely, consider the two projection maps:
\[
\begin{tikzcd}
\; & X^n \times \overline{\mathcal{M}}_{g,n}\arrow{dl}{p}\arrow{dr}{q} & \\ X^n & & \overline{\mathcal{M}}_{g,n}.
\end{tikzcd}
\]
We require that:
\[	I^X_{g,n,\beta}(\gamma_1, \dots, \gamma_n) = q_*\big(C^X_{g,n,\beta} \wedge p^* (\gamma_1\otimes \dots \otimes \gamma_n)\big).\]
This axiom is motivated as follows in \cite{km_gw}, 2.3.8.
Suppose we construct a good compactification $\overline{\mathcal{M}}_{0,n}(X,\beta)$,
together with a virtual fundamental class $[\overline{\mathcal{M}}_{0,n}(X,\beta)]$. Consider then the map:
\begin{align*}
\alpha :\overline{\mathcal{M}}_{0,n}(X,\beta) &\to X^n \times \overline{\mathcal{M}}_{0,n} \\
(C, x_1, \dots, x_n, f) &\mapsto \big(f(x_1), \dots, f(x_n), (\bar C, x_1, \dots, x_n)\big).
\end{align*}
We would like $\bar C$ to be $C$, but we may need to contract certain components to get a stable curve from a stable map.
Compare definitions \ref{def:stable_curve} and \ref{def:stable_map}. 
Ignoring this for now, we set $C^X_(g,n,\beta) = \alpha_*\big( [\overline{\mathcal{M}}_{0,n}(X,\beta)]\big)$. This
means, roughly speaking, we're integrating over $\overline{\mathcal{M}}_{0,n}(X,\beta)$, like the naive definition
\ref{eq:gw_naive_def} suggests.
\end{enumerate}
\end{defin}

We are mostly interested in codimension zero invariants, which informally are those where we imposed enough constraints
to get a finite number of curves. For example, if we want to count degree $d$ rational curves in $\bbP^2$,
the relevant codimension zero condition says:
\[	\sum_{i=1}^n |\gamma_i| = 2 \int_{d[H]} c_1(\bbP^2) - 2 \dim \bbP^2 = 6d - 4.	\]
For example, we could ask that the curves pass through $n$ given points in $\bbP^2$, then $|\gamma_i| = 4$, so we obtain
$4n = 6d-4$. If the computation were done right, this would be $12d-4$, so that we get $n=3d-1$. So the relevant thing
to count are degree $d$ rational curves passing through $3d-1$ points.\todo{fix this}








\section{Stable Map Compactification}
\label{sect:stable_map}
To give a naive compactification of
$\overline{\mathcal{M}}_{0,0}(\bbP^r,d)$, we could just look at the space $W(r,d)$ of $r+1$-tuples of degree $d$ polynomials
in 2 variables, up to scaling, and take the subset of tuples which don't vanish simultaneously. We get a subset of a projective space:
\[	W(r,d) \subset \bbP\left( \bigoplus_{i=0}^r H^0(\bbP^1,\mathcal{O}(d)) \right).	\]
We need to quotient by $\Aut(\bbP^1)$ to identify maps that differ by a reparametrization; ignoring this for the moment,
one hopes to take the closure of $W(r,d)$ in $\bbP\left( \bigoplus_{i=0}^r H^0(\bbP^1,\mathcal{O}(d)) \right)$ to obtain
a compactification. However, for $g\neq 0$ and $X \neq \bbP^r$, this doesn't work and we need a less ad-hoc approach.

The choice of compactification matters; different choice leads to different numbers. That's
because the numbers now count things in the boundary as well. 

\begin{eg}
In the stable maps compactification that we introduce shortly, which produces Gromov-Witten invariants,
we keep the domain curves well-behaved:
they acquire nodal singularities, but no non-reduced structure. However, the maps themselves can be highly non-injective.
A different choice is the Donaldson-Thomas compactification via Hilbert schemes: here we work with ideal sheaves, which
always represent embeddings, however the domain curve can now be non-reduced or have singularities worse than nodal.
Section $3\frac{1}{2}$ of \cite{PT_counting_curves} illustrates the differences with the following example.
We work locally and consider the family of conics:
\[	C_t = \{ x^2 + ty = 0 \} \subset \C^2,	\]
which becomes singular as $t\to 0$. In the DT compactification, we take the limit in the defining
equation, and get $x^2 = 0$, which is a thickened $y$-axis. In the stable map compactification, we parametrize
the conics:
\[	C_t \longleftrightarrow \xi \mapsto (-\sqrt{t} \xi, \xi^2) . 	\]
This is a parametrization modulo automorphisms of the curve, namely $\xi \leftrightarrow -\xi$. Now as $t\to 0$,
the limiting map is $\xi \mapsto (0, \xi^2)$, which is a double cover of the $y$-axis.
You can't see from this example, but the different choices of compactification actually give different answers
for the counting problem.
\end{eg}

With that in mind, let's finally define stable maps. For reference and comparison we include the definition of stable
curves:

\begin{defin}
\label{def:stable_curve}
\todo{write this up}
\end{defin}

Think about graphs of curves, such that each ``twig'' has no infinitesimal automorphisms. This means that twigs of
genus $g$ must have at least $3 - 2g$ special points, which means either marked points or singular ones.

\todo{figure out an easy way to include the pictures of graphs}

\begin{defin}[2.4.1 in \cite{km_gw}]
A \textbf{stable map} to $X$ is a structure $\big(C, x_1, \dots, x_n, f\big)$ where:
\begin{itemize}
\item $\big(C, x_1, \dots, x_n\big)$ is a connected reduced curve with $n$ pairwise distinct marked non-singular points,
and at worst additional singular double points.
\item $f:C \to X$ is a map with no non-trivial infinitesimal automorphisms. This means that every irreducible component of
$C$ of genus $g$ which is contracted to a point (of degree 0) must have at least $3-2g$ special points.
\end{itemize}
\end{defin}

\begin{rem}
Note that, in the definition of stable maps $\big(C, x_1, \dots, x_n, f\big)$, the underlying curve
$\big(C, x_1, \dots, x_n\big)$ need not be stable. Therefore the forgetful map $\overline{\mathcal{M}}_{g,n}(X,\beta)
\to \overline{\mathcal{M}}_{g,n}$ must contract components of $\big(C, x_1, \dots, x_n\big)$ which have infinitesimal
automorphisms.
\end{rem}

In his talk notes, Mauro provides the following construction of the moduli stacks of stable maps
$\overline{\mathcal{M}}_{g,n}(X,\beta)$. Start from $\overline{\mathcal{M}}_{g,n}$, which are fine moduli spaces of
curves, and therefore admit a universal family $\mathcal{C}_{g,n}$. Then define:
\begin{equation}
\label{underived_moduli_stack}
	\overline{\mathcal{M}}_{g,n}(X) = 
\Map_{\St / \overline{\mathcal{M}}_{g,n}}(\mathcal{C}_{g,n},  X \times \overline{\mathcal{M}}_{g,n}).
\end{equation}
To obtain $\overline{\mathcal{M}}_{g,n}(X, \beta)$, we must take maps $\alpha$ with the additional constraint that
$\alpha_*[\mathcal{C}_{g,n}] = [\beta] \times [\overline{\mathcal{M}}_{g,n}]$. \todo{figure out the actual condition}

\begin{rem}
When we introduce a derived structure on $\overline{\mathcal{M}}_{g,n}(X, \beta)$, we follow the same approach, but take
maps in $\dSt$ instead of $\St$.
\end{rem}

\begin{thm}[3.14 in \cite{bm_stacks}]
$\overline{\mathcal{M}}_{g,n}(X,\beta)$ are proper, algebraic Deligne-Mumford stacks.
\end{thm}
\todo{we should say something about the proof, but the paper is very techinical}

\begin{defin}
A smooth projective scheme $X$ is \textbf{convex} if for every $f:\bbP^1 \to X$, $H^1(\bbP^1, f^*T_X) = 0$.
\footnote{We may want to restrict $f$ to be stable, but we haven't defined this yet, so we'll ignore it for now.}
\end{defin}

For example, $\bbP^r$ is convex for every $r$. This notion is relevant due to:

\begin{prop}
If $X$ is convex, then $\overline{\mathcal{M}}_{0,n}(X,\beta)$ is a smooth, proper Deligne-Mumford stack.
\footnote{Here we are using the compactification by stable maps; this is defined in \ref{def:stable_map}.}
\todo{what's a reference for this? \cite{km_gw} say it's an expectation in 2.4.2, but Mauro's notes imply that it's proved.}
\end{prop}

Thus, in the situation of convex $X$, $[\mathcal{M}_{g,n}(X,\beta)]$ can be taken to be the fundamental class. Otherwise we will
need to build a virtual fundamental class.

One of the most important properties of $\overline{\mathcal{M}}_{g,n}(X,\beta)$ is the recursive structure of the boundary;
this leads to a proof of the splitting and genus lowering axioms. We first do the case $g=0$, which is formula
2.7.3.1 in \cite{invitation}.

Choose a partition $S_1 \cup S_2$ of the marked points, and classes $\beta_1, \beta_2$ such that $\beta_1 + \beta_2 = \beta$.
Let $D(S_1,S_2;\beta_1, \beta_2) \subset \overline{\mathcal{M}}_{0,n}(X,\beta)$ be the boundary divisor consisting of
curves of genus 0 with 2 irreducible components, with marked points $S_i$ and mapping to $\beta_i$ respectively.

\begin{lem}
\label{lem:recursive_structure}
The boundary divisors are given by:
\[	D(S_1,S_2;\beta_1, \beta_2) =  \mathcal{M}_{0,S_1\cup\{x\}}(X,\beta_1) \otimes_{X}  \mathcal{M}_{0,S_2\cup\{x\}}(X,\beta_2).	\]
Inducting on this formula, we obtain the structure of the lower dimensional strata as well; we don't write this down though.
\end{lem}

\begin{rem}
The straight up generalization for curves of any genus would be:
\[\coprod_{g_1 + g_2 = g} \mathcal{M}_{g_1,S_1\cup\{x\}}(X,\beta_1) \otimes_{X}  \mathcal{M}_{g_2,S_2\cup\{x\}}(X,\beta_2).	\]
where $g_1 + g_2 = g$, and $[\beta_1] + [\beta_2] = [\beta]$. I haven't computed the dimensions, though, to see for what values
of $g_1, g_2$ we get codimension 1 strata. Moreover, we have extra contributions from cycles of lower genus curves.
\todo{finish this}
\end{rem}

To illustrate the need for virtual fundamental classes, we look at an example where $\overline{\mathcal{M}}_{g,n}(X,\beta)$
contains strata of higher dimension than $\vdim$; in this case, taking the straight up fundamental class would break
the grading dimension of Kontsevich-Manin.
The following example is worked out in full detail Section 4 of \cite{nabijou}.

\begin{eg}
We compute the dimension and virtual dimension of $\overline{ \mathcal{M}_{0,0}}(X, 3\pi^*H)$,
where $X = \Bl_p \bbP^2$, $\pi: X \to \bbP^2$ is the blowup map, and $[H] \in H_2(\bbP^2,\Z)$ is the hyperplane class.
Using equation \ref{eq:vdim}, we have:
\[	\vdim \overline{ \mathcal{M}_{0,0}}(X, 3\pi^*H) = \int_{3\pi^*H} c_1(T_X) - 1 = 8.	\]
One could look, for example, at rational curves of degree 3 in $\bbP^2$ which avoid $p$, i.e.
$\overline{ \mathcal{M}_{0,0}}(\bbP^2, 3H)$. This is a stratum in $\overline{ \mathcal{M}_{0,0}}(X, 3\pi^*H)$ of the correct dimension 8
(the space of cubics in $\bbP^2$ is 9-dimensional, and we substract 1 for reparametrizations of the domain $\bbP^1$.)
More strata are given by rational cubics in $\bbP^2$ which pass through $p$ with multiplicity $k$, and therefore 
lift to a curve in $X$ of class $3\pi^*H - rE$, where $E \subset X$ is the exceptional divisor. To obtain
a stable map in the appropriate class $3\pi^*H$, we add $r$ components isomorphic to $\bbP^1$ which map to $E$.
The dimension of this stratum is:
\[	\dim \overline{ \mathcal{M}_{0,0}}(X,3\pi^*H - rE) + \dim \overline{ \mathcal{M}_{0,0}}(\bbP^1,r) = (8-r) + (2r-2) = 6+r.	\]
The farthest we can go while keeping $[\beta]$ effective (that is, $\beta . K_X \leq 0$) is $r=3$. This gives a stratum
(supposedly a boundary stratum!) of dimension $9>8$.
\end{eg}


\chapter{Obstruction Theories and Virtual Fundamental Classes}
\label{ch2:obs}
Talk by Benedict Morrissey.

Given a stack $X$, our objective is to construct a virtual fundamental class $[X]$ for it, motivated by the discussion in
\ref{ch:gw}. We will see two ways in which a derived enhancement of $X$ helps achieve this.
We would like $[X]$ to come from an algebraic cycle, i.e. an element of the Chow group.
In this case, given $f: X\to Y$ proper,
there is a well-defined pushforward  $f_*[X] \to [Y]$, which induces a pushforward
$f_*[X]^H \to [Y]^H$ on the images $[X]^H, [Y]^H$ of the VFCs in any Weil cohomology theory $H$.

However, derived Chow groups have yet to be defined, so we start with a piecemeal approach, by
defining a class in G-theory only.


\section{Construction from G-Theory}
\begin{defin}
The \textbf{G-theory} $G_0(X)$ of a classical stack $X$ is defined as the K-theory of the category of coherent sheaves on $X$:
\footnote{One can also define higher G-theory $G_i$, but we won't need this.}
\[	G_0(X) := K_0(\Coh(X)) .	\]
If $\tilde X$ is a derived stack, we set $G_0(\tilde X) = K_0((\Coh \tilde X)^{\heartsuit})$.
\end{defin}

\begin{defin}
A \textbf{derived enhancement} of a stack $X$ is a derived stack $\tilde X$ such that $t_0(\tilde X) = X$.
\end{defin}

There is a natural inclusion, left-adjoint to the truncation, which we denote $j: X \to \tilde X$.
Using the fact that pushforwards of coherent sheaves by proper maps are coherent,
\todo{check if there are other conditions, and whether $\tilde X$ derived changes anything}
we obtain $j_* : G_0(X) \to G_0(\tilde X)$.

\begin{prop}
If $X$ is quasi-compact, then $j_* : G_0(X) \to G_0(\tilde X)$ is a bijection. In this case we define:
\[	 [X]^{\vir} := j_*^{-1}[\mathcal{O}_{\tilde X}].	\]
\end{prop}

\begin{proof}
The identification actually works on the full spectrum of $G$-theory. We're using the theorem of 
the heart for $K$-theory. The identification is done as follows.
\begin{enumerate}
\item Theorem of the heart for $K$-theory. (Due to Quillen, and Batwick in the DG category setting.) If you have $\cC$ a stable
$\infty$-category, idempotent complete, with $t$-structure, and every object in the heart is bounded, then $K(\cC) = K(\cC^{\heartsuit})$.
\item $\Coh(\tilde X)^{\heartsuit} \simeq \Coh(X)^{\heartsuit}$, which follows from descent and the analogous result for
derived affines, which was proved during the first semester, in the talk on Stable $\infty$-categories.
\footnote{Throughout when we write $\Coh$ we mean $\Coh^b$.}
\end{enumerate}
\end{proof}

\begin{thm}
\label{thm:ox_bounded}
For $\tilde X$ quasi-compact,\footnote{Note that we don't need to assume that $X$ is quasi-compact.}
$\mathcal{O}_{\tilde X}$ is bounded. It follows that the following sum is finite:
\[	j_*^{-1}[\mathcal{O}_{\tilde X}] = \sum_{i=0}^{\infty} (-1)^i [H^i(\mathcal{O}_{\tilde X})],	\]
so it defines an element in $G_0(X)$.
\end{thm}

\begin{rem}
Note that the cohomology in Theorem \ref{thm:ox_bounded} is just the cohomology of the complex, NOT sheaf cohomology. 
Moreover it wouldn't make sense to
use $K$ theory instead of $G$ theory, because even if $\mathcal{O}_{\tilde X}$ is perfect, the kernels and cokernels 
of the various differentials don't need to be.
\end{rem}

\begin{proof}
We start with a vague understanding of why the theorem may be true. The counterexample is $\Spec (\Sym k[2])$, 
where the cotangent complex is unbounded. But if it's in amplitude [-1,0], it's like an exterior algebra and it works.

We work locally, $\Spec B \to \Spec A$, $\supp \bbL_{B/A} \subset[-1,0]$. $B$ is a derived lci over $A$, so the cotangent complex
is perfect, so there's a theorem that says that $B$ is homotopically of finite type over $A$. These can be constructed by attaching
finitely many cells:
\[	A = B_0 \to B_1 \to B_2 \to \dots \to B_k = B.	\]
Attaching the $i+1^{th}$ cells of $B$ looks like:
\[
\begin{tikzcd}
B_i \arrow{r} & B_{i+1} \\
\bigotimes A[\p \Delta^{i+1}]\arrow{u}\arrow{r} & A[\Delta^{i+1}]\arrow{u} .
\end{tikzcd}
\]
$B_1$ is obtained by attaching cells in degree 1. The map $B_1 \to B$ is an isomorphism on $\pi_0$. \todo{review this proof}

There's another proof by Lowrey and Sch\"urg, in \cite{derived_GRR}, which is more intuitive. Having a quasi-smooth structure 
allows one to describe
the derived space locally as the derived zero locus of a section of a vector bundle. Then the derived intersection can be computed
as a Koszul resolution, so $\mathcal{O}_{\tilde X}$ behaves like an exterior algebra, which means it's bounded. Here the quasi-compactness
is used in order to reduce to finitely many local charts, which means that the bound on $\mathcal{O}_{\tilde X}$ is uniform.
\end{proof}

\begin{rem}
The idea behind the proof of Lowrey and Sch\"urg is also that of
\textbf{Kuranishi structures}. These are essentially a machinery for working with derived stacks which remembers the 
local description as zero locus, in order to avoid using the machinery of derived geometry.
In DAG quasi-smoothness is an intrinsic property that one can check at the level of the cotangent complex, so that one doesn't
need to remember the local descriptions, which are cumbersome and don't glue well.
\end{rem}

The VFC in ordinary cohomology is defined by Konstevich to be:
\begin{equation}
\label{eq:G_definition_VFC}
	[X]^{\vir} = \Ch([X]_G^{\vir}) \Td(j^* \bbT_{\tilde X}).
\end{equation}

\begin{conj}
Definition \ref{eq:G_definition_VFC} agrees with the construction of Behrend-Fantechi, \ref{}. \todo{ref this}
\end{conj}

The conjecture has been verified for schemes (not stacks) by Ciocan-Fontanine and Kapranov, in
\cite{ciocan_fontanine_kapranov}, using the additional 
assumption (which is made in Behrend Fantechi anyway) that the cotangent complex admits a global resolution by vector bundles.




\section{Obstruction Theories}
We introduce the alternative construction of VFCs, following \cite{Behrend_Intrinsic_normal_cone_1997}. In the words of
Mauro, we want to use this as a black box which achieves:
\[	\text{Obstruction Theory} \Longrightarrow \text{VFC}.	\]

Throughout we will use $X,Y$ for underived stacks, and $\tilde X, \tilde Y$ for their derived enhancements.

\begin{defin}
An \textbf{obstruction theory} for $X$ is a morphism $\phi : E \to \bbL_X$ in $D(\Coh(X))$, such that:
\begin{align*}
&h^0(\phi) : H^0(E) \to H^0(\bbL_X) \text{ is an isomorphism,} \\
&h^{-1}(\phi) : H^{-1}(E) \to H^{-1}(\bbL_X) \text{ is surjective,} \\
&H^i(E) = 0 \text{ for } i \neq -1,0.
\end{align*} 
\end{defin}

\begin{defin}
A \textbf{perfect obstruction theory} is an obstruction theory such that $E$ is in perfect amplitude [-1,0], which means that
locally $E$ is isomorphic to a 2-term complex of vector bundles $[E^{-1} \to E^0]$.
\end{defin}

The link to derived geometry is as follows. 
\begin{prop}
Given a derived enhancement $j:X \to \tilde X$, with $\tilde X$ a quasi-smooth DM stack, there is a perfect obstruction theory:
\[	j^* \bbL_{\tilde X} \to \bbL_X.	\]
\end{prop}
\begin{proof}
By descent we reduce this to the case of affines, and we need only consider $A \to t_0(A)$. We have the fiber sequence:
\[	j\bbL_A \to \bbL_{\pi_0(A)} \to \bbL_{\pi_0(A)/A}.	\]
Due to the connectivity estimates, which we introduced last semester in the talk about the cotangent complex,
$\bbL_{\pi_0(A)/A}$ is 2-connective. Indeed, the fiber of $A \to \pi_0(A)$ is 1-connective,
so the cofiber, which is the shift of the fiber by 1, is 2-connective.\footnote{In order to relate the fiber and cofiber of the
morphism $A \to \pi_0(A)$, we use the fact that we are working in the stable $\infty$-category $\pi_0(A)\Mod$, and not 
in $\pi_0(A)-\Alg$.}
\end{proof}

Throughout the rest of the talk, the goal is to describe how to construct a VFC, starting with an obstruction theory. In the
smooth case, if you take the $G$-construction we did earlier, you'd get the same answer. 

We also want to describe functoriality properties for the VFC, and to that effect we introduce compatibility data
between obstruction theories. 
During the check that Kontsevich-Manin axioms are satisfied, we will need to use functoriality a lot.
The following is Definition 5.8
in \cite{Behrend_Intrinsic_normal_cone_1997}.


\begin{defin}
\label{defin:compatibility_datum}
Let $u:X' \to X$ be a morphism.
A \textbf{compatibility datum between obstruction theories} $E$ for $X$ and $ F$ for $X'$ is a choice of embeddings
$f: X \to Y$, $g: X' \to Y'$ into smooth stacks, such that the following diagrams commute:
\[
\begin{tikzcd}
X'\arrow{r}{u} \arrow[swap]{d}{g} & X\arrow{d}{f} \\ Y'\arrow{r}{v} & Y,
\end{tikzcd}
\]
\[
\begin{tikzcd}
u^* E \arrow{d}\arrow{r}{\phi} & F \arrow{r}{\psi}\arrow{d} & g^* \bbL_{Y'/Y}\arrow{d} \\
u^* \bbL_X \arrow{r} & \bbL_{X'} \arrow{r} & \bbL_{X'/X} .
\end{tikzcd}
\]
Moreover, we require the two rows to be fibration sequences in $D(\Coh(X'))$.
\end{defin}

Behrend and Fantechi prove:

\begin{prop}
Given compatibility data between obstruction theories $E$ for $X$ and $F$ for $X'$, it follows that 
$u^*[X]^{\vir, E} = [X']^{\vir, F}$.
\end{prop}

For us obstruction theories come from derived enhancements $\tilde X, \tilde X'$.
In this case, we obtain the functoriality of VFCs in a cleaner way, by giving a morphism between derived
enhancements $w: \tilde X' \to \tilde X$, fitting in the commutative diagram:
\[
\begin{tikzcd}
\; & \tilde X'\arrow{rr}{w}\arrow{ddl}{\tilde g} & & \tilde X\arrow{ddl}{\tilde f} \\
X'\arrow{rr}{u}\arrow[swap]{d}{g}\arrow{ur}{j} & & X\arrow[swap]{d}{f}\arrow{ur}{i} & \\
Y'\arrow{rr}{v} & & Y &
\end{tikzcd}
\]
Moreover, we require that the top and back square are homotopy pullbacks.

\begin{rem}
I was hoping that the top square would be enough. Unfortunately, we still need the choice of ambient 
spaces $Y, Y'$, as well as morphisms $\tilde g$, $\tilde f$, and the data for the homotopy commutativity of the back square.
However Mauro says:
\begin{enumerate}
\item In the applications we care about (stable maps), the entire back square will be there naturally.
\item Working with the derived compatibility data is still easier, in practice, than with the fibration sequences in
Definition \ref{defin:compatibility_datum}.
\end{enumerate}
\end{rem}

Let us see why the derived compatibility data implies the diagram between fibration sequences in 
Definition \ref{defin:compatibility_datum}.
The assumption is that $E = i^* \bbL_{\tilde X}$ and $F = j^*\bbL_{\tilde X'}$. We first need the map:
\[		\phi	: u^* i^* \bbL_{\tilde X} \to j^* \bbL_{\tilde X'} .	\] 
This is just given by $w$. More precisely, the commutativity of the top square gives the map on the left 
in the following diagram, and we define the top map as the composition:
\[
\begin{tikzcd}
u^*i^*\bbL_{\tilde X}\arrow{r}\arrow{d} & j^* \bbL_{\tilde X'} \\
j^*w^*\bbL_{\tilde X}\arrow{ur} &
\end{tikzcd}
\]
To get $\psi$, which must be such that the row is a fiber sequence, we make use of the maps $\tilde g, \tilde f$.
Question: how to identify $j^*\bbL_{\tilde X'/\tilde X} $ with $g^* \bbL_{Y'/Y}$? Since the back square is a pullback, we have
a canonical identification $\bbL_{\tilde X'/\tilde X}  \simeq \tilde g^* \bbL_{Y'/Y}$, and this gives:
\begin{equation}
\label{eq:get_psi}
	j^*\bbL_{\tilde X'/\tilde X} \simeq j^*	\tilde g^* \bbL_{Y'/Y} \simeq  g^* \bbL_{Y'/Y} .
\end{equation}
We take this composition to be $\psi$. Note that this chain of equivalences depends very much on the extra data of
the homotopy commutative back square.

\begin{rem}
Throughout, we want $Y', Y$ to be smooth, and $\tilde f, \tilde g$ to be quasi-smooth. Therefore, if $X, X'$ are not
smooth, we cannot expect $f, g$ to be just identity maps. In fact, the point that Behrend-Fantechi make is that $Y$ and $Y'$ 
should only be expected to exist locally.
\end{rem}

\begin{defin}
A \textbf{local embedding} $(U,M)$ of $X$ is the data of $U\to X$ an \'etale map and $U\to M$ a local immersion, 
where $M$ smooth affine $k$-scheme of finite type.
Given a local embedding, the associated \textbf{normal bundle} is $N_{U|M} := \Spec_M(\Sym (I/I^2))$. Inside this 
we have the \textbf{normal cone} $C_{U/M} = \Spec_M(\oplus_{n\geq 0} I^n/I^{n+1})$.
The ring homomorphism $\Sym (I/I^2) \to \oplus_{n\geq 0} I^n/I^{n+1}$ is surjective, so the map
$C_{U/M}	\to N_{U|M}$ is a closed embedding.
\end{defin}

Given an obstruction theory $E \to \bbL_X$, if we can write $E = [F_{-1} \to F_0]$ globally, and define $F^1 := F_{-1}^*$, 
then we get the pullback diagram:
\[
\begin{tikzcd}
C(F^{\bullet})\arrow{r}\arrow{d} & F_1\arrow{d} \\ C_X \arrow{r} & N_X .
\end{tikzcd}
\]

\begin{defin}
Let $0: X \to F_1$ be the zero section.
The \textbf{virtual fundamental class} of $X$ induced by the obstruction theory $E$ is the intersection of
$[C(F^{\bullet})] \in \Chow(F_1)$ with the zero section, i.e. $[X]^{\vir, E} := o^{!}[C(F^{\bullet})]$.
\todo{why is this shriek and not star?}
\end{defin}


\chapter{Geometricity of Mapping Stacks}
\label{ch3:geom}

This chapter is somewhat tangential to our concrete goals for the semester. However we thought that $\R\Map_{g,n}(X,\beta)$
provides a good opportunity to understand Artin-Lurie representability and how it can be used to prove that certain
mapping stacks are geometric.


\section{Using Artin-Lurie representability for Mapping Stacks}
The representability theorem says:

\begin{thm}[Artin-Lurie representability, Theorem 3.2.1 in \cite{DAG-XIV}]
\cite{thm:lurie_representability}
Let $X: \cdga_k^{\leq 0}\to \cS$ be a functor, and suppose we are given a natural transformation $f:X \to \Spec R$.
 Then $X$ is representable by a derived Deligne-Mumford $n$-stack locally almost of finite
presentation over $R$ if and only if the following are satisfied:
\begin{enumerate}
\item For every discrete commutative ring $A$, the space $X(A)$ is n-truncated.
\item $X$ is a sheaf for the \'etale topology.
\item $X$ is nilcomplete, infinitesimally cohesive and integrable. These mean:
\begin{itemize}
\item $X$ commutes with Postnikov towers;
\item $X$ commutes with pullback squares $B \times_A C$, under the assumption that $\pi_0(B) \to \pi_0(A)$ and $\pi_0(C) \to \pi_0(A)$
are surjective with nilpotent kernel;
\item for $A$ a complete local ring, $X(A) \simeq \varprojlim X(A/\fr m^n)$; loosely speaking, every formal $A$-point of $X$
integrates to give a point of $X$.
\end{itemize}
\item $f:X \to R$ admits a connective relative cotangent complex $\bbL_{X/R}$.
\item $f:X \to R$ is locally almost of finite presentation.
\end{enumerate}
\end{thm}

\begin{rem}
(2) is obvious, (5) ensures that the DM stack is locally almost of finite presentation. (3) and (4) ensure that $X$ has good
local behavior, in particular a good deformation theory. The existence of the relative cotangent complex is conceptually
the most important condition, and the one we will put the most effort into verifying. Finally, (1) encodes the geometricity
of the representing DM stack. $n$-stacks are defined to be those for which condition (1) holds; it is then true that: \todo{Mauro
said so; maybe also find a reference}
\begin{itemize}
\item $n$-geometric implies $n+1$-stack;
\item $m$-geometric for some $m$ and $n$-stack implies that, at worst, $m=n+1$.
\end{itemize}
\todo{this is not completely satisfactory: does Lurie representability guarantee that we get $m$-geometric for some $m$?}
\end{rem}

As an application of this, we want to prove the geometricity of mapping stacks.

\begin{thm}
\label{thm:mapping_stack_geometric}
Let $g: X \to Z$ be a morphism of derived stacks which is geometric and of finite type. Let $f:Y\to Z$ be a morphism of stacks
which is representable by proper flat schemes. \footnote{We could replace the condition on $f$ with something slightly more
general, such as representable by quasi-compact quasi-separated algebraic spaces of finite tor amplitude.} Then 
the mapping stack $\Map_{/Z}(Y,X)$ is geometric
over $Z$, i.e. the morphism $\Map_{/Z}(Y,X) \to Z$ is geometric.
\end{thm}
\begin{proof}
Recall that the mapping stack is defined by the functor of points:
\[	 \Map_{/Z}(Y,X) (T) = \Map_{\dSt}(T\times_Z Y, X). 	\]
We first reduce to $Z$ affine, so that Theorem \ref{thm:lurie_representability} applies.
Then, since $g$ is assumed geometric, it satisfies conditions (3) of the Theorem \ref{thm:lurie_representability}. 
It follows by elementary manipulation
of the diagrams that $\Map_{/Z}(Y,X) \to Z$ also has these properties; see Proposition 3.3.6 in \cite{DAG-XIV}.
The most important issue is the existence of a relative cotangent complex for $\Map_{/Z}(Y,X) \to Z$. 
Recalling the definition, we need to construct cotangent complexes $\bbL_{\Map_{/Z}(Y,X),x}$ at each point 
$x: \Spec A \to \Map_{/Z}(Y,X)$, and then make sure that they glue; this will be diagram \ref{} below.

$\bbL_{\Map_{/Z}(Y,X),x}$ is supposed to be an object that represents the functor of derivations over $\Map_{/Z}(Y,X)$:
\[	\Map_{A\Mod}(\bbL_{\Map_{/Z}(Y,X),x}, M) = \Der_{\Map_{/Z}(Y,X)}(A,M) .	\]
The latter is defined as the homotopy pullback:
\[
\begin{tikzcd}
\Der_{\Map_{/Z}(Y,X)}(A,M)\arrow{r}\arrow{d} & \Map(\Spec(A\oplus M) \times_Z Y, X)\arrow{d} \\
\Spec k\arrow{r} & \Map(\Spec A \times_Z Y, X).
\end{tikzcd}
\]
Let $q: \Spec A \times_Z Y \to \Spec A$ denote the projection. Then $\Spec(A\oplus M) \times_Z Y$ coincides with the extension
$(\Spec A \times_Z Y)[q^*M]$ by the pullback $q^*M$.\footnote{Work locally, take 
$\Spec(A \otimes_{\mathcal{O}_Z} \mathcal{O}(Y) \oplus M\otimes_{\mathcal{O}_Z} \mathcal{O}(Y)$ over each affine piece and glue.}
Therefore the pullback diagram becomes:
\[
\begin{tikzcd}
\Der_{X}(A,q^*M)\arrow{r}\arrow{d} & \Map((\Spec A \times_Z Y) [q^*M], X)\arrow{d} \\
\Spec k\arrow{r} & \Map(\Spec A \times_Z Y, X).
\end{tikzcd}
\]
Now the top left is equivalent to $\Map_{\Spec A \times_Z Y}(f_x^*\bbL_{X/Z}, q^*M)$. So the existence of a cotangent complex at the
point $x$ is reduced to:
\[	\Map_{A\Mod}(\bbL_{\Map_{/Z}(Y,X),x}, M) \simeq \Map_{\Spec A \times_Z Y}(f_x^*\bbL_{X/Z}, q^*M).	\]
Thus, we need a left adjoint for $q^*$; this is the map $q_+$ introduced in \ref{} below. Then we can define:
\begin{equation}
\label{eq:mapping_local_cotangent}
	\bbL_{\Map_{/Z}(Y,X)/Z,x} := q_+ f_x^*\bbL_{X/Z}.
\end{equation}

Finally, we address the gluing of these cotangent complexes. Assume that we have a morphism $g: \Spec B \to \Spec A$. We define
a point $y: \Spec B \to \Map_{/Z}(Y,X)$ by requiring the following diagram to commute:
\[
\begin{tikzcd}
\; & & X \\
Y \times_Z \Spec B \arrow{r} \arrow{d}{q_B} \arrow{urr}{f_y} & Y\times_Z \Spec A \arrow{d}{q_A}\arrow[swap]{ur}{f_x} & \\
\Spec B\arrow{r}{g} & \Spec A &
\end{tikzcd}
\]
From the commutativity of the upper triangle we obtain $f_y = f_x \circ 1\times g$, so that:
\[	q_{B+} f_y^* \bbL_X \simeq q_{B+}(1\times g)^*f_x^*\bbL_X .	\]
Our goal is to show that gluing works, which means:
\[	q_{B+} f_y^* \bbL_X \simeq g^* q_{A*} f_x^* \bbL_X.	\]
Therefore it suffices to prove that $q_+$ has the base change property $q_{B+}(1\times g)^* \simeq g^* q_{A+}$. This
is the object of Lemma \ref{lem:q+_base}, while the construction of $q_+$ is Lemma \ref{lem:q+}.
\end{proof}

\begin{rem}
\label{rem:mapping_global_cotangent}
The tangent complex of the mapping stack, when it exists, can be obtained more easily from the diagram:
\[
\begin{tikzcd}
\Map_{/Z}(Y,X) \times_Z Y \arrow{d}{\pi}\arrow{dr}{ev} & \\ \Map_{/Z}(Y,X) & X.
\end{tikzcd}
\]
Then:
\[	\bbT_{\Map_{/Z}(Y,X)/Z} = \pi_* \ev^* \bbT_{X/Z}.	\]
We would like to dualize and obtain:
\begin{equation}
\label{eq:mapping_global_cotangent}
	\bbL_{\Map_{/Z}(Y,X)/Z} = (\pi_* \ev^* \bbL_{X/Z}^{\vee})^{\vee} = \pi_+ \ev^* \bbL_{X/Z}.
\end{equation}
In Theorem \ref{thm:mapping_stack_geometric}, we actually prove that $\bbL_{\Map_{/Z}(Y,X)/Z}$ exists, by constructing
it locally and then showing that the construction glues. The result of the gluing must then be \ref{eq:mapping_global_cotangent},
as can be seen from the extended diagram:
\[
\begin{tikzcd}
\Spec A \times_Z Y\arrow{r}{x\times 1}\arrow{d}{q} &\Map_{/Z}(Y,X) \times_Z Y \arrow{d}{\pi}\arrow{dr}{ev} & \\
\Spec A\arrow{r}{x} & \Map_{/Z}(Y,X) & X.
\end{tikzcd}
\]
Since the square is a homotopy pullback, we apply base change for $q_+$ (see Lemma \ref{lem:q+_base}):
\[	x^* \pi_+ \ev^* \bbL_{X/Z} \simeq q_+ (x\times 1)^* \ev^* \bbL_{X/Z} \simeq q_+ f_x^*\bbL_{X/Z},	\]
which agrees with what we called $\bbL_{\Map_{/Z}(Y,X)/Z,x}$ in \ref{eq:mapping_local_cotangent}. It follows that
$\bbL_{\Map_{/Z}(Y,X)/Z} \simeq \pi_+ \ev^* \bbL_{X/Z}$.
\end{rem}


\section{Stable Maps}
We apply Theorem \ref{thm:mapping_stack_geometric} and Remark \ref{rem:mapping_global_cotangent} to the derived 
moduli space of stable maps on a smooth projective
variety $X$:
\[	\R\mathcal{M}_{g,k}(X) = \Map_{dSt/\mathcal{M}_{g,k}}(\mathcal{C}_{g,k},X\times \mathcal{M}_{g,k}).	\]
According to Remark \ref{rem:mapping_global_cotangent}, and using the notation therein:
\begin{equation}
\label{eq:cotangent_stable_prelim}
	\bbL_{\R\mathcal{M}_{g,k}(X)/\mathcal{M}_{g,k}} = \pi_+ \ev^* \bbL_{X\times \mathcal{M}_{g,k}/\mathcal{M}_{g,k}}.
\end{equation}
We can simplify this expression using the pullback diagram:
\[
\begin{tikzcd}
X \times \mathcal{M}_{g,k}\arrow{r}{p}\arrow{d} & X\arrow{d} \\ \mathcal{M}_{g,k}\arrow{r} & \Spec k .
\end{tikzcd}
\]
The diagram implies $\bbL_{X\times \mathcal{M}_{g,k}/\mathcal{M}_{g,k}} \simeq p^* \bbL_X$, so \ref{eq:cotangent_stable_prelim}
reduces to:
\begin{equation}
\label{eq:cotangent_stable}
	\bbL_{\R\mathcal{M}_{g,k}(X)/\mathcal{M}_{g,k}} = \pi_+ \ev^* p^* \bbL_{X}.
\end{equation}

\begin{prop}
The natural map $\R \mathcal{M}_{g,k}(X) \to \mathcal{M}_{g,k}$ is quasi-smooth.
\end{prop}
\begin{proof}
We need to show that $\bbL_{\R\mathcal{M}_{g,k}(X)/\mathcal{M}_{g,k}} = \pi_+ \ev^* p^* \bbL_{X}$ has cohomological amplitude
$[-1,0]$. $X$ is a smooth variety, so $\bbL_X$ is in amplitude $[0,0]$. Pullbacks preserve cohomological amplitude, because
they only involve tensoring with locally free sheaves. \todo{Need any assumption on the maps?} $\pi_*$ may increase cohomological
amplitude, because of higher direct image sheaves. However, that the fibers of $\pi : \R\mathcal{M}_{g,k}(X) 
\times_{\mathcal{M}_{g,k}} \mathcal{C}_{g,k} \to \mathcal{M}_{g,k}$ are curves, so the cohomological amplitude of
$\pi_* (\ev^* p^* \bbL_{X})^{\vee}$ is at most $[0,1]$. Dualizing again brings $\bbL_{\R\mathcal{M}_{g,k}(X)/\mathcal{M}_{g,k}}$
to amplitude $[-1,0]$.
\end{proof}

Together with the fact that $\mathcal{M}_{g,k}$ is smooth, this implies that $\R\mathcal{M}_{g,k}(X)$ is quasi-smooth.
Proposition \ref{prop:derived_obstruction_theory} implies the following.

\begin{cor}
The derived enhancement $j: \mathcal{M}_{g,k}(X) \to \R\mathcal{M}_{g,k}(X)$ determines a perfect obstruction theory
on $\mathcal{M}_{g,k}(X)$:
\[	j^* \bbL_{\R\mathcal{M}_{g,k}(X)} \to \bbL_{\mathcal{M}_{g,k}(X)}.	\]
\end{cor}

Expression \ref{eq:cotangent_stable} for the cotangent complex of $\R\mathcal{M}_{g,k}(X)$ shows that the obstruction theory
is the same as that considered by \cite{Behrend_Intrinsic_normal_cone_1997} and introduced in 
\todo{reference once the chapter is edited}.




\section{The + Pushforward Functor}

\begin{lem}[3.3.22 and 3.3.23 in \cite{DAG-XII}]
\label{lem:q+}
Suppose that $q: Y \to S$ is perfect, i.e. $q_*$ preserves perfect complexes. Then $q^* : \QCoh(S) \to \QCoh(Y)$ has a left
adjoint $q_+$.
\end{lem}
\begin{proof}
Let $F \in \Perf(Y)$ and $G \in \QCoh(S)$. Then:
\begin{align*}
	&\Map_{\QCoh(S)} \big( (q_*F^{\vee})^{\vee}, G\big) \simeq \Map_{\QCoh(S)} \big( \mathcal{O}_S, q_*F^{\vee} \otimes G\big) \\
\simeq& \Map_{\QCoh(S)} \big(\mathcal{O}_S, q_*(F^{\vee} \otimes q^* G)  \big) \simeq \Map_{\QCoh(Y)}(\mathcal{O}_Y,F^{\vee} \otimes q^*G)
 \simeq \Map_{\QCoh(Y)}(F,q^*G).
\end{align*}
We have used the fact that perfect complexes are dualizable and the projection formula $q_*(F^{\vee} \otimes q^* G)
\simeq q_*F^{\vee} \otimes G$. This means that, for $F$ perfect, we can use $q_+F = (q_*F^{\vee})^{\vee}$. Now if $S$ and $q$
are quasi-compact and quasi-separated, $\QCoh(Y) = \Ind \Perf(Y)$. Remarking that $q_+ : \Perf(Y) \to \QCoh(S)$ is a left
adjoint, it commutes with colimits, and so there exists a unique extension $q_+ : \Ind \Perf(Y) \to \QCoh(S)$ which commutes with
colimits. It also follows that the extension is a left adjoint.
\end{proof}

\begin{lem}[3.3.23 in \cite{DAG-XII}]
\label{lem:q+_base}
Suppose given a pullback diagram of DM stacks, \todo{can we do better?} with $f, f'$ perfect.
\[
\begin{tikzcd}
X'\arrow{r}{g'}\arrow{d}{f'} & X\arrow{d}{f} \\ Y' \arrow{r}{g} & Y
\end{tikzcd}
\]
Then the canonical map $\lambda: f'_+ \circ g'^* \to g^* f_+$ is an equivalence. 
(We say that the + pushforward satisfies base change.)
\end{lem}
\begin{proof}
On perfect objects $F$, $\lambda_F$ is the dual of:
\[	g^*f_* F \to f'_*g'^* F.	\]
We have used the fact that pullbacks preserve duals. This is an isomorphism due to base change for the pusforward $f_*$.
To conclude, both $f_+$ and $g^*$ are now left adjoints, which means they preserve all colimits. It follows that
$f'_+ \circ g'^* \simeq g^* f_+$ extends to $\QCoh \simeq \Ind \Perf$.
\end{proof}

Note that we have used the assumption that $q: Y \to S$ is perfect. It remains, then, to prove that the map $q : \Spec A \times_Z Y
\to \Spec A$ from the proof of Theorem \ref{thm:mapping_stack_geometric} is perfect. We do this in Lemma \ref{lem:q_perfect}
below, after introducing some terminology.

\begin{defin}
A map $q:Y \to S$ is \textbf{categorically proper}, also called \textbf{of finite cohomological dimension}, if
$q_* : \QCoh(Y) \to \QCoh(S)$ increases cohomological dimension by a uniform finite amount.
\end{defin}

\begin{defin}
A map $q:Y \to S$ is \textbf{of finite tor amplitude} if locally $q:\Spec B \to \Spec A$ and $B$ is of finite tor amplitude
as an object of $A\Mod$.
\end{defin}

Note that in our example $q: \Spec A \times_Z Y \to \Spec A$, $q$ is:
\begin{itemize}
\item proper, because we assumed that $Y \to Z$ is proper, and this is stable under base change;
\item categorically proper, because we can compute $q_*$ by a (uniformly) finite \v{C}ech resolution, due to the assumption
that $Y \to Z$ is representable by proper schemes.
\item of finite tor amplitude, because this is a consequence of flatness. We have assumed that $Y\to Z$ is flat,
and flatness is stable under base change.
\end{itemize}

\begin{lem}
\label{lem:q_perfect}
Let $q: Y \to S$ be a map which is proper, categorically proper and of finite tor amplitude. Then $q$ is perfect.
\end{lem}
\begin{proof}
First, we claim that the first two assumptions imply that $q_* \Coh^-(X) \to \Coh^-(S)$. This argument uses the Leray spectral
sequence. \todo{fill this in} Next, note that $\Perf(X) \subset \Coh^-(X)$ is characterized as the full subcategory of complexes
with finite tor amplitude. So it remains to prove that, if $F \in \Coh^-(X)$ has finite tor amplitude, then so does $q_*F$.

Take a Zariski affine cover for $X$; due to the quasi-compactness assumption this can be taken finite. Then the \v{C}ech nerve
$U^{\bullet}$ is a finite complex. Since $q_*$ is a right adjoint, it commutes with limits, and we have:
\[	q_* F = \varprojlim q|_{U^{\bullet}*} F|_{U^{\bullet}}.	\]
Note that the maps $U_i \to X$ are not, in general, proper, so $q|_{U^{\bullet}*} F|_{U^{\bullet}}$ needn't be coherent. However,
on affines $q_*$ is just a forgetful functor on modules. Therefore the coherence of $q|_{U^{\bullet}*} F|_{U^{\bullet}}$
follows from the fact that $F|_{U^{\bullet}}$ is coherent and the map on rings is finitely generated. \todo{explain more}

For finite tor amplitude, it suffices to check that, for every $M$ discrete, $q_*F \otimes_{\mathcal{O}_S} M$ is cohomologically
supported in $[-m,\infty)$ for some $m$. (We already know that $q_*F$ is bounded above.) Since the \v{C}ech nerve is finite,
we have:
\[	q_*F \otimes_{\mathcal{O}_S} M = \varprojlim q|_{U^{\bullet}*} F|_{U^{\bullet}} \otimes_{\mathcal{O}_S} M.	\]
The inclusion $\Coh^{-,[-m,\infty)}(S) \subset \Coh^-(S)$ commutes with limits, which gives the result.
\end{proof}




\section{Application: Weil Restriction}
Weil restriction is, roughly speaking, an adjoint for base change. We sketch the treatment that Lurie gives in \cite{DAG-XIV}.

\begin{defin}
Let $\phi : Y \to Z$ and $X \to Y$ be maps of derived stacks. A \textbf{Weil restriction} for $X$ along $\phi$ is a
stack $\Res_{Y/Z}X \to Z$, equipped with a morphism $\rho_X : \Res_{Y/Z}X \times_Z Y \to X$ over $Y$, such that composition
with $\rho$ determines a homotopy equivalence:
\begin{equation}
\label{eq:weil_restriction}
	\Map_{\dSt/Z}( - , \Res_{Y/Z}X) \simeq \Map_{\dSt/Y}(- \times_Z Y, X).
\end{equation}
\end{defin}

\begin{eg}
In arithmetic geometry $\phi: Y \to Z$ is taken to be a field extension $\phi : \Spec L \to \Spec k$; then the adjunction
\ref{eq:weil_restriction} gives a bijection between $L$-points of $X$ and $k$-points of $\Res_{\Spec L/\Spec k}X$. To illustrate
this without getting too much out of our comfort zone, we take $k = \R$, $L = \C$, and start with $X$ an affine variety over $\C$,
given as a subset of $\C^n$ by equations $f_i(z_1, \dots, z_n) = 0$. Then $\Res{\Spec L/\Spec k}X$ is an affine variety over $\R$,
given as a subset of $\R^{2n}$ by equations $\Re f_i(x_1 + i y_1, \dots, x_n + i y_n) = 0$, 
$\Im f_i(x_1 + i y_1, \dots, x_n + i y_n) = 0$.
\end{eg}

Lurie proves the following existence result.

\begin{thm}

\end{thm}
\begin{proof}
The basic idea is to define $\Res_{Y/Z}X$ as the homotopy pullback:
\[
\begin{tikzcd}
\Res_{Y/Z}X\arrow{r}\arrow{d} & \Map_{/Z}(Y,X)\arrow{d} \\
Z\arrow{r} & \Map_{/Z}(Y,Y)
\end{tikzcd}
\]
\end{proof}



\chapter{Reduced Gromov Witten Invariants for K3 Surfaces}
\label{ch4:k3}

(Talk by Benedict Morrissey)  In chapter \ref{ch3:geom}, we constructed a quasismooth derived enhancement of the moduli stack of stable maps. The machinery of chapter \ref{ch2:obs} can then be used to construct virtual fundamental classes, and from there Gromov--Witten invariants. In this chapter we consider the case where the target varietry $X$ is a K3 surface.  In this case Gromov--Witten invariants defined in this fashion vanish. In this chapter, following \cite{schurg2015derived}, we provide an alternative quasismooth derived enhancement. The virtual fundamental classes obtained were earlier described in \cite{maulik2007gromov, maulik2010curves, okounkov2010quantum} (henceforth referred to as OMPT).

\section{Non Reduced Gromov--Witten Invariants for K3 Surfaces}
Note that GW invariants are deformation invariant, but in the moduli space of K3 surfaces there's a dense locus of non-algebraic
K3s. These don't admit any (1,1) classes in $H_2$. It follows that the deformation of the class $[\beta]$ is not effective,
and the corresponding invariants must vanish.

Somehow, passing to the reduced obstruction theory restricts to algebraic deformations only, and this problem disappears.
\todo{understand why this happens}


\todo{Work out where I can find a proof that all these invariants disappear!}

\section{$\R Pic$ for K3 Surfaces}

The point of this section is to give a canonical identification $\R Pic(X)\xrightarrow{\cong}Pic(X)\times \R Spec(Sym(H^{0}(X,K_{X})[1]))$.  Recall that $\R Pic(X)$ is a (locally of finite presentation) derived group stack.

\begin{thm}[Prop. 4.5 in \cite{schurg2015derived}]
\label{thm:derivedgroup}
A locally of finite presentation group stack $G$ over a field $k$, with identity $e: \Spec k\rightarrow G$, and Lie algebra $\mathfrak{g}=T_{e}G$,\footnote{This is the tangent space at the identity, so it's a dg Lie algebra.} has a canonical map
\[\gamma(G):t_{0}(G)\times \R \Spec(A)\rightarrow G,\]
where $A=k\oplus (\mathfrak{g}^{\vee})_{<0}$.
\end{thm}

\begin{proof}
The projection map $A\rightarrow k$ gives a $k$-point $x_{0}:\Spec(k)\rightarrow \R \Spec(A)$.

We wish to find a commuting diagram 
\[
\begin{tikzcd}
B \arrow{rr}{a} && C\\
 & A \arrow{ur}{d} \arrow{ul}{x_{0}}
\end{tikzcd}
\]
this is equivalent to giving a morphism $a':\bbL_{G,e} \cong \mathfrak{g}^{\vee}\rightarrow (\mathfrak{g}^{\vee})_{<0}$ (due to our choice of $A$), hence taking the truncation map $\tau_{< 0}$ (using the standard t-structure on the stable category of (dg) vector spaces) gives 
this map.

We then take the composition
\[t_{0}(G)\times \R \Spec(A)\xrightarrow{j\times a}G\times G\xrightarrow{\times}G,\]
where the final map uses the group product in $G$.
\end{proof}

We now apply Theorem \ref{thm:derivedgroup} to the group stack $\R \Pic(X)$ for a K3 surface $X$.

\begin{thm}
The map $\gamma_{\R \Pic(X)}$ for $X$ a K3 surface gives an isomorphism of derived stacks
\[\R \Pic(X)\xrightarrow{cong} \Pic(X)\times \R \Spec(\Sym(H^{0}(X,K_{X})[1])).\]
\end{thm}

\begin{proof}
We note first that this map clearly provides an isomorphism on truncations.  Hence as $\R Pic(X)$ is a derived group stack, we need only show that it is \'{e}tale at $e$, that is to say 
\[\bbT_{t_{0}(e),x_{0}}(\gamma_{\R \Pic(X)}):\bbT_{t_{0}(e),x_{0}}(\Pic(X)\times \R \Spec(\Sym(H^{0}(X,K_{X})[1])))\rightarrow T_{e}G\]
is an isomorphism of dg k-vector spaces.

Note that, since $H^1(X,\mathcal{O}_X) = 0$: 
\[T_{e}G=\mathfrak{g}=\R \Gamma(X,\mathcal{O}_{X})[1]\cong H^{0}(X,\mathcal{O}_{X})[1]\oplus H^{2}(X,\mathcal{O}_{X})[-1].\] 

Hence 
\[A=\C\oplus (\mathfrak{g}^{\vee})_{<0}=\C\oplus H^{2}(X,\mathcal{O}_{X})[1]\cong \C \oplus H^{0}(X,K_{X})[1]\cong Sym(H^{0}(X,K_{X})[1]\]
where the final step follows because $H^{0}(X,K_{X})$ is free of dimension 1.

Clearly $T_{t_{0}(e),x_{0}}(\gamma_{\R Pic(X)})$ is an isomoprhism.
\end{proof}

This identification allows the definition of a projection $pr_{der}:\R Pic(X)\rightarrow \R Spec(Sym(H^{0}(X,K_{X})[1])$.

\section{The reduced Moduli Space $\R M_{g,n}(X,\beta)^{red}$.}
\label{sec:reduced moduli space}

Recall the map $x_{0}:Spec(\C)\rightarrow  \R Spec(Sym(H^{0}(X,K_{X})[1])$.

We defined the reduced derived enhancement as follows:
\begin{defin}
The \textbf{reduced stack of $n$-pointed stable maps} of genus $g$, class $\beta$ to a K3 surface $X$ is given by the pullback
\[
\begin{tikzcd}
\R M_{g,n}^{red}(X,\beta)\arrow{r}\arrow{d} & \R M_{g,n}(X,\beta)\arrow{d}{\delta_{1}^{der}(X,\beta)} \\
Spec(\C)\arrow{r}{x_{0}} & \R Spec(Sym(H^{0}(X,K_{X})[1])
\end{tikzcd}
\]
where we define the map $\delta_{1}^{der}(X,\beta)$ by the composition
\[\R M_{g,n}(X,\beta)\rightarrow \R M_{g,n}(X)\xrightarrow{a} Perf(X)\xrightarrow{det} \R Pic(X)\xrightarrow{pr_{det}} \R Spec(Sym(H^{0}(X,K_{X})[1]),\]

where the map $a$ is induced by the perfect complex $\R \pi_{*}(\mathcal{O}_{\R \mathcal{C}_{g,X}})$, (that is this specifies a perfect complex over $X\times M_{g,n}(X,\beta)$, and hence a map  $M_{g,n}(X,\beta)\rightarrow Perf(X)$).
\end{defin}

\begin{rem}
This derived enhancement is quasismooth, and we will show that the induced obstruction theory on $M_{g,n}(X,\beta)$ agrees with that of OMPT.
\end{rem}

\begin{prop}
The derived stack $\R M_{g,n}^{red}(X,\beta)$ is quasismooth.
\end{prop}

\begin{proof}
The cotangent complex relative to $M_{g,n}$ is:
\[ \bbT_{\R M_{g,n}(X,\beta)/M_{g,n}, f:C \to X} = \R \Gamma(C,f^*\bbT_X) . \]
The absolute one is:
\[	\bbT_{\R M_{g,n}(X,\beta), f:C \to X} = \R \Gamma\big(C,\text{Cone}(\bbT_C(-\sum x_i) \to f^*\bbT_X)\big) .	\]
We use the pullback square:
\[
\begin{tikzcd}
\R M_{g,n}^{red}(X,\beta) \arrow{r}\arrow{d} & \R M_{g,n}(X,\beta)\arrow{d} \\
\Spec \C \arrow{r} & \R \Spec (\Sym H^0(X,K_X)[1]) .
\end{tikzcd}
\]
This gives the following distinguished triangle of complexes:
\[	\bbT^{red} := \bbT(\R M_{g,n}^{red}(X,\beta)) \to \R \Gamma \big(C,\text{Cone}(\bbT_C(-\sum x_i) \to f^*\bbT_X)\big)
\to H^2(X,\mathcal{O}_X)[-1] .	\]
Looking at the associated long exact sequence of cohomology, it suffices to show that:
\[	H^1 \R \Gamma \big(C,\text{Cone}(\bbT_C(-\sum x_i) \to f^*\bbT_X)\big) \to H^2(X,\mathcal{O}_X)	\]
is surjective. We do this by precomposing with the map:
\[	H^1(C,f^*\bbT_X) \to  H^1 \R \Gamma \big(C,\text{Cone}(\bbT_C(-\sum x_i) \to f^*\bbT_X)\big) ,	\]
and proving that the resulting map is surjective. For the latter, we use the commutative diagram:
\[
\begin{tikzcd}
H^1(X,\bbT_X)\arrow{d}\arrow{r} & \Ext^2_X(\R f_*\mathcal{O}_C, \R f_*\mathcal{O}_C) \arrow{r} & H^2(X,\mathcal{O}_X)[-1] \\
H^1(C,f^*\bbT_X) \arrow{r} & H^1(X,\text{Cone}(\bbT_C(-\sum x_i) \to f^*\bbT_X)) \arrow{u} & 
\end{tikzcd}
\]
\todo{Finish This}
\end{proof}

\section{The Resultant Obstruction Theories}

In this section we compare the obstruction theories obtained from the reduced derived enhancement with those obtained in OMPT.

\todo{Want THM 4.8 from STV}

\todo{Write an explicit description}

The explicit description is as follows.
\begin{align*}
	H^0(\bbT^{red}_f) &\cong H^0\big(C,\text{Cone}(\bbT_C \to f^*\bbT_X)\big)	\\
	H^1(\bbT^{red}_f) &\cong \Ker\big(H^1(\Theta_f) : H^1\big(C,\text{Cone}(\bbT_C \to f^*\bbT_X)\big) \to H^2(X,\mathcal{O}_X)\big)
\end{align*}


\section{The Link to Donaldson--Thomas Theory}

Recall that in section \ref{sec: reduced moduli space} we made use of a map $\R Map (X,\beta)\rightarrow Perf(X) $.  Recall that the stack $Perf(X)$ is used in Donaldson--Thomas theory, a curve counting theory that uses a different compactification of the space of curves in $X$ that the stable curves used in Gromov--Witten invariants.  There are conjectured to be various relationships between Gromov--Witten and Donaldson--Thomas invariants as developed in \cite{maulik2006gromovI, maulik2006gromovII}.

The compactification for DT is as follows. For $C\subset X$ an embedded curve, $f_*\mathcal{O}_C \in \Perf(X)$. More precisely,
$f_*\mathcal{O}_C$ lands in a component (or union thereof) $\Perf^{si, \geq 0, \beta}$. The latter is the DT stack, and we want to
have a theory of integration over it.

For a start, fix $\mathcal{L}$ a line bundle on $X$, corresponding to a map $\Spec \C \to \R \Pic(X)$. Then we define the stack
of perfect complexes with fixed determinant:
\[
\begin{tikzcd}
\Perf(X)_{\mathcal{L}} \arrow{r}\arrow{d} & \Perf(X)\arrow{d}{\det} \\
\Spec \C \arrow{r} & \R\Pic(X).
\end{tikzcd}
\]
We have another constraint:
\[	\Ext^i(F,F) = 0, \forall i<0,	\]
and moreover $\Ext^0(F,F) \simeq H^0(X,\mathcal{O}_X)$, which gives $\Perf(X)^{si \geq 0}$.

For a K3 surface, we can consider again the reduced theory:
\[
\begin{tikzcd}
\Perf(X)^{red} \arrow{r}\arrow{d} & \Perf(X)\arrow{d} \\
\Spec \C \arrow{r} & \R \Spec (\Sym H^0(X,K_X)[1]) .
\end{tikzcd}
\]

In the end we put all of these constraints together to form $\Perf(X)^{si \geq 0, \mathcal{L}, red}$.

\todo{sb draw big diagram, I got lazy}

These triangles are precisely the diagram necessary for functoriality of obstruction theories. Of course, this is rather silly:
we showed that if you restrict to the open subset, the choice of compactification doesn't influence the computation.

\todo{talk about all of S5}


\chapter{Abelian Threefold}

Let's denote by $\eta : \mathcal{O}_M \to \bbL_M \otimes \bbT_M$ the counit and $\epsilon : \bbL_M \otimes \bbT_M \to \mathcal{O}_M$
the unit.
The most natural expressions for $\omega$ and $\iota_{\xi}$ are as follows.
\[
\begin{tikzcd}
p^*\fr g \otimes \wedge^2 \bbL_M \arrow{r}{act \otimes 1} \arrow[bend left]{rr}{\iota} &
\bbT_M \otimes \wedge^2 \bbL_M\arrow{r}{\epsilon} & \bbL_M
\end{tikzcd}
\]
\[
\begin{tikzcd}
\bbT_M\otimes \bbT_M \arrow{r}\arrow[bend left]{rrrr}{\omega} & p_*\mathcal{A} \otimes p_*\mathcal{A}[2] \arrow{r} & 
p_*(\mathcal{A}\otimes \mathcal{A})[2]
\arrow{r} & p_*\mathcal{O}_{M \times A}[2] \arrow{r} & \mathcal{O}_M[2-3]
\end{tikzcd}
\]

The diagram we want to find a lift in can therefore be written as:
\[
\begin{tikzcd}
\; & & & \mathcal{O}_M[-1]\arrow{d}{d_{dR}} \\
p^* \fr g \otimes \mathcal{O}_M \arrow[swap]{r}{\wedge^2 \eta}\arrow[dashed]{urrr}{\mu} & 
p^* \fr g \otimes \wedge^2 \bbT_M \otimes \wedge^2 \bbL_M \arrow[swap]{r}{\omega} &
p^* \fr g \otimes \wedge^2 \bbL_M \otimes \mathcal{O}_M[-1]\arrow[swap]{r}{\iota} & \bbL_M \otimes \mathcal{O}[-1] .
\end{tikzcd}
\]


\section{An Introduction to Derived Symplectic Reduction}





\bibliographystyle{plain}
\bibliography{dahema}

\end{document}




