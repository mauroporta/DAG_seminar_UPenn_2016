\pdfoutput=1
%The other issue is that some packages, such as microtype, produce different output under pdflatex. By default the arXiv goes from dvi to ps to pdf, so if you need pdflatex you have to set the \pdfoutput flag in the TeX file.
\newif\ifpersonal
\newif\ifarxiv
\personaltrue % comment to remove personal notes
\arxivtrue % comment to display shortened version for journal submissions
\RequirePackage[l2tabu, orthodox]{nag} %detect whether obsolete packages are used
\documentclass[10pt,a4paper,reqno,oneside]{book} %reqno places equation numbers on the right
\linespread{1.2}
%\allowdisplaybreaks[1]
\usepackage{calligra}
\usepackage{amsmath,amsthm,amssymb,mathrsfs,mathtools,bm,eucal,tensor} % math related
\usepackage{microtype,fixltx2e} % latex technical issues
\usepackage[scaled]{beramono,berasans}
\usepackage{enumerate,comment,braket,xspace,tikz-cd} %utilities
\usepackage[all,cmtip]{xy} % utilities
\usepackage[utf8]{inputenc} % input encoding
\usepackage[T1]{fontenc} % font encoding
\usepackage{lmodern}
\definecolor{linkcolor}{HTML}{005050}
\usepackage[centering,vscale=0.7,hscale=0.7]{geometry}
\usepackage{hyperref}
\usepackage[capitalize]{cleveref}
\usepackage{graphicx}
\usepackage{xparse}
\usepackage{url}

%\makeevenhead{headings}{\thepage}{}{\leftmark}
%\setlrmarginsandblock{3cm}{3.5cm}{*}
%\setlength\marginparwidth{2.5cm}
%\checkandfixthelayout
%
%\setlength\headheight{24pt}

\usepackage{vmargin}
\setpapersize{A4}
\setmarginsrb{25mm}{10mm}{25mm}{10mm}%
{12mm}{10mm}{5mm}{10mm}

\usepackage{fancyhdr}
\pagestyle{fancy}
%%%Settings
\renewcommand{\chaptermark}[1]{\markboth{#1}{}}
\renewcommand{\sectionmark}[1]{\markright{\thesection\ #1}}
\fancyhf{}
\fancyhead[LE,RO]{\bfseries\thepage}
\fancyhead[RE]{\bfseries\footnotesize\nouppercase{\leftmark}}
\fancyhead[LO]{\bfseries\footnotesize\nouppercase{\rightmark}}

\theoremstyle{plain}
\newtheorem{thm-intro}{Theorem}
\newtheorem{thm}{Theorem}[section]
\newtheorem*{thm*}{Theorem}
\newtheorem{lem}[thm]{Lemma}
\newtheorem*{lem*}{Lemma}
\newtheorem{prop}[thm]{Proposition}
\newtheorem{conj}[thm]{Conjecture}
\newtheorem{cor}[thm]{Corollary}
\newtheorem{cor-intro}[thm-intro]{Corollary}
\newtheorem{assumption}[thm]{Assumption}
\theoremstyle{definition}
\newtheorem{defin}[thm]{Definition}
\newtheorem{question}[thm]{Question}
\newtheorem{exercise}[thm]{Exercise}
\newtheorem{defin-intro}[thm-intro]{Definition}
\newtheorem{notation}[thm]{Notation}
\theoremstyle{remark}
\newtheorem*{rem*}{Remark}
\newtheorem{eg}[thm]{Example}
\newtheorem{eg-intro}[thm-intro]{Example}
\newtheorem{rem}[thm]{Remark}
\newtheorem{rem-intro}[thm-intro]{Remark}
\numberwithin{equation}{section}
\newtheorem{construction}[thm]{Construction}

% personal remarks

\ifpersonal
\newcommand*{\personal}[1]{\textcolor[rgb]{0.6,0.6,1}{(Personal: #1)}}
\newcommand*{\todo}[1]{\textcolor{red}{(Todo: #1)}}
\else
\newcommand*{\personal}[1]{\ignorespaces}
\newcommand*{\todo}[1]{\ignorespaces}
\fi

% Fonts
\newcommand{\C}{\mathbb C}
\newcommand{\CP}{\mathbb{CP}}
\newcommand{\F}{\mathbb F}
\newcommand{\Q}{\mathbb Q}
\newcommand{\R}{\mathbb R}
\newcommand{\Z}{\mathbb Z}
\newcommand{\N}{\mathbb N}

\newcommand{\rB}{\mathrm B}
\newcommand{\rD}{\mathrm D}
\newcommand{\rH}{\mathrm H}
\newcommand{\rI}{\mathrm I}
\newcommand{\rL}{\mathrm L}
\newcommand{\rP}{\mathrm P}
\newcommand{\rQ}{\mathrm Q}
\newcommand{\rR}{\mathrm R}
\newcommand{\rb}{\mathrm b}
\newcommand{\rd}{\mathrm d}
\newcommand{\rh}{\mathrm h}
\newcommand{\rs}{\mathrm s}
\newcommand{\rt}{\mathrm t}


\newcommand{\fA}{\mathfrak A}
\newcommand{\fB}{\mathfrak B}
\newcommand{\fC}{\mathfrak C}
\newcommand{\fD}{\mathfrak D}
\newcommand{\fH}{\mathfrak H}
\newcommand{\fS}{\mathfrak S}
\newcommand{\fT}{\mathfrak T}
\newcommand{\fU}{\mathfrak U}
\newcommand{\fV}{\mathfrak V}
\newcommand{\fX}{\mathfrak X}
\newcommand{\fY}{\mathfrak Y}
\newcommand{\fZ}{\mathfrak Z}
\newcommand{\ff}{\mathfrak f}
\newcommand{\fm}{\mathfrak m}
\newcommand{\fn}{\mathfrak n}
\newcommand{\fs}{\mathfrak s}
\newcommand{\ft}{\mathfrak t}

\newcommand{\cA}{\mathcal A}
\newcommand{\cB}{\mathcal B}
\newcommand{\cC}{\mathcal C}
\newcommand{\cD}{\mathcal D}
\newcommand{\cE}{\mathcal E}
\newcommand{\cF}{\mathcal F}
\newcommand{\cH}{\mathcal H}
\newcommand{\cG}{\mathcal G}
\newcommand{\cI}{\mathcal I}
\newcommand{\cJ}{\mathcal J}
\newcommand{\cK}{\mathcal K}
\newcommand{\cL}{\mathcal L}
\newcommand{\cM}{\mathcal M}
\newcommand{\cN}{\mathcal N}
\newcommand{\cO}{\mathcal O}
\newcommand{\cP}{\mathcal P}
\newcommand{\cR}{\mathcal R}
\newcommand{\cS}{\mathcal S}
\newcommand{\cT}{\mathcal T}
\newcommand{\cU}{\mathcal U}
\newcommand{\cV}{\mathcal V}
\newcommand{\cW}{\mathcal W}
\newcommand{\cX}{\mathcal X}
\newcommand{\cY}{\mathcal Y}
\newcommand{\cZ}{\mathcal Z}
\DeclareFontFamily{U}{BOONDOX-calo}{\skewchar\font=45 }
\DeclareFontShape{U}{BOONDOX-calo}{m}{n}{<-> s*[1.05] BOONDOX-r-calo}{}
\DeclareFontShape{U}{BOONDOX-calo}{b}{n}{<-> s*[1.05] BOONDOX-b-calo}{}
\DeclareMathAlphabet{\mathcalboondox}{U}{BOONDOX-calo}{m}{n}
%\DeclareMathAlphabet{\mathcalligra}{T1}{calligra}{m}{n}
\newcommand{\cf}{\mathcalboondox f}

\newcommand{\bbA}{\mathbb A}
\newcommand{\bbD}{\mathbb D}
\newcommand{\bbG}{\mathbb G}
\newcommand{\bbL}{\mathbb L}
\newcommand{\bbP}{\mathbb P}
\newcommand{\bbT}{\mathbb T}
\newcommand{\bbV}{\mathbb V}

\newcommand{\bA}{\mathbf A}
\newcommand{\bD}{\mathbf D}
\newcommand{\bP}{\mathbf P}
\newcommand{\bQ}{\mathbf Q}
\newcommand{\bT}{\mathbf T}
\newcommand{\bX}{\mathbf X}
\newcommand{\bY}{\mathbf Y}
\newcommand{\be}{\mathbf e}
\newcommand{\br}{\mathbf r}
\newcommand{\bu}{\mathbf u}
\newcommand{\balpha}{\bm{\alpha}}
\newcommand{\bDelta}{\bm{\Delta}}
\newcommand{\brho}{\bm{\rho}}

\newcommand{\sC}{\mathscr C}
\newcommand{\sX}{\mathscr X}
\newcommand{\sD}{\mathscr D}
\newcommand{\sU}{\mathscr U}


% Decorations

% Definition of \widebar from http://tex.stackexchange.com/questions/16337/can-i-get-a-widebar-without-using-the-mathabx-package/60253#60253
\makeatletter
\let\save@mathaccent\mathaccent
\newcommand*\if@single[3]{%
	\setbox0\hbox{${\mathaccent"0362{#1}}^H$}%
	\setbox2\hbox{${\mathaccent"0362{\kern0pt#1}}^H$}%
	\ifdim\ht0=\ht2 #3\else #2\fi
}
%The bar will be moved to the right by a half of \macc@kerna, which is computed by amsmath:
\newcommand*\rel@kern[1]{\kern#1\dimexpr\macc@kerna}
%If there's a superscript following the bar, then no negative kern may follow the bar;
%an additional {} makes sure that the superscript is high enough in this case:
\newcommand*\widebar[1]{\@ifnextchar^{{\wide@bar{#1}{0}}}{\wide@bar{#1}{1}}}
%Use a separate algorithm for single symbols:
\newcommand*\wide@bar[2]{\if@single{#1}{\wide@bar@{#1}{#2}{1}}{\wide@bar@{#1}{#2}{2}}}
\newcommand*\wide@bar@[3]{%
	\begingroup
	\def\mathaccent##1##2{%
		%Enable nesting of accents:
		\let\mathaccent\save@mathaccent
		%If there's more than a single symbol, use the first character instead (see below):
		\if#32 \let\macc@nucleus\first@char \fi
		%Determine the italic correction:
		\setbox\z@\hbox{$\macc@style{\macc@nucleus}_{}$}%
		\setbox\tw@\hbox{$\macc@style{\macc@nucleus}{}_{}$}%
		\dimen@\wd\tw@
		\advance\dimen@-\wd\z@
		%Now \dimen@ is the italic correction of the symbol.
		\divide\dimen@ 3
		\@tempdima\wd\tw@
		\advance\@tempdima-\scriptspace
		%Now \@tempdima is the width of the symbol.
		\divide\@tempdima 10
		\advance\dimen@-\@tempdima
		%Now \dimen@ = (italic correction / 3) - (Breite / 10)
		\ifdim\dimen@>\z@ \dimen@0pt\fi
		%The bar will be shortened in the case \dimen@<0 !
		\rel@kern{0.6}\kern-\dimen@
		\if#31
		\overline{\rel@kern{-0.6}\kern\dimen@\macc@nucleus\rel@kern{0.4}\kern\dimen@}%
		\advance\dimen@0.4\dimexpr\macc@kerna
		%Place the combined final kern (-\dimen@) if it is >0 or if a superscript follows:
		\let\final@kern#2%
		\ifdim\dimen@<\z@ \let\final@kern1\fi
		\if\final@kern1 \kern-\dimen@\fi
		\else
		\overline{\rel@kern{-0.6}\kern\dimen@#1}%
		\fi
	}%
	\macc@depth\@ne
	\let\math@bgroup\@empty \let\math@egroup\macc@set@skewchar
	\mathsurround\z@ \frozen@everymath{\mathgroup\macc@group\relax}%
	\macc@set@skewchar\relax
	\let\mathaccentV\macc@nested@a
	%The following initialises \macc@kerna and calls \mathaccent:
	\if#31
	\macc@nested@a\relax111{#1}%
	\else
	%If the argument consists of more than one symbol, and if the first token is
	%a letter, use that letter for the computations:
	\def\gobble@till@marker##1\endmarker{}%
	\futurelet\first@char\gobble@till@marker#1\endmarker
	\ifcat\noexpand\first@char A\else
	\def\first@char{}%
	\fi
	\macc@nested@a\relax111{\first@char}%
	\fi
	\endgroup
}
\makeatother


\newcommand{\oDelta}{\widebar\Delta}
\newcommand{\oGamma}{\widebar\Gamma}
\newcommand{\oSigma}{\widebar\Sigma}
\newcommand{\oalpha}{\widebar\alpha}
\newcommand{\obeta}{\widebar\beta}
\newcommand{\otau}{\widebar\tau}
\newcommand{\oC}{\widebar C}
\newcommand{\oD}{\widebar D}
\newcommand{\oE}{\widebar E}
\newcommand{\oG}{\widebar G}
\newcommand{\oM}{\widebar M}
\newcommand{\oR}{\widebar R}
\newcommand{\oS}{\widebar S}
\newcommand{\oU}{\widebar U}
\newcommand{\oW}{\widebar W}
\newcommand{\oX}{\widebar X}
\newcommand{\oY}{\widebar Y}
\newcommand{\oPhi}{\overline{\Phi}}


\newcommand{\ok}{\widebar k}
\newcommand{\ov}{\widebar v}
\newcommand{\ox}{\widebar x}
\newcommand{\oy}{\widebar y}
\newcommand{\oz}{\widebar z}

\newcommand{\hh}{\widehat h}
\newcommand{\hf}{\widehat f}
\newcommand{\hA}{\widehat A}
\newcommand{\hB}{\widehat B}
\newcommand{\hC}{\widehat C}
\newcommand{\hE}{\widehat E}
\newcommand{\hF}{\widehat F}
\newcommand{\hI}{\widehat I}
\newcommand{\hL}{\widehat L}
\newcommand{\hU}{\widehat U}
\newcommand{\hZ}{\hat Z}
\newcommand{\hbeta}{\widehat\beta}
\newcommand{\hGamma}{\widehat\Gamma}
\newcommand{\hPhi}{\widehat{\Phi}}
\newcommand{\hPsi}{\widehat{\Psi}}

\newcommand{\hrP}{\widehat \rP}

\newcommand{\tw}{\widetilde w}
\newcommand{\tW}{\widetilde W}
\newcommand{\tk}{\tilde k}
\newcommand{\tv}{\tilde v}
\newcommand{\tB}{\widetilde B}
\newcommand{\tD}{\widetilde D}
\newcommand{\tI}{\widetilde I}
\newcommand{\tM}{\widetilde M}
\newcommand{\tN}{\widetilde N}
\newcommand{\tP}{\widetilde P}
\newcommand{\tR}{\widetilde R}
\newcommand{\tX}{\widetilde X}
\newcommand{\tfX}{\widetilde{\fX}}
\newcommand{\tfB}{\widetilde{\fB}}
\newcommand{\tsX}{\widetilde{\sX}}
\newcommand{\tH}{\widetilde H}
\newcommand{\tY}{\widetilde Y}
\newcommand{\tbeta}{\widetilde{\beta}}
\newcommand{\tphi}{\widetilde{\phi}}
\newcommand{\ttau}{\widetilde{\tau}}

% Global tropicalization
\newcommand{\Ih}{I^\mathrm{h}}
\newcommand{\Iv}{I^\mathrm{v}}
\newcommand{\IX}{I_\fX}
\newcommand{\IY}{I_\fY}
\newcommand{\SD}{S_\fD}
\newcommand{\SX}{S_\fX}
\newcommand{\SsXH}{S_{(\sX,H)}}
\newcommand{\SY}{S_\fY}
\newcommand{\CsXH}{C_{(\sX,H)}}
\newcommand{\oF}{\overline{F}}
\newcommand{\oP}{\overline{P}}
\newcommand{\oSX}{\overline{\SX}}
\newcommand{\oIX}{\overline{\IX}}


% Vanishing cycles
\newcommand{\fXe}{\fX_\eta}
\newcommand{\fXs}{\fX_s}
\newcommand{\ofX}{\widebar{\fX}}
\newcommand{\ofXs}{\widebar{\fX}_s}
\newcommand{\fYe}{\fY_\eta}
\newcommand{\fYs}{\fY_s}
\newcommand{\fXbs}{\fX_{\bar s}}
\newcommand{\fXbe}{\fX_{\bar\eta}}
\newcommand{\fDe}{\fD_\eta}
\newcommand{\LX}{\Lambda_{\fX}}
\newcommand{\LXe}{\Lambda_{\fX_\eta}}
\newcommand{\LXs}{\Lambda_{\fXbs}}
\newcommand{\QXe}{\Q_{\ell,\fX_\eta}}
\newcommand{\QXbs}{\Q_{\ell,\fXbs}}
\newcommand{\sXe}{\sX_\eta}
\newcommand{\sXs}{\sX_s}
\newcommand{\LUe}{\Lambda_{\fU_\eta}}
\newcommand{\fCbs}{\fC_{\bar s}}
\newcommand{\QUe}{\Q_{\ell,\fU_\eta}}
\newcommand{\QCe}{\Q_{\ell,\fC_\eta}}
\newcommand{\QCs}{\Q_{\ell,\fCbs}}

% stacks

\newcommand{\hcC}{\mathrm h\cC}
\newcommand{\hcD}{\mathrm h\cD}
\newcommand{\PSh}{\mathrm{PSh}}
\newcommand{\Sh}{\mathrm{Sh}}
\newcommand{\Shv}{\mathrm{Shv}}
\newcommand{\Tuupperp}{\tensor*[^\cT]{u}{^p}}
\newcommand{\Tulowerp}{\tensor*[^\cT]{u}{_p}}
\newcommand{\Tuuppers}{\tensor*[^\cT]{u}{^s}}
\newcommand{\Tulowers}{\tensor*[^\cT]{u}{_s}}
\newcommand{\pu}{\tensor*[_p]{u}{}}
\newcommand{\su}{\tensor*[_s]{u}{}}
\newcommand{\Tpu}{\tensor*[^\cT_p]{u}{}}
\newcommand{\Tsu}{\tensor*[^\cT_s]{u}{}}
\newcommand{\Dfpull}{\tensor*[^\cD]{f}{^{-1}}}
\newcommand{\Dfpush}{\tensor*[^\cD]{f}{_*}}
\newcommand{\Duuppers}{\tensor*[^\cD]{u}{^s}}
\newcommand{\Dulowers}{\tensor*[^\cD]{u}{_s}}
\newcommand{\Geom}{\mathrm{Geom}}
\newcommand{\LPr}{\mathcal{P}\mathrm{r}^\rL}
\newcommand{\RPr}{\mathcal{P}\mathrm{r}^\rR}
\newcommand{\LPromega}{\mathcal{P}\mathrm{r}^{\rL, \omega}}
\newcommand{\LPromegast}{\mathcal{P}\mathrm{r}^{\rL, \omega}_{\mathrm{Ex}}}
\newcommand{\CX}{\cC_{/X}}
\newcommand{\CY}{\cC_{/Y}}
\newcommand{\CXP}{(\cC_{/X})_{\bP}}
\newcommand{\GeomXP}{(\mathrm{Geom}_{/X})_\bP}
\newcommand{\GeomYP}{(\mathrm{Geom}_{/Y})_\bP}
\newcommand{\infcat}{$\infty$-category\xspace}
\newcommand{\infcats}{$\infty$-categories\xspace}
\newcommand{\infsite}{$\infty$-site\xspace}
\newcommand{\infsites}{$\infty$-sites\xspace}
\newcommand{\inftopos}{$\infty$-topos\xspace}
\newcommand{\inftopoi}{$\infty$-topoi\xspace}
\newcommand{\pres}{{}^{\mathrm L} \mathcal P \mathrm{res}}
\newcommand{\Grpd}{\mathrm{Grpd}}
\newcommand{\sSet}{\mathrm{sSet}}
\newcommand{\rSet}{\mathrm{Set}}
\newcommand{\Ab}{\mathrm{Ab}}
\newcommand{\DAb}{\cD(\Ab)}
\newcommand{\tauan}{\tau_\mathrm{an}}
\newcommand{\qet}{\mathrm{q\acute{e}t}}
\newcommand{\tauet}{\tau_\mathrm{\acute{e}t}}
\newcommand{\tauqet}{\tau_\mathrm{q\acute{e}t}}
\newcommand{\bPsm}{\bP_\mathrm{sm}}
\newcommand{\bPqsm}{\bP_\mathrm{qsm}}
\newcommand{\Modh}{\textrm{-}\mathrm{Mod}^\heartsuit}
\newcommand{\Mod}{\textrm{-}\mathrm{Mod}}
\newcommand{\Coh}{\mathrm{Coh}}
\newcommand{\Cohb}{\mathrm{Coh}^\mathrm{b}}
\newcommand{\Cohh}{\mathrm{Coh}^\heartsuit}
\newcommand{\QCohh}{\mathrm{QCoh}^\heartsuit}
\newcommand{\RcHom}{\rR\!\mathcal H\!\mathit{om}}
\newcommand{\kfiltered}{$\kappa$-filtered\xspace}
\newcommand{\Stn}{\mathrm{Stn}}
\newcommand{\Sch}{\mathrm{Sch}}
\newcommand{\FSch}{\mathrm{FSch}}
\newcommand{\Aff}{\mathrm{Aff}}
\newcommand{\Afflfp}{\mathrm{Aff}^{\mathrm{lfp}}}
\newcommand{\An}{\mathrm{An}}
\newcommand{\Afd}{\mathrm{Afd}}
\newcommand{\Top}{\mathcal T\mathrm{op}}
\newcommand{\bfMap}{\mathbf{Map}}



% DAnG

\newcommand{\dAnk}{\mathrm{dAn}_k}
\newcommand{\Ank}{\mathrm{An}_k}
\newcommand{\cTan}{\cT_{\mathrm{an}}}
\newcommand{\cTannc}{\cT_{\mathrm{an}}^{\mathrm{nc}}}
\newcommand{\cTank}{\cT_{\mathrm{an}}(k)}
\newcommand{\cTdisck}{\cT_{\mathrm{disc}}(k)}
\newcommand{\cTet}{\cT_{\mathrm{\acute{e}t}}}
\newcommand{\cTetnc}{\cTet^{\mathrm{nc}}}
\newcommand{\cTetk}{\cT_{\mathrm{\acute{e}t}}(k)}
\newcommand{\Strloc}{\mathrm{Str}^\mathrm{loc}}
\newcommand{\RTop}{\tensor*[^\rR]{\Top}{}}
\newcommand{\LTop}{\tensor*[^\rL]{\Top}{}}
\newcommand{\RHTop}{\tensor*[^\rR]{\mathcal{H}\Top}{}}
\newcommand{\LRT}{\mathrm{LRT}}
\newcommand{\Tor}{\mathrm{Tor}}
\newcommand{\dAfd}{\mathrm{dAfd}}
\newcommand{\dAfdk}{\mathrm{dAfd}_k}
\newcommand{\biget}{\mathrm{big,\acute{e}t}}
\newcommand{\trunc}{\mathrm{t}_0}
\newcommand{\Hyp}{\mathrm{Hyp}}
\newcommand{\HSpec}{\mathrm{HSpec}}
\newcommand{\CAlg}{\mathrm{CAlg}}
\newcommand{\trunctopoi}{\Spec^{\cG_{\mathrm{an}}^{\le 0}(k)}_{\cG_{\mathrm{an}(k)}}}

% Formal Gluing

\newcommand{\IndPro}[1]{\mathrm{Ind}(\mathrm{Pro}(#1))}
\newcommand{\GFRings}{\mathrm{GFRings}}
\newcommand{\Pro}{\mathrm{Pro}}
\newcommand{\Ind}{\mathrm{Ind}}
\newcommand{\preNbd}{\mathrm{PNbd}}
\newcommand{\Nbd}{\mathrm{Nbd}^{\circ}}
\newcommand{\cHom}{\cH \mathrm{om}}
\newcommand {\D} {\mathsf{L}}
\newcommand{\St}{\mathbf{St}}
\newcommand{\dSt}{\mathbf{dSt}}
\newcommand{\Tw}{\mathrm{Tw}}
\newcommand{\Lan}{\mathrm{Lan}}
\newcommand{\IndCoh}{\mathrm{IndCoh}}
\newcommand{\QCoh}{\mathrm{QCoh}}
\newcommand{\Perf}{\mathrm{Perf}}
\newcommand{\lex}{\mathrm{lex}}
\newcommand{\Dsing}{\rD_\mathrm{sing}}
\newcommand{\fib}{\mathrm{fib}}
\newcommand{\cofib}{\mathrm{cofib}}
\newcommand{\stMap}{\mathrm{Map}^{\mathrm{st}}}
\newcommand{\Zar}{\mathrm{Zar}}
\newcommand{\Cat}{\mathrm{Cat}}
\newcommand{\AbCat}{\mathrm{AbCat}}
\newcommand{\bfCoh}{\mathbf{Coh}}
\newcommand{\bfPerf}{\mathbf{Perf}}
\newcommand{\bfQCoh}{\mathbf{QCoh}}
\newcommand{\Catst}{\Cat_\infty^{\mathrm{Ex}}}
\newcommand{\Catstidem}{\Cat_\infty^{\mathrm{Ex}, \mathrm{idem}}}
\newcommand{\Catstlc}{\Cat_\infty^{\mathrm{Ex}, \mathrm{l.c.}}}
\newcommand{\Catstlb}{\Cat_\infty^{\mathrm{Ex}, \mathrm{l.b.}}}

\newcommand{\bfBun}{\operatorname{\mathbf{Bun}}}
\newcommand{\Bun}{\operatorname{\mathrm{Bun}}}
\newcommand{\Bunhat}{\operatorname{\mathbf{B\widehat{un}}}}

% Special symbols
\newcommand{\bcM}{\widebar{\mathcal M}}
\newcommand{\bcC}{\widebar{\mathcal C}}
\newcommand{\bcMgn}{\widebar{\mathcal M}_{g,n}}
\newcommand{\bcMol}{\widebar{\mathcal M}_{0,1}}
\newcommand{\bcMot}{\widebar{\mathcal M}_{0,3}}
\newcommand{\bcMof}{\widebar{\mathcal M}_{0,4}}
\newcommand{\bcMon}{\widebar{\mathcal M}_{0,n}}
\newcommand{\bcMgnprime}{\widebar{\mathcal M}_{g,n'}}
\newcommand{\bcMgnijprime}{\widebar{\mathcal M}_{g_{ij},n'_{ij}}}
\newcommand{\bMgnt}{\widebar{M}^\mathrm{trop}_{g,n}}
\newcommand{\Mmdisc}{M_{m\textrm{-disc}}}
\newcommand{\Gm}{\mathbb G_{\mathrm m}}
\newcommand{\Gmk}{\mathbb G_{\mathrm m/k}}
\newcommand{\Gmkprime}{\mathbb G_{\mathrm m/k'}}
\newcommand{\Gmnan}{(\Gm^n)\an}
\newcommand{\Gmknan}{(\Gmk^n)\an}
\newcommand{\Lin}{\mathit{Lin}}
\newcommand{\Simp}{\mathit{Simp}}
\newcommand{\vol}{\mathit{vol}}
\newcommand{\LanD}{\mathcal L_{an}^D}

% Categories


% Shorthands
\newcommand{\kc}{k^\circ}
\newcommand{\llb}{[\![}
\newcommand{\rrb}{]\!]}
\newcommand{\llp}{(\!(}
\newcommand{\rrp}{)\!)}
\newcommand{\an}{^\mathrm{an}}
\newcommand{\alg}{^\mathrm{alg}}
\newcommand{\loweralg}{_\mathrm{alg}}
\newcommand{\bad}{^\mathrm{bad}}
\newcommand{\ess}{^\mathrm{ess}}
\newcommand{\ness}{^\mathrm{ness}}
\newcommand{\et}{_\mathrm{\acute{e}t}}
\newcommand{\Et}{_\mathrm{\acute{E}t}}
\newcommand{\ev}{\mathrm{ev}}
%\newcommand{\eistar}{\mathbf e_i^*}
%\newcommand{\ejstar}{\mathbf e_j^*}
%\newcommand{\ekstar}{\mathbf e_k^*}
\newcommand{\mult}{\mathit{mult}}
\newcommand{\inv}{^{-1}}
\newcommand{\id}{\mathrm{id}}
\newcommand{\gn}{$n$-pointed genus $g$ }
\newcommand{\gnprime}{$n'$-pointed genus $g$ }
\newcommand{\GW}{\mathrm{GW}}
\newcommand{\GWon}{\GW_{0,n}}
\newcommand{\canal}{$\mathbb C$-analytic\xspace}
\newcommand{\nanal}{non-archimedean analytic\xspace}
\newcommand{\kanal}{$k$-analytic\xspace}
\newcommand{\ddim}{$d$-dimensional\xspace}
\newcommand{\ndim}{$n$-dimensional\xspace}
\newcommand{\narch}{non-archimedean\xspace}
\newcommand{\nminusone}{$(n\!-\!1)$}
\newcommand{\nminustwo}{$(n\!-\!2)$}
\newcommand{\red}{^\mathrm{red}}
\renewcommand{\th}{^\mathrm{\tiny th}}
\newcommand{\Wall}{\mathit{Wall}}
\newcommand{\vlb}{virtual line bundle\xspace}
\newcommand{\mvlb}{metrized \vlb}
\newcommand{\wrt}{with respect to\xspace}
\newcommand{\Zaffine}{$\mathbb Z$-affine\xspace}
\newcommand{\sw}{^\mathrm{sw}}
\newcommand{\Trop}{\mathrm{Trop}}
\newcommand{\trop}{^\mathrm{trop}}
\newcommand{\op}{^\mathrm{op}}
\newcommand{\Cech}{\check{\mathcal C}}
\newcommand{\DM}{Deligne-Mumford\xspace}
\providecommand{\abs}[1]{\lvert#1\rvert}
\providecommand{\norm}[1]{\lVert#1\rVert}
\newcommand{\fr}{\mathfrak}
\newcommand{\p}{\partial}



% Arrows
\newcommand*{\longhookrightarrow}{\ensuremath{\lhook\joinrel\relbar\joinrel\rightarrow}}
\newcommand*{\DashedArrow}[1][]{\mathbin{\tikz [baseline=-0.25ex,-latex, dashed,#1] \draw [#1] (0pt,0.5ex) -- (1.3em,0.5ex);}}

\usetikzlibrary{decorations.markings} %arrows for open immersions and closed immersions
\tikzset{
  closed/.style = {decoration = {markings, mark = at position 0.5 with { \node[transform shape, xscale = .8, yscale=.4] {/}; } }, postaction = {decorate} },
  open/.style = {decoration = {markings, mark = at position 0.5 with { \node[transform shape, scale = .7] {$\circ$}; } }, postaction = {decorate} }
}


%Operators
\DeclareMathOperator{\Alg}{Alg}
\DeclareMathOperator{\Anc}{Anc}
\DeclareMathOperator{\Area}{Area}
\DeclareMathOperator{\at}{at}
\DeclareMathOperator{\Aut}{Aut}
\DeclareMathOperator{\Bl}{Bl}
\DeclareMathOperator{\cdga}{cdga}
\DeclareMathOperator{\CH}{CH}
\DeclareMathOperator{\Ch}{Ch}
\DeclareMathOperator{\Chow}{Chow}
\DeclareMathOperator{\Coker}{Coker}
\DeclareMathOperator{\codim}{codim}
\DeclareMathOperator{\cosk}{cosk}
\DeclareMathOperator{\Der}{Der}
\DeclareMathOperator{\dgVect}{dgVect}
\DeclareMathOperator{\Div}{Div}
\DeclareMathOperator{\dist}{dist}
\DeclareMathOperator{\dMan}{dMan}
\DeclareMathOperator{\End}{End}
\DeclareMathOperator{\Ext}{Ext}
\DeclareMathOperator{\Fun}{Fun}
\DeclareMathOperator{\FunR}{Fun^R}
\DeclareMathOperator{\FunL}{Fun^L}
\DeclareMathOperator{\Gal}{Gal}
\DeclareMathOperator{\Hom}{Hom}
\DeclareMathOperator{\Image}{Im}
\DeclareMathOperator{\Int}{Int}
\DeclareMathOperator{\Isom}{Isom}
\DeclareMathOperator{\Ker}{Ker}
\DeclareMathOperator{\KurNbd}{KurNbd}
\DeclareMathOperator{\loc}{loc}
\DeclareMathOperator{\LocTopInf}{LocTopInf}
\DeclareMathOperator{\Map}{Map}
\DeclareMathOperator{\Mor}{Mor}
\DeclareMathOperator{\NE}{NE}
\DeclareMathOperator{\oStar}{\widebar{\Star}}
\DeclareMathOperator{\pt}{pt}
\DeclareMathOperator{\Pic}{Pic}
\DeclareMathOperator{\Proj}{Proj}
\DeclareMathOperator{\rank}{rank}
\DeclareMathOperator{\Res}{Res}
\DeclareMathOperator{\RHom}{RHom}
\DeclareMathOperator{\Sp}{Sp}
\DeclareMathOperator{\Spa}{Spa}
\DeclareMathOperator{\SpB}{Sp_\mathrm{B}}
\DeclareMathOperator{\Spec}{Spec}
\DeclareMathOperator{\Spf}{Spf}
\DeclareMathOperator{\Star}{Star}
\DeclareMathOperator{\supp}{supp}
\DeclareMathOperator{\Sym}{Sym}
\DeclareMathOperator{\Symp}{Symp}
\DeclareMathOperator{\Td}{Td}
\DeclareMathOperator{\Tdisc}{T_{\text{disc}}}
\DeclareMathOperator{\Tr}{Tr}
\DeclareMathOperator{\tr}{tr}
\DeclareMathOperator{\val}{val}
\DeclareMathOperator{\vdim}{vdim}
\DeclareMathOperator{\Vect}{Vect}
\DeclareMathOperator{\vir}{vir}

\DeclareMathOperator*{\hofib}{hofib}
\DeclareMathOperator*{\hocofib}{hocofib}
\DeclareMathOperator*{\colim}{colim}
\DeclareMathOperator*{\holim}{holim}
\DeclareMathOperator*{\hocolim}{hocolim}
\DeclareMathOperator*{\cotimes}{\widehat{\otimes}}


\title{Derived Algebraic Geometry Seminar: UPenn 2017}

\begin{document}

\maketitle
\tableofcontents

\chapter*{Introduction}
\addcontentsline{toc}{chapter}{Introduction} \markboth{INTRODUCTION}{}

This contains notes for the Derived Algebraic Geometry Seminar currently being held at the University of Pennsylvania 
math department in the 2016-17 academic year. Having introduced the machinery of Derived Algebraic Geometry the previous semester,
we investigate its applications to producing Virtual Fundamental Cycles. Initially we will focus on moduli spaces of stable
maps, with various boundary conditions, and how VFCs for these can be used to construct Gromov-Witten invariants and Floer-type
theories.

This is a draft and errors should be expected.




\chapter{Stable Maps and Gromov-Witten Invariants}
\label{ch:gw}

(Talk by Matei Ionita)

\section{The Counting Problem}
\label{sect:counting}
Basic idea of ennumerative geometry, as explained in \cite{invitation} 3.1: set up a moduli space M
for the objects, e.g. curves,
one wants to count: $\mathcal{M}_{g,n}(X,\beta)$, equipped with (flat) evaluation maps
$\nu_i : \mathcal{M}_{g,n}(X,\beta) \to X$, given by $\big(C, p_1, \dots p_n, \mu\big) \mapsto \mu(p_i)$. 
Each constraint $\nu_i \in \Gamma_i$, where $\Gamma_i \in H_*(X,\Z)$, gives a subscheme, of $\mathcal{M}_{g,n}(X,\beta)$.
We take the intersection of all these: 
\begin{equation}
\label{eq:intersection_count}
	\bigcap_{i=1}^m \nu_i^* \Gamma_i .
	\footnote{This pullback is an umkehr map and we need some assumptions; is properness of $\mu_i$ enough?}
\end{equation}
If the intersections are transverse and the result has dimension 0, can count the number of points. We would like to set up $\Gamma_i$
such that:
\[	\sum_{i=1}^m \codim \Gamma_i = \dim \mathcal{M}_{g,n}(X,\beta).	\]
Thus the ennumerative problem is reduced to intersection theory in M. In order to
do intersection theory successfully, M needs to be compact (proper), and we need to understand
its Chow ring, where the subschemes live.

A first modification: in order to drop the transversality assumption on $\Gamma_i$, we replace them with the Poincar\'e dual
cohomology classes $\gamma_i$, and take cup products then \ref{eq:intersection_count} is replaced by a first naive definition
of the \textbf{Gromov-Witten invariants}:
\begin{equation}
\label{eq:gw_naive_def}
	I_{g,n,\beta} := \int_{[\mathcal{M}_{g,n}(X,\beta)]} \bigwedge_i \nu_i^* \gamma_i.
\end{equation}
If $\mathcal{M}_{g,n}(X,\beta)$ is smooth and proper, then $[\mathcal{M}_{g,n}(X,\beta)]$ is the fundamental class, against
which it makes sense to evaluate cohomology classes. $I_{g,n,\beta}$ is defined to be 0 unless $\sum_i \deg \gamma_i =
\dim \mathcal{M}_{g,n}(X,\beta)$.





\section{Axiomatic Definition of GW}
\label{sect:axiom}

The axiomatic approach of Kontsevich and Manin in \cite{km_gw} is as follows. Let $\overline{\mathcal{M}}_{g,n}$ denote the
Deligne-Mumford compactification by stable curves of the moduli stack of genus $g$ curves with $n$ marked points.
We take this as a well-understood object and explain the rest.

\begin{defin}[2.2 in \cite{km_gw}]
A \textbf{system of Gromov-Witten classes for X} is a family of linear maps:
\[	I_{g,n,\beta}^X : H^*(X,\Q)^{\otimes n} \to H^*(\overline{\mathcal{M}}_{g,n},\Q)	\]
defined for $n+2g-3\geq 0$, and satisfying the following axioms.
\begin{enumerate}
\item \textbf{Effectivity}: $I_{g,n,\beta} = 0$ for $\beta$ non-effective, i.e. not in the dual of the K\"ahler cone.

\item $S_n$\textbf{-covariance}: equivariant with respect to the obvious $S_n$ action on the domain and target.

\item \textbf{Grading}: $ \deg I_{g,n,\beta} =  - 2 \int_{\beta} c_1(X) + (2-2g) \dim X$. More precisely, this means that
we set $|\gamma| = i$ for $\gamma \in H^i(X,\Q)$ and we require that:
\[	\left| I_{g,n,\beta}^X(\gamma_1, \dots, \gamma_m)\right| = \sum_{j=1}^m |\gamma_j| - 2 \int_{\beta} c_1(X) + (2g-2) \dim X. 	\]
Some comments on the grading axiom:
\begin{itemize}
\item Following the convention in \cite{km_gw}, we use the real, not complex, dimension.
\item Informally we think of $I_{g,n,\beta}^X(\gamma_1, \dots, \gamma_m)$ as obtained by pushing forward via the natural map:
\[	\mathcal{M}_{g,n}(X,\beta) \to \mathcal{M}_{g,n} .	\]
As a result, its degree is an expectation for $\dim \mathcal{M}_{g,n}- \dim \mathcal{M}_{g,n}(X,\beta)$. 
We know that $\dim \mathcal{M}_{g,n} = 2(3g-3+n)$. By deformation theory we also compute	$\vdim \mathcal{M}_{g,n}(X,\beta)$,
called the \textbf{virtual dimension}, the expected dimension whenever first-order deformations are unobstructed.

The tangent space to $\mathcal{M}_{g,n}(X, \beta)$ at a point $(C,p_1, \dots, p_n, \mu)$ is:
\[	H^1(C,T_C(-p_1-\dots-p_n)) \oplus H^0(C,\mu^*T_X).	\]
By Serre duality this is:
\[	H^0(C,\Omega^{\otimes 2}_C(p_1+\dots+p_n))^{\vee} \oplus H^0(C,\mu^*T_X).	\]
Approximating the dimensions with the Euler characteristic, we get via Riemann-Roch:
\begin{equation}
\label{eq:vdim}
	\vdim \mathcal{M}_{g,n}(X, \beta) = 2(\dim X - 3)(1-g) + 2\int_{\beta} c_1(T_X) + 2n .
\end{equation}
Substracting these we get what the grading axiom requires:
\[	\dim \mathcal{M}_{g,n}- \dim \mathcal{M}_{g,n}(X,\beta) = 2 \int_{\beta} c_1(X) - (2-2g) \dim X.	\]
\item Assume that $I_{g,n,\beta}^X(\gamma_1, \dots, \gamma_m)$ is of \textbf{codimension zero}, i.e. that:
\begin{equation}
\label{eq:codimension_zero}
\sum_{j=1}^n |\gamma_j| = 2 \int_{\beta} c_1(X) - (2-2g) \dim X.
\end{equation}
Then $\left| I_{g,n,\beta}^X(\gamma_1, \dots, \gamma_m)\right| = \dim \overline{\mathcal{M}}_{g,n}$.
We can integrate this against the fundamental class of $\overline{\mathcal{M}}_{g,n}$, which is a proper smooth
Deligne-Mumford stack. \todo{reference?} We obtain a finite number, which we take as the result of the curve count.
\end{itemize}

\item \textbf{Fundamental class.} We introduce some more terminology. Call a class \textbf{basic} if it has the smallest
$n$ which makes sense, namely:
\[	I_{0,3,\beta}^X(\gamma_1, \gamma_2, \gamma_3) \hspace{2cm} I_{1,1,\beta}^X(\gamma_1) \hspace{2cm} I_{g,0,\beta}^X
\text{ for } g\geq 2.	\]
Let $\pi: \overline{\mathcal{M}}_{g,n} \to \overline{\mathcal{M}}_{g,n-1}$ be the projection that forgets the last marked point.
Let $e_X^0 \in H^0(X,\Q)$ be the identity of the cohomology ring.
Unless the class on the LHS is basic, we require that:
\[	I^X_{g,n,\beta}(\gamma_1, \dots, \gamma_{n-1}, e_X^0) = \pi^* I^X_{g,n-1,\beta}(\gamma_1, \dots, \gamma_{n-1}).	\]
In addition, we set:
\[	I^X_{0,3,\beta}(\gamma_1, \gamma_2, e_X^0) = \left\{ \begin{array} {ll} \int_X \gamma_1 \wedge \gamma_2, & \text{if } \beta = 0, \\
0, & \text{if } \beta \neq 0. \end{array} \right.	\]

\item \textbf{Divisor.} In the case $|\gamma_n| = 2$, i.e. $\gamma_n$ is the Poincar\'e dual class of a divisor,
and if the LHS is a non-basic class, we require:
\[	\pi_{n*}I_{g,n,\beta}^X(\gamma_1, \dots, \gamma_n) = \int_{\beta}\gamma_n I_{g,n-1,\beta}^X(\gamma_1, \dots, \gamma_{n-1}).	\]

\item \textbf{Splitting.} This axiom and the next are very important: they postulate a manageable structure of the boundary
of the compactification $\overline{\mathcal{M}}_{g,n}(X,\beta)$, compatible with that of the boundary of $\overline{\mathcal{M}}_{g,n}$.
One way to get boundary maps is to let the curves have 2 irreducible components, with genera $g_1, g_2$ and marked points
$n_1 +1, n_2+1$ such that $g = g_1 + g_2$, $n = n_1+n_2$. The extra marked point on each irreducible component is where we glue them;
they become one singular point in the resulting reducible curve. For $S$ some partition of the $n$ marked points into 2 sets of
cardinality $n_1$ and $n_2$, we let $\phi_S : \overline{\mathcal{M}}_{g_1,n_1+1} \times \overline{\mathcal{M}}_{g_2,n_2+1} \to
\overline{\mathcal{M}}_{g,n}$ be the gluing map. Choose a basis $\{\Delta_a\}$ of $H^*(X,\Q)$ and define $g_{ab} = \int_V
\Delta_a \wedge \Delta_b$; let $(g^{ab}) = (g_{ab})^{-1}$ denote the entries of the inverse matrix. Then:
\[	\phi_S^* I^X_{g,n,\beta}(\gamma_1, \dots, \gamma_n) = (-1)^S \sum_{\beta_1 + \beta_2 = \beta} \sum_{a,b}
I^X_{g_1,n_1+1,\beta_1}(\otimes_{j\in S_1} \gamma_j \otimes \Delta_a) g^{ab} \otimes I^X_{g_2,n_2+1,\beta_2}(\Delta_b
\otimes \otimes_{j\in S_2} \gamma_j). 	\]
Roughly speaking, we need to introduce $\sum_{a,b}(\Delta_a \otimes \Delta_b)$ to account for the position of the extra marked points.
Integrating over these produces a factor $g_{a,b}$ that wasn't there on the LHS, so we need to multiply by $g^{ab}$ to
compensate for it.

\item \textbf{Genus reduction.} Let $\psi : \overline{\mathcal{M}}_{g-1,n+2}\to \overline{\mathcal{M}}_{g,n}$ be the map
which glues together the last 2 marked points. Then:
\[	\psi^*	I^X_{g,n,\beta}(\gamma_1, \dots, \gamma_n) = \sum_{a,b} 
I^X_{g-1,n+2,\beta}(\gamma_1, \dots, \gamma_n, \Delta_a,\Delta_b) g^{ab}.\]

The splitting and genus reduction axioms motivate the choice of stable maps compactification, see \ref{rem:boundary}.

\item \textbf{Motivic axiom.} The maps $I^X_{g,n,\beta}$ are induced by correspondences in the Chow rings:
\[	C^X_{g,n,\beta} \in C^*(X^n \times \overline{\mathcal{M}}_{g,n}).	\]
Namely, consider the two projection maps:
\[
\begin{tikzcd}
\; & X^n \times \overline{\mathcal{M}}_{g,n}\arrow{dl}{p}\arrow{dr}{q} & \\ X^n & & \overline{\mathcal{M}}_{g,n}.
\end{tikzcd}
\]
We require that:
\[	I^X_{g,n,\beta}(\gamma_1, \dots, \gamma_n) = q_*\big(C^X_{g,n,\beta} \wedge p^* (\gamma_1\otimes \dots \otimes \gamma_n)\big).\]
This axiom is motivated as follows in \cite{km_gw}, 2.3.8.
Suppose we construct a good compactification $\overline{\mathcal{M}}_{0,n}(X,\beta)$,
together with a virtual fundamental class $[\overline{\mathcal{M}}_{0,n}(X,\beta)]$. Consider then the map:
\begin{align*}
\alpha :\overline{\mathcal{M}}_{0,n}(X,\beta) &\to X^n \times \overline{\mathcal{M}}_{0,n} \\
(C, x_1, \dots, x_n, f) &\mapsto \big(f(x_1), \dots, f(x_n), (\bar C, x_1, \dots, x_n)\big).
\end{align*}
We would like $\bar C$ to be $C$, but we may need to contract certain components to get a stable curve from a stable map.
Compare definitions \ref{def:stable_curve} and \ref{def:stable_map}. 
Ignoring this for now, we set $C^X_(g,n,\beta) = \alpha_*\big( [\overline{\mathcal{M}}_{0,n}(X,\beta)]\big)$. This
means, roughly speaking, we're integrating over $\overline{\mathcal{M}}_{0,n}(X,\beta)$, like the naive definition
\ref{eq:gw_naive_def} suggests.
\end{enumerate}
\end{defin}

We are mostly interested in codimension zero invariants, which informally are those where we imposed enough constraints
to get a finite number of curves. For example, if we want to count degree $d$ rational curves in $\bbP^2$,
the relevant codimension zero condition says:
\[	\sum_{i=1}^n |\gamma_i| = 2 \int_{d[H]} c_1(\bbP^2) - 2 \dim \bbP^2 = 6d - 4.	\]
For example, we could ask that the curves pass through $n$ given points in $\bbP^2$, then $|\gamma_i| = 4$, so we obtain
$4n = 6d-4$. If the computation were done right, this would be $12d-4$, so that we get $n=3d-1$. So the relevant thing
to count are degree $d$ rational curves passing through $3d-1$ points.\todo{fix this}








\section{Stable Map Compactification}
\label{sect:stable_map}
To give a naive compactification of
$\overline{\mathcal{M}}_{0,0}(\bbP^r,d)$, we could just look at the space $W(r,d)$ of $r+1$-tuples of degree $d$ polynomials
in 2 variables, up to scaling, and take the subset of tuples which don't vanish simultaneously. We get a subset of a projective space:
\[	W(r,d) \subset \bbP\left( \bigoplus_{i=0}^r H^0(\bbP^1,\mathcal{O}(d)) \right).	\]
We need to quotient by $\Aut(\bbP^1)$ to identify maps that differ by a reparametrization; ignoring this for the moment,
one hopes to take the closure of $W(r,d)$ in $\bbP\left( \bigoplus_{i=0}^r H^0(\bbP^1,\mathcal{O}(d)) \right)$ to obtain
a compactification. However, for $g\neq 0$ and $X \neq \bbP^r$, this doesn't work and we need a less ad-hoc approach.

The choice of compactification matters; different choice leads to different numbers. That's
because the numbers now count things in the boundary as well. 

\begin{eg}
In the stable maps compactification that we introduce shortly, which produces Gromov-Witten invariants,
we keep the domain curves well-behaved:
they acquire nodal singularities, but no non-reduced structure. However, the maps themselves can be highly non-injective.
A different choice is the Donaldson-Thomas compactification via Hilbert schemes: here we work with ideal sheaves, which
always represent embeddings, however the domain curve can now be non-reduced or have singularities worse than nodal.
Section $3\frac{1}{2}$ of \cite{PT_counting_curves} illustrates the differences with the following example.
We work locally and consider the family of conics:
\[	C_t = \{ x^2 + ty = 0 \} \subset \C^2,	\]
which becomes singular as $t\to 0$. In the DT compactification, we take the limit in the defining
equation, and get $x^2 = 0$, which is a thickened $y$-axis. In the stable map compactification, we parametrize
the conics:
\[	C_t \longleftrightarrow \xi \mapsto (-\sqrt{t} \xi, \xi^2) . 	\]
This is a parametrization modulo automorphisms of the curve, namely $\xi \leftrightarrow -\xi$. Now as $t\to 0$,
the limiting map is $\xi \mapsto (0, \xi^2)$, which is a double cover of the $y$-axis.
You can't see from this example, but the different choices of compactification actually give different answers
for the counting problem.
\end{eg}

With that in mind, let's finally define stable maps. For reference and comparison we include the definition of stable
curves:

\begin{defin}
\label{def:stable_curve}
\todo{write this up}
\end{defin}

Think about graphs of curves, such that each ``twig'' has no infinitesimal automorphisms. This means that twigs of
genus $g$ must have at least $3 - 2g$ special points, which means either marked points or singular ones.

\todo{figure out an easy way to include the pictures of graphs}

\begin{defin}[2.4.1 in \cite{km_gw}]
A \textbf{stable map} to $X$ is a structure $\big(C, x_1, \dots, x_n, f\big)$ where:
\begin{itemize}
\item $\big(C, x_1, \dots, x_n\big)$ is a connected reduced curve with $n$ pairwise distinct marked non-singular points,
and at worst additional singular double points.
\item $f:C \to X$ is a map with no non-trivial infinitesimal automorphisms. This means that every irreducible component of
$C$ of genus $g$ which is contracted to a point (of degree 0) must have at least $3-2g$ special points.
\end{itemize}
\end{defin}

\begin{rem}
Note that, in the definition of stable maps $\big(C, x_1, \dots, x_n, f\big)$, the underlying curve
$\big(C, x_1, \dots, x_n\big)$ need not be stable. Therefore the forgetful map $\overline{\mathcal{M}}_{g,n}(X,\beta)
\to \overline{\mathcal{M}}_{g,n}$ must contract components of $\big(C, x_1, \dots, x_n\big)$ which have infinitesimal
automorphisms.
\end{rem}

In his talk notes, Mauro provides the following construction of the moduli stacks of stable maps
$\overline{\mathcal{M}}_{g,n}(X,\beta)$. Start from $\overline{\mathcal{M}}_{g,n}$, which are fine moduli spaces of
curves, and therefore admit a universal family $\mathcal{C}_{g,n}$. Then define:
\begin{equation}
\label{underived_moduli_stack}
	\overline{\mathcal{M}}_{g,n}(X) = 
\Map_{\St / \overline{\mathcal{M}}_{g,n}}(\mathcal{C}_{g,n},  X \times \overline{\mathcal{M}}_{g,n}).
\end{equation}
To obtain $\overline{\mathcal{M}}_{g,n}(X, \beta)$, we must take maps $\alpha$ with the additional constraint that
$\alpha_*[\mathcal{C}_{g,n}] = [\beta] \times [\overline{\mathcal{M}}_{g,n}]$. \todo{figure out the actual condition}

\begin{rem}
When we introduce a derived structure on $\overline{\mathcal{M}}_{g,n}(X, \beta)$, we follow the same approach, but take
maps in $\dSt$ instead of $\St$.
\end{rem}

\begin{thm}[3.14 in \cite{bm_stacks}]
$\overline{\mathcal{M}}_{g,n}(X,\beta)$ are proper, algebraic Deligne-Mumford stacks.
\end{thm}
\todo{we should say something about the proof, but the paper is very techinical}

\begin{defin}
A smooth projective scheme $X$ is \textbf{convex} if for every $f:\bbP^1 \to X$, $H^1(\bbP^1, f^*T_X) = 0$.
\footnote{We may want to restrict $f$ to be stable, but we haven't defined this yet, so we'll ignore it for now.}
\end{defin}

For example, $\bbP^r$ is convex for every $r$. This notion is relevant due to:

\begin{prop}
If $X$ is convex, then $\overline{\mathcal{M}}_{0,n}(X,\beta)$ is a smooth, proper Deligne-Mumford stack.
\footnote{Here we are using the compactification by stable maps; this is defined in \ref{def:stable_map}.}
\todo{what's a reference for this? \cite{km_gw} say it's an expectation in 2.4.2, but Mauro's notes imply that it's proved.}
\end{prop}

Thus, in the situation of convex $X$, $[\mathcal{M}_{g,n}(X,\beta)]$ can be taken to be the fundamental class. Otherwise we will
need to build a virtual fundamental class.

One of the most important properties of $\overline{\mathcal{M}}_{g,n}(X,\beta)$ is the recursive structure of the boundary;
this leads to a proof of the splitting and genus lowering axioms. We first do the case $g=0$, which is formula
2.7.3.1 in \cite{invitation}.

Choose a partition $S_1 \cup S_2$ of the marked points, and classes $\beta_1, \beta_2$ such that $\beta_1 + \beta_2 = \beta$.
Let $D(S_1,S_2;\beta_1, \beta_2) \subset \overline{\mathcal{M}}_{0,n}(X,\beta)$ be the boundary divisor consisting of
curves of genus 0 with 2 irreducible components, with marked points $S_i$ and mapping to $\beta_i$ respectively.

\begin{lem}
\label{lem:recursive_structure}
The boundary divisors are given by:
\[	D(S_1,S_2;\beta_1, \beta_2) =  \mathcal{M}_{0,S_1\cup\{x\}}(X,\beta_1) \otimes_{X}  \mathcal{M}_{0,S_2\cup\{x\}}(X,\beta_2).	\]
Inducting on this formula, we obtain the structure of the lower dimensional strata as well; we don't write this down though.
\end{lem}

\begin{rem}
The straight up generalization for curves of any genus would be:
\[\coprod_{g_1 + g_2 = g} \mathcal{M}_{g_1,S_1\cup\{x\}}(X,\beta_1) \otimes_{X}  \mathcal{M}_{g_2,S_2\cup\{x\}}(X,\beta_2).	\]
where $g_1 + g_2 = g$, and $[\beta_1] + [\beta_2] = [\beta]$. I haven't computed the dimensions, though, to see for what values
of $g_1, g_2$ we get codimension 1 strata. Moreover, we have extra contributions from cycles of lower genus curves.
\todo{finish this}
\end{rem}

To illustrate the need for virtual fundamental classes, we look at an example where $\overline{\mathcal{M}}_{g,n}(X,\beta)$
contains strata of higher dimension than $\vdim$; in this case, taking the straight up fundamental class would break
the grading dimension of Kontsevich-Manin.
The following example is worked out in full detail Section 4 of \cite{nabijou}.

\begin{eg}
We compute the dimension and virtual dimension of $\overline{ \mathcal{M}_{0,0}}(X, 3\pi^*H)$,
where $X = \Bl_p \bbP^2$, $\pi: X \to \bbP^2$ is the blowup map, and $[H] \in H_2(\bbP^2,\Z)$ is the hyperplane class.
Using equation \ref{eq:vdim}, we have:
\[	\vdim \overline{ \mathcal{M}_{0,0}}(X, 3\pi^*H) = \int_{3\pi^*H} c_1(T_X) - 1 = 8.	\]
One could look, for example, at rational curves of degree 3 in $\bbP^2$ which avoid $p$, i.e.
$\overline{ \mathcal{M}_{0,0}}(\bbP^2, 3H)$. This is a stratum in $\overline{ \mathcal{M}_{0,0}}(X, 3\pi^*H)$ of the correct dimension 8
(the space of cubics in $\bbP^2$ is 9-dimensional, and we substract 1 for reparametrizations of the domain $\bbP^1$.)
More strata are given by rational cubics in $\bbP^2$ which pass through $p$ with multiplicity $k$, and therefore 
lift to a curve in $X$ of class $3\pi^*H - rE$, where $E \subset X$ is the exceptional divisor. To obtain
a stable map in the appropriate class $3\pi^*H$, we add $r$ components isomorphic to $\bbP^1$ which map to $E$.
The dimension of this stratum is:
\[	\dim \overline{ \mathcal{M}_{0,0}}(X,3\pi^*H - rE) + \dim \overline{ \mathcal{M}_{0,0}}(\bbP^1,r) = (8-r) + (2r-2) = 6+r.	\]
The farthest we can go while keeping $[\beta]$ effective (that is, $\beta . K_X \leq 0$) is $r=3$. This gives a stratum
(supposedly a boundary stratum!) of dimension $9>8$.
\end{eg}


\chapter{Obstruction Theories and Virtual Fundamental Classes}
\label{ch2:obs}
Talk by Benedict Morrissey.

Given a stack $X$, our objective is to construct a virtual fundamental class $[X]$ for it, motivated by the discussion in
\ref{ch:gw}. We will see two ways in which a derived enhancement of $X$ helps achieve this.
We would like $[X]$ to come from an algebraic cycle, i.e. an element of the Chow group.
In this case, given $f: X\to Y$ proper,
there is a well-defined pushforward  $f_*[X] \to [Y]$, which induces a pushforward
$f_*[X]^H \to [Y]^H$ on the images $[X]^H, [Y]^H$ of the VFCs in any Weil cohomology theory $H$.

However, derived Chow groups have yet to be defined, so we start with a piecemeal approach, by
defining a class in G-theory only.


\section{Construction from G-Theory}
\begin{defin}
The \textbf{G-theory} $G_0(X)$ of a classical stack $X$ is defined as the K-theory of the category of coherent sheaves on $X$:
\footnote{One can also define higher G-theory $G_i$, but we won't need this.}
\[	G_0(X) := K_0(\Coh(X)) .	\]
If $\tilde X$ is a derived stack, we set $G_0(\tilde X) = K_0((\Coh \tilde X)^{\heartsuit})$.
\end{defin}

\begin{defin}
A \textbf{derived enhancement} of a stack $X$ is a derived stack $\tilde X$ such that $t_0(\tilde X) = X$.
\end{defin}

There is a natural inclusion, left-adjoint to the truncation, which we denote $j: X \to \tilde X$.
Using the fact that pushforwards of coherent sheaves by proper maps are coherent,
\todo{check if there are other conditions, and whether $\tilde X$ derived changes anything}
we obtain $j_* : G_0(X) \to G_0(\tilde X)$.

\begin{prop}
If $X$ is quasi-compact, then $j_* : G_0(X) \to G_0(\tilde X)$ is a bijection. In this case we define:
\[	 [X]^{\vir} := j_*^{-1}[\mathcal{O}_{\tilde X}].	\]
\end{prop}

\begin{proof}
The identification actually works on the full spectrum of $G$-theory. We're using the theorem of 
the heart for $K$-theory. The identification is done as follows.
\begin{enumerate}
\item Theorem of the heart for $K$-theory. (Due to Quillen, and Batwick in the DG category setting.) If you have $\cC$ a stable
$\infty$-category, idempotent complete, with $t$-structure, and every object in the heart is bounded, then $K(\cC) = K(\cC^{\heartsuit})$.
\item $\Coh(\tilde X)^{\heartsuit} \simeq \Coh(X)^{\heartsuit}$, which follows from descent and the analogous result for
derived affines, which was proved during the first semester, in the talk on Stable $\infty$-categories.
\footnote{Throughout when we write $\Coh$ we mean $\Coh^b$.}
\end{enumerate}
\end{proof}

\begin{thm}
\label{thm:ox_bounded}
For $\tilde X$ quasi-compact,\footnote{Note that we don't need to assume that $X$ is quasi-compact.}
$\mathcal{O}_{\tilde X}$ is bounded. It follows that the following sum is finite:
\[	j_*^{-1}[\mathcal{O}_{\tilde X}] = \sum_{i=0}^{\infty} (-1)^i [H^i(\mathcal{O}_{\tilde X})],	\]
so it defines an element in $G_0(X)$.
\end{thm}

\begin{rem}
Note that the cohomology in Theorem \ref{thm:ox_bounded} is just the cohomology of the complex, NOT sheaf cohomology. 
Moreover it wouldn't make sense to
use $K$ theory instead of $G$ theory, because even if $\mathcal{O}_{\tilde X}$ is perfect, the kernels and cokernels 
of the various differentials don't need to be.
\end{rem}

\begin{proof}
We start with a vague understanding of why the theorem may be true. The counterexample is $\Spec (\Sym k[2])$, 
where the cotangent complex is unbounded. But if it's in amplitude [-1,0], it's like an exterior algebra and it works.

We work locally, $\Spec B \to \Spec A$, $\supp \bbL_{B/A} \subset[-1,0]$. $B$ is a derived lci over $A$, so the cotangent complex
is perfect, so there's a theorem that says that $B$ is homotopically of finite type over $A$. These can be constructed by attaching
finitely many cells:
\[	A = B_0 \to B_1 \to B_2 \to \dots \to B_k = B.	\]
Attaching the $i+1^{th}$ cells of $B$ looks like:
\[
\begin{tikzcd}
B_i \arrow{r} & B_{i+1} \\
\bigotimes A[\p \Delta^{i+1}]\arrow{u}\arrow{r} & A[\Delta^{i+1}]\arrow{u} .
\end{tikzcd}
\]
$B_1$ is obtained by attaching cells in degree 1. The map $B_1 \to B$ is an isomorphism on $\pi_0$. \todo{review this proof}

There's another proof by Lowrey and Sch\"urg, in \cite{derived_GRR}, which is more intuitive. Having a quasi-smooth structure 
allows one to describe
the derived space locally as the derived zero locus of a section of a vector bundle. Then the derived intersection can be computed
as a Koszul resolution, so $\mathcal{O}_{\tilde X}$ behaves like an exterior algebra, which means it's bounded. Here the quasi-compactness
is used in order to reduce to finitely many local charts, which means that the bound on $\mathcal{O}_{\tilde X}$ is uniform.
\end{proof}

\begin{rem}
The idea behind the proof of Lowrey and Sch\"urg is also that of
\textbf{Kuranishi structures}. These are essentially a machinery for working with derived stacks which remembers the 
local description as zero locus, in order to avoid using the machinery of derived geometry.
In DAG quasi-smoothness is an intrinsic property that one can check at the level of the cotangent complex, so that one doesn't
need to remember the local descriptions, which are cumbersome and don't glue well.
\end{rem}

The VFC in ordinary cohomology is defined by Konstevich to be:
\begin{equation}
\label{eq:G_definition_VFC}
	[X]^{\vir} = \Ch([X]_G^{\vir}) \Td(j^* \bbT_{\tilde X}).
\end{equation}

\begin{conj}
Definition \ref{eq:G_definition_VFC} agrees with the construction of Behrend-Fantechi, \ref{}. \todo{ref this}
\end{conj}

The conjecture has been verified for schemes (not stacks) by Ciocan-Fontanine and Kapranov, in
\cite{ciocan_fontanine_kapranov}, using the additional 
assumption (which is made in Behrend Fantechi anyway) that the cotangent complex admits a global resolution by vector bundles.




\section{Obstruction Theories}
We introduce the alternative construction of VFCs, following \cite{Behrend_Intrinsic_normal_cone_1997}. In the words of
Mauro, we want to use this as a black box which achieves:
\[	\text{Obstruction Theory} \Longrightarrow \text{VFC}.	\]

Throughout we will use $X,Y$ for underived stacks, and $\tilde X, \tilde Y$ for their derived enhancements.

\begin{defin}
An \textbf{obstruction theory} for $X$ is a morphism $\phi : E \to \bbL_X$ in $D(\Coh(X))$, such that:
\begin{align*}
&h^0(\phi) : H^0(E) \to H^0(\bbL_X) \text{ is an isomorphism,} \\
&h^{-1}(\phi) : H^{-1}(E) \to H^{-1}(\bbL_X) \text{ is surjective,} \\
&H^i(E) = 0 \text{ for } i \neq -1,0.
\end{align*} 
\end{defin}

\begin{defin}
A \textbf{perfect obstruction theory} is an obstruction theory such that $E$ is in perfect amplitude [-1,0], which means that
locally $E$ is isomorphic to a 2-term complex of vector bundles $[E^{-1} \to E^0]$.
\end{defin}

The link to derived geometry is as follows. 
\begin{prop}
Given a derived enhancement $j:X \to \tilde X$, with $\tilde X$ a quasi-smooth DM stack, there is a perfect obstruction theory:
\[	j^* \bbL_{\tilde X} \to \bbL_X.	\]
\end{prop}
\begin{proof}
By descent we reduce this to the case of affines, and we need only consider $A \to t_0(A)$. We have the fiber sequence:
\[	j\bbL_A \to \bbL_{\pi_0(A)} \to \bbL_{\pi_0(A)/A}.	\]
Due to the connectivity estimates, which we introduced last semester in the talk about the cotangent complex,
$\bbL_{\pi_0(A)/A}$ is 2-connective. Indeed, the fiber of $A \to \pi_0(A)$ is 1-connective,
so the cofiber, which is the shift of the fiber by 1, is 2-connective.\footnote{In order to relate the fiber and cofiber of the
morphism $A \to \pi_0(A)$, we use the fact that we are working in the stable $\infty$-category $\pi_0(A)\Mod$, and not 
in $\pi_0(A)-\Alg$.}
\end{proof}

Throughout the rest of the talk, the goal is to describe how to construct a VFC, starting with an obstruction theory. In the
smooth case, if you take the $G$-construction we did earlier, you'd get the same answer. 

We also want to describe functoriality properties for the VFC, and to that effect we introduce compatibility data
between obstruction theories. 
During the check that Kontsevich-Manin axioms are satisfied, we will need to use functoriality a lot.
The following is Definition 5.8
in \cite{Behrend_Intrinsic_normal_cone_1997}.


\begin{defin}
\label{defin:compatibility_datum}
Let $u:X' \to X$ be a morphism.
A \textbf{compatibility datum between obstruction theories} $E$ for $X$ and $ F$ for $X'$ is a choice of embeddings
$f: X \to Y$, $g: X' \to Y'$ into smooth stacks, such that the following diagrams commute:
\[
\begin{tikzcd}
X'\arrow{r}{u} \arrow[swap]{d}{g} & X\arrow{d}{f} \\ Y'\arrow{r}{v} & Y,
\end{tikzcd}
\]
\[
\begin{tikzcd}
u^* E \arrow{d}\arrow{r}{\phi} & F \arrow{r}{\psi}\arrow{d} & g^* \bbL_{Y'/Y}\arrow{d} \\
u^* \bbL_X \arrow{r} & \bbL_{X'} \arrow{r} & \bbL_{X'/X} .
\end{tikzcd}
\]
Moreover, we require the two rows to be fibration sequences in $D(\Coh(X'))$.
\end{defin}

Behrend and Fantechi prove:

\begin{prop}
Given compatibility data between obstruction theories $E$ for $X$ and $F$ for $X'$, it follows that 
$u^*[X]^{\vir, E} = [X']^{\vir, F}$.
\end{prop}

For us obstruction theories come from derived enhancements $\tilde X, \tilde X'$.
In this case, we obtain the functoriality of VFCs in a cleaner way, by giving a morphism between derived
enhancements $w: \tilde X' \to \tilde X$, fitting in the commutative diagram:
\[
\begin{tikzcd}
\; & \tilde X'\arrow{rr}{w}\arrow{ddl}{\tilde g} & & \tilde X\arrow{ddl}{\tilde f} \\
X'\arrow{rr}{u}\arrow[swap]{d}{g}\arrow{ur}{j} & & X\arrow[swap]{d}{f}\arrow{ur}{i} & \\
Y'\arrow{rr}{v} & & Y &
\end{tikzcd}
\]
Moreover, we require that the top and back square are homotopy pullbacks.

\begin{rem}
I was hoping that the top square would be enough. Unfortunately, we still need the choice of ambient 
spaces $Y, Y'$, as well as morphisms $\tilde g$, $\tilde f$, and the data for the homotopy commutativity of the back square.
However Mauro says:
\begin{enumerate}
\item In the applications we care about (stable maps), the entire back square will be there naturally.
\item Working with the derived compatibility data is still easier, in practice, than with the fibration sequences in
Definition \ref{defin:compatibility_datum}.
\end{enumerate}
\end{rem}

Let us see why the derived compatibility data implies the diagram between fibration sequences in 
Definition \ref{defin:compatibility_datum}.
The assumption is that $E = i^* \bbL_{\tilde X}$ and $F = j^*\bbL_{\tilde X'}$. We first need the map:
\[		\phi	: u^* i^* \bbL_{\tilde X} \to j^* \bbL_{\tilde X'} .	\] 
This is just given by $w$. More precisely, the commutativity of the top square gives the map on the left 
in the following diagram, and we define the top map as the composition:
\[
\begin{tikzcd}
u^*i^*\bbL_{\tilde X}\arrow{r}\arrow{d} & j^* \bbL_{\tilde X'} \\
j^*w^*\bbL_{\tilde X}\arrow{ur} &
\end{tikzcd}
\]
To get $\psi$, which must be such that the row is a fiber sequence, we make use of the maps $\tilde g, \tilde f$.
Question: how to identify $j^*\bbL_{\tilde X'/\tilde X} $ with $g^* \bbL_{Y'/Y}$? Since the back square is a pullback, we have
a canonical identification $\bbL_{\tilde X'/\tilde X}  \simeq \tilde g^* \bbL_{Y'/Y}$, and this gives:
\begin{equation}
\label{eq:get_psi}
	j^*\bbL_{\tilde X'/\tilde X} \simeq j^*	\tilde g^* \bbL_{Y'/Y} \simeq  g^* \bbL_{Y'/Y} .
\end{equation}
We take this composition to be $\psi$. Note that this chain of equivalences depends very much on the extra data of
the homotopy commutative back square.

\begin{rem}
Throughout, we want $Y', Y$ to be smooth, and $\tilde f, \tilde g$ to be quasi-smooth. Therefore, if $X, X'$ are not
smooth, we cannot expect $f, g$ to be just identity maps. In fact, the point that Behrend-Fantechi make is that $Y$ and $Y'$ 
should only be expected to exist locally.
\end{rem}

\begin{defin}
A \textbf{local embedding} $(U,M)$ of $X$ is the data of $U\to X$ an \'etale map and $U\to M$ a local immersion, 
where $M$ smooth affine $k$-scheme of finite type.
Given a local embedding, the associated \textbf{normal bundle} is $N_{U|M} := \Spec_M(\Sym (I/I^2))$. Inside this 
we have the \textbf{normal cone} $C_{U/M} = \Spec_M(\oplus_{n\geq 0} I^n/I^{n+1})$.
The ring homomorphism $\Sym (I/I^2) \to \oplus_{n\geq 0} I^n/I^{n+1}$ is surjective, so the map
$C_{U/M}	\to N_{U|M}$ is a closed embedding.
\end{defin}

Given an obstruction theory $E \to \bbL_X$, if we can write $E = [F_{-1} \to F_0]$ globally, and define $F^1 := F_{-1}^*$, 
then we get the pullback diagram:
\[
\begin{tikzcd}
C(F^{\bullet})\arrow{r}\arrow{d} & F_1\arrow{d} \\ C_X \arrow{r} & N_X .
\end{tikzcd}
\]

\begin{defin}
Let $0: X \to F_1$ be the zero section.
The \textbf{virtual fundamental class} of $X$ induced by the obstruction theory $E$ is the intersection of
$[C(F^{\bullet})] \in \Chow(F_1)$ with the zero section, i.e. $[X]^{\vir, E} := o^{!}[C(F^{\bullet})]$.
\todo{why is this shriek and not star?}
\end{defin}


\chapter{Geometricity of Mapping Stacks}
\label{ch3:geom}

This chapter is somewhat tangential to our concrete goals for the semester. However we thought that $\R\Map_{g,n}(X,\beta)$
provides a good opportunity to understand Artin-Lurie representability and how it can be used to prove that certain
mapping stacks are geometric.


\section{Using Artin-Lurie representability for Mapping Stacks}
The representability theorem says:

\begin{thm}[Artin-Lurie representability, Theorem 3.2.1 in \cite{DAG-XIV}]
\cite{thm:lurie_representability}
Let $X: \cdga_k^{\leq 0}\to \cS$ be a functor, and suppose we are given a natural transformation $f:X \to \Spec R$.
 Then $X$ is representable by a derived Deligne-Mumford $n$-stack locally almost of finite
presentation over $R$ if and only if the following are satisfied:
\begin{enumerate}
\item For every discrete commutative ring $A$, the space $X(A)$ is n-truncated.
\item $X$ is a sheaf for the \'etale topology.
\item $X$ is nilcomplete, infinitesimally cohesive and integrable. These mean:
\begin{itemize}
\item $X$ commutes with Postnikov towers;
\item $X$ commutes with pullback squares $B \times_A C$, under the assumption that $\pi_0(B) \to \pi_0(A)$ and $\pi_0(C) \to \pi_0(A)$
are surjective with nilpotent kernel;
\item for $A$ a complete local ring, $X(A) \simeq \varprojlim X(A/\fr m^n)$; loosely speaking, every formal $A$-point of $X$
integrates to give a point of $X$.
\end{itemize}
\item $f:X \to R$ admits a connective relative cotangent complex $\bbL_{X/R}$.
\item $f:X \to R$ is locally almost of finite presentation.
\end{enumerate}
\end{thm}

\begin{rem}
(2) is obvious, (5) ensures that the DM stack is locally almost of finite presentation. (3) and (4) ensure that $X$ has good
local behavior, in particular a good deformation theory. The existence of the relative cotangent complex is conceptually
the most important condition, and the one we will put the most effort into verifying. Finally, (1) encodes the geometricity
of the representing DM stack. $n$-stacks are defined to be those for which condition (1) holds; it is then true that: \todo{Mauro
said so; maybe also find a reference}
\begin{itemize}
\item $n$-geometric implies $n+1$-stack;
\item $m$-geometric for some $m$ and $n$-stack implies that, at worst, $m=n+1$.
\end{itemize}
\todo{this is not completely satisfactory: does Lurie representability guarantee that we get $m$-geometric for some $m$?}
\end{rem}

As an application of this, we want to prove the geometricity of mapping stacks.

\begin{thm}
\label{thm:mapping_stack_geometric}
Let $g: X \to Z$ be a morphism of derived stacks which is geometric and of finite type. Let $f:Y\to Z$ be a morphism of stacks
which is representable by proper flat schemes. \footnote{We could replace the condition on $f$ with something slightly more
general, such as representable by quasi-compact quasi-separated algebraic spaces of finite tor amplitude.} Then 
the mapping stack $\Map_{/Z}(Y,X)$ is geometric
over $Z$, i.e. the morphism $\Map_{/Z}(Y,X) \to Z$ is geometric.
\end{thm}
\begin{proof}
Recall that the mapping stack is defined by the functor of points:
\[	 \Map_{/Z}(Y,X) (T) = \Map_{\dSt}(T\times_Z Y, X). 	\]
We first reduce to $Z$ affine, so that Theorem \ref{thm:lurie_representability} applies.
Then, since $g$ is assumed geometric, it satisfies conditions (3) of the Theorem \ref{thm:lurie_representability}. 
It follows by elementary manipulation
of the diagrams that $\Map_{/Z}(Y,X) \to Z$ also has these properties; see Proposition 3.3.6 in \cite{DAG-XIV}.
The most important issue is the existence of a relative cotangent complex for $\Map_{/Z}(Y,X) \to Z$. 
Recalling the definition, we need to construct cotangent complexes $\bbL_{\Map_{/Z}(Y,X),x}$ at each point 
$x: \Spec A \to \Map_{/Z}(Y,X)$, and then make sure that they glue; this will be diagram \ref{} below.

$\bbL_{\Map_{/Z}(Y,X),x}$ is supposed to be an object that represents the functor of derivations over $\Map_{/Z}(Y,X)$:
\[	\Map_{A\Mod}(\bbL_{\Map_{/Z}(Y,X),x}, M) = \Der_{\Map_{/Z}(Y,X)}(A,M) .	\]
The latter is defined as the homotopy pullback:
\[
\begin{tikzcd}
\Der_{\Map_{/Z}(Y,X)}(A,M)\arrow{r}\arrow{d} & \Map(\Spec(A\oplus M) \times_Z Y, X)\arrow{d} \\
\Spec k\arrow{r} & \Map(\Spec A \times_Z Y, X).
\end{tikzcd}
\]
Let $q: \Spec A \times_Z Y \to \Spec A$ denote the projection. Then $\Spec(A\oplus M) \times_Z Y$ coincides with the extension
$(\Spec A \times_Z Y)[q^*M]$ by the pullback $q^*M$.\footnote{Work locally, take 
$\Spec(A \otimes_{\mathcal{O}_Z} \mathcal{O}(Y) \oplus M\otimes_{\mathcal{O}_Z} \mathcal{O}(Y)$ over each affine piece and glue.}
Therefore the pullback diagram becomes:
\[
\begin{tikzcd}
\Der_{X}(A,q^*M)\arrow{r}\arrow{d} & \Map((\Spec A \times_Z Y) [q^*M], X)\arrow{d} \\
\Spec k\arrow{r} & \Map(\Spec A \times_Z Y, X).
\end{tikzcd}
\]
Now the top left is equivalent to $\Map_{\Spec A \times_Z Y}(f_x^*\bbL_{X/Z}, q^*M)$. So the existence of a cotangent complex at the
point $x$ is reduced to:
\[	\Map_{A\Mod}(\bbL_{\Map_{/Z}(Y,X),x}, M) \simeq \Map_{\Spec A \times_Z Y}(f_x^*\bbL_{X/Z}, q^*M).	\]
Thus, we need a left adjoint for $q^*$; this is the map $q_+$ introduced in \ref{} below. Then we can define:
\begin{equation}
\label{eq:mapping_local_cotangent}
	\bbL_{\Map_{/Z}(Y,X)/Z,x} := q_+ f_x^*\bbL_{X/Z}.
\end{equation}

Finally, we address the gluing of these cotangent complexes. Assume that we have a morphism $g: \Spec B \to \Spec A$. We define
a point $y: \Spec B \to \Map_{/Z}(Y,X)$ by requiring the following diagram to commute:
\[
\begin{tikzcd}
\; & & X \\
Y \times_Z \Spec B \arrow{r} \arrow{d}{q_B} \arrow{urr}{f_y} & Y\times_Z \Spec A \arrow{d}{q_A}\arrow[swap]{ur}{f_x} & \\
\Spec B\arrow{r}{g} & \Spec A &
\end{tikzcd}
\]
From the commutativity of the upper triangle we obtain $f_y = f_x \circ 1\times g$, so that:
\[	q_{B+} f_y^* \bbL_X \simeq q_{B+}(1\times g)^*f_x^*\bbL_X .	\]
Our goal is to show that gluing works, which means:
\[	q_{B+} f_y^* \bbL_X \simeq g^* q_{A*} f_x^* \bbL_X.	\]
Therefore it suffices to prove that $q_+$ has the base change property $q_{B+}(1\times g)^* \simeq g^* q_{A+}$. This
is the object of Lemma \ref{lem:q+_base}, while the construction of $q_+$ is Lemma \ref{lem:q+}.
\end{proof}

\begin{rem}
\label{rem:mapping_global_cotangent}
The tangent complex of the mapping stack, when it exists, can be obtained more easily from the diagram:
\[
\begin{tikzcd}
\Map_{/Z}(Y,X) \times_Z Y \arrow{d}{\pi}\arrow{dr}{ev} & \\ \Map_{/Z}(Y,X) & X.
\end{tikzcd}
\]
Then:
\[	\bbT_{\Map_{/Z}(Y,X)/Z} = \pi_* \ev^* \bbT_{X/Z}.	\]
We would like to dualize and obtain:
\begin{equation}
\label{eq:mapping_global_cotangent}
	\bbL_{\Map_{/Z}(Y,X)/Z} = (\pi_* \ev^* \bbL_{X/Z}^{\vee})^{\vee} = \pi_+ \ev^* \bbL_{X/Z}.
\end{equation}
In Theorem \ref{thm:mapping_stack_geometric}, we actually prove that $\bbL_{\Map_{/Z}(Y,X)/Z}$ exists, by constructing
it locally and then showing that the construction glues. The result of the gluing must then be \ref{eq:mapping_global_cotangent},
as can be seen from the extended diagram:
\[
\begin{tikzcd}
\Spec A \times_Z Y\arrow{r}{x\times 1}\arrow{d}{q} &\Map_{/Z}(Y,X) \times_Z Y \arrow{d}{\pi}\arrow{dr}{ev} & \\
\Spec A\arrow{r}{x} & \Map_{/Z}(Y,X) & X.
\end{tikzcd}
\]
Since the square is a homotopy pullback, we apply base change for $q_+$ (see Lemma \ref{lem:q+_base}):
\[	x^* \pi_+ \ev^* \bbL_{X/Z} \simeq q_+ (x\times 1)^* \ev^* \bbL_{X/Z} \simeq q_+ f_x^*\bbL_{X/Z},	\]
which agrees with what we called $\bbL_{\Map_{/Z}(Y,X)/Z,x}$ in \ref{eq:mapping_local_cotangent}. It follows that
$\bbL_{\Map_{/Z}(Y,X)/Z} \simeq \pi_+ \ev^* \bbL_{X/Z}$.
\end{rem}


\section{Stable Maps}
We apply Theorem \ref{thm:mapping_stack_geometric} and Remark \ref{rem:mapping_global_cotangent} to the derived 
moduli space of stable maps on a smooth projective
variety $X$:
\[	\R\mathcal{M}_{g,k}(X) = \Map_{dSt/\mathcal{M}_{g,k}}(\mathcal{C}_{g,k},X\times \mathcal{M}_{g,k}).	\]
According to Remark \ref{rem:mapping_global_cotangent}, and using the notation therein:
\begin{equation}
\label{eq:cotangent_stable_prelim}
	\bbL_{\R\mathcal{M}_{g,k}(X)/\mathcal{M}_{g,k}} = \pi_+ \ev^* \bbL_{X\times \mathcal{M}_{g,k}/\mathcal{M}_{g,k}}.
\end{equation}
We can simplify this expression using the pullback diagram:
\[
\begin{tikzcd}
X \times \mathcal{M}_{g,k}\arrow{r}{p}\arrow{d} & X\arrow{d} \\ \mathcal{M}_{g,k}\arrow{r} & \Spec k .
\end{tikzcd}
\]
The diagram implies $\bbL_{X\times \mathcal{M}_{g,k}/\mathcal{M}_{g,k}} \simeq p^* \bbL_X$, so \ref{eq:cotangent_stable_prelim}
reduces to:
\begin{equation}
\label{eq:cotangent_stable}
	\bbL_{\R\mathcal{M}_{g,k}(X)/\mathcal{M}_{g,k}} = \pi_+ \ev^* p^* \bbL_{X}.
\end{equation}

\begin{prop}
The natural map $\R \mathcal{M}_{g,k}(X) \to \mathcal{M}_{g,k}$ is quasi-smooth.
\end{prop}
\begin{proof}
We need to show that $\bbL_{\R\mathcal{M}_{g,k}(X)/\mathcal{M}_{g,k}} = \pi_+ \ev^* p^* \bbL_{X}$ has cohomological amplitude
$[-1,0]$. $X$ is a smooth variety, so $\bbL_X$ is in amplitude $[0,0]$. Pullbacks preserve cohomological amplitude, because
they only involve tensoring with locally free sheaves. \todo{Need any assumption on the maps?} $\pi_*$ may increase cohomological
amplitude, because of higher direct image sheaves. However, that the fibers of $\pi : \R\mathcal{M}_{g,k}(X) 
\times_{\mathcal{M}_{g,k}} \mathcal{C}_{g,k} \to \mathcal{M}_{g,k}$ are curves, so the cohomological amplitude of
$\pi_* (\ev^* p^* \bbL_{X})^{\vee}$ is at most $[0,1]$. Dualizing again brings $\bbL_{\R\mathcal{M}_{g,k}(X)/\mathcal{M}_{g,k}}$
to amplitude $[-1,0]$.
\end{proof}

Together with the fact that $\mathcal{M}_{g,k}$ is smooth, this implies that $\R\mathcal{M}_{g,k}(X)$ is quasi-smooth.
Proposition \ref{prop:derived_obstruction_theory} implies the following.

\begin{cor}
The derived enhancement $j: \mathcal{M}_{g,k}(X) \to \R\mathcal{M}_{g,k}(X)$ determines a perfect obstruction theory
on $\mathcal{M}_{g,k}(X)$:
\[	j^* \bbL_{\R\mathcal{M}_{g,k}(X)} \to \bbL_{\mathcal{M}_{g,k}(X)}.	\]
\end{cor}

Expression \ref{eq:cotangent_stable} for the cotangent complex of $\R\mathcal{M}_{g,k}(X)$ shows that the obstruction theory
is the same as that considered by \cite{Behrend_Intrinsic_normal_cone_1997} and introduced in 
\todo{reference once the chapter is edited}.




\section{The + Pushforward Functor}

\begin{lem}[3.3.22 and 3.3.23 in \cite{DAG-XII}]
\label{lem:q+}
Suppose that $q: Y \to S$ is perfect, i.e. $q_*$ preserves perfect complexes. Then $q^* : \QCoh(S) \to \QCoh(Y)$ has a left
adjoint $q_+$.
\end{lem}
\begin{proof}
Let $F \in \Perf(Y)$ and $G \in \QCoh(S)$. Then:
\begin{align*}
	&\Map_{\QCoh(S)} \big( (q_*F^{\vee})^{\vee}, G\big) \simeq \Map_{\QCoh(S)} \big( \mathcal{O}_S, q_*F^{\vee} \otimes G\big) \\
\simeq& \Map_{\QCoh(S)} \big(\mathcal{O}_S, q_*(F^{\vee} \otimes q^* G)  \big) \simeq \Map_{\QCoh(Y)}(\mathcal{O}_Y,F^{\vee} \otimes q^*G)
 \simeq \Map_{\QCoh(Y)}(F,q^*G).
\end{align*}
We have used the fact that perfect complexes are dualizable and the projection formula $q_*(F^{\vee} \otimes q^* G)
\simeq q_*F^{\vee} \otimes G$. This means that, for $F$ perfect, we can use $q_+F = (q_*F^{\vee})^{\vee}$. Now if $S$ and $q$
are quasi-compact and quasi-separated, $\QCoh(Y) = \Ind \Perf(Y)$. Remarking that $q_+ : \Perf(Y) \to \QCoh(S)$ is a left
adjoint, it commutes with colimits, and so there exists a unique extension $q_+ : \Ind \Perf(Y) \to \QCoh(S)$ which commutes with
colimits. It also follows that the extension is a left adjoint.
\end{proof}

\begin{lem}[3.3.23 in \cite{DAG-XII}]
\label{lem:q+_base}
Suppose given a pullback diagram of DM stacks, \todo{can we do better?} with $f, f'$ perfect.
\[
\begin{tikzcd}
X'\arrow{r}{g'}\arrow{d}{f'} & X\arrow{d}{f} \\ Y' \arrow{r}{g} & Y
\end{tikzcd}
\]
Then the canonical map $\lambda: f'_+ \circ g'^* \to g^* f_+$ is an equivalence. 
(We say that the + pushforward satisfies base change.)
\end{lem}
\begin{proof}
On perfect objects $F$, $\lambda_F$ is the dual of:
\[	g^*f_* F \to f'_*g'^* F.	\]
We have used the fact that pullbacks preserve duals. This is an isomorphism due to base change for the pusforward $f_*$.
To conclude, both $f_+$ and $g^*$ are now left adjoints, which means they preserve all colimits. It follows that
$f'_+ \circ g'^* \simeq g^* f_+$ extends to $\QCoh \simeq \Ind \Perf$.
\end{proof}

Note that we have used the assumption that $q: Y \to S$ is perfect. It remains, then, to prove that the map $q : \Spec A \times_Z Y
\to \Spec A$ from the proof of Theorem \ref{thm:mapping_stack_geometric} is perfect. We do this in Lemma \ref{lem:q_perfect}
below, after introducing some terminology.

\begin{defin}
A map $q:Y \to S$ is \textbf{categorically proper}, also called \textbf{of finite cohomological dimension}, if
$q_* : \QCoh(Y) \to \QCoh(S)$ increases cohomological dimension by a uniform finite amount.
\end{defin}

\begin{defin}
A map $q:Y \to S$ is \textbf{of finite tor amplitude} if locally $q:\Spec B \to \Spec A$ and $B$ is of finite tor amplitude
as an object of $A\Mod$.
\end{defin}

Note that in our example $q: \Spec A \times_Z Y \to \Spec A$, $q$ is:
\begin{itemize}
\item proper, because we assumed that $Y \to Z$ is proper, and this is stable under base change;
\item categorically proper, because we can compute $q_*$ by a (uniformly) finite \v{C}ech resolution, due to the assumption
that $Y \to Z$ is representable by proper schemes.
\item of finite tor amplitude, because this is a consequence of flatness. We have assumed that $Y\to Z$ is flat,
and flatness is stable under base change.
\end{itemize}

\begin{lem}
\label{lem:q_perfect}
Let $q: Y \to S$ be a map which is proper, categorically proper and of finite tor amplitude. Then $q$ is perfect.
\end{lem}
\begin{proof}
First, we claim that the first two assumptions imply that $q_* \Coh^-(X) \to \Coh^-(S)$. This argument uses the Leray spectral
sequence. \todo{fill this in} Next, note that $\Perf(X) \subset \Coh^-(X)$ is characterized as the full subcategory of complexes
with finite tor amplitude. So it remains to prove that, if $F \in \Coh^-(X)$ has finite tor amplitude, then so does $q_*F$.

Take a Zariski affine cover for $X$; due to the quasi-compactness assumption this can be taken finite. Then the \v{C}ech nerve
$U^{\bullet}$ is a finite complex. Since $q_*$ is a right adjoint, it commutes with limits, and we have:
\[	q_* F = \varprojlim q|_{U^{\bullet}*} F|_{U^{\bullet}}.	\]
Note that the maps $U_i \to X$ are not, in general, proper, so $q|_{U^{\bullet}*} F|_{U^{\bullet}}$ needn't be coherent. However,
on affines $q_*$ is just a forgetful functor on modules. Therefore the coherence of $q|_{U^{\bullet}*} F|_{U^{\bullet}}$
follows from the fact that $F|_{U^{\bullet}}$ is coherent and the map on rings is finitely generated. \todo{explain more}

For finite tor amplitude, it suffices to check that, for every $M$ discrete, $q_*F \otimes_{\mathcal{O}_S} M$ is cohomologically
supported in $[-m,\infty)$ for some $m$. (We already know that $q_*F$ is bounded above.) Since the \v{C}ech nerve is finite,
we have:
\[	q_*F \otimes_{\mathcal{O}_S} M = \varprojlim q|_{U^{\bullet}*} F|_{U^{\bullet}} \otimes_{\mathcal{O}_S} M.	\]
The inclusion $\Coh^{-,[-m,\infty)}(S) \subset \Coh^-(S)$ commutes with limits, which gives the result.
\end{proof}




\section{Application: Weil Restriction}
Weil restriction is, roughly speaking, an adjoint for base change. We sketch the treatment that Lurie gives in \cite{DAG-XIV}.

\begin{defin}
Let $\phi : Y \to Z$ and $X \to Y$ be maps of derived stacks. A \textbf{Weil restriction} for $X$ along $\phi$ is a
stack $\Res_{Y/Z}X \to Z$, equipped with a morphism $\rho_X : \Res_{Y/Z}X \times_Z Y \to X$ over $Y$, such that composition
with $\rho$ determines a homotopy equivalence:
\begin{equation}
\label{eq:weil_restriction}
	\Map_{\dSt/Z}( - , \Res_{Y/Z}X) \simeq \Map_{\dSt/Y}(- \times_Z Y, X).
\end{equation}
\end{defin}

\begin{eg}
In arithmetic geometry $\phi: Y \to Z$ is taken to be a field extension $\phi : \Spec L \to \Spec k$; then the adjunction
\ref{eq:weil_restriction} gives a bijection between $L$-points of $X$ and $k$-points of $\Res_{\Spec L/\Spec k}X$. To illustrate
this without getting too much out of our comfort zone, we take $k = \R$, $L = \C$, and start with $X$ an affine variety over $\C$,
given as a subset of $\C^n$ by equations $f_i(z_1, \dots, z_n) = 0$. Then $\Res{\Spec L/\Spec k}X$ is an affine variety over $\R$,
given as a subset of $\R^{2n}$ by equations $\Re f_i(x_1 + i y_1, \dots, x_n + i y_n) = 0$, 
$\Im f_i(x_1 + i y_1, \dots, x_n + i y_n) = 0$.
\end{eg}

Lurie proves the following existence result.

\begin{thm}

\end{thm}
\begin{proof}
The basic idea is to define $\Res_{Y/Z}X$ as the homotopy pullback:
\[
\begin{tikzcd}
\Res_{Y/Z}X\arrow{r}\arrow{d} & \Map_{/Z}(Y,X)\arrow{d} \\
Z\arrow{r} & \Map_{/Z}(Y,Y)
\end{tikzcd}
\]
\end{proof}



\chapter{Reduced Gromov Witten Invariants for K3 Surfaces}
\label{ch4:k3}

(Talk by Benedict Morrissey)  In chapter \ref{ch3:geom}, we constructed a quasismooth derived enhancement of the moduli stack of stable maps. The machinery of chapter \ref{ch2:obs} can then be used to construct virtual fundamental classes, and from there Gromov--Witten invariants. In this chapter we consider the case where the target varietry $X$ is a K3 surface.  In this case Gromov--Witten invariants defined in this fashion vanish. In this chapter, following \cite{schurg2015derived}, we provide an alternative quasismooth derived enhancement. The virtual fundamental classes obtained were earlier described in \cite{maulik2007gromov, maulik2010curves, okounkov2010quantum} (henceforth referred to as OMPT).

\section{Non Reduced Gromov--Witten Invariants for K3 Surfaces}
Note that GW invariants are deformation invariant, but in the moduli space of K3 surfaces there's a dense locus of non-algebraic
K3s. These don't admit any (1,1) classes in $H_2$. It follows that the deformation of the class $[\beta]$ is not effective,
and the corresponding invariants must vanish.

Somehow, passing to the reduced obstruction theory restricts to algebraic deformations only, and this problem disappears.
\todo{understand why this happens}


\todo{Work out where I can find a proof that all these invariants disappear!}

\section{$\R Pic$ for K3 Surfaces}

The point of this section is to give a canonical identification $\R Pic(X)\xrightarrow{\cong}Pic(X)\times \R Spec(Sym(H^{0}(X,K_{X})[1]))$.  Recall that $\R Pic(X)$ is a (locally of finite presentation) derived group stack.

\begin{thm}[Prop. 4.5 in \cite{schurg2015derived}]
\label{thm:derivedgroup}
A locally of finite presentation group stack $G$ over a field $k$, with identity $e: \Spec k\rightarrow G$, and Lie algebra $\mathfrak{g}=T_{e}G$,\footnote{This is the tangent space at the identity, so it's a dg Lie algebra.} has a canonical map
\[\gamma(G):t_{0}(G)\times \R \Spec(A)\rightarrow G,\]
where $A=k\oplus (\mathfrak{g}^{\vee})_{<0}$.
\end{thm}

\begin{proof}
The projection map $A\rightarrow k$ gives a $k$-point $x_{0}:\Spec(k)\rightarrow \R \Spec(A)$.

We wish to find a commuting diagram 
\[
\begin{tikzcd}
B \arrow{rr}{a} && C\\
 & A \arrow{ur}{d} \arrow{ul}{x_{0}}
\end{tikzcd}
\]
this is equivalent to giving a morphism $a':\bbL_{G,e} \cong \mathfrak{g}^{\vee}\rightarrow (\mathfrak{g}^{\vee})_{<0}$ (due to our choice of $A$), hence taking the truncation map $\tau_{< 0}$ (using the standard t-structure on the stable category of (dg) vector spaces) gives 
this map.

We then take the composition
\[t_{0}(G)\times \R \Spec(A)\xrightarrow{j\times a}G\times G\xrightarrow{\times}G,\]
where the final map uses the group product in $G$.
\end{proof}

We now apply Theorem \ref{thm:derivedgroup} to the group stack $\R \Pic(X)$ for a K3 surface $X$.

\begin{thm}
The map $\gamma_{\R \Pic(X)}$ for $X$ a K3 surface gives an isomorphism of derived stacks
\[\R \Pic(X)\xrightarrow{cong} \Pic(X)\times \R \Spec(\Sym(H^{0}(X,K_{X})[1])).\]
\end{thm}

\begin{proof}
We note first that this map clearly provides an isomorphism on truncations.  Hence as $\R Pic(X)$ is a derived group stack, we need only show that it is \'{e}tale at $e$, that is to say 
\[\bbT_{t_{0}(e),x_{0}}(\gamma_{\R \Pic(X)}):\bbT_{t_{0}(e),x_{0}}(\Pic(X)\times \R \Spec(\Sym(H^{0}(X,K_{X})[1])))\rightarrow T_{e}G\]
is an isomorphism of dg k-vector spaces.

Note that, since $H^1(X,\mathcal{O}_X) = 0$: 
\[T_{e}G=\mathfrak{g}=\R \Gamma(X,\mathcal{O}_{X})[1]\cong H^{0}(X,\mathcal{O}_{X})[1]\oplus H^{2}(X,\mathcal{O}_{X})[-1].\] 

Hence 
\[A=\C\oplus (\mathfrak{g}^{\vee})_{<0}=\C\oplus H^{2}(X,\mathcal{O}_{X})[1]\cong \C \oplus H^{0}(X,K_{X})[1]\cong Sym(H^{0}(X,K_{X})[1]\]
where the final step follows because $H^{0}(X,K_{X})$ is free of dimension 1.

Clearly $T_{t_{0}(e),x_{0}}(\gamma_{\R Pic(X)})$ is an isomoprhism.
\end{proof}

This identification allows the definition of a projection $pr_{der}:\R Pic(X)\rightarrow \R Spec(Sym(H^{0}(X,K_{X})[1])$.

\section{The reduced Moduli Space $\R M_{g,n}(X,\beta)^{red}$.}
\label{sec:reduced moduli space}

Recall the map $x_{0}:Spec(\C)\rightarrow  \R Spec(Sym(H^{0}(X,K_{X})[1])$.

We defined the reduced derived enhancement as follows:
\begin{defin}
The \textbf{reduced stack of $n$-pointed stable maps} of genus $g$, class $\beta$ to a K3 surface $X$ is given by the pullback
\[
\begin{tikzcd}
\R M_{g,n}^{red}(X,\beta)\arrow{r}\arrow{d} & \R M_{g,n}(X,\beta)\arrow{d}{\delta_{1}^{der}(X,\beta)} \\
Spec(\C)\arrow{r}{x_{0}} & \R Spec(Sym(H^{0}(X,K_{X})[1])
\end{tikzcd}
\]
where we define the map $\delta_{1}^{der}(X,\beta)$ by the composition
\[\R M_{g,n}(X,\beta)\rightarrow \R M_{g,n}(X)\xrightarrow{a} Perf(X)\xrightarrow{det} \R Pic(X)\xrightarrow{pr_{det}} \R Spec(Sym(H^{0}(X,K_{X})[1]),\]

where the map $a$ is induced by the perfect complex $\R \pi_{*}(\mathcal{O}_{\R \mathcal{C}_{g,X}})$, (that is this specifies a perfect complex over $X\times M_{g,n}(X,\beta)$, and hence a map  $M_{g,n}(X,\beta)\rightarrow Perf(X)$).
\end{defin}

\begin{rem}
This derived enhancement is quasismooth, and we will show that the induced obstruction theory on $M_{g,n}(X,\beta)$ agrees with that of OMPT.
\end{rem}

\begin{prop}
The derived stack $\R M_{g,n}^{red}(X,\beta)$ is quasismooth.
\end{prop}

\begin{proof}
The cotangent complex relative to $M_{g,n}$ is:
\[ \bbT_{\R M_{g,n}(X,\beta)/M_{g,n}, f:C \to X} = \R \Gamma(C,f^*\bbT_X) . \]
The absolute one is:
\[	\bbT_{\R M_{g,n}(X,\beta), f:C \to X} = \R \Gamma\big(C,\text{Cone}(\bbT_C(-\sum x_i) \to f^*\bbT_X)\big) .	\]
We use the pullback square:
\[
\begin{tikzcd}
\R M_{g,n}^{red}(X,\beta) \arrow{r}\arrow{d} & \R M_{g,n}(X,\beta)\arrow{d} \\
\Spec \C \arrow{r} & \R \Spec (\Sym H^0(X,K_X)[1]) .
\end{tikzcd}
\]
This gives the following distinguished triangle of complexes:
\[	\bbT^{red} := \bbT(\R M_{g,n}^{red}(X,\beta)) \to \R \Gamma \big(C,\text{Cone}(\bbT_C(-\sum x_i) \to f^*\bbT_X)\big)
\to H^2(X,\mathcal{O}_X)[-1] .	\]
Looking at the associated long exact sequence of cohomology, it suffices to show that:
\[	H^1 \R \Gamma \big(C,\text{Cone}(\bbT_C(-\sum x_i) \to f^*\bbT_X)\big) \to H^2(X,\mathcal{O}_X)	\]
is surjective. We do this by precomposing with the map:
\[	H^1(C,f^*\bbT_X) \to  H^1 \R \Gamma \big(C,\text{Cone}(\bbT_C(-\sum x_i) \to f^*\bbT_X)\big) ,	\]
and proving that the resulting map is surjective. For the latter, we use the commutative diagram:
\[
\begin{tikzcd}
H^1(X,\bbT_X)\arrow{d}\arrow{r} & \Ext^2_X(\R f_*\mathcal{O}_C, \R f_*\mathcal{O}_C) \arrow{r} & H^2(X,\mathcal{O}_X)[-1] \\
H^1(C,f^*\bbT_X) \arrow{r} & H^1(X,\text{Cone}(\bbT_C(-\sum x_i) \to f^*\bbT_X)) \arrow{u} & 
\end{tikzcd}
\]
\todo{Finish This}
\end{proof}

\section{The Resultant Obstruction Theories}

In this section we compare the obstruction theories obtained from the reduced derived enhancement with those obtained in OMPT.

\todo{Want THM 4.8 from STV}

\todo{Write an explicit description}

The explicit description is as follows.
\begin{align*}
	H^0(\bbT^{red}_f) &\cong H^0\big(C,\text{Cone}(\bbT_C \to f^*\bbT_X)\big)	\\
	H^1(\bbT^{red}_f) &\cong \Ker\big(H^1(\Theta_f) : H^1\big(C,\text{Cone}(\bbT_C \to f^*\bbT_X)\big) \to H^2(X,\mathcal{O}_X)\big)
\end{align*}


\section{The Link to Donaldson--Thomas Theory}

Recall that in section \ref{sec: reduced moduli space} we made use of a map $\R Map (X,\beta)\rightarrow Perf(X) $.  Recall that the stack $Perf(X)$ is used in Donaldson--Thomas theory, a curve counting theory that uses a different compactification of the space of curves in $X$ that the stable curves used in Gromov--Witten invariants.  There are conjectured to be various relationships between Gromov--Witten and Donaldson--Thomas invariants as developed in \cite{maulik2006gromovI, maulik2006gromovII}.

The compactification for DT is as follows. For $C\subset X$ an embedded curve, $f_*\mathcal{O}_C \in \Perf(X)$. More precisely,
$f_*\mathcal{O}_C$ lands in a component (or union thereof) $\Perf^{si, \geq 0, \beta}$. The latter is the DT stack, and we want to
have a theory of integration over it.

For a start, fix $\mathcal{L}$ a line bundle on $X$, corresponding to a map $\Spec \C \to \R \Pic(X)$. Then we define the stack
of perfect complexes with fixed determinant:
\[
\begin{tikzcd}
\Perf(X)_{\mathcal{L}} \arrow{r}\arrow{d} & \Perf(X)\arrow{d}{\det} \\
\Spec \C \arrow{r} & \R\Pic(X).
\end{tikzcd}
\]
We have another constraint:
\[	\Ext^i(F,F) = 0, \forall i<0,	\]
and moreover $\Ext^0(F,F) \simeq H^0(X,\mathcal{O}_X)$, which gives $\Perf(X)^{si \geq 0}$.

For a K3 surface, we can consider again the reduced theory:
\[
\begin{tikzcd}
\Perf(X)^{red} \arrow{r}\arrow{d} & \Perf(X)\arrow{d} \\
\Spec \C \arrow{r} & \R \Spec (\Sym H^0(X,K_X)[1]) .
\end{tikzcd}
\]

In the end we put all of these constraints together to form $\Perf(X)^{si \geq 0, \mathcal{L}, red}$.

\todo{sb draw big diagram, I got lazy}

These triangles are precisely the diagram necessary for functoriality of obstruction theories. Of course, this is rather silly:
we showed that if you restrict to the open subset, the choice of compactification doesn't influence the computation.

\todo{talk about all of S5}


\chapter{Abelian Threefold}

Let's denote by $\eta : \mathcal{O}_M \to \bbL_M \otimes \bbT_M$ the counit and $\epsilon : \bbL_M \otimes \bbT_M \to \mathcal{O}_M$
the unit.
The most natural expressions for $\omega$ and $\iota_{\xi}$ are as follows.
\[
\begin{tikzcd}
p^*\fr g \otimes \wedge^2 \bbL_M \arrow{r}{act \otimes 1} \arrow[bend left]{rr}{\iota} &
\bbT_M \otimes \wedge^2 \bbL_M\arrow{r}{\epsilon} & \bbL_M
\end{tikzcd}
\]
\[
\begin{tikzcd}
\bbT_M\otimes \bbT_M \arrow{r}\arrow[bend left]{rrrr}{\omega} & p_*\mathcal{A} \otimes p_*\mathcal{A}[2] \arrow{r} & 
p_*(\mathcal{A}\otimes \mathcal{A})[2]
\arrow{r} & p_*\mathcal{O}_{M \times A}[2] \arrow{r} & \mathcal{O}_M[2-3]
\end{tikzcd}
\]

The diagram we want to find a lift in can therefore be written as:
\[
\begin{tikzcd}
\; & & & \mathcal{O}_M[-1]\arrow{d}{d_{dR}} \\
p^* \fr g \otimes \mathcal{O}_M \arrow[swap]{r}{\wedge^2 \eta}\arrow[dashed]{urrr}{\mu} & 
p^* \fr g \otimes \wedge^2 \bbT_M \otimes \wedge^2 \bbL_M \arrow[swap]{r}{\omega} &
p^* \fr g \otimes \wedge^2 \bbL_M \otimes \mathcal{O}_M[-1]\arrow[swap]{r}{\iota} & \bbL_M \otimes \mathcal{O}[-1] .
\end{tikzcd}
\]


\section{An Introduction to Derived Symplectic Reduction}
\chapter{DT-Theory and stable pairs}
In this chapter, we introduce two other ways of curve counting, especially in projective 3-fold, called Donaldson-Thomas theory and Pandharipande-Thomas theory(Stable pairs). Comparison of these invariants with GW-invariants is mainstream of enumerative geometry and we explain several conjectures and results. Next, we formalize it in the derived setting. More generally, we introduce compactly supported integration along the fiber with which we could define shifted symplectic structure on the derived mapping stack with non-proper source.

\section{Donaldson-Thomas Theory}
The idea of Donaldson-Thomas theory is to count curves "embedded" in $X$, instead of counting stable maps from curve to $X$. One might be able to take Hilbert scheme to compactify such embedded curves as follows. Let $n \in \Z, \beta \in H_2(X;\Z)$. There exists a projective scheme $Hilb_n(X,\beta)$ compactifying the moduli space of embedded curves in $X$. 
\begin{equation*}
    Hilb_n(X, \beta) = \{C \hookrightarrow X | dimC \leq 1, \chi(\mathcal{O}_C)=n, [C]=\beta \}
\end{equation*}
However, it does not admit perfect obstruction theory because deformation and obstruction space don't behave nicely enough. In order to remedy this, we should interpret it as a moduli of (stable) sheaves on $X$. 
\begin{lem}
     Let $I_n(X, \beta)$ be a moduli space of rank 1 torsion free sheaf $I$ in $X$ with chern character $ch(I)=(1,0,-\beta, -n) \in H^*(X;\Z)$ and trivial determinant. Then, $Hilb_n(X;\beta) \simeq I_n(X, \beta)$
\end{lem}
\begin{proof}
    The map $Hilb_n(X;\beta) \to I_n(X,\beta)$ is given by $Z \mapsto I_Z$ where $I_Z$ is an ideal sheaf of the subscheme $Z$. $I_Z$ is torsion-free because it is a subsheaf of coherent locally free sheaf $\mathcal{O}_X$. The Chern character part is trivial. Conversely, any element $I \in I_n(X,\beta)$ gives an ideal sheaf of $\mathcal{O}_X$ from a natural map to reflexive hull $I^{\vee\vee} \simeq \mathcal{O}_X$. Since $I$ is torsion-free, $det(I) \simeq I^{\vee\vee} \simeq \mathcal{O}_X$.
\end{proof}
\begin{rem}
    This is a special case of a moduli of stable sheaves. If we choose polarization $\omega$ on $X$, then we can put ($\omega$)-stability condition on the category of coherent sheaves on $X$. To be specific, $E \in \Coh(X)$ is $\omega$-(semi)stable if 
    \begin{itemize}
        \item $E$ is pure
        \item For any subsheaf $0 \neq F \subsetneq E$, $P_F(m)<(\leq)P_E(m)$ where $P_E(m)=\chi(E \otimes \mathcal{O}_X(m))$ modulo a leading coefficent. 
    \end{itemize} 
    Hence, for fixed Chern character $\mu \in H^*(X;\Q)$, $\mathfrak{M}^{s(s)}_{\omega}(\mu)$ is defined to be a moduli stack of $\omega$-(semi)stable sheaf with fixed Chern character. Clearly, $\mathfrak{M}^s_{\omega}(\mu) \subset \mathfrak{M}^{ss}_{\omega}(\mu)$ and there exists a quasi-projective scheme $M_{\omega}(\mu)$ with a $\C^*$-gerbe $\mathfrak{M}^s_{\omega}(\mu)$ because automorphism in $\mathfrak{M}^s_{\omega}(\mu)$ is just $\C^*$. In particular, \\
    if $\mathfrak{M}^s_{\omega}(\mu)=\mathfrak{M}^{ss}_{\omega}(\mu)$, then $M_{\omega}(\mu)$ is a projective scheme. In our case, for $\mu=(1, 0, -\beta, -n)$, this condition holds and $M_{\omega}(\mu)$ is isomorphic to $Hilb_n(X,\beta)$.
\end{rem}
The moduli space of sheaves $I_n(X,\beta)$ admits a virtual class. The main point is that deformations and obstructions are governed by $\text{Ext}^1(I_C, I_C)_0, \text{Ext}^2(I_C, I_C)_0$ respectively, where the subscript 0 denotes the trace-free part governing deformations with fixed determinant $\mathcal{O}_X$. Since $\text{Hom}(I_C, I_C)=\C$ consists of only the scalars, the trace-free part vanishes. By Serre duality, 
\begin{equation*}
    \Ext^3(I_C, I_C) \cong \Hom(I_C, I_C \otimes K_X)^* \cong H^0(K_X) \cong H^3(\mathcal{O}_X)
\end{equation*}
It vanishes when we take trace-free parts. Hence, there is no higher obstruction spaces except\\ $\text{Ext}^1(I_C, I_C)_0, \text{Ext}^2(I_C, I_C)_0$ and they govern a perfect obstruction theory of virtual dimension equals to 
\begin{equation*}
    ext^1(I_C, I_C)_0-ext^2(I_C, I_C)_0= \int_{\beta}c_1(X)
\end{equation*}
In CY-case, we get the following definition.
\begin{defin}
    Let $X$ be a projective Calabi-Yau 3-fold. Donaldson-Thomas invariant $I_{n, \beta}$ is defined to be
    \begin{equation*}
        I_{n,\beta}=\int_{[I_n(X,\beta)]^{vir}}1
    \end{equation*}
\end{defin}
\begin{rem}
    By Serre duality, we can see that it indeed admits a symmetric obstruction theory. By Berhend, when the moduli space is equipped with symmetric obstruction theory, we can construct a function $\nu:I_n(X,\beta) \to \Z$ which is called Berhend function. Then 
    \begin{equation*}
        I_{n,\beta} = \int_{[I_n(X,\beta)]}\nu de =\sum_{m\in \Z} me(\nu^{-1}(m))
    \end{equation*}
    In particular, if $I_n(X,\beta)$ is non-singular and connected, then it is just the topological Euler characteristic of $I_{n,\beta}$
\end{rem}

\begin{eg}\label{DTex}
    Consider $C_t=\{x=z=0\} \cup \{y=0,z=t\} \subset \C^3$. As a subscheme, the limit goes to $C_0$ whose ideal sheaf is generated by $(x,z)(y,z)=(xy,xz,yz,z^2) \varsubsetneq (xy,z)$. This limit ideal does not contain $(z)$, so as a scheme, the limit curve $C_0$ is given by $\{xy=0=z\}$ with a scheme-theoretic point added at the origin. Moreover, this embedded point can break off as follows. Consider the flat family $C_{\epsilon}=\{xy=\epsilon, z=0\}$. $C_0$ can be smoothed to $C_\epsilon$ with higher genus. In this case, the origin is not an embedded point anymore, but free point not on the curve. 
\end{eg}
This example reveals some disadvantages of DT-invariant in counting curves. Free and embedded points distract us from counting curves. 
\begin{defin}
    For fixed curve class $\beta \in H_2(X, \Z)$, the DT partition function is
    \begin{equation*}
        \mathsf{Z}^{DT}_{\beta}(q)=\sum_n I_{n,\beta}q^n.
    \end{equation*}
\end{defin}
Since $I_n(X, \beta)$ is easily seen to be empty for sufficiently negative $n$, the partition function is a Laurent series in $q$. In order to count just curves, not points and curves, we define the reduced generating function by dividing by contribution of just points:
\begin{equation*}
    \mathsf{Z}^{red}_{\beta}(q)=\frac{\mathsf{Z}^{DT}_{\beta}(q)}{\mathsf{Z}^{DT}_0(q)}
\end{equation*}
Note that $I_n(X, 0)$ is a moduli space of points with length $n$. Here, we gather some properties of partition functions which is not obvious at all.
\begin{prop}
    \begin{enumerate}
        \item $\mathsf{Z}^{DT}_{0}(q)=M(-q)^{e(x)}$ where $M$ is the MacMahon function, \\
        $M(q)= \prod_{n \geq 1}(1-q^n)^{-n}$ the generating function for 3d partitions. 
        \item $\mathsf{Z}^{red}_{\beta}(q)$ is the Laurent expansion of a rational function in $q$, invariant under $q \leftrightarrow q^{-1}$. 
    \end{enumerate}
\end{prop}
Therefore, we can substitute $q=-e^{iu}$ to get a real-valued function of $u$. The main conjecture of MNOP in the CY-case is the following,

\begin{conj}
    $\mathsf{Z}^{GW}_{\beta}(u)=\mathsf{Z}^{red}_{\beta}(-e^{iu})$
\end{conj}

\section{Stable Pairs}
Let's start with an example.
\begin{eg}\label{PDex}
    Unlike \ref{DTex}, we take different a limit of $C_t$ as follows. Denote each component by $C^1_t, C^2_t$. Consider the sheaf map $s_t:\mathcal{O}_{\C^3} \to \mathcal{O}_{C^1_t} \oplus \mathcal{O}_{C^2_t}$ and take the limit $t \to 0$ of this map. Then we get the following exact sequence of sheaves.
    \begin{equation*}
        0 \to \Ker(s) \to \mathcal{O}_{\C^3} \xrightarrow{s} \mathcal{O}_{C^1} \oplus \mathcal{O}_{C^2} \to \Coker(s) \to 0
    \end{equation*}
    In this case, $\Ker(s)$ is ideal of subscheme without an embedded point and $Im(s)$ is indeed a structure sheaf of $C_0=\{xy=0=z\}$. The lost data of intersection is encoded in $\Coker(s)$ which is a skyscraper sheaf at the origin. As you can see, the difference from \ref{DTex} is the surjectivity of $s$. 
\end{eg}
In \ref{PDex}, $Im(s)$ is a pure sheaf of rank 1 which means that any subsheaf of $Im(s)$ has 1-dimensional support(i.e. no embedded points). Also, the origin can not break off the curve because it is constrolled by the sheaf map. (different limit determines different choice of point.) It justifies the following definition.
\begin{defin}
    A stable pair on $X$ is $(F, s)$ such that 
    \begin{enumerate}
        \item $F$ is a pure sheaf of rank 1
        \item $s:\mathcal{O}_X \to F$ has 0-dimensional cokernel.
    \end{enumerate}
\end{defin}
\begin{eg}
Consider a divisor $D \subset C$ where $C$ is a curve embedded in $X$. Then, the natural map 
\begin{equation*}
    \mathcal{O}_X \hookrightarrow i_*\mathcal{O}_C \hookrightarrow i_*\mathcal{O}_C(D)
\end{equation*}
is a stable pair.
\end{eg}
\begin{lem}
    Giving a stable pair $(F,s)$ is equivalent to choosing 1-dimensional subscheme $C$ of $X$ with a maximal ideal $m \subset \mathcal{O}_C$
\end{lem}
\begin{proof}
    Given a stable pair $(F,s)$, $\ker(s)$ determines ideal sheaf associated a certain 1-dimensional subscheme $C$ whose struture sheaf is given by $Im(s)$. The support $\Coker(s)$ corresponds to a set of points on $C$, hence determining a maximal ideal $m \subset \mathcal{O}_C$ associated to these points. The converse also holds. \todo{reference}
\end{proof}

\begin{rem}
   Similar to moduli space of sheaves, we can put stability condition to justify the word "stable" in the definition.
\end{rem}
As before, we can define the moduli space of stable pairs on $X$, denoted by $P_n(X,\beta)$. However, again, it does not admit virtual class. The way to remedy this is to consider a pair as an object $I^{\bullet}=[\mathcal{O}_X \xrightarrow{s} F]$ in derived category $D^b(X)$. Deformation and obstruction spaces with fixed determinant is given by $\text{Ext}^1(I^{\bullet}, I^{\bullet})_0, \text{Ext}^2(I^{\bullet}, I^{\bullet})_0$ governing a perfect obstruction theory of virtual dimension equals to 
\begin{equation*}
    ext^1(I^{\bullet}, I^{\bullet})_0-ext^2(I^{\bullet}, I^{\bullet})_0= \int_{\beta}c_1(X)
\end{equation*}
\begin{defin}
Let $X$ be a projective Calabi-Yau 3-fold. Pandharipande-Thomas invariant $P_{n, \beta}$ is given by
    \begin{equation*}
        P_{n,\beta}=\int_{[P_n(X,\beta)]^{vir}}1
    \end{equation*}
    Also, for fixed curve class $\beta \in H_2(X,\Z)$, the stable pairs partition function is defined to be 
    \begin{equation*}
        \mathsf{Z}^{PD}_{\beta}(q)=\sum_n P_{n,\beta}q^n
    \end{equation*}
\end{defin}
\begin{thm}
    \[\mathsf{Z}^{PD}_{\beta}(q)= \mathsf{Z}^{red}_{\beta}(q)\]
\end{thm}

\section{Derived Version}
    In this section, we introduce derived variants of $I_n(X,\beta), P_n(X, \beta)$ and prove that they admit $(-1)$-shifted symplectic structure. More generally, we construct shifted symplectic structure on the mapping stack with non-proper source and apply this to the case $X$ is non-proper CY variety.
\begin{defin}
    Let $X$ be a variety and $\mathcal{L}=\mathcal{O}_X[+d] \in \Pic^{gr}X$ be the trivial line in grading $d \neq 0$. The stack of perfect complexes on $X$ with fixed determinant $\mathcal{L}$, is $\Perf^{\mathcal{L}}(X)=\Perf(X) \times _{\Pic^{gr}(X)}\{\mathcal{L}\}$.
\end{defin}
In particular, if $X$ is a CY 3-fold and $d=1$, then we can define derived analogue of $I_n(X,\beta), P_n(X, \beta)$. For simplicity, we ignore $n, \beta$ and denote them by $I(X), P(X)$, respectively. 
\begin{enumerate}
    \item $I(X)$ is a derived stack of torsion free sheaves of rank 1 with fixed determinant. It is open substack of $\Perf^{\mathcal{O}_X[+1]}$.
    \item $P(X)$ is a derived stack of stable pairs. It is open substack of $\Perf^{\mathcal{O}_X[+1]}$ as well.
\end{enumerate}
\todo{There must be concrete ways to define both stacks, but I don't know at this point.} For grading, we can think it as follows. Consider a vector bundle $E$ of rank $d$ on $X$. The associated determinant bundle of $E$ is $\det(E) \cong \bigwedge^d E$. In the derived setting, taking wedge product is the same as taking symmetric product followed by shifting the degree by 1. So, $\det(E) \cong \Sym^d(E[1])$ which is place at degree $-d$. In order to compare with $\mathcal{O}_X$ we should take shifting by $d$. 

If $X$ is compact, then $(-1)$ shifted symplectic structure $\omega$ on $\Perf(X)$ can be restricted to a closed 2-form on $\Perf^{\mathcal{O}_X[+1]}$ which is still non-degenerate. The proof is similar to our main theorem\ref{mainthm}. Now, we are interested in the case $X$ is not proper. The first ingredient we need is the notion of compactly supported integration along the fiber.
\begin{defin}
    Suppose that $\mathcal{X}$ is a derived pre-stack and $K \subset \mathcal{X}$ a closed subset. 
    \begin{enumerate}
    \item Define the (filtered) chain complex of relative de Rham cochains to be 
    \begin{equation*}
        F^k\textbf{DR}(\mathcal{X}, \mathcal{X} \setminus K ) := fib\{i^*: F^k\textbf{DR}(\mathcal{X}) \to F^k\textbf{DR}(\mathcal{X} \setminus K )\}
    \end{equation*}
    \item The compactly supported de Rham cochains are defined as the directed colimit
    \begin{equation*}
        F^k\textbf{DR}_c(\mathcal{X})=\varinjlim_{K \subset  \mathcal{X}} F^k\textbf{DR}(\mathcal{X}, \mathcal{X} \setminus K )
    \end{equation*}
    over all closed subsets $K \subset \mathcal{X}$ which are proper over the base.
    \item If $\mathcal{X}$ is an $S$-prestack, then a relative variant is defined to be 
    \begin{equation*}
        F^k\textbf{DR}_{c/S}(\mathcal{X})=\varinjlim_{K \subset  \mathcal{X}} F^k\textbf{DR}(\mathcal{X}, \mathcal{X} \setminus K )
    \end{equation*}
    \end{enumerate}
\end{defin}

\begin{thm}\label{intalongfiber}
    Suppose that $X$ is a smooth $d$-dimensional scheme, and that $\mathcal{F}$ is a derived pre-stack almost of finite presentation over $k$. A choice of volume form $\text{vol}_X: \mathcal{O}_X \to \Omega^d_X$ gives rise to an integration map of filtered complexes
    \begin{equation*}
        \int_{X}\text{vol}_X \wedge-:F^{\bullet}\textbf{DR}_{c/\mathcal{F}}(X \times \mathcal{F}) \to F^{\bullet}\textbf{DR}(\mathcal{F})[-d]
    \end{equation*}
    such that the induced map on associated graded pieces is the Grothendieck-Serre trace map.
\end{thm}
\todo{proof, mention Grothendieck trace map}
This map comes from the composition of the following maps:
\begin{equation*}
    F^{\bullet}\textbf{DR}_{c/\mathcal{F}}(X \times \mathcal{F}) \to F^{\bullet}\textbf{DR}(\mathcal{F}) \otimes \Gamma(X,\mathcal{O}_X) \to F^{\bullet}\textbf{DR}(\mathcal{F}) \otimes \Gamma(X, \Omega^d_X) \to F^{\bullet}\textbf{DR}(\mathcal{F}) \otimes k[-d]
\end{equation*}
By assumption this map factors through the space of global sections with compact support, denoted by $F^{\bullet}\textbf{DR}(\mathcal{F}) \otimes \Gamma_c(X, \mathcal{O}_X)$. Now the issue is the existence of relative version of Grothendieck-Serre trace map. \todo{reference}
\begin{thm}\label{mainthm}
    Suppose that $X$ is a variety and that $\mathcal{L}=\mathcal{O}_X[+d] \in \Pic^{gr}X$ is trivial line in grading $d \neq 0$. Let $\mathfrak{U} \subset \Perf^{\mathcal{L}}(X)$ be an oper sub-stack statisfying the following properness condition:
    \begin{center}
    For any ring $R$ and $R$-point $\eta: \Spec R \to \mathfrak{U}$, let $\mathcal{F} \in \Perf(X_R)$ be the perfect complex classified by $\eta$. Then, we require that the cone of the trace map of sheaves on $X_R:=X \times \Spec R$
    \begin{equation*}
        \mathcal{RH}om_{X_R}(\mathcal{F}, \mathcal{F}) \xrightarrow{tr} \mathcal{O}_{X_R}
    \end{equation*}
    have support proper over $\Spec R$. 
    \end{center}
    Then, $\mathfrak{U} \subset \Perf^{\mathcal{L}}(X)$ carries a $(2-d)$ shifted symplectic structure. Furthermore, this is natural for open inclusion of substacks satisfying the above condition.
\end{thm}
Now we get our motivating examples.
\begin{cor}
    Suppose that $X$ is a (not necessarily compact) 3-CY variety and $\mathcal{L}=\mathcal{O}_X[1+]$. Let $\mathfrak{U} \subset \Perf^{\mathcal{L}}(X)$ be the locus classifying ideal sheaves of proper subvarieties. In our case $\mathfrak{U}=I(X)$. Then, $\mathfrak{U}$ admits $(-1)$ shifted symplectic structure.
\end{cor}
\begin{proof}
    It is enough to show that the condition holds. Take $\mathcal{E} \in \Perf(X_R)$ with an identification $\det(\mathcal{E}) \cong \mathcal{O}_X[+1]$ corresponding to a point $\eta: \Spec R \to \mathfrak{U}$. The natural map 
    \begin{equation*}
        \mathcal{E} \to \mathcal{E}^{\vee\vee} \cong (\det(\mathcal{E})[-1] \cong \mathcal{O}_X
    \end{equation*}
    exhibits it as an ideal sheaf with the cone having proper support by assumption. Since this map is the same with the trace map 
    \begin{equation*}
        tr: \RHom(\mathcal{E}, \mathcal{E}) \to \mathcal{O}_X
    \end{equation*}
    is isomorphism away from this proper support as well.
\end{proof}
\begin{cor}
    Under the same assumption, consider a derived moduli of stable pairs $P(X) \subset \Perf^{\mathcal{L}}(X)$. It admits $(-1)$ shifted symplectic structure as well. 
\end{cor}
\begin{proof}
    In this case, the trace map is an isomorphism.\todo{finish}
    Let $I^{\bullet}=[\mathcal{O}_X \xrightarrow{s} F]$ be a stable pair in $\Perf^{\mathcal{L}}$. We have the following exact triangles associated to $I^{\bullet}$:
    \begin{equation}\label{exacttriangle}
        F[-1] \to I^{\bullet} \to \mathcal{O}_X \xrightarrow{s} F \to \cdots
    \end{equation}
    Applying $\mathcal{H}om(-,\mathcal{O}_X)$ to \ref{exacttriangle} yields 
    \begin{equation*}
        \mathcal{H}om(F,\mathcal{O}_X) \to \mathcal{H}om(\mathcal{O}_X,\mathcal{O}_X) \to \mathcal{H}om(I^{\bullet},\mathcal{O}_X) \to \mathcal{E}xt^1(F,\mathcal{O}_X)
    \end{equation*}
    The first and last term vanish because $F$ has support of codimension 2. The canonical map $I^{\bullet} \to \mathcal{O}_X$ generates the third term, so we get $\mathcal{H}om(I^{\bullet},\mathcal{O}_X) \cong \mathcal{O}_X$. In fact, it is the image of the identity in the exact sequence 
    \begin{equation*}
        \mathcal{E}xt^{-1}(I^{\bullet},F) \to \mathcal{H}om(I^{\bullet},I^{\bullet}) \to \mathcal{H}om(I^{\bullet},\mathcal{O}_X)
    \end{equation*}
    obtained from \ref{exacttriangle} by applying $\mathcal{H}om(I^{\bullet},-)$. Therefore, in order to show $\mathcal{H}om(I^{\bullet},I^{\bullet} \cong \mathcal{O}_X$, we need only prove the vanishing of $\mathcal{E}xt^{-1}(I^{\bullet},F)$. But, it turns out to be $\mathcal{H}om(\Coker(s),F)$ which vanishes due to purity of $F$.
\end{proof}
\begin{proof}[proof of \ref{mainthm}]
    We divide this into four steps.
    \begin{enumerate}
        \item Pull back the universal form from $\Perf$ to $X \times \mathfrak{U}$\\
        Consider the followng sequence of derived stacks
        \begin{equation*}
            X_{\mathfrak{U}}=X \times \mathfrak{U} \xrightarrow{j} X \times \Perf^{\mathcal{L}}(X) \xrightarrow{i} X \times \Perf(X) \xrightarrow{ev} \Perf
        \end{equation*}
        Let $\mathcal{E}$ be the universal perfect complex and 
        \begin{equation*}
            \omega_{\Perf} = ch(\mathcal{E})_2 \in H^0(F^2\textbf{DR}(\Perf)[2])
        \end{equation*}
        be the 2-shifted symplectic form constructed in PTVV. Since pullback on derived de Rham complexes commutes with filtration, we obtain a class 
        \begin{equation*}
            \omega_{X \times \mathfrak{U}}=j^*i^*ev^*(\omega_{\Perf}) =ch(j^*i^*ev^*\mathcal{E})_2 \in H^0(F^2\textbf{DR}(X \times \mathfrak{U})[2])
        \end{equation*}
        The last assertion follows from the functoriality of Chern character.
        \item Lift the form to compactely supported cochain level over $\mathfrak{U}$.\\
        Let $\mathcal{F}=j^*i^*ev^*\mathcal{E} \in \Perf(X \times \mathfrak{U})$ and $K \subset X \times \mathfrak{U}$ be the support of the cone of the trace map given by
        \begin{equation*}
            \mathcal{RH}om_{X \times \mathfrak{U}}(\mathcal{F}, \mathcal{F}) \xrightarrow{tr} \mathcal{O}_{X \times \mathfrak{U}}
        \end{equation*}The assumption require $K$ to be proper over $\mathfrak{U}$. By setting $V=X \times \mathfrak{U} \setminus K$, we claim that the restriction of the symplectic form $\omega_{X \times \mathfrak{U}}|_V$ vanishes. On $V$, trace map 
        \begin{equation*}
            tr: \RHom_V(\mathcal{F}|_V, \mathcal{F}|_V) \xrightarrow{\cong} \mathcal{O}_V
        \end{equation*}
        is an isomorphism so that we get the natural isomorphism 
        \begin{equation*}
            \mathcal{F}|_V \cong (\det \mathcal{F})|_V \cong \det(\mathcal{F|_V}) \cong \mathcal{O}_V[+d]
        \end{equation*}
        Hence, 
        \begin{equation*}
            \omega_{X \times \mathfrak{U}}|_V = ch(\mathcal{F}|_V)_2=ch(\mathcal{O}_V[+d])_2=0
        \end{equation*}
        \item Integration along the fiber\\
        Applying \ref{intalongfiber} we can integrate $\omega_{X \times \mathfrak{U}} \in H^0(F^2\textbf{DR}_{c/\mathfrak{U}}(X \times \mathfrak{U})[+2]$ to obtain
        \begin{equation*}
          \omega_\mathfrak{U}=\int_{[X]}\omega_{X \times \mathfrak{U}} \in H^0(F^2\textbf{DR}(\mathfrak{U})[2-d])
        \end{equation*}
        \item $\omega_{\mathfrak{U}}$ is non-degenerate.\\
        Fix an $R$-point classifying a perfect complex $\mathcal{E} \in \Perf(X_R)$ with trivial determinant; the tangent space at this point is 
        \begin{equation*}
            \RHom_{X_R}(\mathcal{E}, \mathcal{E})_0=\fib\{tr: \RHom_{X_R}(\mathcal{E}, \mathcal{E}) \to \Gamma(X_R, O_{X_R})\}
        \end{equation*}
        At a sheaf level, we can define $\mathcal{RH}om_{X_R}(\mathcal{E}, \mathcal{E})_0$ and by assumption, it also admits support proper over $R$. The 2-form $\omega_{\mathcal{U}}$ at this point is nothing but a composition map
        \begin{equation*}
            \RHom_{X_R}(\mathcal{E}, \mathcal{E})_0^{\otimes 2} \to \RHom_{X_R}(\mathcal{E}, \mathcal{E})^{\otimes 2} \to \RHom_{X_R}(\mathcal{E}, \mathcal{E}) \to \Gamma(X_R, \mathcal{O}_{X_R})
        \end{equation*}
        By assumption, it factors through $\Gamma_c(X_R, \mathcal{O}_{X_R})$. Now non-degeneracy of $\omega_{\mathfrak{U}}$ follows from the property of the relative Grothendieck-Serre trace map.
    \end{enumerate}
\end{proof}

\chapter{An Introduction to Pardon's approach to Virtual Fundamental Classes}
\label{chap: pardon}

\section{Atlases}

\section{Virtual Fundamental Chain}

\section{Applications to Gromov--Witten invariants and to Hamiltonian Floer Homology}
\chapter{Derived Manifold}
In this chapter, we introduce a notion of derived smooth manifold by Spivak. Since moduli spaces with Floer theoretic origins forms a smooth manifold or orbifold, it is inevitable to use a derived smooth manifold(more precisely, derived smooth stack if possible) to give a natural derived enhancement of such moduli spaces. \\

Like DAG, one of the crucial motivation is the intersection theory in the smooth manifolds. Classically, for a smooth manifold $M$ and submanifolds $A, B \subset M$, $A \cap B$ becomes a submanifold if they intersect transversally. This transversality condition can be ignored once we have a bigger category, which we will call $\textbf{dMan}$ where the intersection is well defined. 

\section{Construction of $\textbf{dMan}$}
Here, we use Lawvere Theory to construct $\textbf{dMan}$. The idea is to follow the analogue construction in Classical Algebraic Geometry. It would be helpful to keep the following correspendence.

\begin{align*}
    Ring &
    \Longleftrightarrow C^{\infty}-Ring \\
    Ringed space &
    \Longleftrightarrow  C^{\infty}-Ringed space \\
    Locally ringed space &
    \Longleftrightarrow 
    Locally C^{\infty}-ringed space \\
    (Affine) Scheme &
    \Longleftrightarrow 
    (Affine) Derived scheme
\end{align*}

We first recall the Lawvere Theory. Take a category of affine spaces with products $T^{disc}=\{\mathbb{A}^i_k | \text{with products}, i \in \N\}$ where $k$ is a field(of characteristic 0). We can identify a category of rings with product preserving functor from $T^{disc}$ to a category of sets. Similarly, we can describe a category of simplicial rings $sRing$ by replacing lax functor to $sSet$. Then, we get a category of locally ringed space $LRS$ whose object is of form $(X, \mathcal{O_X}$ where $\mathcal{O_X} \in Shv(X, sRing)$.

\begin{defin}
    Let $E \subset Man$ be a subcategory whose objects are Euclidean spaces and morphisms are smooth maps. 
    \begin{enumerate}
        \item We define a category of $C^{\infty}$ rings,  $C^{\infty}-Rings$ to be $\Fun^{\times}(E, Set)$
        \item Similarly, $sC^{\infty} := \Fun^{\otimes}(E, sSet)$ a category of lax monoidal functor from $E$ to $sSet$. 
        \item Let $E^{alg} \subset E$ be a full-subcategory whose morphisms are algebraic maps. Then, there exists a natural functor $U:sC^{\infty} \to \Fun^{\otimes}(E^{alg}, sSet)$. We say $C^{\infty}$ is a local if its underlying discrete $\R$-algebra $\pi_0(U(F))$ is local in usual sense. 
        \item We also construct a $\infty$ category of sheaf of $C^{\infty})$-rings over $X$, $shv(X,sC^{\infty})$ where $X$ is compactly generated topological space. $F \in shv(X,sC^{\infty})$ is called sheaf of $\textbf{local} C^{\infty}$-ring if for every point $p:* \to X$, the stalk $p^*F$ is local $C^{\infty}$-ring.
    \end{enumerate}
\end{defin}

Let's denote $C^{\infty}LRS$ by a $\inf$-category of pairs whose object is $(X, \cO_X)$ where $X$ is (compactly generated Hausdorff)topological space and $\cO_X$ is a sheaf of local $C^{\infty}$-ring over $X$. Morphism between $(X, \cO_X), (Y, \cO_Y)$ is given as follows.
\begin{equation*}
    Map_{C^{\infty}LRS}((X, \cO_X),(Y, \cO_Y)):= \coprod_{f \in Hom(X,Y)}Map_{sC^{\infty}}(f^*\cO_Y, \cO_X)
\end{equation*}

\begin{prop}\label{emb}
    There exists a fully faithful embedding $i:Man \to C^{\infty}LRS$ which sends $M$ to $(M, C^{\infty}_M)$ where $C^{\infty}_M(U)(\R^n):=Hom_{Man}(U, \R^n) \in sSet$
\end{prop}

\begin{rem}
    Gluing is possible in $C^{\infty}LRS$ as is in $LRS$. 
\end{rem}

The following proposition gives a notion of global section in $C^{\infty}LRS$.

\begin{prop}
    Let $(X, \cO_X)$ be a local $C^{\infty}$-ringed space. There exists an isomorphism between simplicial sets; 
    \begin{equation*}
        Map_{C^{\infty}LRS}((X, \cO_X), (\R, C^{\infty}_{\R})) \cong |\cO_X(X)|:=\cO_X(X)(\R)
    \end{equation*}
\end{prop}

\begin{defin}
    \begin{enumerate}
    \item $\mathcal{X}=(X, \cO_X)$ is an affine derived maniofld if it is given by vanishing locus of a smooth map.
    \item $\mathcal{X}$ is a derived (smooth) scheme if there exists open cover $\{U_i\}$ of $X$ such that $(U_i, \cO_X|_{U_i})$ is an affine derived manifold.
    \item We define $\textbf{dMan}$ a $\infty$ category of derived manifold as a subcategory of $C^{\infty}LRS$
    \end{enumerate}
\end{defin}

\begin{thm}
    The canonical embedding $i$ in \ref{emb} factors through $dMan$.
\end{thm}

\section{Properties of $\textbf{dMan}$}
Here, we collect some of properties $\textbf{dMan}$ satisfies. The full lists are in Spivak's paper.\\
As mentioned in the introduction, the main goal for constructing $\textbf{dMan}$ is to do intersection theory without transversality condition. Since $\textbf{dMan}$ have a (homotopy)pullback, we can define $A \cap B$ to be a homotopy pullback of $i(A)$ and $i(B)$ where $A, B$ are submanifolds of a given manifold, say $M$ and $i$ is a canonical embedding in \ref{emb}. Moreover, we require it to have a notion of "derived" cobordism ring $\Omega^{der}$ associated to each objects which is isomorphic to ordinary cobordism ring$\Omega$ of underlying object. The motivation is that we need address where virutal fundamental class lives. Precisely, For each manifold $T$, there exists a ring $\Omega^{der}(T)$ where $i:Man \to \textbf{dMan}$ induces a homomorphism of cobordism rings over $T$, $i_*\Omega(T) \to \Omega^{der}(T)$. In particular, it is an isomorphism.\\

We first give some notation. Let $\Top$ be a ($\infty$)-category of compactly generated Hausdorff spaces. There exists a natural forgetful functor $U:\textbf{dMan} \to \Top$.

\begin{defin}
    \begin{enumerate}
        \item Suppose that $\mathcal{X} \in \textbf{dMan}$ is an object with underlying space $X=U(\mathcal{X})$ and $j:V \to X$ is an open inclusion. We say a map $k:\cV \to \mathcal{X}$ in $\textbf{dMan}$ is an open subobject over $j$ if it is Cartesian over j.
        \item An object $\cU \in \textbf{dMan}$  is a local model if there exists a smooth function $f:\R^n \to \R^k$ such that $\cU \cong \R^n_{f=0}$. The virtual dimension of $\cU$ is $n-k$. 
        \item For any map $f:\mathcal{Y} \to \R^n \in \textbf{dMan}$, the canonical inclusion of the zeroset $\mathcal{Y}_{f=0} \to \mathcal{Y}$ is called a model imbedding. 
        \item A map $g:\mathcal{X} \to \mathcal{Y}$ is called an embedding if there is a cover of $\mathcal{Y}$ by open subobjects $\mathcal{Y}_i$ such that if the induced maps $g|_{\mathcal{X}_i}:\mathcal{X}_i \to \mathcal{Y}_i$ are model embeddings where $\mathcal{X}_i=g^{-1}(0)$. 
    \end{enumerate}
\end{defin}

\begin{prop}\label{properties}
    \begin{enumerate}
        \item ((Reasonable)Finite limits. Given objects $\mathcal{X}, \mathcal{Y} \in \textbf{dMan}$, a smooth manifold $M$ with $f:\mathcal{X} \to M, g:\mathcal{Y} \to M$, there exists an object $\mathcal{Z} \in \textbf{dMan}$ which is given by homotopy pullback.
        \todo{need proof. 9.15}
        \item (Local models) For any object $\mathcal{X} \in \textbf{dMan}$, underlying space $X=U(\mathcal{X})$ can be covered by open subsets in such a way that the corresponding open subobjects are all local models. More generally, any open cover of $U(\mathcal{X})$ can be refined to an open cover whose corresponding open subobjects are local models.
        \item (Normal bundle) Let $M$ be a smooth manifold and $\mathcal{X} \in \textbf{dMan}$ with underlying space $X=U(\mathcal{X})$. If $g:\mathcal{X} \to M$ is an embedding, then there exists an open neighborhood $U \subset M$ of $\mathcal{X}$, a real vector bundle $E \to U$, and a section $s:U \to E$ such that the following diagram commutes up to homotopy.
        \[
    \begin{tikzcd}
    \mathcal{X} \arrow{r}{g}\arrow{d}{g} & U\arrow{d}{s} \\
    U \arrow{r}{z} & E.
    \end{tikzcd}
    \]
        where $z$ is a zero section.
    \end{enumerate}
\end{prop}
\begin{rem}
\begin{itemize}
    \item This proposition tells us about Kuranishi structure of an derived manifold. For given $\mathcal{X} \in \textbf{dMan}$, we get an open cover which are zero-sets of vector space. This might allows to build $\inf$-categorical formalism of virtual fundamental class by John Parden.
    \item We intenionally put the word "reasonable" in 1 in \ref{properties}. This is because, there does not exists a finite limit in $\textbf{dMan}$. We can also construct a slightly bigger category which is universal by requiring it to have certain type of limit(e.g. finite limit). We will come back to this point in the next section.
\end{itemize}
\end{rem}

Let's switch the gear to cotangent complex and quasi-smoothness of $\textbf{dMan}$. Given $f:\mathcal{X} \to \mathcal{Y}$ in $\textbf{dMan}$, as a morphism of $C^{\inf}$-ringed spaces, we can construct cotangent complex of $f$. On the other hand, we can do the same game by regarding $f$ as a morphism of underlying ringed spaces. These two cotangent complexes are not quite the same. The former one enjoys more natural properties but difficult to compute and we need more set up to define it. However, the latter one, which we call $\mathbb{L}_f$, is easier to compute and be defined. Fortunately, it has all the properties we need. Since definitions and properties of cotangent complexes are rather tautological, we skip this part.\\

\begin{defin}
    Let $\mathcal{X}=(X, \cO_X)$ be a derived manifold and $\mathbb{L}_{\mathcal{X}}$ its cotangent complex. The pointwise Euler characteristic $e:X \to \Z$ is a function whose value at a point $x \in X$ is the Euler characteristic of $\mathbb{L}_{\mathcal{X}, x}$. We also call it a virtual dimension of $\mathcal{X}$ at $x$, denoted by $dim_x(\mathcal{X})$.
\end{defin}
\begin{prop}
    $e:X \to \Z$ is continuous(i.e. locally constant), and for all $i \geq 2$, we have $H^i(\mathbb{L}_{\mathcal{X}})=0$. 
\end{prop}
\begin{proof}
    Without loss of generality, we assume that $\mathcal{X}$ is an affine derived manifold: i.e. there is a homotopy limit squre of the form 
    \[
    \begin{tikzcd}
    \mathcal{X} \arrow{r}{t}\arrow{d}{i} & \R^0 \arrow{d}{0} \\
    \R^n \arrow{r}{f} & \R^k.
    \end{tikzcd}
    \]
    Note that $\mathbb{L}_{\mathcal{X}}:=\mathbb{L}_t$ are $\mathbb{L}_f$ are weak equivalent. It suffices to show that $\mathbb{L}_f$ is constant on $\mathcal{X}$. \\
    The composable pair of morphisms $\R^n \rightarrow{f} \R^k \rightarrow{0} \R^0$ induces an exact triangle
    \[ f^* \mathbb{L}_{\R^k} \to \mathbb{L}_{R^n} \to \mathbb{L}_f \]
    By taking cohomology, we get
    \[ 0 \to H^{-1}(\mathbb{L}_f) \to \R^k \to \R^n \to H^0(\mathbb{L}_f) \to 0 \]
    Both assertions follow from this sequence. In particular, $\text{rank}(H^0(\mathbb{L}_f)) - \text{rank}(H^{-1}(\mathbb{L}_f))=n-k$ at all points in $\mathcal{X}$.
    
\end{proof}

This proposition tells that every object in $\textbf{dMan}$ is quasi-smooth. However, not every morphism is quasi-smooth in general. 

\begin{rem}
    As mentioned earlier, $\textbf{dMan}$ does not have finite limit. We can define more general category $\C$ which contains $\textbf{dMan}$ as a full subcategory. One might argue that it is the same as category of quasi-smooth objects in $\C$. 
\end{rem}

\section{Universal construction of category of derived manifold}
\chapter{VFC in Derived Differential/Algebraic Geometry}
\label{ch:infinity}

Talk by Benedict Morrissey.

\section{Big Picture}
We construct a functor $\dMan \to \Cat_{\infty}$, which associates to every derived manifold $M$ its category of Kuranishi
neighborhoods, i.e. pullback diagrams of the form:
\[
\begin{tikzcd}
U \arrow{r}\arrow{d} & \bbA^n \arrow{d}{f} \\
\pt \arrow{r} & \bbA^m .
\end{tikzcd}
\]
Here $U$ is an open chart of $M$ which can be locally written as $f^{-1}(0)$.

Then we construct the following chain of functors:
\[
\begin{tikzcd}
\KurNbd_M\arrow{r}\arrow[bend left]{rr}{F} & \LocTopInf\arrow{r} & \Map ( I, \dgVect ) .
\end{tikzcd}
\]
We define:
\[	\{C^{\vir}_{\bullet}(M) \to C^{\loc}_{\bullet}(M)\} := \colim F .\]
$I = \Delta^1$ is the $\infty$-category with 2 objects and 1 non-trivial 1-morphism.
\todo{Wasn't sure how to write the construction in a way that's manifestly functorial in $M$. Benedict's formulation uses diagrams
in $\LocTopInf$, $\Map(I,\dgVect)$. However the shape of the diagram depends on $M$, so I don't know how to write
this as a functor from $\dMan$.}

Now there's a map $C^{\loc}_{\bullet}(M) \to \Z$ \todo{is it a quasi-isomorphism?}. We define the virtual
fundamental class $[M]^{\vir}$ as the composition $C^{\vir}_{\bullet}(M)\to \Z$. (But see Remark \ref{rem:poincare}.)



\section{Construction}
\begin{defin}
The category of \textbf{Kuranishi neighborhoods} of $M$ is:
\[	\KurNbd_M = \dMan_{/M} \times_{\dMan} \Map(\Box, \dMan) \times_{\Map(\_, \dMan)} \Map(\pt, \Tdisc) 
\times_{\Map(\cdot, \dMan)} \Tdisc .	\]
\end{defin}

\todo{We need to define these squares and intervals precisely, with direction of arrows. Also, we need to quote the relevant
strictification results which give the higher homotopies for free once we specify data for the vertices and edges of the squares.}

Note that $M$ only appears in the leftmost factor, $\dMan_{/M}$.
Therefore the assignment $M \mapsto \KurNbd_M$ is an $\infty$-functor,
which follows from the fact that $M \mapsto \dMan_{/M}$ and fiber product are $\infty$-functors.

\begin{defin}
The category of \textbf{local topological information} $\LocTopInf$ is actually $\Map(\Box, \cS)$, which morally speaking
consists of maps between pairs of topological spaces $(A,B) \to (C,D)$. Here ``pairs'' is used in the sense of algebraic
topology, i.e. $B\to A$ and $D \to C$ are cofibrations.
\end{defin}

We construct a functor $\KurNbd_M \to \LocTopInf$ by sending each Kuranishi atlas to the map of pairs:
\[
\begin{tikzcd}
\big( U(\bbA^n), U(\bbA^n \setminus f^{-1}(0))\big)\arrow{r}{f} & \big( U(\bbA^m), U(\bbA^m \setminus \{0\})\big).
\end{tikzcd}
\]
Here $U: \dMan \to \cS$ is the forgetful functor; this will exist in whatever category we end up working in.

Finally, we construct a functor $\LocTopInf \to \Hom(I, \dgVect)$ by sending the map of pairs $f: (A,B) \to (C,D)$
to the induced morphism on singular chains:
\[
\begin{tikzcd}
C_{\bullet}(A,B) \arrow{r}{f_*} & C_{\bullet}(C,D) .
\end{tikzcd}
\]
\todo{We actually want $n-\bullet$ in the first factor and $m-\bullet$ in the second. How do we obtain the shifts by different
amounts? Look at how Pardon does it.}

\begin{lem}
This is an $\infty$-functor, which moreover preserves all colimits.
\end{lem}
\begin{proof}
In brief, $C_{\bullet}$ is the unique functor $\cS \to \dgVect$ which preserves colimits and sends $\pt \mapsto k$. That's
because $\cS$ is generated under colimits by $\pt$. In the case of pairs of spaces, the functor of relative singular
chains is defined by:
\[
\begin{tikzcd}
\Fun(I, \cS) \arrow{r}{C_{\bullet}} & \Fun(I, \dgVect) \arrow{r}{\text{cone}} & \dgVect .
\end{tikzcd}
\]
Here $C_{\bullet}$ is computed pointwise. Since $C_{\bullet}$ and $\text{cone}$ are $\infty$-functors, so are relative chains.
Moreover, colimits of presheaves can be computed pointwise, so relative chains also preserve colimits.
\end{proof}

\begin{rem}
This construction actually gives a cosheaf on $M$. That's because, for $V \to M$ a chart, $C_{\bullet}^{\vir}(V)$ will be computed
by a colimit over the category of Kuranishi neighborhoods which factor through $V$. This colimit maps naturally to the
colimit over $\KurNbd_M$, which computes $C_{\bullet}^{\vir}(M)$.
\end{rem}
\todo{Actually we need to prove that it satisfies descent.}


\section{Recovery of the colimit from finite atlas}
\begin{conj}
Assume that $M \in \dMan$ admits a Kuranishi atlas consisting of a single chart:
\[
\begin{tikzcd}
M\arrow{r}{i_M}\arrow{d} & \bbA^n\arrow{d}{f} \\ \pt \arrow{r} & \bbA^m .
\end{tikzcd}
\]
This chart is a final object in $\KurNbd_M$, and as such we have $C^{\vir}_{\bullet}(M) = C_{\bullet}\big(U(\bbA^n), 
U(\bbA^n \setminus f^{-1}(0)) \big)$ and $C^{\loc}_{\bullet}(M) = C_{\bullet}\big(U(\bbA^m), 
U(\bbA^m \setminus \{0\}) \big)$.
\end{conj}
\begin{proof}[Attempt at Proof -- Does not work]
Consider any other chart, given by $p: U \to M$ and:
\[
\begin{tikzcd}
U\arrow{r}{i_U}\arrow{d} & \bbA^{n_U}\arrow{d}{f_U} \\ \pt \arrow{r} & \bbA^{m_U} .
\end{tikzcd}
\]
We need to produce a map between the two charts. The idea should be to consider:
\[
\begin{tikzcd}
U\arrow{r}{(i_U,i_M \circ p)}\arrow{d} & \bbA^{n_U+n}\arrow{d}{f_U,f} \\ \pt \arrow{r} & \bbA^{m_U+m} ,
\end{tikzcd}
\]
and then project to the $\bbA^n$ and $\bbA^m$ factors. Unfortunately the composition is 0 on the rightmost column.
\todo{Work this out.}
\end{proof}

In general, $\KurNbd_M$ is a category which doesn't admit a final object. However, there exists a final system of objects
 - i.e. finitely many charts $U_i$ which cover the entire $M$. We should prove that the colimit of $F$ over the entire $\KurNbd_M$
can be recovered from the value of $F$ on these objects.



\section{To do}
To do ASAP:
\begin{enumerate}
\item In the Big picture section, understand how to write the VFC as a functor from $\dMan$.
\item Prove that we can recover the colimit over $\KurNbd_M$ from a finite atlas.
\item Fill in more details, as outlined in the to do notes. What else needs to be filled in?
\end{enumerate}

Then:

\begin{rem}
\label{rem:poincare}
So far we only defined $[M]^{\vir}$ as a map $C^{\vir}_{\bullet}(M)\to \Z$. We would like it to be an element in a
homology group of $M$, so that we can use it to integrate cohomology classes. John Pardon achieves this via
a Poincar\'e duality type quasi-isomorphism $H^{\vir}_{\bullet}(M) \simeq \big(\check{H}^{\bullet}(M)\big)^{\vee}$.
See the results of Section 4.3 in \cite{Pardon_An_algebraic_approach_2016}.
Can we prove this in our setting?
\end{rem}

Finally, the medium-term things which are more or less logically independent from the material of this chapter:
\begin{enumerate}
\item Define the appropriate category to replace $\dMan$ in all the above. This should:
\begin{itemize}
\item be small enough so that every object is quasi-smooth;
\item be large enough to accomodate certain mapping stacks, such as the moduli space of stable maps.
\end{itemize}
It suffices to construct a theory of derived DM stacks in smooth manifolds. We don't need Artin stacks, or indeed want them
- it's not clear that they admit VFCs, since nobody has come up with a theory that achieves this.
\item In this context, we need some version of Artin-Lurie representability to hold. This would guarantee the geometricity of
the mapping stacks we define.
\item Generalize to manifolds with boundary, maybe even corners.
\item Now we should be able to define Floer moduli spaces as objects in this category.
\item World peace?
\end{enumerate}

\chapter{Pregeometries and Geometries}
\label{ch:geometries}

Talk by Matei Ionita



\begin{center}
\begin{tabular}{ c c c }
  & $dMan_{C^{\infty}}$ & $dDM_{C^{\infty}}$ \\ 
 Pre-geometry & $T_{disc}=\{\bbR^{n}, \mathrm{smooth maps}\}$ & $T_{disc}, T_{etale}$?? \\  
 Geometric Envelope & $sC^{\infty}=Fun^{\otimes}(T_{disc}, sSets)$ & cell9   \\
 & $(sC^{\infty})^{op}=C^{\infty}$-affines & \\
 & Restrict to finitely presented objects $(sC^{\infty, fp})^{op}$, only finite limits &\\
 & Restrict to quasismooth objects $(sC^{\infty, qs})^{op}$, objects are fiber products &
\end{tabular}
\end{center}




Structured Topoi/spaces:L

\[LRS(sC^{\infty}):=Fun^{lex}((sC^{\infty})^{op}, \overline{S})\times_{Fun^{lex}((sC^{\infty})^{op},S)}S\]

Here $\overline{S}$ is given by the Grothendieck construction applied to the Cartesiion firbation $S\rightarrow Cat_{\infty}$, $X\mapsto Sh_{S(X)}$.  We should see this as giving an embedding of topological spaces into Topoi, the fiber of $\overline{S}\rightarrow S$ over a topological space $X$ is the topos associated to the topological space $X$.

So objects of $LRS(sC^{\infty})$ are pairs $(X,\cO_{X})$, with $X$ a space and $\cO_{X}:(sC^{\infty})^{op}\rightarrow Sh_{S}(X)$.
Here $\cO_{X}$ is a sheaf of $C^{\infty}$ rings on $X$ \todo{explain this is equivalent data to the obvious definition}.

Geometric objects:

There is an underlying space functor
\[U:(sC^{\infty})^{op}\rightarrow S\]

Yoneda Embedding:

\[T_{disc}\rightarrow (sC^{\infty})^{op}\rightarrow LRS(sC^{\infty})\]

\todo{work out how the second map works}

The underlying space functor is the composition of the above with the forgetful map $LRS(sC^{\infty})\rightarrow S$.

Let us now consider some hypercover $(U_{S})_{S\subset I}$ of a topological space $X$, together with compatible lifts $\tilde{U_{S}}_{S\subset I}$ in $(sC^{\infty})^{op}$.  We call this a scheme atlas for $X$.

We can take the colimit of $\tilde{U_{S}}_{S\subset I}$ in $LRS(sC^{\infty})$, to give a derived scheme in the sense of Macpherson.

Macpherson defines

\[dSch_{C^{\infty}}=\{(X, \tilde{U_{S}}_{S\subset I}), X\in S, \tilde{U_{S}}_{S\subset I} a scheme atlas\}\]

\[dMan_{C^{\infty}}\subset dSch_{C^{\infty}}\]

is the subset generated by objects where $X$ is second countable and Hausdorff.

\begin{prop}
$dMan^{qs}_{C^{\infty}}$ coincides with Spivaks construction.
\end{prop}

We now wish to describe Deligne--Mumford stacks.  We replace $\overline{S}$ by a category $^{L}Top$ of infinity topoi, and again apply the Grothendieck construction to gain $\overline{^{L}Top}$ -- intuitively objects here are pairs $(\cX,\cO_{\cX})$, where $\cX$ is a topos, and $\cO_{\cX}$ is a sheaf on the topos $\cX$.  

\[^{L}Top(sC^{\infty})\subset Fun^{lex}((sC^{\infty})^{op}, \overline{^{L}Top})\times_{Fun^{lex}((sC^{\infty})^{op},^{L}Top)}^{L}Top\]

where the subcategory is defined by the subset of objects satisfying \todo{TO BE SPECIFIED, localic?} conditions.

As before we have a forgetful map to the underlying topos $U:(sC^{\infty})^{op}\rightarrow ^{L}Top$ (remark, this factors through the previous underlying space map).

We can now define a "stacky?" atlas, by giving a hypercovering (of etale\todo{???} maps) of some topos $\cX$, with the property that we lift the hypercover to the category $(sC^{\infty})^{op}$.  We then take the colimit of this category in the category $^{L}Top(sC^{\infty})$.







\bibliographystyle{plain}
\bibliography{dahema}

\end{document}




