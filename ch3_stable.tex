\chapter{Stable $\infty$-categories}
Talk by Michael Gerapetritis.

\section{Motivation}
\label{sect:motivation}

In the 1-categorical setting, if $\cC$ is a category, we may require that $\cC(A,B)$ be a set. 
To get particularly well-behaved categories, namely the additive categories, we require that $\cC(A,B)$ is actually
an abelian group. 

We try to replicate this in the $\infty$-category setting.
Let $\cC$ be an $\infty$-category, then $\cC(X,Y)$ is a space. We want to discover what is the good extra structure to have on 
this space; we will call the corresponding $\infty$-categories stable.


\section{Stable $\infty$-categories and triangulated 1-categories}

\begin{defin}
An $\infty$-category $\cC$ is \textbf{stable} if:
\begin{itemize}
\item $\cC$ is pointed, i.e. it has a zero object;
\item every morphism $f : X \to Y$ admits fibers and cofibers;
\item a triangle is a fiber iff it is a cofiber.
\end{itemize}
\end{defin}

Recall that a \textbf{triangle} in $\cC$ is a map of simplicial sets $\Delta^1 \times \Delta^1 \to \cC$,
i.e. a homotopy commutative diagram with the zero object in the bottom-left corner:
\[
\begin{tikzcd}
X\arrow{r}{f}\arrow{d} & Y\arrow{d}{g} \\
0\arrow{r} & Z
\end{tikzcd}
\]
The triangle is a \textbf{fiber} if it is a pullback square, and a \textbf{cofiber} if it is a pushout square. We say
that $f: X \to Y$ admits a fiber (resp. cofiber) when $\exists W$ (resp $Z$) such that:
\[
\begin{tikzcd}
W\arrow{r}\arrow{d} & X\arrow{d}{f} \\
0\arrow{r} & Y
\end{tikzcd}
\]
is a pullback square (or, respectively:
\[
\begin{tikzcd}
X\arrow{r}{f}\arrow{d} & Y\arrow{d} \\
0\arrow{r} & Z
\end{tikzcd}
\]
is a pushout square).

\begin{rem}
Note that the data of a triangle consists not only of homotopy commutative diagrams as above, but also of choices of 
homotopies between the branches. This is crucial, since it ensures that cones are
functorial at the level of the homotopy category. This functoriality does not hold in a general triangulated category. 
(See Theorem \ref{thm:stable_triang} for the relation between stable
$\infty$-categories and triangulated 1-categories.)
\end{rem}

\begin{eg}
Our two main examples are $\infty$-categories of spectra (see Section \ref{sect:spectra}) and of modules over a CDGA or SCR
(see Section \ref{sect:modules}).
\end{eg}

Recall the data for a triangulated category.

\begin{defin}
A category $\cD$ is triangulated if:
\begin{enumerate}
\item $\cD$ is additive;
\item \label{item:translation_equiv}
 $\cD$ admits a translation functor $T : \cD \overset{\simeq}{\to} \cD$;
\item $\cD$ has a collection of distinguished triangles:
\[
\begin{tikzcd}
X\arrow{r} & Y \arrow{r} & Z \arrow{r} & X[1]
\end{tikzcd}
\]
\end{enumerate}
This data is required to satisfy some axioms, but we won't go into details here.
\end{defin}

\begin{thm}
\label{thm:stable_triang}
If $\cC$ is a stable $\infty$-category, then $h\cC$ is triangulated.
\end{thm}

For a proof see \cite{Lurie_Higher_algebra}. We won't go over it, let's just say that translation is given by $\Sigma$,
and distinguished triangles are precisely the images of fiber sequences (or equivalently, cofiber sequences), as resulting
from the following diagram.
\[
\begin{tikzcd}
X\arrow{r}\arrow{d} & Y \arrow{r}\arrow{d} & 0\arrow{d} \\
0\arrow{r} & Z \arrow{r} & X[1]
\end{tikzcd}
\]

\begin{prop}
$\cC$ is stable iff the following hold:
\begin{enumerate}
\item $\cC$ admits finite limits and colimits;
\item any square is a pushout iff it is a pullback.
\end{enumerate}
\end{prop}
\begin{proof}
Again, we don't give a full proof. Let's just see why products and coproducts must exist in a stable $\infty$-category. 
Note first that $\Sigma$ is an equivalence of
$\infty$-categories. Indeed, $\Sigma$ is a left adjoint functor; moreover, the unit and counit of the adjunction become 
isomorphisms in the homotopy category, due to condition \ref{item:translation_equiv} in the definition of a triangulated
category. Then we use the following diagram.
\[
\begin{tikzcd}
\Omega(X)\arrow{d}\arrow{r} & 0\arrow{d}\arrow{r} & Y\arrow{d} \\
0\arrow{r} & X\arrow{r} & X \oplus Y
\end{tikzcd}
\]
We have defined $X \oplus Y$ as the cofiber of $\Omega(X) \overset{0}{\to} Y$, which is postulated to exist in a stable $\infty$
-category. This turns the outer rectangle into a pushout square, and it follows that the square on the right is also a pushout
square. Thus $X\oplus Y$ is the coproduct of $X$ and $Y$. We reason dually to obtain products.
\end{proof}

\begin{defin}
Let $\cC, \cC'$ be stable $\infty$-categories, and $F: \cC \to \cC'$ an $\infty$-functor which maps 0 objects to 0 objects. 
Equivalently, $F$ maps triangles to triangles.
If $F$ maps fiber sequences to fiber sequences, we say that $F$ is \textbf{exact}.
\end{defin}

\begin{lem}
\label{lem:exact}
TFAE:
\begin{enumerate}
\item $F$ is exact;
\item $F$ is right-exact, i.e. commutes with finite colimits;
\item $F$ is left-exact, i.e. commutes with finite limits.
\end{enumerate}
\end{lem}

This is very useful: sometimes it's really easy to check that a functor is right or left exact, e.g. if it's a left or right
adjoint, respectively.


\section{Modules}
\label{sect:modules}

For a useful example of the result in Lemma \ref{lem:exact}, we look at $\cC = A-\cM od$, where $A$ is a CDGA or SCR over $k$.
(By $A-\cM od$ we mean the unbounded derived category.) The easiest way
to see $A-\cM od$ as an $\infty$-category is to put a model structure on chain complexes, say the projective one, and then
take the underlying $\infty$-category.
We claim that $A-\cM od$ is a stable $\infty$-category. Using the theorem
Mauro talked about in Lecture 1, limits and colimits exists in the $\infty$-category iff they exist in the model category.
\todo{reference theorem} It remains to prove the following.

\begin{lem}
A triangle in $A-\cM od$ is a fiber iff it is a cofiber.
\end{lem}
\begin{proof}
We prove one direction; the other argument is dual to this one. Assume that $f:M^{\bullet} \to N^{\bullet}$ 
is the fiber of a map $g$. Take a cofibrant replacement of $f$, get $\tilde M, \tilde N$ cofibrant and a homotopy
pullback square:
\todo{figure out how to do the cartesian symbol in tikz}
\[
\begin{tikzcd}
\tilde M^{\bullet}\arrow[hook]{r}{\tilde f}\arrow{d} & \tilde N^{\bullet}\arrow{d}{\tilde g} \\
0\arrow{r}  & P^{\bullet} .
\end{tikzcd}
\]
$\tilde f$ is cofibrant, so it's a degree-wise injection. Then $g$ is a degreewise surjection, and it follows that the 
square is a strict pushout. \todo{wait, how did this work again?}
\end{proof}

Now suppose we have $f: A \to B$ a morphism of $CDGA_k^{\leq 0}$. It induces the adjunction of model categories:
\[
\begin{tikzcd}
A-Mod\arrow[shift left]{r}{f^*} & B-Mod\arrow[shift left]{l}{f_*},
\end{tikzcd}
\]
where $f_*$ is the forgetful functor, and $f^*(M) = M \otimes_A B$. So this gives an adjunction of $\infty$-categories:
\footnote{Here we use $L$ and $R$ to indicate that the functors are derived.
In later talks derived functors will be the default, and we will omit the symbols $L$ and $R$.}
\[
\begin{tikzcd}
\label{tikz:adjunction}
A-\cM od\arrow[shift left]{r}{Lf^*} & B-\cM od\arrow[shift left]{l}{Rf_*}.
\end{tikzcd}
\]
Explicitly, $Lf^*$ is constructed by first choosing a cofibrant replacement $\tilde M$ for $M$, 
and then taking $\tilde M \otimes_A B$. The
answer doesn't depend on cofibrant replacement, up to coherent isomorphism. Then $Lf^*$ is a left adjoint
functor, so it follows from general nonsense that it's right exact. Lemma \ref{lem:exact} then implies that $Lf^*$
is also left exact and exact.

\begin{rem}
\label{rem:exact_t}
If $f$ is not flat in the sense of Definition \ref{def:fav_morphisms}, then the exactness of
$Lf^*$ comes at the price of losing t-exactness. To explain what we mean, pick $M \in A-Mod$, such that
$H^i(M) = 0$ unless $i = 0$. But then $Lf^*(M) = M \otimes_A^{\bbL} B$, and $H^{-i}(M\otimes_A^{\bbL}B) = \Tor_i^A(M,B)$, which
is $\neq 0$ in general, because $f$ is not flat. So even though $M$ was homologically concentrated in degree 0, $Lf^*(M)$ may
not be. In other words, the failure of a functor of (Grothendieck) abelian categories to preserve limits translates into a lack 
of t-exactness of the derived functor. In the following section we define t-structures and t-exactness for $\infty$-categories.
\end{rem}


\section{t-structures}
\label{sect:t-struct}

\begin{defin}
If $\cC$ is a stable $\infty$-category, a \textbf{t-structure}\footnote{$t$ stands for truncation} on 
$\cC$ is the data of two full subcategories of $\cC$, $\cC^{\leq 0}$ and
$\cC^{\geq 0}$, \footnote{Note that we use cohomological notation, while Lurie in \cite{Lurie_Higher_algebra} uses homological
notation. Therefore gradings have opposite signs in this seminar and in \cite{Lurie_Higher_algebra}.} such that:

\begin{enumerate}
\item \label{item:no_morph_right}
$\pi_0 \Map_{\cC}(X,Y[-1]) = 0$ if $X \in \cC^{\leq 0}$ and $Y \in \cC^{\geq 0}$.
\footnote{In a stable $\infty$-category, we sometimes use the shift notation $[n]$ to denote the $|n|$-fold iterated application
of the $\Sigma$ functor (if $n$ is positive) or the $\Omega$ functor (if $n$ is negative). This notation is justified by
Proposition \ref{prop:stable_shift}.}

\item $X \in \cC^{\leq 0}, X[1]  \in \cC^{\leq 0}$;

\item \label{item:fiber_seq}
$\forall X$, $\exists$ fiber sequence $X' \to X \to X''$, where $X' \in \cC^{\leq 0}$, $X'' \in \cC^{\geq 1}$.
\end{enumerate}
\end{defin}

\begin{rem}
Condition \ref{item:no_morph_right} has the following intuitive meaning in the case $\cC = A-\cM od$. $0$-morphisms in 
$\cC$ are chain maps which preserve degree, while
higher morphisms are homotopies which shift the degree to the left; morphisms that shift degree to the right are not allowed.
Then, if $X \in \cC^{\leq 0}$ and $Y \in \cC^{\geq 0}$, no nonzero morphisms should be allowed between $X$ and $Y[-1]$:
\[
\begin{tikzcd}
\dots \arrow{r} & X_{-2} \arrow{r} & X_{-1} \arrow{r} & X_0 \arrow{r} & 0\arrow{r} & 0\arrow{r} & 0\arrow{r} & \dots \\
\dots \arrow{r} & 0 \arrow{r} & 0 \arrow{r} & 0 \arrow{r} & Y_0\arrow{r} & Y_1\arrow{r} & Y_2\arrow{r}  & \dots
\end{tikzcd}
\]
\end{rem}

\begin{rem}
$X'$ and $X''$ are uniquely determined by $X$.
\end{rem}

\begin{thm}
The inclusion $\cC^{\leq 0} \to \cC$ has a right adjoint, which we denote $\tau_{\leq 0} : \cC \to \cC^{\leq 0}$. Similarly
we get $\tau_{\geq 0} : \cC \to \cC^{\geq 0}$.
\end{thm}

\begin{cor}
For all $X \in \cC$, the fiber sequence of \ref{item:fiber_seq} is just:
\[	\tau_{\leq 0} X \to X \to \tau_{\geq 1} X.	\]
\end{cor}

\begin{prop}
Denote by $\cC^{\heartsuit} := \cC^{\leq 0} \cap \cC^{\geq 0}$, the \textbf{heart} or \textbf{core} of the t-structure. 
It is an abelian 1-category.
\end{prop}

\begin{prop}
\label{prop:stable_les_homology}
Let $\cC$ be stable. Then if:
\[	X \to Y \to Z	\]
is a fiber sequence, then we have a long exact sequence of $H^i$, where $H^i(X) := \tau_{\geq i} \circ \tau_{\leq i} (X)$.
\end{prop}

Putting the last few results together, from $\cC$ a presentable stable $\infty$-category with $t$-structure, the heart 
is Grothendieck abelian. Write $A = \cC^{\heartsuit}$. Then we can form $\cD(A)$, the $\infty$-derived category of $A$.
The next theorem describes the relationship between $\cC$ and $\cD(A)$.

\begin{thm}[Lurie]
$\cD(A)$ has a universal property which produces an $\infty$-functor:
\[	\cD(A) \to \cC.	\]
In general this is very far from being an equivalence.
\end{thm}

\begin{eg}
Let $A \in CDGA^{\leq 0}_k$. The theorem gives a map:
\begin{equation}
\label{eq:map_heart}
	(A-\cM od)^{\heartsuit} \to  (H^0(A)-Mod)^{\heartsuit}.
\end{equation}
This is one of the most important facts
in DAG, because it reduces problems about the $\infty$-category of $A$-modules to problems in classical categories of modules,
where one can work with generators and relations. The map in \ref{eq:map_heart} is an equivalence iff $A \simeq H^0(A)$ 
are quasi-isomorphic.\todo{figure out what's the precise relationship here}
\end{eg}

\begin{defin}
Let $\cC, \cD$ be stable $\infty$-categories with $t$-structures. Then an exact functor $F : \cC \to \cD$ is:
\begin{enumerate}
\item \textbf{left t-exact} if $F(\cC^{\leq 0}) \subset \cD^{\leq 0}$;
\item \textbf{right t-exact} if $F(\cC^{\geq 0}) \subset \cD^{\geq 0}$;
\item \textbf{t-exact} if both.
\end{enumerate}
\end{defin}

\begin{eg}
\label{eg:modules_t_exact}
For $A,B \in CDGA_k^{\leq 0}$, $f:A \to B$, we have the adjunction:
\[
\begin{tikzcd}
A-\cM od\arrow[shift left]{r}{Lf^*} & B-\cM od\arrow[shift left]{l}{Rf_*}.
\end{tikzcd}
\] 
Every object is fibrant, so we don't need to
derive the functors. $Rf_*$ is both left and right t-exact. $Lf^*$ is not right t-exact, because of nontrivial $\Tor^i$
terms; see \ref{rem:exact_t}. However, $Lf^*$ is right t-exact: 
morally speaking, Projective resolution only puts stuff in negative degrees. We give an $\infty$-categorical proof.

Pick $M \in A-Mod^{\geq 0}$. We want $Lf^*(M) \in B-Mod^{\leq 0}$. To check this is the same as checking that
$\forall N \in B-Mod^{\geq 1}$, $\Map_{B-Mod}(Lf^*M,N) \cong 0$. But this is $Map_{A-Mod}(M,Rf_*N) \cong 0$, which
follows since $Rf_*$ was t-exact.
\end{eg}


\section{Spectra}
\label{sect:spectra}

Going back to the question left unanswered in Section \ref{sect:motivation}, the extra structure we want on
morphism spaces of stable $\infty$-categories is $\Map_{\cC}(X,Y) \in \Sp^{\leq 0}$.

\begin{defin}
\textbf{Spectra} are sequences $\{F_i\}$ of objects in $\cC$ such that $F_n \simeq \Omega F_{n+1}$. Alternatively,
we identify them with objects of the homotopy limit:
\[	\dots \overset{\Omega}{\to} \cC \overset{\Omega}{\to} \cC \overset{\Omega}{\to} \dots	\]
\end{defin}

\begin{rem}
We must be careful with defining morphisms between spectra: we want squares to commute up to coherent homotopy.
Moreover, it's hard to get a monoidal model structure on the category of spectra: this was done only in the 2000s, after Hovey 
introduced symmetric spectra. Lurie has a very categorical and very nice way of putting a monoidal structure 
at the level of the $\infty$-category
directly. See the last chapter of \cite{groth}, and also 4.8.2 of \cite{Lurie_Higher_algebra}.
\end{rem}

\begin{thm}
$Sp(\cC)$ is stable.
\end{thm}

This gives a canonical stabilization for every $\infty$-category. The proof of the theorem follows from the following characterization
of stable $\infty$-categories,
and the fact that $\Omega:Sp(\cC) \to Sp(\cC)$ is an equivalence.

\begin{prop}
\label{prop:stable_shift}
$\cC$ is a pointed $\infty$-category. TFAE:
\begin{enumerate}
\item $\cC$ is stable;
\item $\cC$ admits colimits and $\Sigma : \cC \to \cC$ is an equivalence;
\item $\cC$ admits limits and $\Omega : \cC \to \cC$ is an equivalence;
\end{enumerate}
\end{prop}

%\begin{thm}
%Let $K$ be a simplicial set, and $F : K \to \cC at_{\infty}$. These correspond to cartesian fibrations, by HTT chapter 3,
%as explained by Mauro in the first lecture. We find a cartesian fibration $\mathscr{P} : \mathscr{X} \to K$.
%\todo{insert diagram}
%We have $\N \to \Cat_{\infty}$, we associate $\mathscr{X} \overset{\mathscr{P}}{\to} \N$. Look at section $s
%: \N \to \mathscr{X}$. $n \to m$ goes to:\todo{complete the diagram}
%\[
%\begin{tikzcd}
%s(n)\arrow{d}\arrow{r} & s(m)\arrow{d} \\
%n \arrow{r} & m
%\end{tikzcd}
%\]
%Thus we get a cocartesian fibration. Conclusion: let $\cD$ be the full subcategory of $\Map_{/K}(K,\mathscr{X})$ 
%spanned by cocartesian sections. Then
%$\cD$ is equivalent to $invlim_{\cC at_{\infty}} K$. In Theorem 3.3.3.1 of \cite{HTT}, Lurie has $\cD = \Map^{\flat}_{K^{\sharp}}
%(K^{\sharp}, \mathscr{X}^{fillthisinhere})$.
%\end{thm}

%Upshot: we need $sSet^+_{/K}$, where some collection of arrows is marked. We think of the marked ones as the ones which
%will become cocartesian.
%\[	sSet^+ = \{ (L,S) | L \in sSet, S\subset \Fun(\Delta^1, L), S \text{ contains the identities }\}	\]
%We have two functors $sSet \to sSet^+$. One is $K \mapsto K^{\flat} = (K, \text{identities})$. The other is
%$K \mapsto (K,\text{all morphisms}) = K^{\sharp}$. $\mathscr{P} : \mathscr{X} \to K$ cartesian, so
%$\mathscr{X}^{fillthisin} = (\mathscr{X}, \text{p-cartesian edges})$.

%We have an adjunction $Forget: sSet^+ \to sSet_{Joyal} : (-)^{\sharp}$. What is $L(Forget)$? Cocartesian morphisms
%over a point are just equivalences. So $L(Forget)$ is $\infty$-categorical localization.





