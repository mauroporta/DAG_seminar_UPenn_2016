\chapter{The Cotangent Complex}
Talk by Sukjoo Lee.

\section{Motivation}
\label{sect:cc_motivation}

We recall from classical AG: if $f:A \to B$ is a homomorphism between commutative rings and $M$ is a $B$-module, 
an A-\textbf{derivation} of $B$ into $M$ is a map $d : B \to M$ such that:
\begin{itemize}
\item $d\big( f(a)\big) = 0$, for all $a\in A$;
\item $d(bb') = db\; b' + b\; db'$ (Leibniz rule).
\end{itemize}
We denote by $Der_A(B,M)$ the set of all derivations of $B$ into $M$. There is also an absolute version, where
we take $f:0 \to A$, and the first condition is automatic.

\begin{defin}
The \textbf{module of relative K\"ahler differentials} of $B$ over $A$ is a derivation $(\Omega^{1}_{B/A}, d_A)$ 
over $A$ satisfying the universal property:
\[
\begin{tikzcd}
B \arrow{r}{d_A}\arrow[swap]{rd}{d'} & \Omega^1_{B/A}\arrow[dotted]{d}{\exists !} \\
 & M.
\end{tikzcd}
\]
Equivalently, $\Hom_{B-Mod}(\Omega^1_{B/A}, M) \simeq Der_A(B,M)$. (+ absolute version).
\end{defin}

\begin{prop}
If $A \to B \to C$ is a sequence of maps of commutative rings, then the following sequence of $C$-modules is exact:
\begin{equation}
\label{eq:ses_omegas}
	\Omega^1_{B/A} \otimes_B C \to \Omega^1_{C/A} \to \Omega^1_{C/B} \to 0.
\end{equation}
\end{prop}

One of the goals for this talk is to extend the sequence to the left. If $CRing$ was an Abelian category, we would
attempt to derive the functor $\Omega^1$; however, this is not the case. Instead, what we do is generalize the notion of
K\"ahler differential to the $\infty$-categorical setting, and show that this gives an extension to the left of the sequence
\ref{eq:ses_omegas}. Slogan: ``$\infty$-category theory allows us to do derived functors in a non-linear setting''. 



\section{Generalization and definition}
\label{sect:cc_def}

Note that generalizing the Leibniz rule to the $\infty$-category setting is hard, because we'd have to replace the 
equality with a homotopy.
Instead, consider the following idea. For a ring homomorphism $\phi : A \to B$, we want a new homomorphism $\phi': A \to B$ 
``sufficiently close'' to $\phi$. For example, take $I \subset B$ an ideal
with $I^2 = 0$. Then ``sufficiently close'' means that $\phi' : A \to B$ is congruent to $\phi$ modulo $I$, i.e.:
\[	\forall\; a \in A,\; \phi(a) - \phi'(a) \in I.	\]
For a fixed $\phi$, we have a bijective correspondence:
\begin{equation}
\label{eq:corresp_derivations}
\left\{ \begin{array} {c} \phi': A \to B \text{ such that} \\ \phi' \equiv \phi \text{ mod } I \end{array} \right\}
\longleftrightarrow
\left\{ \begin{array} {c} d: A \to I \text{ satisfying} \\ \text{the Leibniz rule} \end{array} \right\} .
\end{equation}


If $M$ an $A$-module, take $B := A \oplus M$ equipped with 
the ring structure such that $M^2 = 0$:
\[	(a_1, m_1) (a_2,m_2) = (a_1 a_2, a_1 \dot m_2 + a_2\dot m_1).	\]
We fix $\phi : A \to B$ the natural inclusion of $A$ into the coproduct (as abelian groups); with the given ring structure,
$\phi$ is also a ring homomorphism. Take the ideal $I = M$. Then the correspondence \ref{eq:corresp_derivations} becomes:
\begin{equation}
\label{eq:derivations_classical}
\Map_{/A}(A, A\oplus M) \cong Der(A,M).
\end{equation}


This is something we can generalize. We work with $A \in SCR_k$, and the category of $A-\cM od$; all categories in the rest of
the talk are $\infty$-categories. Take $M \in A-\cM od$ and construct $A \oplus M$, whose underlying simplicial set is the
coproduct, and whose ring structure is defined levelwise (see \cite{HAG-II}, 1.2.1.1 for details).

We adapt equation \ref{eq:derivations_classical} to this setting, by defining:
\[	\cD er(A,M) = \Map_{/A}(A,A\oplus M) \in \cS.	\]
Moving from the absolute version of derivations to the relative one, for $f : A \to B$ in $SCR_k$ and $M \in B-\cM od$, we 
define: 
\[	\cD er_A (B,M) = \Map_{A-Alg/B}(B, B\oplus M)\in \cS.	\]


We obtain functors $\cD er(A,-) : A-Mod \to \cS$ and $\cD er_A(B,-) : B-Mod \to \cS$. We claim that these functors are corepresentable,
and call the corepresenting objects the \textbf{absolute cotangent complex} $\bbL_A$ and \textbf{relative cotangent complex}
$\bbL_{B/A}$, respectively. Equivalently, this means:
\begin{align*}
\Map_{A-Mod}(\bbL_{A}, M) &\cong \Map_{/A}(A, A\oplus M), \\
\Map_{B-Mod}(\bbL_{B/A}, M) &\cong \Map_{A-Alg/B}(B, B\oplus M).
\end{align*}

The proof of corepresentability relies on the following lemma.
\begin{lem}
\item $Der_A(B,-)$ and $Der(A,-)$ commute with limits and $K$-filtered colimits.\footnote{
$\omega$-filtered would mean that the representing guy can be given by presentation with gen and relation. Otherwise we just
mean for everything strictly less than cardinality $K$, which could be continuum or more.} \todo{explain the footnote better}
\end{lem}

Then the result follows by Theorem 5.5.2.7 in \cite{HTT}, which we reproduce here.\footnote{Related to this is the Adjoint
Functor Theorem 5.5.2.9, which is one of the most important theorems in $\infty$-category theory. It's also the reason
we love presentable $\infty$-categories.}

\begin{thm}
Let $\cC$ be a presentable $\infty$-category and $F : \cC \to \cS$ a functor. Then $F$ is corepresentable by an object of
$\cC$ if and only if $F$ preserves $K$-filtered colimits and all small limits.
\end{thm}

\begin{rem}
Note that preserving small limits is obviously necessary in order to be corepresentable, since $\Hom$ is a right adjoint,
and thus preserves small limits.\todo{think more about the small hypothesis} Let's explain this in more detail
for 1-category theory. We have the diagram:
\[
\begin{tikzcd}
A-Mod \arrow{r}{\simeq} & Ab(CRing_{/A})\arrow{d}{\text{forget}} & \\
 & CRing_{/A} \arrow{r}{\Hom(A, -)} & Set .
\end{tikzcd}
\]
The equivalence on the first line works by sending an $A$-module $M$ to $A \oplus M$, and a ring $B$ over $A$ to the kernel
of $B \to A$. \todo{finish this}

For $\infty$-category theory, the relevant diagram is:
\[
\begin{tikzcd}
A-\cM od \arrow{r}{\simeq} & \cS p(SCR_{/A})\arrow{d}{\text{forget}} & \\
 & SCR_{/A} \arrow{r}{\Map_{/A}(A, -)} & \cS .
\end{tikzcd}
\]
The reference for this is \cite{Lurie_Higher_algebra}, section 7.4.\todo{wait, where exactly? can't find it}
\end{rem}


In \cite{HAG-II}, Chapter 1, we find an explicit (although not useful in practice, according to Mauro) model 
for $\bbL_A$. Take a simplicial resolution $\tilde A \to A$, which is also a cofibrant replacement. Then we have:
\[	 \Omega_{\tilde A}^1 \otimes^{\bbL}_{\tilde A} A \in A-Mod	\]
is a model for $\bbL$, where the complex $\Omega_{\tilde A}^1$ is build by taking K\"ahler differentials degree-wise:
\[	(\Omega_{\tilde A}^1)_{\bullet} := \Omega^1_{\tilde A_{\bullet}}.	\]





\section{Examples and Properties}
\label{sect:cc_eg}

In this section we compute $\bbL_{k[x]/k}$ and $\bbL_{k/k[x]}$, where $x$ is in degree -1, as opposed to 1, by our 
cohomological convention. (See remark \ref{rem:degree_convention} for what this implies; $k[x]$ is not what it seems.)
In the process we go over some of the properties of cotangent complexes.

\begin{lem}
Let $A \in SCR_k$ and $M \in A-\cM od$. The cotangent complex of $f: A \to \Sym_A M$ is:
\[	\bbL_{\Sym_A M /A}\cong M \otimes^{\bbL}_A \Sym_A M .	\]
\end{lem}
\begin{proof}
For all $\Sym_A M$-module $N$,
\begin{align*}
	\Map_{\Sym_A M -\cM od}(\bbL_{\Sym_A M/A}, N) &\cong \Map_{A-\cM od}(\Sym, \Sym\oplus N ) \cong \Map_{A-\cM od}(M, f_* N) \\
&\cong \Map_{\Sym_A M - \cM od}(M \otimes^{\bbL}_A \Sym_A M ,N).
\end{align*}
Here the first equivalence is definitional, the second follows from the universal property of $\Sym_A M$, and the third is
the adjunction \ref{tikz:adjunction}.
\end{proof}

For our first example, note that $k[x] \cong \Sym_k(k[1])$. Then the answer is $k[1] \otimes^{\bbL}_k k[x]$, which is just 
$k[x]$ concentrated in degree $-1$.

\begin{rem}
\label{rem:degree_convention}
Note that, since $k[1]$ is concentrated in degree $-1$, so is $\Sym_k(k[1])$; it does not have information in all 
nonnegative degrees, as the notation may mislead one into thinking. We just get a copy of $k$ in degree 0 and one
in degree -1,
and this is what we call $k[x]$. This is because multiplication in the symmetric algebra is graded commutative:
\[	x y = (-1)^{|x||y|} yx,	\]
so in particular for $x$ of degree 1 we get $x^2 = -x^2 = 0$. If we started with $k[2]$ instead,
$\Sym_k(k[2])$ would be nontrivial in all negative even degrees and commutative in the classical sense. More generally, 
$\Sym(k[n])$ gives what we would classically
call a symmetric algebra if $n$ is even, or an alternating algebra if $n$ is odd.
\end{rem}

Some properties of cotangent complex:
\begin{prop}
\;
\begin{enumerate}
\item For $A \to B \to C$ in $SCR_k$, there is a homotpy cofiber sequence in $C-Mod$:
\[	\bbL_{B/A} \otimes^{\bbL}_B C \to \bbL_{C/A} \to \bbL_{C/B}.	\]
\item Base change: given a homotopy pullback square
\[
\begin{tikzcd}
A\arrow{r}\arrow{d} & B\arrow{d} \\
A'\arrow{r} & B',
\end{tikzcd}
\]
there is an equivalence $\bbL_{B/A} \otimes^{\bbL}_B B' \simeq \bbL_{B'/A'}$.
\end{enumerate}
\end{prop}

To compute $\bbL_{k/k[x]}$, use the cofiber sequence associated to the sequence of maps $k \to k[x] \to k$. We get the
cofiber sequence in $k-\cM od$:
\begin{equation}
\label{eq:cofiber_poly}
	\bbL_{k[x]/k} \otimes^{\bbL}_{k[x]} k \to \bbL_{k/k} \to \bbL_{k/k[x]}.
\end{equation}
By our previous computation, the first term is:
\[ \bbL_{k[x]/k} \otimes^{\bbL}_{k[x]} k \simeq k[1] \otimes_k^{\bbL} k[x] \otimes^{\bbL}_{k[x]} k \simeq k[1].	\]
(Using associativity for derived tensor product.) The second term in \ref{eq:cofiber_poly} is 0, so the cofiber sequence
is actually a suspension diagram. 
\[
\begin{tikzcd}
k[1]\arrow{r}\arrow{d} & 0\arrow{d} \\
0\arrow{r} & \bbL_{k/k[x]}
\end{tikzcd}
\]
Then $\bbL_{k/k[x]}\simeq k[2]$. 
%That this is consistent with our intuition from the classical case: 
%if, to begin with, $k$ is in degree 0 to begin with, and
%$\bbL_{k/k[x]} = k[1]$, concentrated in degree -1. It's great that it doesn't have anything in degree 0, since Kahler differentials
%of a closed immersion are 0.

Going back to Section \ref{sect:cc_motivation}, we complete the exact sequence \ref{eq:ses_omegas} on the left. 
Using stability of $C-\cM od$, the cofiber sequence gives a long exact sequence on homology (recall proposition
\ref{prop:stable_les_homology}; in particular, $H^i(X) = \tau^{\geq i} \circ \tau^{\leq i}(X)$).
\[
\begin{tikzpicture}[descr/.style={fill=white,inner sep=1.5pt}]
\label{tikz:les}
        \matrix (m) [
            matrix of math nodes,
            row sep=1em,
            column sep=2.5em,
            text height=1.5ex, text depth=0.25ex
        ]
        { \dots & H^{-1}(\bbL_{B/A} \otimes^{\bbL}_{B} C) & H^{-1}(\bbL_{C/A}) & H^{-1}(\bbL_{C/B}) & \\
            & H^0(\bbL_{B/A} \otimes^{\bbL}_{B} C) & H^0(\bbL_{C/A}) & H^0(\bbL_{C/B}) & \\
            & H^1(\bbL_{B/A} \otimes^{\bbL}_{B} C) & H^1(\bbL_{C/A}) & H^1(\bbL_{C/B}) & \dots \\
        };

        \path[overlay,->, font=\scriptsize,>=latex]
        (m-1-1) edge (m-1-2)
        (m-1-2) edge (m-1-3)
        (m-1-3) edge (m-1-4)
        (m-1-4) edge[out=350,in=170]  (m-2-2)
        (m-2-2) edge (m-2-3)
        (m-2-3) edge (m-2-4)
        (m-2-4) edge[out=350,in=170]  (m-3-2)
        (m-3-2) edge (m-3-3)
        (m-3-3) edge (m-3-4)
				(m-3-4) edge (m-3-5);
\end{tikzpicture}
\]
We claim that:
\begin{enumerate}
\item \label{item:h0_underived}
\textit{For underived rings}, $H^i(\bbL_{B/A}) = 0$ if $i>0$;
\item \textit{For underived rings}, $H^0(\bbL_{B/A}) \simeq \Omega^1_{B/A}$;
\item \textit{In general}, $H^0(\bbL_{B/A}) \simeq \Omega^1_{\pi_0(B)/\pi_0(A)}$.
\end{enumerate}
An application of these facts is the desired extension
to the left of the sequence \ref{eq:ses_omegas}. The facts are proved in \cite{Lurie_Higher_algebra}, section 7.4.3. 
We will say more about claim \ref{item:h0_underived}, but first we need to talk about connectivity. 



\section{Connectivity}
\label{sect:cc_connectivity}

\begin{defin}
A space $X$ is \textbf{n-connective} if $\pi_i(X,x) = 0$ for all $x \in X$ and $i<n$. We say $X$ is \textbf{connective}
if it's 0-connective, \textbf{connected} if it's 1-connective. $f:X \to Y$ is \textbf{n-connective} if fiber$(f)$ is n-connective.
\end{defin}

The following is in \cite{Lurie_Higher_algebra}, 7.4.3.2, and it's VERY important.
\begin{thm}[Connectivity estimate]
\label{thm:connectivity_estimate}
Assume $f : A \to B$ is a map in $SCR_k$ and cofib$(f)$ is $n$-connective. Then there exists a map:
\[	\mathcal{E}_f: B \otimes_A^{\bbL} Cofib(f) \to \bbL_{B/A}	\]
in $B-Mod$, which is $2n$-connective.
\end{thm}

\begin{rem}
The proof is not hard; the only difficulty is constructing the map, which we can do after we learn Postnikov towers.
\todo{reference once we have the postnikov notes}
\end{rem}

\begin{cor}
The hypothesis of Theorem \ref{thm:connectivity_estimate} implies $\bbL_{B/A}$ is $n$-connective.
\end{cor}
\begin{proof}
We look at the fiber sequence:
\[	fib(\mathcal{E}_f) \to B\otimes^{\bbL}_A cofib(f) \to \bbL_{B/A},	\]
and get a long exact sequence of homotopy groups. So it suffices to show that:
\begin{enumerate}
\item \label{item:cofiber_connective}
 $B\otimes_A^{\bbL} Cofib(f)$ is $n$-connective;
\item \label{item:epsilon_connective}
$fib(\mathcal{E}_f)$ is $n-1$-connective.
\end{enumerate}
\ref{item:epsilon_connective} is implied by Theorem \ref{thm:connectivity_estimate}; note that theorem is actually considerably
stronger. Property \ref{item:cofiber_connective} is proved in \cite{Sq0}. The proof there uses a spectral sequence due to 
Quillen: for $M,N \in A-Mod$, $A \in SCR_k$,
\[	\Tor_p^{\pi_q(A)}(\pi_qM, \pi_qN) \Longrightarrow \pi_{p+q}(M \otimes^{\bbL}_A N).	\]
\end{proof}

\begin{rem}
In particular, cotangent complexes are 0-connective for commutative rings. This gives a proof of fact
\ref{item:h0_underived} at the end of the previous section.
\end{rem}

\begin{cor}
For $A \in SCR_k$, $\bbL_A$ is 1-connective. Moreover, $f:A \to \pi_0(A)$ is 1-connective, so $\bbL_{\pi_0(A)/A}$ is
1-connective.
\end{cor}

The most important corollary:
\begin{cor}
$f: A \to B$ is an equivalence iff $\pi_0(f) : \pi_0(A) \to \pi_0(B)$ is and $\bbL_{B/A} \simeq 0$. One direction obvious, the
other comes from the fact that $\bbL_{B/A}$ is $n$-connected for all $n$.
\end{cor}

\begin{rem}
Slogan: ``DAG = classical AG + DDT''.
Lurie's representability theorm is a great example of the philosophy: it says that a derived stack is representable iff
its truncation is representable and its cotangent complex is nice enough. We won't get to see this in the seminar, since we'll
change course towards structured DAG instead.
\end{rem}


\begin{rem}
Cotangent complexes we glue for free, which was not possible before $\infty$-categories. This allows to reduce many questions
to the affine setting, where we may have to do actual computations if things go wrong.
\end{rem}

We have one talk on Postnikov tower, and one on perfect complexes, then we leave the affine setting forever.

\todo{look at last 2 exercises from stable $\infty$-category}


