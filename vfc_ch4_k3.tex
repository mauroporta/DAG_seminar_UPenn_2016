\chapter{Reduced Gromov Witten Invariants for K3 Surfaces}
\label{ch4:k3}

(Talk by Benedict Morrissey)  In chapter \ref{ch3:geom}, we constructed a quasismooth derived enhancement of the moduli stack of stable maps.  The machiner of chapter \ref{ch2:obs} can then be used to construct virtual fundamental classes, and from there Gromov--Witten invariants.  In this chapter we consider the case where the target varietry $X$ is a K3 surface.  In this case Gromov--Witten invariants defined in this fashion vanish.  In this chapter following \cite{schurg2015derived} provide an alternative quasismooth derived enhancement.  The virtual fundamental classes obtained were earlier described in \cite{maulik2007gromov, maulik2010curves, okounkov2010quantum} (henceforth referred to as OMPT).

\section{Non Reduced Gromov--Witten Invariants for K3 Surfaces}

\todo{Work out where I can find a proof that all these invariants disappear!}

\section{$\R Pic$ for K3 Surfaces}

The point of this section is to give a canonical identification $\R Pic(X)\xrightarrow{\cong}Pic(X)\times \R Spec(Sym(H^{0}(X,K_{X})[1]))$.  Recall that $\R Pic(X)$ is a (locally of finite presentation) derived group stack.

\begin{thm}[Prop. 4.5 in \cite{schurg2015derived}]
\label{thm:derivedgroup}
A locally of finite presentation group stack $G$ over a field $k$, with identity $e: Spec k\rightarrow G$, and Lie algebra $\mathfrak{g}=T_{e}G$, has a canonical map
\[\gamma(G):t_{0}(G)\times \R Spec(A)\rightarrow G,\]
where $A=k\oplus (\mathfrak{g}^{\vee})_{<0}$.
\end{thm}

\begin{proof}
The projection map $A\rightarrow k$ gives a $k$-point $x_{0}:Spec(k)\rightarrow \R Spec(A)$.

We wish to find a commuting diagram 
\[
\begin{tikzcd}
B \arrow{rr}{a} && C\\
 & A \arrow{ur}{d} \arrow{ul}{x_{0}}
\end{tikzcd}
\]
this is equivalent to giving a morphism $a':\L_{G,e} \cong \mathfrak{g}^{\vee}\rightarrow (\mathfrak{g}^{\vee})_{<0}$ (due to our choice of $A$), hence taking the truncation map $\tau_{< 0}$ (using the standard t-structure on the stable category of (dg) vector spaces) gives 
this map.

We then take the composition
\[t_{0}(G)\times \R Spec(A)\xrightarrow{j\times a}G\times G\xrightarrow{\times}G,\]
where the final map uses the group product in $G$.
\end{proof}

We now apply Theorem \ref{thm:derivedgroup} to the group stack $\R Pic(X)$ for a K3 surface $X$.

\begin{thm}
The map $\gamma_{\R Pic(X)}$ for $X$ a K3 surface gives an isomorphism of derived stacks
\[\R Pic(X)\xrightarrow{cong} Pic(X)\times \R Spec(Sym(H^{0}(X,K_{X})[1])).\]
\end{thm}

\begin{proof}
We note first that this map clearly provides an isomorphism on truncations.  Hence as $\R Pic(X)$ is a derived group stack, we need only show that it is \'{e}tale at $e$, that is to say 
\[T_{t_{0}(e),x_{0}}(\gamma_{\R Pic(X)}):T_{t_{0}(e),x_{0}}(Pic(X)\times \R Spec(Sym(H^{0}(X,K_{X})[1])))\rightarrow T_{e}G\]
is an isomorphism of dg k-vector spaces.

Note that 
\[T_{e}G=\mathfrak{g}=\R \Gamma(X,\mathcal{O}_{X})[1]\cong H^{0}(X,\mathcal{O}_{X})[1]\oplus H^{2}(X,\mathcal{O}_{X})[-1].\] \todo{why? is the cohomology of a K3 like this?}

Hence 
\[A=\C\oplus (\mathfrak{g}^{\vee})_{<0}=\C\oplus H^{2}(X,\mathcal{O}_{X})[1]\cong H^{0}(X,K_{X})[1]\cong Sym(H^{0}(X,K_{X})[1]\]
where the final step follows because $H^{0}(X,K_{X})$ is free of dimension 1.

Clearly $T_{t_{0}(e),x_{0}}(\gamma_{\R Pic(X)})$ is an isomoprhism.
\end{proof}

This identification allows the definition of a projection $pr_{der}:\R Pic(X)\rightarrow \R Spec(Sym(H^{0}(X,K_{X})[1])$.

\section{The reduced Moduli Space $\R M_{g,n}(X,\beta)^{red}$.}
\label{sec:reduced moduli space}

Recall the map $x_{0}:Spec(\C)\rightarrow  \R Spec(Sym(H^{0}(X,K_{X})[1])$.

We defined the reduced derived enhancement as follows:
\begin{defin}
The \textbf{reduced stack of $n$-pointed stable maps} of genus $g$, class $\beta$ to a K3 surface $X$ is given by the pullback
\[
\begin{tikzcd}
\R M_{g,n}^{red}(X,\beta)\arrow{r}\arrow{d} & \R M_{g,n}(X,\beta)\arrow{d}{\delta_{1}^{der}(X,\beta)} \\
Spec(\C)\arrow{r}{x_{0}} & \R Spec(Sym(H^{0}(X,K_{X})[1])
\end{tikzcd}
\]
where we define the map $\delta_{1}^{der}(X,\beta)$ by the composition
\[\R M_{g,n}(X,\beta)\rightarrow \R M_{g,n}(X)\xrightarrow{a} Perf(X)\xrightarrow{det} \R Pic(X)\xrightarrow{pr_{det}} \R Spec(Sym(H^{0}(X,K_{X})[1]),\]

where the map $a$ is induced by the perfect complex $\R \pi_{*}(\mathcal{O}_{\R \mathcal{C}_{g,X}})$, (that is this specifies a perfect complex over $X\times M_{g,n}(X,\beta)$, and hence a map  $M_{g,n}(X,\beta)\rightarrow Perf(X)$).
\end{defin}

\begin{rem}
This derived enhancement is quasismooth, and we will show that the induced obstruction theory on $M_{g,n}(X,\beta)$ agrees with that of OMPT.
\end{rem}

\begin{prop}
The derived stack $\R M_{g,n}^{red}(X,\beta)$ is quasismooth.
\end{prop}

\begin{proof}
\todo{Write This}
\end{proof}

\section{The Resultant Obstruction Theories}

In this section we compare the obstruction theories obtained from the reduced derived enhancement with those obtained in OMPT.

\todo{Want THM 4.8 from STV}

\todo{Write an explicit description}

\section{The Link to Donaldson--Thomas Theory}

Recall that in section \ref{sec: reduced moduli space} we made use of a map $\R Map (X,\beta)\rightarrow Perf(X) $.  Recall that the stack $Perf(X)$ is used in Donaldson--Thomas theory, a curve counting theory that uses a different compactification of the space of curves in $X$ that the stable curves used in Gromov--Witten invariants.  There are conjectured to be various relationships between Gromov--Witten and Donaldson--Thomas invariants as developed in \cite{maulik2006gromovI, maulik2006gromovII}.

\todo{talk about all of S5}

