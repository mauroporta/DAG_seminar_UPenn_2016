\chapter{Geometric Stacks and Gluing}
Talk by Mauro Porta.


We finally leave the affine world! Only took us 2 months. But first, we mention a
correction to Lemma \ref{lem:equiv_heart} from the previous talk.
The statement ``$F^{\heartsuit}$ is an equivalence'' should be taken as ``$F^{\heartsuit}$ is essentially surjective and 
$F|_{\cC^{\heartsuit}}$ is
fully faithful''. This means we have to take into account $\Ext$ between discrete objects; an equivalence at the level of
hearts is not strong enough to guarantee an equivalence on the entire categories.

Otherwise, there's a counterexample to the lemma. Take $A \in cdga^{\leq 0}$, we have the pullback  functor 
$A-\cM od \overset{f^*}{\to} H^0(A)-\cM od$, $M\mapsto M \otimes_A H^0(A)$, which is not t-exact, so the lemma makes no statement
about it. However, the forgetful functor $f_* : H^0(A) - \cM od \to A-\cM od$ is t-exact, but it fails to be an equivalence, even if
$f_*^{\heartsuit}$ is an equivalence between the categories of discrete modules.


\section{Gluing: problems and approaches}
The problem for today is: how do we patch together derived affines? This is difficult for two reasons:
\begin{enumerate}
\item If a derived scheme is to be thought of as a gluing of derived affines, we should be able to produce many gluing
diagrams in $d \cA ff_k$. This means functors $I \to d \cA ff_k$, which is difficult because the latter is an $\infty$-category,
so we need to specify higher coherencies when defining functors.
In particular, to define $\bbP^1$, we have:
\[
\begin{tikzcd}
\; & \bbA^1\arrow{dr} &  \\
\bbG_m \arrow{ur}\arrow{dr} & & \bbP^1 \\
& \bbA^1\arrow{ur} &
\end{tikzcd}
\] 
as well as a homotopy between the branches. \todo{Am I interpreting this correctly?}
This is not an existential threat, because we can choose a model-categorical presentation for $d \cA ff_k$.
\item But we really need an environment category where the gluing is performed; constructing this is tricky. There are 2 ways.
\begin{itemize}
\item Structured spaces, i.e. the environment is the category of locally ringed topoi, or similar.
\item Functor of points, i.e. the environment is the category of presheaves of spaces, $\Fun(d\cA ff \to \cS)$.
\end{itemize}
Today we want to address both and compare them.
\end{enumerate}

First, we look at pros and cons for both:
\begin{itemize}
\item Structured spaces. It's fairly easy to think about objects in this way: they are $(X, \mathcal{O}_X)$, where $X$ is some
sort of topological space (actually $\infty$-topos), and $\mathcal{O}_X$ is a sheaf of cdga's on $X$. Note how similar this is
to how we think about underived schemes. Moreover, the subdivision $(X, \mathcal{O}_X)$ makes it easy to distinguish the
derived information from the underived one. For example, $\pi_0(X,\mathcal{O}_X) = (X, \pi_0(\mathcal{O}_X))$, and the Postnikov
tower discussion carries over to derived schemes:
\[	\dots \to (X, \tau_{\leq 2} \mathcal{O}_X) \to (X, \tau_{\leq 1} \mathcal{O}_X) \to (X, \pi_0 \mathcal{O}_X).	\]
Cons: maps are difficult to understand, and only Deligne-Mumford stacks can be described in this way. 
\footnote{Any structured space has connected cotangent complex, but Artin stacks don't need to. The smooth \'etale
site is not canonical, while the \'etale site is.}
Note that the stack of
perfect complexes, as well as the Eilenberg-Maclane stacks are Artin, but not Deligne-Mumford.

\item Functor of points. It can deal with Artin stacks, and then some more. Sometimes in life we want to deal with objects 
which are stacks,
but not geometric stacks; the easiest example is $QCoh : d Aff \to Cat_{\infty}$. We saw last time it has descent, so it's
a stack, but it's not geometric.\footnote{I.e. it doesn't have an appropriate atlas, we'll see shortly what this means.} 
We only have one con, but it's pretty bad: it's not clear at all,
in this language, why schemes should be simpler than Artin stacks. In other words, techniques which hold for schemes but don't
hold for Artin stacks are obscured.
\end{itemize}


\section{Structured spaces}
We want a category $\cC$ with the following properties:
\begin{enumerate}
\item \label{env1}
$\cC$ contains $d \cA ff_k$ in a fully faithful way.
\item \label{env2}
$\cC$ is big enough to contain all the gluings we'll make. For example, we'd be happy with $\cC$ closed under colimits.
\item \label{env3}
$\cC$ is small enough to have a good notion of Grothendieck topology. (Note that any co-complete category has a Grothendieck
topology, called the \textbf{canonical topology}: for an object $X$, and a collection $\{U\}$ of objects mapping to $X$, 
call it a covering if
the geometric realization of the $\check{C}$ech nerve of $\{U\}$ is equivalent to $X$. But this is not a very useful topology; when
we say ``good Grothendieck topology'', we want to have a better grasp on the coverings: describe them using words such as
\'etale, smooth, flat, open immersion etc.)
\end{enumerate}


To exemplify this philosophy, recall what we do in classical AG. That is, start with $\cA ff_k$ instead of $d \cA ff_k$. 
Then we have two choices for $\cC$.
\begin{itemize}
\item $(X,\mathcal{O}_X)$ locally ringed spaces. Property \ref{env1} we all know. For \ref{env2}, note that LRS doesn't admit
all colimits (you can't talk about $\bbP^{\infty}$, for example; that's an IndScheme), but admits enough of them
to describe schemes. For \ref{env3}, the Grothendieck topology is as follows. We say $(X, \mathcal{O}_X) \to (Y,\mathcal{O}_Y)$ is an
\textbf{open immersion} if it's an open inclusion at the level of topological spaces, and the induced map $f^{\sharp}
: f^* \mathcal{O}_Y \to \mathcal{O}_X$ is an isomorphism. Then a collection $\{X_i\}$ is a \textbf{covering} of $Y$ if
each $X_i \to Y$ is an open immersion, and moreover the induced map $\coprod X_i \to Y$ is surjective at the level of topological
spaces. \todo{is this definition of covering correct?}
Then we can define schemes as objects in LRS which are covered, in the above sense, by objects in the essential
image of $\cA ff_k \to LRS$.

\item Alternatively, we can take locally ringed 1-topoi, in which case we get Deligne-Mumford stacks. Below we make a short
summary of locally ringed 1-topoi.
\end{itemize}

\begin{defin}
A \textbf{Grothendieck site} is a category $\mathcal{D}$ together with a Grothendieck topology. A \textbf{1-topos} $X$
is a category equivalent to $\Sh(\mathcal{D})$ for some Grothendieck site $\mathcal{D}$.
\end{defin}

If the gluing environment $\cC$ is the category of locally ringed topoi, then \ref{env1}
and \ref{env2} are again easy. For the Grothendieck topology on $\cC$, we make the following definitions.
\footnote{Don't confuse the Grothendieck topology on the category $\cC$ of locally ringed topoi, which we're trying to
define now, with the Grothendieck topology on the underlying site $\cD$ of an individual topos $\mathcal{X}$. Sometimes
both topologies have the same name, e.g. \'etale, but it should be clear that they're completely different notions.}

\begin{defin}
For $\mathcal{X}$, $\mathcal{Y}$ 1-topoi, a \textbf{geometric morphism} $\mathcal{Y} \to \mathcal{X}$ is an adjoint pair:
\[	f^{-1} : \mathcal{X} \to \mathcal{Y} : f_*,	\] 
where moreover $f^{-1}$ preserves finite limits.
\end{defin}

Note that, despite the notation, the geometric morphism goes from $\mathcal{Y}$ to
$\mathcal{X}$; the ``inverse image'' $f^{-1}$ happens to be the left adjoint. This is motivated by the following example.
If $X, Y$ topological spaces, take $\mathcal{X} = \Sh(X)$ and $\mathcal{Y} = \Sh(Y)$.
For any $f: Y \to X$, the usual inverse and direct image functors on sheaves:
\[	f^{-1}: \Sh(X) \to \Sh(Y) : f_*	\]
form a geometric morphism. Recall that $f^{-1}$ is defined as a colimit indexed by open subsets containing
the image of $f$, followed by sheafification. Both these operations are filtered colimits, therefore commute with finite
limits. Hence $f^{-1}$ commutes with finite limits, and the pair $(f^{-1},f_*)$ is a geometric morphism.

\begin{defin}
\label{defin:etale_topoi}
A geometric morphism $f^{-1} : \mathcal{X} \to \mathcal{Y} : f_*$ is \textbf{\'etale} if there exists $U \in \mathcal{X}$
and an equivalence $\mathcal{X}_{/U}$ making the following diagram commute:
\[
\begin{tikzcd}
\mathcal{X}\arrow{r}{\times U}\arrow[shift left]{dr}{f^{-1}} & \mathcal{X}_{/U}\arrow{d}{\simeq} \\
 & \mathcal{Y}\arrow[shift left]{ul}{f_*} .
\end{tikzcd}
\]
\end{defin}

To get a feel for this definition, note the following examples.

\begin{lem}
\label{lem:etale_topoi}
If $f:Y \to X$ is a local homeomorphism of topological spaces, then the standard $f^{-1}, f_*$ is \'etale. Moreover, if
$f:Y \to X$ is an \'etale map of schemes, then $f^{-1} : \Sh_{Set}(X_{\'et}, T_{\'et}) \to 
\Sh_{Set}(Y_{\'et}, T_{\'et})$ is \'etale.\footnote{By $X_{\'et}$ we mean the \'etale site of the scheme $X$, which is the
subcategory of the comma category $\Sch_X$ given by \'etale morphisms $U \to X$. We endow this site with the Grothendieck
topology $T_{\'et}$ induced from the \'etale topology on $\cA ff$.}
\end{lem}

In fact, we can uprgrade the second statement of Lemma \ref{lem:etale_topoi} to a characterization of \'etale morphisms of
schemes.

\begin{prop}
$f : X \to Y$ in $Sch_k$ is \'etale iff the induced $(f^{-1},f_*)$ morphism of topoi is \'etale
and $f^{-1} \mathcal{O}_Y \to \mathcal{O}_X$ is an equivalence.
\end{prop}

\begin{rem}
We elaborate on the element $U \in \mathcal{X}$ which appears in Definition \ref{defin:etale_topoi}. If $Y \subset X$ is an
open subset of the topological space $X$, then $Y$ defines by Yoneda a sheaf $h_Y \in \mathcal{X} = \Sh(X)$, such that:
\[
h_Y(V) = \left\{ \begin{array} {ll} * & ,V \subset Y, \\ \emptyset &, V\not \subset Y. \end{array}\right.
\]
Then we take $U = h_Y$, and note that we have $\Sh(Y) \simeq \Sh(X)_{/U}$.

We can relax the assumptions and take $f:Y \to X$ a local homeomorphism. Then we define a sheaf $h_f$ by
$h_f(V) = \Hom_{/V}(V,V \otimes_X Y)$.\footnote{In other words, this is the sheaf of sections of $f:Y \to X$.}
 This generalizes the sheaf $h_Y$ from the previous paragraph, in the sense that,
if $f$ is an open immersion, $\Hom_{/V}(V,V \otimes_X Y)$ is nonempty iff $V\cap Y = V$, i.e. $V \subset Y$.
Moreover, it's a fact that local homeomorphisms are characterized by the property $\Sh(Y) \simeq \Sh(X)_{/h_f}$. So
we take $U = h_f$.
\end{rem}

We move on from the idea of gluing in the ambient category of locally ringed 1-topoi to the derived analog. This
will involve the theory of $\infty$-topoi, which allows us to replace sheaves of sets with sheaves of spaces,
and obtain higher DM stacks. This is necessary, for example, in order to talk about $K(G,n)$, for $G$ finite abelian
and $n>1$.\footnote{In algebraic topology, we obtain $K(G,n)$ by de-looping $K(G,n-1)$, a process which almost never
returns a scheme. So when doing geometry we need to replace de-looping with working with higher groupoids; morally
speaking, de-looping $n$ times in topology corresponds to increasing the stack level by $n$. For example, the de-looping
$BG$ of a topological group $G$ corresponds to the stack $BG$ in geometry.}


\section{A primer on $\infty$-topoi}
Note first that the HAG framework \cite{HAG-I} involves hypercomplete $\infty$-topoi, while the DAG framework
\cite{Lurie_DAG_V} involves non-hypercomplete 
$\infty$-topoi. The latter make for a nicer theory, but in practice often need to be reduced to the hypercomplete case.

The main difference is that the following hold if $\mathcal{X}$ is hypercomplete, but don't need to otherwise.
\begin{itemize}
\item $\mathcal{X} = \Sh(\cC, \cT)$, where $(\cC, \cT)$ is an $\infty$-Grothendieck site.\footnote{From now on,
sheaves are sheaves of spaces, unless we explicitly say otherwise.}
\item For $f: \mathcal{F} \to \mathcal{G}$ a morphism in $\mathcal{X}$, $f$ is an equivalence
iff $\pi_i(f)$ are isomorphisms for all $i$.
\end{itemize}

\begin{rem}
\begin{enumerate}
\item If $\mathcal{X}$ is a 1-topos, viewing it as an $\infty$-category (i.e. taking the nerve) \emph{does not} get us
an $\infty$-topos. Instead, if $\mathcal{X} = \Sh_{Set}(\cC, \cT)$, we need to replace it with $\Sh(\cC,\cT) :=\Sh_{\cS}(\cS,\cT)$,
which is an $\infty$-topos. Moreover, we have a fully faithful embedding $\mathcal{X} \subset \Sh(\cC,\cT)$.

\item If $\mathcal{X}$ is an $\infty$-topos, we can look at $\mathcal{X}^{\leq n}$, the $\infty$-category of $n$-truncated 
objects in $\mathcal{X}$. $\mathcal{F} \in \mathcal{X}$
is \textbf{n-truncated} if for all $\mathcal{G} \in \mathcal{X}$, $\Map_{\mathcal{X}}(\mathcal{G},\mathcal{F})$ is n-truncated
as an $\infty$-category. Every $\mathcal{F} \in \mathcal{X}^{\leq n}$ is hypercomplete, i.e. equivalent to the limit of its
Postnikov tower.

We say that $\mathcal{X}$ is \textbf{n-localic} if we can recover it from $\mathcal{X}^{\leq n}$. More precisely, this means that
for all $\infty$-topoi $\mathcal{Y}$, $\Map_*(\mathcal{Y}, \mathcal{X}) \simeq \Map_*(\mathcal{Y}^{\leq n}, \mathcal{X}^{\leq n})$.
The equivalence is implemented as follows. Since $f_*$ is a right adjoint, it preserves truncated objects, which gives the
bottom arrow in the diagram:
\[
\begin{tikzcd}
\mathcal{Y}\arrow{r}{f_*} & \mathcal{X} \\
\mathcal{Y}^{\leq n}\arrow[hook]{u}\arrow[dashed]{r} & \mathcal{X}^{\leq n}\arrow[hook]{u}.
\end{tikzcd}
\]

\item Sometimes, $\infty$-topoi are ``naturally'' hypercomplete. For example:
\begin{itemize}
\item If $X$ is a topological space with finite covering dimension, such as locally compact Hausdorff, then $\Sh(X)$ is
hypercomplete.
\item If $X$ is a quasi-compact, quasi-separated, locally Noetherian scheme, then sheaves on $(X_{Zar},\cT_{Zar})$
are a hypercomplete $\infty$-topos.
\item With $X$ as before $\Sh(X_{Nis},\cT_{Nis})$ is hypercomplete.
\item With $X$ as nice a scheme as you want, even a field, $\Sh(X_{\'et},\cT_{\'et})$ is \emph{not} hypercomplete. But it's
always 1-localic.
\end{itemize}
\end{enumerate}
\end{rem}

\begin{defin}
An $\mathbf{\infty}$\textbf{-topos} is a left exact, accessible localization of a presentable $\infty$-category.
(For example, a presentable $\infty$-category of presheaves of spaces.)
\end{defin}

\begin{rem}
There exists an analog of Giraud's characterization, giving necessary and sufficient intrinsic conditions for an $\infty$-category
to be an $\infty$-topos. They are somewhere in \cite{HAG-I}.
\end{rem}

\begin{rem}
$\mathcal{X}$ is n-localic iff there exists an n-cateogry $\cC$ such that $\mathcal{X} = \Sh(\cC,\cT)$.
\end{rem}

We want to construct the category of locally ringed $\infty$-topoi; a reference for this is Chapter 3 of \cite{DAG-V}.
Recall the definition of derived affines via
Lawvere theory, explained in \ref{sect:lawvere}: $d \cA ff = \Fun^{\times}(T_{disc}(k), \cS)$, where $T_{disk}(k) = \{\bbA_k^n\}$
is the $\infty$-category of affine spaces with morphisms of schemes. To get locally ringed $\infty$-topoi, we change the target $\cS$
to something else. Let $^L Top$ be the $\infty$-category of $\infty$-topoi with geometric morphisms; there is a forgetful
functor $^L Top \to \cC at_{\infty}$. Using the Grothendieck construction, this buys us a Cartesian fibration
$\overline{^L Top} \to ^L Top$.

\begin{defin}
The $\infty$-category of locally ringed $\infty$-topoi is a subcategory:
\[	^L Top(T_{disc}(k)) \subset \Fun(T_{disc}(k), \overline{^L Top}) \times_{\Fun(T_{disc}(k),^L Top)} {^L Top}.	\]
Its objects are pairs $(\mathcal{X}, \mathcal{O} : T_{disc}(k) \to \mathcal{X})$, where $\mathcal{O}$ commutes with products
and $\pi_0(\mathcal{O})$ is a sheaf of local rings. Its morphisms are pairs $(f,f^{\sharp}) : (\mathcal{X},\mathcal{O}_{\mathcal{X}})
\to (\mathcal{X},\mathcal{O}_{\mathcal{X}})$, such that $f^{\sharp} : f^{-1} \mathcal{O}_{\mathcal{Y}} \to \mathcal{O}_{\mathcal{X}}$
induces a local morphism on $\pi_0$.
\end{defin}

\begin{defin}
A \textbf{derived Deligne-Mumford stack} is a pair $(\mathcal{X},\mathcal{O}_{\mathcal{X}}) \in ^L Top(T_{disc}(k))$,
such that locally $(\mathcal{X},\mathcal{O}_{\mathcal{X}})$ is of the form $\Spec A = (\Sh(A_{\'et}), \mathcal{O}_A)$,
for some derived ring $A$.
\end{defin}

In practice, $(\mathcal{X},\mathcal{O}_{\mathcal{X}})$ is a derived DM stack if there exist $U_i \in \mathcal{X}$ such that:
\begin{enumerate}
\item $(\mathcal{X}_{/U_i}, \mathcal{O}_{\mathcal{X}}|_{U_i}) \cong \Spec A_i$;
\item $U= \coprod_i U_i \to 1_{\mathcal{X}}$ is an effective epimorphism (i.e. $\pi_*(U) \to \pi_*(1_{\mathcal{X}})$ is surjective).
\end{enumerate}

\begin{defin}
We say that $(\mathcal{X},\mathcal{O}_{\mathcal{X}})$ is a \textbf{derived algebraic space} if we can take $U_i$ as above such that
$U_i \to 1_{\mathcal{X}}$ is a monomorphism for all $i$.
\end{defin}

A key property of maps of derived schemes:
\begin{prop}
The following diagram is a homotopy pullback:
\[
\begin{tikzcd}
\Map_{\Sh_{cdga}(\mathcal{X})} (f^{-1}\mathcal{O}_{\mathcal{Y}}, \mathcal{O}_{\mathcal{X}}) \arrow{r}\arrow{d} &
\Map_{d Sch} \big( (\mathcal{X},\mathcal{O}_{\mathcal{X}}), (\mathcal{Y},\mathcal{O}_{\mathcal{Y}}) \big)\arrow{d} \\
(f^{-1},f) \arrow{r} & \Map_{^L Top}(\mathcal{X}, \mathcal{Y})
\end{tikzcd}
\]
\end{prop}

So far so good; but recall that requirement \ref{env1} for the environment gluing category was that it admits a fully
faithful embedding from the affine category. The following, Theorem 2.1.12 in \cite{DAG-V}, addresses this. Note that
the proof is difficult.\todo{understand and say why}

\begin{thm}
The embedding $\Spec : sc\cA lg_k \to ^L Top(T_{disc}(k))$ is fully faithful.\footnote{Essentially one needs to prove that
$\Map(\mathcal{X}, \Spec A) \simeq \Map(A, \Gamma(\mathcal{O}_{\mathcal{X}}))$.}
\end{thm}



\section{Functor of points}
Recall that we defined $^R Top(T_{disc})$ the $\infty$-category of locally ringed $\infty$-topoi. Then we defined the
full subcategory: $DM-Stacks \subset {^R Top}(T_{disc})$ with objects $(\mathcal{X},\mathcal{O}_{\mathcal{X}})$ satisfying:
there exist objects $U_i \in \mathcal{X}$ such that:
\begin{enumerate}
\item $\coprod U_i \to 1_{\mathcal{X}}$ is an effective epimorphism;
\item $(\mathcal{X}_{/U_i}, \mathcal{O}_{\mathcal{X}/U_i}) \simeq \Spec (A_i)$, for $A_i \in sCRing$.
\end{enumerate}

Recall also that, if $A \in sCRing$, then $\Spec A$ was defined as:
\begin{enumerate}
\item $\mathcal{X}_A = \Sh (A_{et}, \tau_{et})$;
\item $\mathcal{O}_A : A_{et} \to scRing_k$
\end{enumerate}

\begin{thm}[DAG-V, Theorem 2.12]
If $(\mathcal{X},\mathcal{O}_{\mathcal{X}}) \in {^R Top}^{loc}(T_{disc})$, there is a canonical equivalence:
\[	\Map_{{^R Top}^{loc}(T_{disc})}\big( (\mathcal{X},\mathcal{O}_{\mathcal{X}}), \Spec A\big)
\simeq \Map_{sCRing_k}\big.(A,\Gamma(\mathcal{O}_{\mathcal{X}})\big)	\]
\end{thm}

\begin{cor}
$\Spec : sCRing_k \to {^R Top}^{loc}(T_{disc})$ is fully faithful.
\end{cor}

\begin{rem}
If $(\mathcal{X},\mathcal{O}_{\mathcal{X}})$ is a dDM, then it is a derived algebraic space if the following equivalent
conditions are satisfied:
\begin{enumerate}
\item $U_i \to 1_{\mathcal{X}}$ are homotopy monomorphisms (think inclusion of open sets);
\item $\mathcal{X}$ is the \'etale topos of an ordinary algebraic space.
\end{enumerate}
\end{rem}

\begin{rem}
There is no internal characterization of schemes, as opposed to algebraic spaces. The reason is that with these definitions
the underlying $\infty$-topos of $\Spec A$ is 1-localic and not 0-localic.
However, note the following.
What we called $\Spec$ is usually denoted $\Spec^{\'et}$, as opposed to $\Spec^{Zar}$, defined as $\Spec^{Zar}(A) 
= (\mathcal{X}_A^{Zar},\mathcal{O}_{A})$, where $\mathcal{X}_A^{Zar} = \Sh(A_{Zar}, \tau_{Zar})$. If we use $\Spec^{Zar}$,
derived schemes can be characterized as derived DM stacks which are covered by monomorphisms.
\end{rem}

\begin{rem}
There's a way of taking the topology into account inside $T_{disc}$. That is, there is a modified version of Lawvere theory
which takes into consideration also the Grothendieck topology. This is called (pre)geometry. It gives extra flexibility
that allows to freely switch between different topologies. This language is also used in the analytic setting. (Complex or
non-archimedian.) Then we get:
\[
\begin{tikzcd}
^R Top(T_{Zar})\arrow{r}{\Spec_{Zar}^{\'et}} & ^R Top(T_{\'et})\arrow{r}{\Spec_{\'et}^{an}} & ^R Top(T_{an}(\C)) \\
d Sch\arrow[hook]{u} & dDM\arrow[hook]{u} & d An_{\C}\arrow[hook]{u}
\end{tikzcd}
\]
Moreover, the ``\'etalification'' functor takes $\Spec^{Zar}$ to $\Spec^{\'et}$. 
Note also that the analytification functor can be constructed by hand, but we need the abstract machinery to prove
that the construction is correct. More details about this are in \cite{DAG-V}.
\end{rem}
\todo{the stuff so far should be merged with the previous section}


Now we actually move on to the functor of points. Say $\cC$ is a category of (derived) affines. We want to enlarge 
$\cC$ in order to allow general gluings. This can be performed in 3 steps:
\begin{enumerate}
\item Add all colimits to $\cC$, i.e. take the $\infty$-category $\PSh(\cC)$.
\item Realize that this destroys all geometric information in $\cC$.
\item Replace presheaves $\PSh(\cC)$ with sheaves $\Sh(\cC,\tau)$, for an appropriate Grothendieck topology $\tau$.
\end{enumerate}

For an example of why presheaves lose geometric information in $\cC$, consider $X, Y \in \cC$, and suppose 
the coproduct $Z = X \coprod Y$ exists.
Then we have $h_X, h_Y, h_Z \in \PSh(\cC)$. Unfortunately, $h_X \coprod h_Y \neq h_Z$. More concretely, take
$X = Y = \Spec k[x]$ to be affine lines. Then $h_X \coprod h_Y = \Map( -, \bbA^1_k) \coprod \Map(-,\bbA^1_k)$. On the other hand, we have
$h_Z = \Map(-, \bbA^1_k \coprod \bbA^1_k)$. Evaluating both on $\Spec k \coprod \Spec k$, $h_X \coprod h_Y$ gives
a point on each affine line, while $h_Z$ allows two points on the same affine line. This problem doesn't arise
when working with sheaves, rather than presheaves, because then $h_X \coprod h_Y$ is defined by applying sheafification
to the coproduct of presheaves.

\begin{rem}
A moment's thought should convince you that these ``geometric relations'' in $\cC$ are the same as a Grothendieck topology. 
The best way to explain this is the following lemma.
\end{rem}

\begin{lem}
Let $(\cC, \tau)$ be a Grothendieck site. Then $\tau$ is subcanonical (every representable preseheaf is a sheaf) if and only if
for all $U^{\bullet} \to X$ $\tau$-cover of $X$ in $\cC$, we have $X \cong \colim U^{\bullet}$.
\end{lem}

\begin{lem}
Let $(\cC, \tau)$ be a Grothendieck site. Then $\Sh(\cC, \tau)$ has the following universal property:
\begin{enumerate}
\item The functor $j:\cC \to \Sh(\cC, \tau)$ sends Cech nerves of $\tau$-covers to colimits.
\item $j$ is universal with property (1), i.e. for all $\infty$-categories $\cD$ and $F: \cC \to \cD$ such that
$F$ sends Cech nerves of $\tau$-covers to colimits, there exists a unique up to
homotopy extension $\tilde F$ making the following diagram commute.
\[
\begin{tikzcd}
\cC \arrow{r}{j}\arrow[swap]{dr}{F} & \Sh(\cC, \tau)\arrow[dashed]{d}{\tilde F} \\ & \cD
\end{tikzcd}
\]
\end{enumerate}
\end{lem}

Thus, we need to replace $\PSh(\cC)$ with $\Sh(\cC,\tau)$. The case we're primarily interested in is
 $\cC = dAff_k$, and $\tau = \tau_{\'et}$.

\begin{rem}
$\tau_{\'et}$ is subcanonical thanks to the descent for $QCoh$ that we discussed.
\end{rem}

Often in $\cC$ there are many interesting geometric notions (e.g. properness). It's not clear how to translate them for
random objects in $\Sh(\cC, \tau)$. Therefore we restrict to a full subcategory $\Geom(\cC,\tau, \bbP) \subset
\Sh(\cC, \tau)$, spanned by ``tame objects'', where the geometric notions transport in
a painless way. The idea is that every sheaf is a colimit of representable ones, but we want to restrict the diagrams which can
idex the colimit.

The original idea of geometric stack is due to Artin; it has been generalized by Simpson. Before starting, we fix a collection
of morphisms $\bbP$ in $\cC$, which are $\tau$\textbf{-local}. I.e. for $f: X \to Y$ in $\cC$, $f\in \bbP$ iff
for all $\tau$-covers $Y_i \to Y$, $X \times_{Y} Y_i \to Y_i$ is in $\bbP$. \footnote{Actually this is local on target,
we may also need local on domain.} In our case, $\tau$ is \'etale and $\bbP$ is smooth morphisms.

\begin{defin}
\begin{enumerate}
\;
\item $F \in \Sh(\cC,\tau)$ is (-1)-geometric if it's representable ($F = \Spec^f(A)$, where $f$ stands for functor).
\item A morphism $f:F\to G$ is (-1)-geometric if for all $X \in \cC$ and all $h_X \to G$, $h_X \times_G F$ is representable.
In other words, there is some $Y$ making the following diagram cartesian.
\[
\begin{tikzcd}
h_Y \arrow{r}\arrow{d} & F\arrow{d} \\ h_X \arrow{r} & G
\end{tikzcd}
\]
\item An $n$-atlas of $F \in \Sh(\cC,\tau)$ is an epimorphism $\pi : U \to F$ such that $U$ is (-1)-geometric,
$\pi$ is $n-1$-geometric and $\pi$ is $n-1$-$\bbP$.
\item $F \in \Sh(\cC,\tau)$ is $n$-geometric if it has an $n$-atlas and $F \to F\times F$ is $n-1$-geometric.
\item A morphism $f:F \to G$ is $n$-geometric if for all $h_X \to G$, $h_X \times_G F$ is $n-1$-geometric.
\end{enumerate}
\end{defin}

\begin{rem}
If you start with affine schemes, 0-geometric is algebraic spaces.
\end{rem}

\begin{rem}
Let $F$ be a geometric stack, and let $\pi : U \to F$ be an $n$-atlas. Then $U^{\bullet} = \check{C}(U\to F)$ is a groupoid,
and each level $U^k$ is $n-1$-geometric. Moreover, $|U^{\bullet}| \simeq F$, and the transition morphisms are in $\bbP$.
The converse is also true: given a groupoid where transition morphisms are in $\bbP$, the geometric realization is a geometric
stack.
\end{rem}

\begin{rem}
In $\cC = d\Aff_k$, it's not necessary to ask $F \to F \times F$ to be $n-1$-geometric.
\end{rem}

\begin{rem}
There's always a cheap way of extending geometric properties of morphisms in $\cC$ to analogous properties of geometric
stacks. This goes by induction. But caution: this is not necessarily the best thing to do, e.g. properness has to be dealt with
more cleverly. 
\end{rem}

\begin{rem}
In the specific case $\cC = d\Aff_k$, given $F:d\Aff_k^{\op} \to \cS$, we consider the composition:
\[
\begin{tikzcd}
\Aff_k^{\op} \arrow[hook]{d} \arrow{r}{t_0(F)} & \cS \\ d\Aff_k^{\op} \arrow[swap]{ur}{F} &
\end{tikzcd}
\]
We call this
\textbf{truncation}. The truncation functor preserves geometric stacks, as well as finite limits. Moreover, $t_0(\Spec(A))
= \Spec^f(\pi_0(A))$.
\end{rem}

\begin{rem}
If $F$ is a derived geometric stack, then in general $F$ is not truncated. For example, take $A = k[x]$, 
$F = \Spec A$, $F(B) = \Map(\Spec B,\Spec A) = \Map_{sCR_k}(A,B) \cong B$, the underlying topological space of $B$. This can
be arbitrarily complicated. However, $t_0(F)$ always factors through $\cS^{\leq n}$ for some $n$. More precisely,
if $F$ is $n$-geometric, then $t_0(F)$ is $n+1$-truncated. The proof is simple, by induction on the geometric level.
\footnote{In fact, this is how most proofs go for geometric stacks.}
\end{rem}

\begin{rem}
When $\bbP$ is smooth, we call $\Geom(d\Aff_k, \tau_{\'et}, \bbP)$ \textbf{derived Artin stacks}. When $\bbP$ is \'etale,
we call $\Geom(d\Aff_k, \tau_{\'et}, \bbP)$ \textbf{ derived Deligne-Mumford stacks}.
\end{rem}

From now on, we only consider the case $\cC = d\Aff_k$.

\begin{lem}
$F \in \Geom(d\Aff_k, \tau_{\'et}, \bbP)$, there exists a functor of the form: 
\begin{align*}
F_{\'et} &\to t_0(F)_{\'et} \\
\big(G \overset{\'et}{\to} F\big) & \mapsto \big( t_0(G) \overset{\'et}{\to} t_0(F)\big),
\end{align*}
which is a Morita equivalence.
\footnote{This means that the categories of sheaves on them are equivalent.} Note that $F_{\'et}$ is the small
\'etale site, which considers derived affines mapping into $F$ via \'etale maps.
\end{lem}

\begin{cor}
\label{cor:stack_ncat}
For $F \in \Geom(d\Aff_k, \tau_{\'et}, \bbP)$, if $F$ is $n$-geometric, then $F_{\'et}$ is an $n$-category.
\end{cor}

\begin{rem}
Let $F,G \in \Geom(d\Aff_k, \tau_{\'et}, \bbP)$ and $f :F \to G$. The pullback $G_{lis-\'et} \to F_{lis-\'et}$
\footnote{These sites are called \textbf{lisse-\'etale}, which means that $\bbP$ is the class of smooth morphisms but the
topology $\tau$ is \'etale.} does not induce a geometric morphism of topoi. There is an adjunction:
\[	f^* : \Sh(G_{lis-\'et}, \tau_{\'et}) \to \Sh(F_{lis-\'et}, \tau_{\'et}) : f_*,	\]
but $f^*$ does not preserve finite limits. The reason for this, approximately, is that certain pullbacks don't exist
in the lisse-\'etale site. If we use the \'etale site, given two \'etale morphisms $X \to G$, $Y\to G$,
any morphism between them is forced to be \'etale as well. But this is no longer true when using the lisse-\'etale site.
This is a thorny problem, but it can be avoided if one is content with working with
coherent sheaves.
\end{rem}


\section{Comparison of approaches}

\begin{thm}
Let $dDM^{loc}$ be the $\infty$-category of structured derived DM-stacks $(\mathcal{X}, \mathcal{O}_{\mathcal{X}})$, where
$\mathcal{X}$ is $n$-localic for some $n$. Let $dDM^f := \Geom(d\Aff_k,\tau_{\'et}, \bbP_{\'et})$ be derived DM-stacks
obtained from the functor of points. Then there is an equivalence:
\[	dDM^{loc} \simeq dDM^f	.\]
\end{thm}
\begin{proof}
The construction $dDM^{loc} \to dDM^f$ takes $(\mathcal{X}, \mathcal{O}_{\mathcal{X}})$ and sends it to the functor
$F: dAff_k^{\op} \to \cS$, which maps $B$ to $\Map(\Spec^{\'et}(B), (\mathcal{X}, \mathcal{O}_{\mathcal{X}}))$.
$\mathcal{X}$ localic means that $F$ satisfies hyperdescent.

In the other direction, pick $F$ and send it to $\big(\Sh(F_{\'et},\tau_{\'et}), \mathcal{O}_F\big)$. Thanks to Corrolary
\ref{cor:stack_ncat} $\Sh(F_{\'et},\tau_{\'et})$ is n-localic.
\end{proof}




\section{Descent and infinitesimal theory}
How do we define $QCoh(F)$ for a sheaf $F$? We would like it to be the dashed arrow in the diagram: 
\[
\begin{tikzcd}
d\Aff_k^{\op}\arrow{d}\arrow{r}{QCoh} & \Pr^L \\
\Sh(d\Aff_k, \tau_{\'et})\arrow[dashed, swap]{ur}{\underline{QCoh}}
\end{tikzcd}
\]
Then one defines $\underline{QCoh}(F)$ as the
\textbf{mapping stack} $\underline{Map}(F,QCoh)$. In turn, this is defined as the adjoint of $- \times F$. 
That is, the functor of points of a mapping stack is:
\[	\Map(\Spec A, \underline{Map}(F,QCoh)) = \Map(\Spec A \times F, QCoh) .	\]

This is the same as setting $\underline{QCoh}(F) := \lim QCoh(A)$, where
the limit is taken over $\Spec(A) \to F$.

\begin{rem}
This makes sense for every sheaf $F$. If $F$ is geometric, then we can restrict to the lisse-\'etale site when taking the limit.
\end{rem}

\begin{defin}
Let $F \in \Sh(d\Aff_k, \tau_{\'et})$. Then:
\begin{enumerate}
\item Let $x : \Spec A \to F$, we say that $F$ \textbf{has a cotangent complex at} $x$ if there exists an object
$\bbL_{F,x} \in QCoh^-(A)$ such that for all $M \in QCoh(A)$, $\Map_{QCoh(A)}(\bbL_{F,x}, M) \simeq Der_F(A;M)$. 
Here the $F$\textbf{-linear derivations} $Der_F(A;M)$ are defined as the pullback:
\[
\begin{tikzcd}
Der_F(A;M)\arrow{d}\arrow{r} & \Map\big(\Spec(A\oplus M), F\big)\arrow{d}{0} \\ * \arrow[swap]{r}{x} & \Map(\Spec(A),F) .
\end{tikzcd}
\]
\item $F$ \textbf{has a global cotangent complex} if it has a cotangent complex at $x$ for every $x : \Spec(A) \to F$, and 
moreover there exists $\bbL_F \in \underline{QCoh}(F)$ such that for all $x$, $x^* \bbL_F \cong \bbL_{F,x}$. 
\end{enumerate}
\end{defin}

\begin{thm}
If $F$ is geometric, then it has a global cotangent complex.
\end{thm}

\begin{rem}
If $F$ is dDM, then $\bbL_F$ is connective, i.e. concentrated in non-positive degrees. However, if $F$ is Artin $n$-geometric,
then $\bbL_F$ is concentrated in degrees $(-\infty, n]$. \todo{$\pm 1$}
\end{rem}

