\chapter{Abelian Threefold}

Let's denote by $\eta : \mathcal{O}_M \to \bbL_M \otimes \bbT_M$ the counit and $\epsilon : \bbL_M \otimes \bbT_M \to \mathcal{O}_M$
the unit.
The most natural expressions for $\omega$ and $\iota_{\xi}$ are as follows.
\[
\begin{tikzcd}
p^*\fr g \otimes \wedge^2 \bbL_M \arrow{r}{\xi \otimes 1} \arrow[bend left]{rr}{\iota_{\xi}} &
\bbT_M \otimes \wedge^2 \bbL_M\arrow{r}{\epsilon} & \bbL_M
\end{tikzcd}
\]
\[
\begin{tikzcd}
\bbT_M\otimes \bbT_M \arrow{r}\arrow[bend left]{rrrr}{\omega} & p_*\mathcal{A} \otimes p_*\mathcal{A}[2] \arrow{r} & 
p_*(\mathcal{A}\otimes \mathcal{A})[2]
\arrow{r} & p_*\mathcal{O}_{M \times A}[2] \arrow{r} & \mathcal{O}_M[2-3]
\end{tikzcd}
\]

The diagram we want to find a lift in can therefore be written as:
\[
\begin{tikzcd}
\; & & & \mathcal{O}_M[-1]\arrow{d}{d_{dR}} \\
p^* \fr g \otimes \mathcal{O}_M \arrow[swap]{r}{\wedge^2 \eta}\arrow[dashed]{urrr}{\mu} & 
p^* \fr g \otimes \wedge^2 \bbT_M \otimes \wedge^2 \bbL_M \arrow[swap]{r}{\omega} &
p^* \fr g \otimes \wedge^2 \bbL_M \otimes \mathcal{O}_M[-1]\arrow[swap]{r}{\iota_{\xi}} & \bbL_M \otimes \mathcal{O}[-1] .
\end{tikzcd}
\]
