\chapter{VFC in Derived Differential/Algebraic Geometry}
\label{ch:infinity}

Talk by Benedict Morrissey.

\section{Big Picture}
We construct a functor $\dMan \to \Cat_{\infty}$, which associates to every derived manifold $M$ its category of Kuranishi
neighborhoods, i.e. pullback diagrams of the form:
\[
\begin{tikzcd}
U \arrow{r}\arrow{d} & \bbA^n \arrow{d}{f} \\
\pt \arrow{r} & \bbA^m .
\end{tikzcd}
\]
Here $U$ is an open chart of $M$ which can be locally written as $f^{-1}(0)$.

Then we construct the following chain of functors:
\[
\begin{tikzcd}
\KurNbd_M\arrow{r}\arrow[bend left]{rr}{F} & \LocTopInf\arrow{r} & \Map ( I, \dgVect ) .
\end{tikzcd}
\]
We define:
\[	\{C^{\vir}_{\bullet}(M) \to C^{\loc}_{\bullet}(M)\} := \colim F .\]
$I = \Delta^1$ is the $\infty$-category with 2 objects and 1 non-trivial 1-morphism.
\todo{Wasn't sure how to write the construction in a way that's manifestly functorial in $M$. Benedict's formulation uses diagrams
in $\LocTopInf$, $\Map(I,\dgVect)$. However the shape of the diagram depends on $M$, so I don't know how to write
this as a functor from $\dMan$.}

Now there's a map $C^{\loc}_{\bullet}(M) \to \Z$ \todo{is it a quasi-isomorphism?}. We define the virtual
fundamental class $[M]^{\vir}$ as the composition $C^{\vir}_{\bullet}(M)\to \Z$. (But see Remark \ref{rem:poincare}.)



\section{Construction}
\begin{defin}
The category of \textbf{Kuranishi neighborhoods} of $M$ is:
\[	\KurNbd_M = \dMan_{/M} \times_{\dMan} \Map(\Box, \dMan) \times_{\Map(\_, \dMan)} \Map(\pt, \Tdisc) 
\times_{\Map(\cdot, \dMan)} \Tdisc .	\]
\end{defin}

\todo{We need to define these squares and intervals precisely, with direction of arrows. Also, we need to quote the relevant
strictification results which give the higher homotopies for free once we specify data for the vertices and edges of the squares.}

Note that $M$ only appears in the leftmost factor, $\dMan_{/M}$.
Therefore the assignment $M \mapsto \KurNbd_M$ is an $\infty$-functor,
which follows from the fact that $M \mapsto \dMan_{/M}$ and fiber product are $\infty$-functors.

\begin{defin}
The category of \textbf{local topological information} $\LocTopInf$ is actually $\Map(\Box, \cS)$, which morally speaking
consists of maps between pairs of topological spaces $(A,B) \to (C,D)$. Here ``pairs'' is used in the sense of algebraic
topology, i.e. $B\to A$ and $D \to C$ are cofibrations.
\end{defin}

We construct a functor $\KurNbd_M \to \LocTopInf$ by sending each Kuranishi atlas to the map of pairs:
\[
\begin{tikzcd}
\big( U(\bbA^n), U(\bbA^n \setminus f^{-1}(0))\big)\arrow{r}{f} & \big( U(\bbA^m), U(\bbA^m \setminus \{0\})\big).
\end{tikzcd}
\]
Here $U: \dMan \to \cS$ is the forgetful functor; this will exist in whatever category we end up working in.

Finally, we construct a functor $\LocTopInf \to \Hom(I, \dgVect)$ by sending the map of pairs $f: (A,B) \to (C,D)$
to the induced morphism on singular chains:
\[
\begin{tikzcd}
C_{\bullet}(A,B) \arrow{r}{f_*} & C_{\bullet}(C,D) .
\end{tikzcd}
\]
\todo{We actually want $n-\bullet$ in the first factor and $m-\bullet$ in the second. How do we obtain the shifts by different
amounts? Look at how Pardon does it.}

\begin{lem}
This is an $\infty$-functor, which moreover preserves all colimits.
\end{lem}
\begin{proof}
In brief, $C_{\bullet}$ is the unique functor $\cS \to \dgVect$ which preserves colimits and sends $\pt \mapsto k$. That's
because $\cS$ is generated under colimits by $\pt$. In the case of pairs of spaces, the functor of relative singular
chains is defined by:
\[
\begin{tikzcd}
\Fun(I, \cS) \arrow{r}{C_{\bullet}} & \Fun(I, \dgVect) \arrow{r}{\text{cone}} & \dgVect .
\end{tikzcd}
\]
Here $C_{\bullet}$ is computed pointwise. Since $C_{\bullet}$ and $\text{cone}$ are $\infty$-functors, so are relative chains.
Moreover, colimits of presheaves can be computed pointwise, so relative chains also preserve colimits.
\end{proof}

\begin{rem}
This construction actually gives a cosheaf on $M$. That's because, for $V \to M$ a chart, $C_{\bullet}^{\vir}(V)$ will be computed
by a colimit over the category of Kuranishi neighborhoods which factor through $V$. This colimit maps naturally to the
colimit over $\KurNbd_M$, which computes $C_{\bullet}^{\vir}(M)$.
\end{rem}
\todo{Actually we need to prove that it satisfies descent.}


\section{Recovery of the colimit from finite atlas}
\begin{lem}
Assume that $M \in \dMan$ admits a Kuranishi atlas consisting of a single chart:
\[
\begin{tikzcd}
M\arrow{r}{i_M}\arrow{d} & \bbA^n\arrow{d}{f} \\ \pt \arrow{r} & \bbA^m .
\end{tikzcd}
\]
This chart is a final object in $\KurNbd_M$, and as such we have $C^{\vir}_{\bullet}(M) = C_{\bullet}\big(U(\bbA^n), 
U(\bbA^n \setminus f^{-1}(0)) \big)$ and $C^{\loc}_{\bullet}(M) = C_{\bullet}\big(U(\bbA^m), 
U(\bbA^m \setminus \{0\}) \big)$.
\end{lem}
\begin{proof}
Consider any other chart, given by $p: U \to M$ and:
\[
\begin{tikzcd}
U\arrow{r}{i_U}\arrow{d} & \bbA^{n_U}\arrow{d}{f_U} \\ \pt \arrow{r} & \bbA^{m_U} .
\end{tikzcd}
\]
We need to produce a map between the two charts. The idea should be to consider:
\[
\begin{tikzcd}
U\arrow{r}{(i_U,i_M \circ p)}\arrow{d} & \bbA^{n_U+n}\arrow{d}{f_U,f} \\ \pt \arrow{r} & \bbA^{m_U+m} ,
\end{tikzcd}
\]
and then project to the $\bbA^n$ and $\bbA^m$ factors. Unfortunately the composition is 0 on the rightmost column.
\todo{Work this out.}
\end{proof}

In general, $\KurNbd_M$ is a category which doesn't admit a final object. However, there exists a final system of objects
 - i.e. finitely many charts $U_i$ which cover the entire $M$. We should prove that the colimit of $F$ over the entire $\KurNbd_M$
can be recovered from the value of $F$ on these objects.



\section{To do}
To do ASAP:
\begin{enumerate}
\item In the Big picture section, understand how to write the VFC as a functor from $\dMan$.
\item Prove that we can recover the colimit over $\KurNbd_M$ from a finite atlas.
\item Fill in more details, as outlined in the to do notes. What else needs to be filled in?
\end{enumerate}

Then:

\begin{rem}
\label{rem:poincare}
So far we only defined $[M]^{\vir}$ as a map $C^{\vir}_{\bullet}(M)\to \Z$. We would like it to be an element in a
homology group of $M$, so that we can use it to integrate cohomology classes. John Pardon achieves this via
a Poincar\'e duality type quasi-isomorphism $H^{\vir}_{\bullet}(M) \simeq \big(\check{H}^{\bullet}(M)\big)^{\vee}$.
See the results of Section 4.3 in \cite{Pardon_An_algebraic_approach_2016}.
Can we prove this in our setting?
\end{rem}

Finally, the medium-term things which are more or less logically independent from the material of this chapter:
\begin{enumerate}
\item Define the appropriate category to replace $\dMan$ in all the above. This should:
\begin{itemize}
\item be small enough so that every object is quasi-smooth;
\item be large enough to accomodate certain mapping stacks, such as the moduli space of stable maps.
\end{itemize}
It suffices to construct a theory of derived DM stacks in smooth manifolds. We don't need Artin stacks, or indeed want them
- it's not clear that they admit VFCs, since nobody has come up with a theory that achieves this.
\item In this context, we need some version of Artin-Lurie representability to hold. This would guarantee the geometricity of
the mapping stacks we define.
\item Generalize to manifolds with boundary, maybe even corners.
\item Now we should be able to define Floer moduli spaces as objects in this category.
\item World peace?
\end{enumerate}
