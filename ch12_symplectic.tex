\chapter{HKR isomorphism and shifted symplectic structures}
Talk by Matei Ionita.

We have reached the ``structured'' part of our ``structured DAG'' plan for the semester. We want to talk about shifted
symplectic structures and give examples, following \cite{PTVV}. But first we discuss some prerequisites.

\section{Affine Stacks}
For this section we follow \cite{vezzosi2013autour}, Expos\'e 10. Recall that we defined derived affine schemes as
the category opposite to $\cdga_k^{\leq 0}$. We have the following adjoint pair:
\[
\begin{tikzcd}
dSt \arrow[shift left]{r}{\mathcal{O}} & \cdga_k^{\leq 0} \arrow[shift left]{l}{\Spec} .
\end{tikzcd}
\]
The functor of global sections $\mathcal{O}$ is defined as the unique functor such that $\mathcal{O}(\Spec(A)) = A$ and
commutes with colimits. We want to extend this adjunction to cdga's that are not necessarily connective:
\[
\begin{tikzcd}
dSt \arrow[shift left]{r}{\mathcal{O}} & \cdga_k \arrow[shift left]{l}{\Spec} .
\end{tikzcd}
\]
This $\Spec$ functor is defined by $\Spec(A)(R) = \Map_{\cdga_k}(R,A)$, and then $\mathcal{O}$ is defined as a left adjoint,
like before.

\begin{defin}
A derived stack is \textbf{affine} if it is in the essential image of $\Spec$. Some authors use \textbf{co-affine}
to refer to the essential image of $\cdga_k^{\geq 0}$ under $\Spec$.
\end{defin}

In particular, we can regard topological spaces as constant sheaves, and thus as elements of $dSt$. Certain topological
spaces are co-affine. Recall, for example, the Eilenber-Maclane spaces $K(\Z,n) \simeq B^n\bbG_a$.

\begin{prop}
$B^n\bbG_a \simeq \Spec(\Sym k[-n])$.
\end{prop}
\begin{proof}
The proof in \cite{vezzosi2013autour} is by induction on pushout squares.
\end{proof}

This leads to a description of $\mathcal{O}(X)$ for $X$ the constant sheaf associated to a CW complex.

\begin{thm}
\label{thm:global_cochain}
If $X$ is a CW complex, then there is a quasi-isomorphism $\mathcal{O}(X) \simeq C^*(X)$, where the latter is the singular
cochain complex.
\end{thm}
\begin{proof}
Write $X$ as the colimit of some diagram of points: $X = \colim \Spec k$. Then consider the map:
\[	C^*(X) \simeq C^*(\colim \Spec(k)) \to \lim C^*(\Spec(k)) \simeq \lim k \simeq \mathcal{O}(X) .	\]
This induces an isomorphism in cohomology:
\[	H^n(X) \cong [X,B^n\bbG_a] \cong \pi_0 \Map_{\cdga_k}(\Sym k[-n], \mathcal{O}(X) ) \simeq H^n(\mathcal{O}(X)).	\]
\end{proof}



\section{HKR isomorphism}
In this section we follow \cite{TV_HKR}; a less technical exposition of these ideas is in \cite{BZN_loop}. We want
to prove an equivalence between cdga's with an $S^1$ action and, roughly speaking, cdga's with an extra derivation.

\begin{defin}
Let $S^1-\cdga_k^{\leq 0}$ denote connective cdga's with an action of $S^1$. This definition can be made rigorous by realizing
the connetive cdga's as simplicial commutative algebras via Dold-Kan, and realizing $S^1$ as a simplicial group, which can
act on the simplicial algebras.
\end{defin}

\begin{eg}
For $A \in \cdga_k^{\leq 0}$, the free $S^1$-algebra is $S^1 \otimes A \simeq A \otimes_{A\otimes A} A$, also known as the
Hochschild chain complex. The functor $A \mapsto S^1 \otimes A$ is left adjoint to the forgetful functor $S^1-\cdga_k^{\leq 0}
\to \cdga_k^{\leq 0}$.
\end{eg}

The second category of interest is the following.

\begin{defin}
Let $\epsilon-\cdga_k^{\leq 0}$ be the category of cdga's equipped with an extra differential $\epsilon : A \to A[-1]$. This
definition can be made more rigorous by considering the over-category $k[\epsilon] - dg-mod_k$, where $|\epsilon| = -1$. (These
are called \textbf{mixed complexes} in \cite{PTVV}.)
This category has a symmetric monoidal structure, arising from the co-multiplication $k[\epsilon] \to k[\epsilon]
\otimes k[\epsilon]$. Then we define $\epsilon-\cdga_k^{\leq 0}$ to be the monoids in $k[\epsilon] - dg-mod_k$.
\end{defin}

\begin{eg}
\label{eg:de_rham}
For $A \in \cdga_k^{\leq 0}$, the free $\epsilon$-algebra is \textbf{the de Rham complex} $DR(A) = \Sym_A(\bbL_A[1])$,
where the extra differential is the de Rham differential.
\[
\begin{tikzcd}
A^{-3}\arrow{d} & \bbL^{-2}\arrow{d} & (\wedge^2 \bbL)^{-1}\arrow{d} & (\wedge^3 \bbL)^0 \\
A^{-2}\arrow{d}\arrow{ur}{d_{dR}} & \bbL^{-1}\arrow{d}\arrow{ur}{d_{dR}} & (\wedge^2 \bbL)^{0}\arrow{ur}{d_{dR}} & \\
A^{-1}\arrow{d}\arrow{ur}{d_{dR}} & \bbL^0\arrow{ur}{d_{dR}} & & \\
A^0\arrow{ur}{d_{dR}} & & &
\end{tikzcd}
\]
The functor $A \mapsto DR(A)$ is left adjoint to the forgetful functor $\epsilon-\cdga_k^{\leq 0}
\to \cdga_k^{\leq 0}$.
\end{eg}

We can now state and sketch the proof of the main theorem in \cite{TV_HKR}.

\begin{thm}
There is an equivalence $\phi$ which commutes with the forgetful functors:
\[
\begin{tikzcd}
S^1 - \cdga_k^{\leq 0}\arrow{rr}{\phi}\arrow{dr} & & \epsilon - \cdga_k^{\leq 0}\arrow{dl} \\
 & \cdga_k^{\leq 0} & 
\end{tikzcd}
\]
\end{thm}
\begin{proof}
   
\begin{enumerate}
\item Note first that $S^1-\cdga_k^{\leq 0} \simeq \Fun(BS^1, \cdga_k^{\leq 0})$, where $BS^1$ is the category with a single
object and 1-morphisms for each element of $S^1$.
\item Passing to global sections gives a functor $\phi_3$:
\[
\begin{tikzcd}
\Fun(BS^1, \cdga_k^{\leq 0})\arrow{r}{\phi_3}\arrow[swap]{d}{\text{eval}} & k[u]-\cdga_k\arrow{d}{k\otimes_{k[u]} -} \\
\cdga_k^{\leq 0}\arrow[hook]{r} & \cdga_k .
\end{tikzcd}
\]
Some comments are necessary here. First, the global sections of $BS^1$ are $k[u]$ with $|u| = 2$, due to Theorem
\ref{thm:global_cochain}. Second, working with non-connective cdga's is necessary in the right hand column, because
$\cdga_k^{\leq 0}$ is not closed under limits. Third, the diagram is \emph{not} commutative, but becomes so when restricted
to $\cdga^+_k$, the category of cdga's which are bounded on the left. The proof of commutativity uses an increasing induction
on Postnikov towers, and the existence of a base case for the induction requires this extra constraint. Moreover,
the restriction $\phi_3^+$ to left-bounded cdga's is fully faithful, with essential image given by $k[u]-\cdga_k^{+,\leq 0}$.
\item $k[\epsilon]$ is Koszul dual to $k[u]$, which gives a map:
\[	\Ext_{k[\epsilon]}(k,-) : k[\epsilon] - dg-mod_k \to k[u] - dg-mod_k.	\]
Moreover, this map preserves the monoidal structure on the two categories, so it induces a map $\phi_4: \epsilon-\cdga_k^{\leq 0}
\to k[u] -\cdga_k$, which fits into a diagram:\todo{do this argument more carefully}
\[
\begin{tikzcd}
\epsilon-\cdga_k^{\leq 0}\arrow{r}{\phi_4}\arrow{d} & k[u]-\cdga_k\arrow{d}{k\otimes_{k[u]} -} \\
\cdga_k^{\leq 0}\arrow[hook]{r} & \cdga_k .
\end{tikzcd}
\]
The comments about commutativity, full faithfulenss and essential image from the previous step also apply here.
\item Putting these together, we obtain a commutative diagram:
\[
\begin{tikzcd}
\Fun(BS^1, \cdga_k^{+,\leq 0})\arrow{r}{\phi_3^+}\arrow[swap]{d}{\text{eval}} & k[u]-\cdga^+_k\arrow{d}{k\otimes_{k[u]} -}
& \epsilon-\cdga_k^{+,\leq 0}\arrow[swap]{l}{\phi_4^+}\arrow{d}{\text{forget}} \\
\cdga_k^{+,\leq 0}\arrow[hook]{r} & \cdga_k^+ & \cdga_k^{+,\leq 0}\arrow[hook]{l} .
\end{tikzcd}
\]
$\phi_3^+$ and $\phi_4^+$ are fully faithful with the same essential image, which gives an equivalence $\phi$ which
fits into the commutative diagram:
\[
\begin{tikzcd}
\Fun(BS^1, \cdga_k^{+,\leq 0})\arrow{r}{\phi^+}\arrow[swap]{d}{\text{eval}} & \epsilon-\cdga_k^{+,\leq 0}\arrow{d}{\text{forget}} \\
\cdga_k^{+,\leq 0}\arrow[hook]{r} &  \cdga_k^{+,\leq 0} .
\end{tikzcd}
\]
$\phi^+$ is an equivalence, so in particular it commutes with colimits. Therefore it extends to an equivalence between
$\Fun(BS^1, \cdga_k^{\leq 0})$ and $\epsilon-\cdga_k^{\leq 0}$.
\end{enumerate}
\end{proof}

\begin{thm}
$\phi$ also commutes with the free functors, i.e. $\phi(S^1\otimes A) \simeq DR(A)$.
\[
\begin{tikzcd}
S^1 - \cdga_k^{\leq 0}\arrow{rr}{\phi} & & \epsilon - \cdga_k^{\leq 0} \\
 & \cdga_k^{\leq 0}\arrow[swap]{ur}{DR}\arrow{ul}{S^1 \otimes} & 
\end{tikzcd}
\]
\end{thm}
\begin{proof}
This just follows from category theory and the fact that $\phi$ commutes with the forgetful functors.
\end{proof}

\begin{cor}[HKR isomorphism]
\label{cor:HKR}
Let $X \in dSt$. \todo{in the paper they state this just for classical schemes, but it clearly works in more generality.
what assumptions do we need on the stack? finite presentation?}
\begin{enumerate}
\item $\mathcal{O}_X \otimes_{\mathcal{O}_X\otimes\mathcal{O}_X} \mathcal{O}_X \simeq \Sym_{\mathcal{O}_X} (\bbL_X[1])$.
\item If $X$ is a smooth truncated scheme, this specializes to $\mathcal{O}_X \otimes_{\mathcal{O}_X\otimes\mathcal{O}_X} 
\mathcal{O}_X \simeq \Sym_{\mathcal{O}_X} (\Omega^1_X[1])$. This is the classical version of the HKR isomorphism, enhanced
to an isomorphism of algebras, as opposed to just complexes.
\item Let $hS^1$ denote homotopy fixed points of the $S^1$ action, and ev denote the even part of de Rham cohomology. Then:
\[	(\mathcal{O}_X \otimes_{\mathcal{O}_X\otimes\mathcal{O}_X} \mathcal{O}_X)^{hS^1} \simeq H_{dR}^{ev}(X).	\]
\end{enumerate}
\end{cor}
\begin{proof}
The first statement is just saying that we can apply the equivalence $\phi$ at the level of sheaves.
\todo{say something about the last one, maybe even negative cyclic complexes}
\end{proof}

\begin{rem}
\label{rem:loop_stack}
For $X \in dSt$, we can define the \textbf{derived loop stack} $\mathscr{L}X$ as the mapping stack $\Map_{dSt}(S^1,X)$.
Using tensor-hom adjunction, $\mathcal{O}(\mathscr{L}X) \simeq S^1 \otimes X$. On the other side of the HKR
isomorphism, $\Sym_{\mathcal{O}_X} (\bbL_X[1])$ has an interpretation as $\mathcal{O}(T[1]X)$, global sections of the
shifted tangent bundle of $X$. This gives an identification $\mathscr{L}X \simeq T[1]X$. 
The slogan is that ``forms on $X$ are functions on $\mathcal{L}X$''. More about this in \cite{BZN_loop}.
\end{rem}


\section{Shifted symplectic structures}
In all remaining sections we follow \cite{PTVV}. For $X$ a smooth scheme over $k$, a symplectic structure is a closed
2-form
$\omega \in H^0(X,\Omega^{2,cl}_X)$, which induces an isomorphism $\Theta_\omega : T_X \simeq \Omega^1_X$. We want to
generalize this for $X$ a derived Artin stack over $k$. Morally speaking, we want an n-shifted symplectic structure
to be an element $\omega \in H^0(X,\wedge^2\bbL_X [2])$, which is closed and induces an isomorphism
$\Theta_\omega : \bbT_X \to \bbL_X[n]$. The main technical difficulty is defining what it means for a form to be closed:
instead of $d_{dR} \omega = 0$, we want $d_{dR} \omega$ to be exact, meaning that it's in the image of the internal
differential of the complex $\wedge^3\bbL_X[3]$. (See the diagram in \ref{eg:de_rham}; the internal differentials are the
vertical maps there.) In particular, ``closed'' is now a structure, instead of a property.\todo{explain this better} 

Using the last part of Corollary \ref{cor:HKR}, as well as remark \ref{rem:loop_stack}
we can translate the closed condition into the condition of being a homotopy fixed point for the action of $S^1$ on $\mathscr{L}X$. 
However, the difficulty in working with $\mathscr{L}X$ is that loops do not satisfy descent in the \'etale (or smooth) topology.
As such, one needs to work with infinitesimal loops, which is achieved in \cite{BZN_loop} by introducing the completion
of $\mathscr{L}X$ around the constant loops, i.e. the zero section $X \to \mathscr{L}X$.

The authors of \cite{PTVV} prefer a different approach, based on the negative cyclic complex, where descent is immediate
from the definition of forms and closed forms, see \ref{prop:forms_descent}.

\begin{defin}
Let $E$ be an $\epsilon$-dg-module (or mixed complex) over $k$. The \textbf{negative cyclic complex} of $E$ is a dg-module $NC(E)$
over $k$, given by:
\[	NC^n(E) = \prod_{i\geq 0} E^{n-2i} .	\]
The differential $D$ is the sum of $\epsilon$ and the internal differential $d$:
\[	D\big(\{m_i\}\big)_j = \epsilon m_{j-1} + d m_j .	\]
\end{defin}

Consider now the \textbf{weight grading} on $E$, which is $E = \oplus_p E(p)$ such that $\epsilon : E(p) \to E(p+1)$.
We consider a variant of the negative cyclic complex which uses the weight grading: $NC^n(E)(p) = \prod_{i\geq 0}
E^{n-2i}(p+i)$.\footnote{When talking about complexes, round parantheses always refer to the weight grading,
while square brackets refer to the cohomological grading. The confusing bit is that $(p)$ means that we're isolating
the weight $p$ subspace, while $[n]$ means that we're shifting the cohomological degree by $n$.}

\begin{defin}
The \textbf{weighted negative cyclic complex} is:
\[	NC^w(E) = \oplus_p NC(E)(p).	\]
In other words, we have a double complex whose degree $n$, weight $p$ part is $\prod_{i\geq 0}
E^{n-2i}(p+i)$.
\end{defin}

\todo{give some intuition about how this is related to the corollary in HKR}

\begin{rem}
\label{rem:underlying_form}
There is a natural map $NC^w(E) \to E$, given by projection to the $i=0$ component:
\[	NC^w(E)^n(p) = \prod_{i\geq 0} E^{n-2i}(p+i) \to E^n(p) .	\]
Shortly we will interpret this as assigning to a closed $p$-form its underlying $p$-form.
\end{rem}

Recall that, for $A \in \cdga_k^{\leq 0}$, we have the de Rham complex $DR(A) = \oplus_p(\wedge^p \bbL_A)[p]$.
\begin{defin}
The space of \textbf{p-forms of degree n} on $A$ is\footnote{We're using Dold-Kan to identify a connective complex
with a simplicial set, and then taking geometric realization to get a space.}:
\[	\mathcal{A}^p(A,n) = \big| \wedge^p \bbL_A [n-p] \big| \in \cS.	\]
The space of \textbf{closed p-forms of degree n} on $A$ is:
\[	\mathcal{A}^{p,cl}(A,n) = \big|NC^w(DR(A))[n-p](p) \big| \in \cS.	\]
\end{defin}

\begin{rem}
One may expect $n$-shifted $p$-forms to require a shift by $n$, instead of the $n-p$ that appears in the definition.
But note that a shift in $p$ is already present in the de Rham complex, as well as its associated weighted negative cyclic
complex. This brings the total shift to $n-p$.\todo{wait, this doesn't actually work yet, fix it}
\end{rem}

\begin{rem}
The map in Remark \ref{rem:underlying_form} induces a map on the geometric realizations $\mathcal{A}^{p,cl}(A,n)
\to \mathcal{A}^p(A,n)$, which associates to a closed $p$-form its underlying $p$-form.
\end{rem}

\begin{defin}
For $w \in \mathcal{A}^p(A,n)$, the homotopy fiber $K(w)$ of $\mathcal{A}^{p,cl}(A,n) \to \mathcal{A}^p(A,n)$ is
\textbf{the space of keys} of $w$.
\end{defin}

As advertised, this formalism makes it easy to prove:
\begin{prop}
\label{prop:forms_descent}
$\mathcal{A}^p(-,n)$ and $\mathcal{A}^{p,cl}(-,n)$ are derived stacks for the \'etale topology.
\end{prop}
\begin{proof}
Backtracking through two pairs of adjoint functors we have:
\[	\mathcal{A}^p(A,n) \simeq \Map_{dg-mod_k}(k, \wedge^p\bbL_A[n]).	\]
This reduces the problem to showing that $A \mapsto \wedge^p\bbL_A[n]$ satisfies descent, which it does, because it's
quasi-coherent. Similarly, letting $k(p)$ denote the mixed complex with just a copy of $k$ in degree 0 and weight $p$,
\[	\mathcal{A}^{p,cl}(A,n) = |NC^w(DR(A))[n-p](p)| \simeq \Hom_{\epsilon-dg-mod_k}\big(k(p),DR(A)[n-p]\big)
= \Hom_{\epsilon-dg-mod_k} \big(k(p), \oplus_q \wedge^q \bbL_A[n-p+q].	\]
The middle equivalence follows from Corollary 1.4 in \cite{PTVV}. Hence this problem is also reduced to descent for the
cotangent complex.
\end{proof}

This proposition allows us to globalize the definitions. For $F\in dSt_k$, let:
\begin{align*}
\mathcal{A}^p(F,n) &= \Map_{dSt_k}\big(F,\mathcal{A}^p(-,n)\big), \\
\mathcal{A}^{p,cl}(F,n) &= \Map_{dSt_k}\big(F,\mathcal{A}^{p.cl}(-,n)\big).
\end{align*}

We also have the following nice description for $\mathcal{A}^p(F,n)$, but unfortunately no such thing exists for
$\mathcal{A}^{p,cl}(F,n)$.

\begin{prop}
\label{prop:forms_maps}
Let $F \in dSt_k$, then $\mathcal{A}^p(F,n) \simeq \Map_{QCoh(F)}\big(\mathcal{O}_F,\wedge^p\bbL_F[n]\big)$.
\end{prop}
\begin{proof}
The idea is to induct on the $m$-geometricity level of $F$. That is, choose an atlas for $F$ in terms of $m-1$-geometric
stacks $\{X_{\alpha}\}$, and look at the following commutative diagram:
\[
\begin{tikzcd}
\Map_{QCoh(F)}\big(\mathcal{O}_F,\wedge^p\bbL_F[n]\big) \arrow{d}\arrow{r} &
\lim \Map_{QCoh(X_{\alpha})}\big(\mathcal{O}_{X_{\alpha}},\wedge^p\bbL_{X_{\alpha}}[n]\big)\arrow{d} \\
\mathcal{A}^p(F,n) \arrow{r} & \lim \mathcal{A}^p(X_{\alpha},n) .
\end{tikzcd}
\]
The bottom map is an equivalence by descent, the right map is an equivalence by the inductive hypothesis. The top map
is also an equivalence, but this is more subtle, see \cite{PTVV} for the details. Therefore the left map is an equivalence.
\end{proof}

Let $F$ be a derived Artin stack, locally finitely presented over $k$. Then $\bbL_F$ is perfect, so in particular it's
dualizable in $QCoh(F)$. The dual is the tangent complex $\bbT_F$.
Using Proposition \ref{prop:forms_maps}, $\omega \in \mathcal{A}^2(F,n)$ determines a map $\mathcal{O}_F \to
\wedge^2 \bbL_F[n]$, which is dual to a map $ \Theta_{\omega}: \bbT_F \to \bbL_F[n]$.

\begin{defin}
$\omega \in \mathcal{A}^2(F,n)$ is \textbf{non-degenerate} if $\Theta_{\omega}$ is a quasi-isomorphism. The space of
\textbf{n-shifted symplectic structures} on $F$ is the pullback of the diagram:
\[
\begin{tikzcd}
Symp(F,n)\arrow{d}\arrow{r} & \mathcal{A}^{2,cl}(F,n)\arrow{d} \\
\mathcal{A}^2(F,n)^{nd}\arrow{r} & \mathcal{A}^2(F,n).
\end{tikzcd}
\]
\end{defin}

\begin{rem}
$\mathcal{A}^2(F,n)^{nd}$ is a union of path components of $\mathcal{A}^2(F,n)$, so symplectic structures are the same as
closed 2-forms whose underlying 2-forms live in these particular components.
\end{rem}

\begin{rem}
If $\bbL_X$ has amplitude $(-m,n)$, then shifted symplectic structures on $X$ can only exist in degree $m-n$.
\end{rem}

\begin{eg}
If $X$ is a smooth underived scheme, then $\bbL_X = \Omega^1_X[0]$, so the only shifted symplectic structures on $X$
are symplectic structures in the usual sense.
\end{eg}

\begin{rem}
For low values of $n$, an $n$-shifted symplectic structure can be seen as pairing the negative degree terms in $\bbL_X$
(the derived structure) with the positive degree terms (the stacky structure).
\end{rem}


\section{Examples: BG and Perf}
For our first first nontrivial example, we take $X = BG$, for $G$ reductive. Let $\fr g$ be the Lie algebra of $G$.

\begin{prop}
There exist 2-shifted symplectic structures on $BG$. Moreover, homotopy classes of such correspond to non-degenerate 
invariant bilinear forms on $\fr g$.
\end{prop}
\begin{proof}
Recall that $\bbL_{BG} \simeq g^* [-1]$, so that $\bbT_{BG} \simeq g[1]$. It's immediate that non-degenerate
$p$-forms of degree $n$ cannot exist unless $n=2$. Let's see what the spaces of forms look like.

We proved that $\mathcal{A}^p(-,n)$ satisfies descent; this is equivalent to the fact that $DR(-)$ satisfies descent.
Therefore we can talk about $DR(BG)$, obtained by taking global sections on $BG$, which corresponds to taking
invariants under $G$. We obtain $DR(BG) \simeq (\Sym_k \fr g^*)^G$, concentrated
in degree 0, and where $\fr g^*$ has weight 1.
\[
\begin{tikzcd}
0 & 0 & 0 & 0 \\
k\arrow{ur}{d_{dR}} & \fr g^*\arrow{ur}{d_{dR}} & \Sym^2 \fr g^*\arrow{ur}{d_{dR}} & \\
0 & 0 & 0 & 
\end{tikzcd}
\]
Now recall that $\mathcal{A}^p(BG,n) = |DR(BG)[n-p](p)|$; in particular:
\[	\pi_i\big(\mathcal{A}^p(BG,n)\big) = H^{-i} \tau^{\leq 0} DR(BG)[n-p](p) = H^{-i} \tau^{\leq 0} (\Sym^p \fr g^*)^G [n-p]	.	\] 
This means that we have:
\begin{equation}
\label{eq:desc_forms}
	\mathcal{A}^p(BG,n) = 
\left\{ \begin{array} {ll} * & \text{if } n<p,  \\ 
K\big((\Sym^p \fr g^*)^G, n-p\big) & \text{if } n\geq p .  \end{array} \right.
\end{equation}
The interesting case is $n=p$, where $\pi_0\big(\mathcal{A}^p(BG,n)\big) \cong (\Sym^p \fr g^*)^G$; in all other cases,
$\pi_0\big(\mathcal{A}^p(BG,n)\big) = 0$.

The de Rham differential is identically 0 for degree reasons. This means that all forms admit a closed structure. However, this
need not be canonical. For example, $\pi_0\big(\mathcal{A}^1(BG,3)\big) = 0$, but $\pi_0\big(\mathcal{A}^{1,cl}(BG,3)\big) \cong 
(\Sym^2 \fr g)^G$. This can be seen from the global sections of the de Rham complex:
\[
\begin{tikzcd}
k & \fr g^* & \Sym^2 \fr g^*\arrow{d} \\
0 & 0 & 0 \\
0 & 0\arrow{ur}{d_{dR}} & 0 
\end{tikzcd}
\]

Let's determine the spaces of closed forms in more detail:
\[	\mathcal{A}^{p,cl}(BG,n) = \big| NC^w(DR(BG))[n-p](p)\big|.	\]
Using the definition of $NC^w$, and then the particular form that $DR(BG)$ takes:
\begin{align*}
	NC^w(DR(BG))^m = \oplus_p \prod_{i\geq 0} DR(BG)^{m-2i}(p+i) &= \left\{ \begin{array} {ll}
0 & \text{for m odd}, \\ \oplus_{p\geq - \frac{m}{2}} DR(BG)^0(p+\frac{m}{2}) & \text{for m even}
\end{array} \right. \\
&= \left\{ \begin{array} {ll}
0 & \text{for m odd}, \\ \oplus_{p\geq - \frac{m}{2}} \big(\Sym^{p+j} \fr g^*\big)^G & \text{for m even}
\end{array} \right.
\end{align*}
We may as well write $m=2j$, and remember that for odd $m$ the result is 0:
\[	NC^w(DR(BG))^{2j} = \oplus_{p\geq -j} \big(\Sym^{p+j} \fr g^*\big)^G[-2j].	\]
The final form of the weighted negative cyclic complex is, then:
\[	NC^w(DR(BG)) = \oplus_j \oplus_{p\geq -j} \big(\Sym^{p+j} \fr g^*\big)^G[-2j].	\]
Finally, we need to shift by $n-p$, take the weight $p$ part, and take geometric realization:
\[	\mathcal{A}^{p,cl}(BG,n) = \big|  \oplus_{j\geq -p} \big(\Sym^{p+j} \fr g^*\big)^G[n-p-2j] \big|. \]

It's a bit hard to give a complete description of these spaces, analogous to \ref{eq:desc_forms}. But we make a few comments:
\begin{enumerate}
\item If $n<p$, the entire negative cyclic complex gets shifted into positive degrees, which are killed by geometric realization.
Hence for $n<p$, $\mathcal{A}^{p,cl}(BG,n) = * = \mathcal{A}^{p}(BG,n)$.
\item To have nontrivial $\pi_0$ we need $n-p \geq 0$ and even. In this case, $\pi_0\big(\mathcal{A}^{p,cl}(BG,n)\big)
\cong \big(\Sym^{\frac{p+n}{2}} \fr g^*\big)^G$. In particular, $\pi_0\big(\mathcal{A}^{2,cl}(BG,2)\big)
\cong \big(\Sym^{2} \fr g^*\big)^G$.
\end{enumerate}
All this was quite messy, but at least the non-degeneracy condition is what you'd expect.
\end{proof}

\begin{rem}
\label{rem:symp_trace}
If $G \subset GL(n)$, there's a canonical 2-shifted symplectic structure coming from the trace $\Tr(\text{mult})$.
\end{rem}

We move on to an example which is universal in some sense. Recall the stack $\Perf$ from 
Chapter \ref{chap:stack_perf}, and in particular the description of its tangent spaces in Theorem \ref{thm:tangent_perf}:
$T_E \Perf \simeq \End(E)[1]$. We use here without proof the global description of the tangent bundle, which is as
follows. Let $\mathcal{E}$ be the universal perfect complex over $\Perf$, which is classified by the identity
map $\Perf \to \Perf$. Then $\bbT_{\Perf} \simeq \End(\mathcal{E}, \mathcal{E})[1]$.

\begin{thm}
\label{thm:symp_perf}
There is a 2-shifted symplectic structure on $\Perf$.
\end{thm}
\begin{proof}
The closed 2-form on $\Perf$ is given by the Chern character, which is constructed in \cite{TV_Chern}. We first discuss
the underlying 2-form and the nondegeneracy condition. 

Recall from \ref{} the Atiyah class:
\[	a_\mathcal{E} : \mathcal{E} \to \mathcal{E} \otimes_{\mathcal{O}_\Perf} \bbL_{\Perf}[1] .	\]
Iterating on this construction, we obtain:
\[	a_\mathcal{E}^i : \mathcal{E} \to \mathcal{E} \otimes_{\mathcal{O}_\Perf} \wedge^i \bbL_{\Perf}[i] .	\]
Since $\mathcal{E}$ is dualizable \todo{say sth about this}, we can consider the dual map, which we denote by the same
symbol:
\[	a_\mathcal{E}^i : \mathcal{O}_{\Perf} \to \mathcal{E}^* \otimes \mathcal{E} \otimes_{\mathcal{O}_\Perf} \wedge^i \bbL_{\Perf}[i].	\]
Composing with the trace map $\mathcal{E}^* \otimes \mathcal{E} \to \mathcal{O}_\Perf$, we obtain:
\[	\Ch_i(\mathcal E) = \frac{\Tr(a_\mathcal{E}^i)}{i!} \in \Hom\big(\mathcal{O}_\Perf,\wedge^i \bbL_{\Perf}[i]\big) 
\cong H^i\big(\wedge^i \bbL_{\Perf}\big) .	\]
Taking $i=2$ we otain an element of $H^2(\wedge^2 \bbL_{\Perf})$, which is the underlying 2-form that we are looking after.

Now we use the fact that $\bbT_{\Perf} \simeq \End(\mathcal{E})[1]$, so that
$\bbL_\Perf[1] \simeq \End(\mathcal{E})$. The Atiyah class is, then, the adjoint of the multiplication morphism:
\[	\text{mult} : \End(\mathcal{E}) \otimes \End(\mathcal{E}) \to \End(\mathcal{E}),	\]
and we can identify $\Ch_2(\mathcal{E})$ with $\Tr(\text{mult})/2$, which is nondegenerate because the trace is. (Recall that
the trace is defined as adjoint to the identity map $\mathcal{E} \to \mathcal{E}$.)

It remains to lift $\Ch_2(\mathcal{E})$ to the negative cyclic complex:
\[
\begin{tikzcd}
\; & NC \arrow{d} \\
\Perf \arrow{r}{\Ch_2}\arrow[dotted]{ur} & DR .
\end{tikzcd}
\]
\todo{finish this}
\end{proof}

\begin{rem}
Note that there is an embedding $i: BGL(n) \to \Perf$, obtaining by regarding vector bundles as a sub-category of the perfect
complexes. Theorem \ref{thm:symp_perf} gives a 2-shifted symplectic structure $\omega_{\Perf}$ on $\Perf$, while 
\ref{rem:symp_trace} gives
an explicit form for a 2-shifted symplectic structure $\omega_{BGL(n)}$ on $BGL(n)$. Up to a numerical factor, both 
expressions are the trace
of a multiplication map between endomorphisms. It follows that $\omega_{BGL(n)} = i^* \omega_{\Perf}$.
\end{rem}



\section{Examples: Mapping Stacks}
For the next example, we build towards the following AKSZ-type statement, made precise in Theorem \ref{thm:symp_map}.
If $F \in dSt_k$ is equipped with an n-shifted symplectic structure, and $X \in dSt_k$ 
is equipped with, roughly, a fundamental class in degree $d$, then the mapping stack $\Map_{dSt}(X,F)$ admits an
$n-d$ shifted symplectic structure. Together with Theorem \ref{thm:symp_perf}, this proves that various moduli stacks of
bundles and complexes admit shifted symplectic structures. (See Corollary \ref{cor:symp_moduli}.)

\begin{rem}
Let us first give a heuristic idea for the AKSZ construction, by working with $C^{\infty}$ manifolds. Let $M$ be a compact
manifold of dimension $d$, and $(N,\omega)$ a symplectic manifold. Consider the evaluation map:
\[
\begin{tikzcd}
	M \times C^{\infty}(M,N) \arrow{r}{eval} & N .
\end{tikzcd}
\]
Pullback and integration along $M$ give a map:
\begin{align*}
\Omega_N^p &\to \Omega^{p-d}_{C^{\infty}(M,N)} \\
\alpha &\mapsto \int_M eval^* \alpha.
\end{align*}
In particular applying it to $\omega$ gives $\int_M eval^*\omega \in \Omega^{2-d}$. \todo{how can this be symplectic if it's no
longer a 2-form?}
\end{rem}

To imitate this strategy in the derived context, we need to define appropriate notions of orientability and integration
along the fibers.

\begin{defin}
For any $X \in dSt_k$, let $X_A$ denote $X \times \Spec A$.
\footnote{If, like me, you're not very good with stacks, here's some intuition for considering $X_A$. When making statements
about the ``points'' of a stack, e.g. in the proof of Theorem \ref{thm:symp_map}, it's not sufficient to consider $k$-valued
points, but rather those valued in an arbitrary derived affine. That's because stacks are defined as functors on
$d\Aff$, so information about them is complete only when we've probed with all derived affines.} 
We say that $X$ is $\mathcal{O}$\textbf{-compact} if:
\begin{enumerate}
\item $\mathcal{O}_{X_A}$ is a compact object in $QCoh(X_A)$;
\item for any perfect complex $E$ on $X_A$, the $A$-dg-module $C(X,E) = \Hom(\mathcal{O}_{X_A},E)$ is perfect.
\footnote{This is the same as $p_* E$, where $p:X\times \Spec A \to \Spec A$ is the projection.}
\end{enumerate}
\end{defin}

The property of being $\mathcal{O}$-compact buys us the following.
\begin{lem}
If $X$ is $\mathcal{O}$-compact, there is a natural transformation:
\[	\kappa_X : DR( - \times X) \to DR(-) \otimes_k C(X,\mathcal{O}_X).	\]
\end{lem}
\begin{proof}
A brief explanation for this is as follows. $C(X,\mathcal{O}_X)$ is perfect over $k$ by the $\mathcal{O}$-compactness
hypothesis, so the functor $E \mapsto \otimes_k C(X,\mathcal{O}_X)$ commutes with limits. Similarly, $DR$ sends colimits
to limits (this is descent). Hence both functors send colimits to limits. Now, since every object of $dSt_k$ is a colimit
of objects in $d\Aff_k$, it suffices to construct a natural transformation between the functors restricted to $d\Aff_k$,
and then take a left Kan extension in the following diagram.
\[
\begin{tikzcd}
dSt_k^{\op} \arrow[shift left]{rr}{DR(-\times X)}\arrow[shift right,swap]{rr}{DR(-)\otimes_k C(X,\mathcal{O}_X)} 
& &\epsilon - dg-mod_k^{gr} \\
d\Aff_k^{\op}\arrow[hook]{u}\arrow[shift left]{urr}\arrow[shift right]{urr} & &
\end{tikzcd}
\]
At the level of derived affines, the natural transformation is obtained essentially from a Kunneth formula: 
$DR(B)\otimes_k DR(C) \simeq DR(B\otimes_k C)$.
\end{proof}

An application of the functor $NC^w$ to $\kappa_X$, together with the fact that $C(X,\mathcal{O}_X)$ is perfect,
gives another natural transformation:
\[	NC^w(- \times X) \to NC^w\big(DR(-) \otimes_k C(X,\mathcal{O}_X)\big) \simeq NC^w(F) \otimes_k C(X,\mathcal{O}_X) .	\]
Moreover these commute with the canonical maps from $NC^w$ to $DR$:
\[
\begin{tikzcd}
NC^w( F \times X) \arrow{r}{\kappa_{F,X}}\arrow{d} & NC^w(F) \otimes_k C(X,\mathcal{O}_X)\arrow{d} \\
DR( F \times X) \arrow[swap]{r}{\kappa_{F,X}} & DR(F) \otimes_k C(X,\mathcal{O}_X) .
\end{tikzcd}
\]

\begin{defin}
Let $X,F \in dSt_k$ with $X$ $\mathcal{O}$-compact. Assume given a morphism $\eta : C(X,\mathcal{O}_X) \to k[-d]$ of
perfect complexes over $k$. The \textbf{integration map along} $\eta$ is the morphism:
\[
\begin{tikzcd}
\int_{\eta} : NC^w(F\times X) \arrow{r}{K_{F,X}} & NC^w(F) \otimes_k C(X,\mathcal{O}_X) \arrow{r}{id \otimes \eta} &
NC^w(F)[-d] .
\end{tikzcd}
\]
\end{defin}
\begin{rem}
We can define the same for $DR$, and the two integration maps are compatible.
\end{rem}

Finally, an additional constraint on the morphism $\eta$ makes it an $\mathcal{O}$-orientation on $X$, which we define now.
For any $E \in \Perf(X)$, we have a natural pairing, which we can compose with $\eta$:
\[
\begin{tikzcd}
C(X,E) \otimes_k C(X,E^*) \arrow{r} & C(X,\mathcal{O}_X)\arrow{r} & k[-d] .
\end{tikzcd}
\]
Dually this gives a ``cap product'' morphism:
\[	\cap \eta : C(X,E) \to C(X,E^*)^*[-d].	\]

\begin{defin}
Let $X \in dSt_k$ $\mathcal{O}$-compact. An $\mathcal{O}$\textbf{-orientation of degree d} on $X$ is a morphism of complexes:
\[	[X] : C(X,\mathcal{O}_X) \to k[-d]	\]
such that, for every $A \in \cdga_k^{\leq 0}$ and any $E \in \Perf(X_A)$, the morphism
\[	\cap [X]_A : C(X_A,E) \to C(X_A,E^*)*[-d]	\]
is a quasi-isomorphism of $A$-dg-modules.
\end{defin}

The main theorem of this section is:

\begin{thm}[Theorem 2.5 in \cite{PTVV}]
\label{thm:symp_map}
Let $F$ be a derived Artin stack equipped with $\omega \in \Symp(F,n)$. Let $X$ be an $\mathcal{O}$-compact derived stack
equipped with an $\mathcal{O}$-orientation of degree $d$:
\[	[X] : C(X,\mathcal{O}_X) \to k[-d].	\]
Assume, moreover, that $\Map_{dSt}(X,F)$ is a derived Artin stack, locally of finite presentation. Then there is a canonical
$n-d$ shifted symplectic structure on $\Map_{dSt}(X,F)$.
\end{thm}

\begin{proof}
Note that $\omega \in \mathcal{A}^{2,cl}(F,n)$ is the same as:
\[	\omega : k[2-n](2) \to NC^w(F) .	\]
Pulling back along the evaluation morphism $\pi : X \to \Map(X,F)$ and integrating on $X$ gives:
\[
\begin{tikzcd}
k[2-n](2) \arrow{r}{\omega} & NC^w(F) \arrow{r}{\pi^*} & NC^w\big(X\times \Map(X,F)\big) \arrow{r}{\int_{[X]}} &
NC^w\big(\Map(X,F)\big)[-d] .
\end{tikzcd}
\]
It remains to see that the underlying 2-form of $\int_{[X]} \pi^* \omega$ is non-degenerate. We can check
this condition locally. Let $f : \Spec A \to \Map(X,F)$ be an $A$-point of $\Map(X,F)$. The tangent complex at $f$ is:
\[	\bbT_f \Map(X,F) \simeq C\big(X \times \Spec A, f^*(\bbT_F)\big).	\]
The underlying 2-form of $\omega$ determines a non-degenerate pairing:
\[	\bbT_F \wedge \bbT_F \to \mathcal{O}_F[n].	\]
By pull-back we obtain a non-degenerate pairing of $A$-dg-modules:
\[	C\big(X \times \Spec A, f^*(\bbT_F)\big) \wedge C\big(X \times \Spec A, f^*(\bbT_F)\big) \to C\big(X \times \Spec A,
\mathcal{O}_{X\times \Spec A}[n]\big) .	\]
Composing with the orientation $[X_A]$ gives a non-degenerate pairing:
\[	C\big(X \times \Spec A, f^*(\bbT_F)\big) \wedge C\big(X \times \Spec A, f^*(\bbT_F)\big) \to A[n-d].	\]
But this is just the pairing induced by the underlying 2-form of $\int_{[X]} \pi^*\omega$.
\end{proof}

The following are examples of stacks $X$ which satisfy the $\mathcal{O}$-compactness and $\mathcal{O}$-orientability
hypotheses of Theorem \ref{thm:symp_map}.

\begin{eg}
\begin{enumerate}
\item (Calabi-Yau) Let $X$ be a smooth and proper DM stack over $\Spec k$ with relative dimension $d$, with connected
geometric fibers. Assume given an isomorphism of line bundles $u:\omega_X \simeq \mathcal{O}_X$. $X$ is $\mathcal{O}$-compact
automatically \todo{why?}. Moreover, the isomorphism $u$ together with Serre duality give an isomorphism:
\[	H^d(X,\mathcal{O}_X) \overset{u}{\to} H^d(X,\omega_X) \cong k ,	\]
which lifts to a quasi-isomorphism of complexes:
\[	C(X,\mathcal{O}_X) \to k[-d] .	\]
Theorem \ref{thm:symp_map} also requires that $\Map(X,F)$ be a derived Artin stack when $F$ is one. This follows from 
Artin-Lurie representability.
\item (Betti) 
\item (de Rham)
\item (Dolbeault) We omit this one.
\end{enumerate}
\end{eg}

We have the following existence statements for various moduli spaces of bundles and complexes.
\begin{cor}[Corollaries 2.6 and 2.13 in \cite{PTVV}]
\label{cor:symp_moduli}
Let $G$ be a reductive group scheme over $k$, and fix $\omega \in (\Sym^2 \fr g) ^G$ non-degenerate.
\begin{enumerate}
\item (Betti) Let $M$ be a compact, orientable topological manifold of degree $d$. A choice of fundamental class $[M] \in H_d(M,k)$
determines a $2-d$ shifted symplectic structure on:
\begin{align*}
LocSys(M) &= \Map(M,BG), \\
\Perf(M) &= \Map(M, \Perf) .
\end{align*}

\item (de Rham) Let $Y$ be a smooth and proper DM stack with connected geometric fibers of relative dimension $d$. A choice of
fundamental class $[Y] \in H^{2d}_{dR}(Y)$ determines a $2-2d$ shifted symplectic structure on the stacks of bundles/perfect
complexes with flat connections:
\begin{align*}
Loc_{dR}(Y) &= \Map(Y_{dR},BG), \\
\Perf_{dR}(Y) &= \Map(Y_{dR}, \Perf) .
\end{align*}

\item (Dolbeault) Let $Y$ be as before. A choice of
fundamental class $[Y] \in H^{2d}_{\p}(Y)$ determines a $2-2d$ shifted symplectic structure on the stacks of bundles/perfect
complexes with Higgs fields:
\begin{align*}
Higgs(Y) &= \Map(Y_{\p},BG), \\
\Perf_{\p}(Y) &= \Map(Y_{\p}, \Perf) .
\end{align*}

\item (Calabi-Yau) Let $Y$ be as before. A choice of trivialization $\omega_{Y/k} \simeq \mathcal{O}_Y$ determines a canonical
$2-d$ shifted symplectic structure on:
\begin{align*}
Bun_G(Y) &= \Map(Y,BG), \\
\Perf(Y) &= \Map(Y, \Perf) .
\end{align*}
\end{enumerate}
\end{cor}

\begin{rem}
For $Y$ a K3 or an elliptic surface, we obtain a 0-shifted symplectic structure on $Bun_G(Y)$. This recovers
a result that was known before the paper \cite{PTVV}, for the locus of simple bundles.\footnote{Simple means that the
only endomorphisms are constants, and they don't have negative self extensions.} But the fact that this classical 
symplectic structure extends to the entire moduli space
is new.
\end{rem}

\todo{some extensions of the mapping stack result in a paper by Calaque and in the thesis of Ted Spaide}


\section{Examples: Lagrangian intersections}
Probably no time for this

