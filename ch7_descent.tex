\chapter{Descent}
Talk by Antonijo Mrcela.

\section{Statement}
\label{sect:cc_statement}

We recall the construction of overcategories, introduced in \ref{sect:cartesian}. In particular, the $\infty$-category of
commutative algebras over $A$ is the homotopy pullback in the following diagram.
\[
\begin{tikzcd}
\cC Alg_{/A} \arrow{r}\arrow{d} & \Fun(\Delta^1 , \cC Alg_k)\arrow{d}{ev_1} \\
\{A\} \arrow{r} & \cC Alg_k
\end{tikzcd}
\]
This is a co-Cartesian fibration. Next, given $f:B \to A$, we get a Cartesian morphism:
\[
\begin{tikzcd}
\cC Alg_{/B} \arrow{d} &  \cC Alg_{/A} \arrow{d}\arrow{l}\\
\{B\} \arrow{r}{f} & \cC \{A\}
\end{tikzcd}
\]
So we can apply the Grothendieck construction to get a functor $\mathcal{F} : \cC Alg_k^{\op}
\to \cC at_{\infty}$. We then take the spectralization to $\cC at_{\infty}^{st}$.

Moreover, $Mod$ lands in the presentable category: $Pr_{st}^R$, where the functors are right adjoint functors. There's also a version
$Pr_{st}^L$, and we have, by \cite{HTT} 5.5.3.4, $[\cS, Pr_{st}^R] \simeq [\cS^{\op}, Pr^L]$. So we can take the adjoint of
$Mod: \cC Alg_k^{\op} \to Pr_{st}^R$ to get $QCoh^{\times} \cC Alg_k \to Pr_{st}^L$.

Note also that at the level of Mod we have a contravariant functoriality, where $f: A \to B$ gets sent to the forgetful functor
$f_* : B-Mod \to A-Mod$. At the level of $QCoh^{\times}$ we have covariant functoriality.

We'll define Grothendieck topologies for $\infty$-categories, and surprisingly get just Grothendieck topologies on the homotopy
category.

\begin{defin}
A \textbf{sieve} on an $\infty$-cat $\cC$ is a full subcategory such that, if $f: C \to D$ and $D \in \cC^0$, then
$C \in \cC^0$. If $C \in \cC$, then a \textbf{sieve over} $C$ is a sieve on $\cC_{/C}$.
\end{defin}

By Remark 6.2.2.3 in \cite{HTT}, a Grothendieck topology on $\cC$ is just one on $h\cC$.

Note, though, that $\eta : h(\cC_{/C}) \to h(\cC)_{/C}$ is not normally an equivalence. This is because in $h(\cC_{/C})$
we also need to specify hte homotopy that makes $A \to B \to C$ commute. However, $\eta$ induces a bijection on sieves.

In $\cC Alg_k$, $S$ the set of faitfully flat morphisms. The following is DAG VII 5.4, 5.1: $S$
determines a Grothendieck topology on $\cC Alg_k$. This is called \textbf{the flat topology}.

\begin{defin}
The \textbf{Cech nerve functor} $B^{\bullet} : \Delta \to \cC Alg_k$, is informally described by $i \to B_i = B^{\otimes_A^{i+1}}$.
To construct it as an $\infty$-functor, we take the left Kan extension of the functor $\Delta^{\leq 1} \to \cC Alg_k$, given
by the morphism $B \to B \otimes_A B$. \todo{Doesn't sound right, figure this out}
\end{defin}


Since $A$ maps to each $B_i$, we obtain an $\infty$-functor $\phi: A-\cM od \to \varinjlim_{\Delta} B^{\otimes_A^{n+1}} - \cM od$.
The \textbf{descent problem} asks if this is an equivalence, and if it is,
whether we can construct some sort of inverse. We give an affirmative answer to the first question, using the following
strategy:
\begin{itemize}
\item The category $A-\cM od$ has a standard t-structure given by the degrees of the modules. We use Lemma \ref{lem:t_filt}
to put a t-structure on $\varinjlim_{\Delta} B^{\otimes_A^{n}} - \cM od$ as well.
\item Due to Lemma \ref{lem:equiv_heart}, $\phi$ is an equivalence if and only if it induces an equivalence on the hearts
of the given t-structures. Due to the equivalence \ref{eq:map_heart}, $\phi$ is an equivalence if and only if it induces
an equivalence $\pi_0(A) \cM od^{\heartsuit} \to \varinjlim_{\Delta} \pi_0(B)^{\otimes_{\pi_0(A)}^{n+1}} - \cM od^{\heartsuit}$.
\item We use a version of Quillen's Theorem A, reproduced in \ref{thm:A}, to show that we can replace the infinite Cech nerve
with a 3-term Cech nerve, without changing the limit. This reduces the problem to Grothendieck's classical formulation of 
descent, which we know to be true.
\end{itemize}


\section{Proof}
\label{sect:descent_proof}
The following lemma is 3.20 in Shennon - Porta - Vezzosi, Formal Gluing along non-linear
flags. \todo{cite this once it appears}

\begin{lem}
\label{lem:t_filt}
Let $p : \mathcal{X} \to \cS$ be a stable filtration, and let $\cS^{\op} \to \cC at_{\infty}^{st}$ be the associated 
$\infty$-functor. Suppose:
\begin{enumerate}
\item For all $s \in \cS$ there is a t-structure $(\mathcal{X}_s^{\leq 0}, \mathcal{X}_s^{\geq 0})$ on $\mathcal{X}_s$.
\item For all edges $f:s \to s'$, the induced functor $f^*$ is t-exact.
\end{enumerate}
Then the stable $\infty$-category $\varprojlim F$ has a (unique) t-structure characterized by:
\[	\forall s \in S, e_s : \varprojlim F \to \mathcal{X}_s 	\]
is $t$-exact.
\end{lem}
\begin{proof}
We can represent the limit as: $\varprojlim F = \Map^{\flat}_S(S^{\sharp},\mathcal{X}^{} )$. \todo{figure out how to do symbol
over $\mathcal{X}$; see \cite{HTT} 3.3.3.2.}
I.e. we map all edges to $p$-Cartesian edges. Define $\cC^{\leq 0}$ to be the full subcategory spanned by $x \in \cC$ such that
$x(s) \in \mathcal{X}_s^{\leq 0}$.
in $\mathcal{X}$.

Let $\mathcal{X}^{\leq 0}$ be the full subcat spanned by $x \in \mathcal{X}$ such that $x \in \mathcal{X}^{\leq 0}_{}$. Let
$j: \mathcal{X}^{\leq 0} \to \mathcal{X}$ be the inclusion.
$p\circ j$ is again a Cartesian fibration, because $f^*$ is exact. Then the inclusion preserves Cartesian edges.
By Proposition 1.2.1.5 in \cite{Lurie_Higher_algebra} (says that $\cC^{\leq n}$ is a localization of $\cC$), which we apply
fiberwise, we get a left adjoint for each fiber. Then apply \cite{Lurie_Higher_algebra} 7.3.2.6 which says the
following. Suppose that we have
a commutative diagram,
\[
\begin{tikzcd}
\cC\arrow{dr}{q} & & \cD\arrow{ll}{G} \arrow{dl}{p} \\
 & \cE & 
\end{tikzcd}
\]
where $p, q$ are locally Cartesian categorical fibrations. Then $G$ admits a left adjoint iff
\begin{enumerate}
\item for every $E \in \mathcal{E}$, the map $G_E : \cD_E \to \cC_E$ admits a left adjoint;
\item $G$ carries locally $p$-Cartesian morphisms in $\cD$ to locally $q$-Cartesian in $\cC$.
\end{enumerate}
This is a ``gluing result for left-adjoints''; not entirely obvious result. But using this gives a global adjoint
$\tau_{\leq 0} : \mathcal{X} \to \mathcal{X}^{\leq 0}$.

Note that, in general, $G$ doesn't have
to preserve Cartesian edges. But we used the fact that $f^*$ is t-exact to deal with this.
\end{proof}

\begin{eg}
Pick $A \to B$ a non-flat morphism. Then we have: \todo{add from paper}
\end{eg}

Now we need to reduce to the problem of descent in the heart. We use Lemma 3.3.7 in the same paper.
\begin{lem}
\label{lem:equiv_heart}
Let $f : \cC \to \cD$ be an exact functor between stable $\infty$-categories. Assume $\cC, \cD$ have t-structures which
are left complete and right bounded, and that with respect to these structures $f$ is t-exact. Then TFAE:
\begin{enumerate}
\item $f$ is an equivalence;
\item $f^{\heartsuit} : \cC^{\heartsuit} \to \cD^{\heartsuit}$ is an equivalence of abelian categories.
\end{enumerate}
\end{lem}
Note that 1 $\Rightarrow$ 2 is obvious, while 2 $\Rightarrow$ 1 is very powerful. This is because we can do many constructions
at the level of the hearts that we can't do at the level of $\infty$-categories.

\begin{proof}
The first step is full faithfulness. For $x,y \in \cC$, there is a canonical transformation $\psi_{x,y} :
Map_{\cC}^{st}(x,y) \to Map_{\cD}^{st}(f(x), f(y))$. Start by fixing $x$ and defining the full subcategory $\cC_x \subset \cC$,
spanned by those $y$ such that $\psi_{x,y}$ is an equivalence. This is closed under loop and suspension, extensions
and retract. So if $\cC^{\heartsuit} \subset \cC_X$, we go by induction on non-vanishing cohomology groups,
to get $\cC^b \subset \cC_X$. Here we use the left complete and right complete assumptions. Now for an arbitrary $y\in \cC$,
$y = \varinjlim \tau_{\geq n} y$. $f$ commutes with this specific colimit, so $f(y) = \varinjlim f(\tau_{\geq n} y)$.
Every map from $x$ to $y$ lands in $\tau_{\geq n} y$ for some $n$, and analogously for maps $f(x)$ to $f(y)$, which reduces
the problem to the case of bounded modules, which is already proved.

Step 2 is essential surjectivity. On the heart it's the hypothesis. Pick $y\in \cD^b$, we have the exact sequence
$\tau_{\leq k} y \to y \to \tau_{>k} y$, with $k$ chosen so that both truncations have fewer cohomology groups than $y$.
Since \todo{fill in from paper}
\end{proof}

\begin{rem}
According to Marci, there are two non-equivalent versions of $D^bCoh$ which have the same heart. We could think about why
this lemma doesn't apply for them.
\end{rem}

\begin{rem}
Mauro says that you can use this statement to prove a bunch of things, for example reduce $\infty$-GAGA to classical GAGA.
\end{rem}

Recall that we were trying to determine whether $A-\cM od \to \varinjlim B^{\otimes_A^{n}} - \cM od$ is an equivalence. We use
Lemma \ref{lem:t_filt} to put a t-structure on the limit, and Lemma \ref{lem:equiv_heart} to show that the problem is equivalent
to that of equivalence of the hearts of the categories. The RHS
becomes:
\[	\varinjlim (B^{\otimes_A^{n}} - \cM od)^{\heartsuit} = \varinjlim \pi_0(B^{\otimes_A^{n}}) - \cM od ,	\]
and the LHS becomes:
\[	\pi_0(A)- \cM od.	\]
This is almost the statement of the classical descent theorem \`{a} la Grothendieck. However, 
in our case the Cech nerve is infinite, instead of having only 3 terms. These two versions are actually
equivalent, due to the following theorem. 

\begin{thm}[Quillen, version of Theorem A]
\label{thm:A}
If $\cC$ is an $n$ category (it's proven in \cite{HTT} that $n$ can be $\infty$), $A : J \to I$ a functor,
if for every object $x \in I$ we have $\pi_i(J_{/x}) = 0$ for $i<n$, then $\lim F = \lim F \circ A$, for all $F : I \to \cC$.
\end{thm}

Here $J_{/x} = J \times_I I_{/x}$. We apply the theorem with $\cC = \cC at \subset \cC at_{\infty}$, which is a 2-category. 
So we need $n=2$.\footnote{For $n=1$, Theorem A is classical; for $n=\infty$, it is proved in \cite{HTT}. For $1<n<\infty$,
we don't think it's written up anywhere, but it should be true.} Furthermore, we use the inclusion $\Delta_s^{\leq 3} \to
\Delta_s$, where the subscript denotes the subcategories with the same objects, but only monomorphisms as morphisms. Define
$F$ as the infinite Cech nerve, $F : \Delta_s \to \cC$, $n \mapsto B^{\otimes_A^n}$; then the restriction $F \circ A$ is
the Cech nerve \`a la Grothendieck:
\[
\begin{tikzcd}
\pi_0(B) - \cM od \arrow[shift left]{r}\arrow[shift right]{r} & \pi_0(B) \otimes_{\pi_0(A)} \pi_0(B) -\cM od
\arrow[shift left = 2]{r}\arrow{r}\arrow[shift right = 2]{r} & \pi_0(B) \otimes_{\pi_0(A)} \pi_0(B) \otimes_{\pi_0(A)} \pi_0(B)
-\cM od .
\end{tikzcd}
\]


\begin{rem}
We motivate the choice of 3 in $\Delta_s^{\leq 3}$ above.
As proved in Exercise 1.5.4, the homotopy type of $(\Delta^{\leq m}_S)_{/m+k}$, with $k\geq 0$,
is a wedge of a number $N_{m,k}$ of $m-1$-spheres. \footnote{Mauro has computed $N_{m,1} = 1$ and $N_{m,2}= 3$; we should see
if we can determine all $N_{m,k}$.} In order for $J = \Delta^{\leq m}_S$ to satisfy the assumptions of Theorem \ref{thm:A} with
$n=2$, we need $m\geq 3$. Therefore the Cech nerve can be reduced to a minimum of 3 terms.
\end{rem}

\begin{rem}
We have:
\[
\begin{tikzcd}
 \varinjlim B^{\otimes_A^{n}} - Mod \arrow{d} \\
A-Mod
\end{tikzcd}
\]
Warning: in non-affine situations, the functor $\lim QCoh(U^n) \to \Fun(\Delta, QCoh(X))$ is highly non-explicit. Given
a descent datum $\{\mathcal{F}^n\}$, we get an $\infty$-functor $\Delta \to QCoh(X)$ which is very lax. In practice one uses
rectification to write $\Fun(\Delta, QCoh(X)) \simeq \infty \Fun(\Delta, Ch(QCoh(X)))$, and use Reed something. The problem is
that the rectification is also very non-explicit.
\end{rem}

