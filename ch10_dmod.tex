\chapter{D-modules}
Talk by Benedict Morrissey.

\section{Introduction and motivation}

We want to talk about differential equations in the setting of AG. It's all classical AG today, but things will go south starting
next week. We also want to talk about quantization: replace functions on the classical phase space ($\mathcal{O}(T^*X)$) with
differential operators on $X$ ($\mathcal{D}(X)$).

There's also motivation from representation theory: Beilinson-Bernstein says roughly that $Rep(\mathcal{U}(\mathfrak g))
\simeq \mathcal{D}_{G/B} - mod$.

The setting is $X$ smooth complex projective variety. We introduce sheaf of noncommutative algebras: $\mathcal{D}_X
 = Der_{\mathcal{C}_X}(\mathcal{O}_X)$. Say $U \subset X$ is a coordinate chart, with coordinates $y_1, \dots, y_n$,
$\mathcal{D}_X$ is generated by $[y_i, \partial_j] = \delta_{ij}$.

First thing to notice is that $\mathcal{D}_X$ acts on $\mathcal{O}_X$ on the left. Therefore $\mathcal{D}_X(U)$ also
acts on the distributions $\mathcal{O}_X(U)^*$ on the right.\footnote{$\mathcal{O}_X(U)$ should be the analytic sheaf.}

Say $f$ is the solution of a differential equation on $X$, e.g.:
\[	(\partial_x - \lambda) f = 0. 	\]
then $\mathcal{D}_X \dot f = \mathcal{D}_X / (\partial_x - \lambda) f$, and we call this $\mathcal{D}$-module
$M_{e^{\lambda x}}$. Then $\Hom_{\mathcal{D}_X -mod}(M_{e^{\lambda x}}, C^{\infty}(\bbA^1))$ is in bijective
correspondence with solutions to the differential equation.

Note also that this sheaf of solutions is a local system. 


\section{Operations on D-modules}
\begin{eg}
We start with pullback. If $f:X \to Y$, and $N \in \mathcal{D}_Y$ is a D-module, we define:
\[	p^*N = \mathcal{O}_X \otimes_{f^{-1} \mathcal{O}_Y} f^{-1} N.	\]
So far, this is just a sheaf of $\mathcal{O}_X$ modules. We also need to define an ation by derivations $\theta \in \mathcal{D}_X$,
as follows. \todo{paper} Taking coordinates $\{y_i\}$ on $Y$,
\[	\theta(\psi \otimes s) = \theta(\psi) \otimes s + \psi \sum_{i=1}^n \theta(y_i \circ f) \otimes \partial_i(s).	\]
Note that, if the D-module is one of functions, this is the same as pulling back the function to $X$, and then acting by differentition.
\end{eg}

Direct image is more annoying; we need to use complexes of D-modules for it to be meaningful. We define:
\[	f_* M = \mathcal{D}_{Y , X} \otimes^{\bbL}_{\mathcal{D}_X} M.	\]
Here $\mathcal{D}_{Y , X}$ is a $\mathcal{D}_Y - \mathcal{D}_X$ bimodule defined as follows. \todo{paper}


Fourier transform of $\mathcal{D}$-modules. Think about functions first: we pull back functions from $\bbA^1$ to
$\bbA^2$, multiply with a function (kernel) there, and then push forward to the second factor of $\bbA^1$.
For $\mathcal{D}$-modules, this is done as follows:
\[	F(M) = p_{2*} \big( (p_1^* M) \otimes M_{e^{2\pi i xy}} \big) . \]


\section{Characteristic Variety}
We next investigate how the $D$-module relates to the geometry of the solution set of the associated differential equation.
We take the associated graded \todo{enter more details here}, and $\Spec (\mathcal{D}_X^{gr}) = T^*X$, as long as $X$ is
smooth. For any D-module $M$, the associated graded $M^{gr}$ is a coherent $\mathcal{O}(T^*X)$-module. Then we associate
the \textbf{character variety} $ch(M) =  sup(M^{gr}) \subset T^*X$.

We have a functor $D_X -mod \to \mathcal{O}_X -mod$, $sol_X (M) = \R \Hom_{\mathcal{D}_X}(M,\mathcal{O}_X)$.
There's an easier functor to work with, the deRHam functor, which does: $DR_X(M) = M \otimes \Omega_X^{\bullet}$.
There's a duality $DR_X(M) = sol(D_X M)$, where \todo{paper}

\begin{defin}
$\eta \in T^*_x X$ is \textbf{non-characteristic} if for any division of $X$ into level sets of $h$ such that $dh = \eta$,
$dR(M)$ is locally constant transverse to level sets.
\end{defin}

\begin{thm}[Cauchy-Kovalevski]
The set $\{\eta \in T^*X \text{ characteristic}\} = ch(M)$.
\end{thm}

We have $\dim(ch(M)) \geq \dim(X)$; if equality holds, we call $M$ \textbf{holonomic}.

If $M$ is holonomic, $dR(M)$ is really nice, in the sense that $H^i(dR(M))$ is locally ocnstant on the strata of some 
stratification of $X$, $X = \bigsqcup_{\alpha \in \Lambda} X_{\alpha}$. Think about solutions to the logarithm,
whose associated D-module is locally constant on the strata $\C = \C^{\times} \sqcup \{0\}$.

\begin{defin}
A D-module $M$ is \textbf{regular} if for all $x\in X$, $\R \Hom_D(M_x, \mathcal{O}_x) \cong \R \Hom_D(M_x, \hat{\mathcal{O}_x})$.
\end{defin}

\begin{rem}
This can be promoted to a categorical statement, whereby regular holonomic $\mathcal{D}$-modules correspond under the deRham functor
to constructible complexes. THis is the Riemann-Hilbert correspondence.
\end{rem}


\section{More general spaces}
We can embed a singular variety $X$ in a nonsingular $Y$. A theorem of Kashiwara tells us that the following is well-defined:
\[	\mathcal{D}_X -mod = \{ U \in \mathcal{D}_Y -mod| supp_{\mathcal{O}_Y}(U) \subset X \}.	\]

Quotient stacks: $X = Y / G$, where $Y$ is smooth and $G$ is a reductive algebraic group. We want to relate D-modules on $X$
to D-modules on $Y$ with a strong equivariance property.

\begin{defin}
A D-module $M$ on $Y$ is \textbf{strongly equivariant} if there's an equivalence $\alpha : act^* M \to p^* M$ of
$\mathcal{D}_G \otimes \mathcal{D}_Y$-modules, where the latter is $p_1^* \mathcal{D}_G \otimes p_2^* \mathcal{D}_Y$.
\footnote{Change to square tensor}. This has to satisfy some compatibility condition \todo{paper}
\end{defin}


\section{Representation theory}
Beilinson-Bernstein theorem. $G$ acts on $G/B = Fl$. Infinitesimally this gives a map $\mathfrak g \to T_p Fl$, ie.
$\mathfrak g \to $ Vector fields on $Fl$. This extends to a map:
\[	\mathcal{U}(\mathfrak g) \to \mathcal{D}_{Fl}.	\]
Then given a line bundle $\mathcal{L}^{\chi}$ on $G/B$, we get $\mathcal{D}^{\chi}$ acting on the sections of $\mathcal{L}$.
Note that $Fl$ is the moduli space of Borels. Given a character $\chi : T \to \C$, I get a character on $B$ by
precomposing with the projection $B\to T$, i.e. the character is trivial on strictly upper-triangular matrices.

There's a theorem by Harish-Chandra saying that, to such a character, we can associate \todo{paper}. Then the theorem says that,
for $\chi$ regular, we have $\mathcal{U}(\mathfrak g)^{\chi}-mod = \mathcal{D}_{FL}^{\chi} - mod$.


\section{Nearby and vanishing cycles}
Say I have a family of algebraic varieties which degenerate at some point. Say I have the family of tori over $\C$ given by
$y^2 = x(x-a)(x-2a)$. But over 0 I have the cuspidal cubic $y^2 = x^3$. We want to use data about the nearby nice objects to
talk about the singular object.

The general setup is:
\[
\begin{tikzcd}
X_0\arrow{r}\arrow{d} & X \arrow{d} & X^{\times} \arrow{l}\arrow{d} \\
\{0\} \arrow{r} & \C & \C^{\times} \arrow{l}.
\end{tikzcd}
\]

Nearby cycle functor: \todo{paper}

Theorem: if $f$ is a proper map, then there exists $U$ of $X_0$ in $X$ such that $q:U \to X_0$ is continuous and homotopes to
identity on $X_0$. 


Silly example: if $X = \C$, $M$ is a $\mathcal{D}$-module on $\C^{\times}$, $K = DR(M)$. If we take $f$ to be the identity, then
$\psi_f(K) \cong K_{X_{\epsilon}}$. We have a monodromy operator $T: \psi_f(K) \to \psi_f(K)$, sending $a \mapsto a + c$.
If $K$ comes from logarithm, this is $a \mapsto a + 2\pi$. 

If $M$ is holonomic, from adjunction we have a unit map $M \mapsto p_* p^* M$ (need holonomic to have $p^* = p^!$).
Then we have an exact triangle in the derived category, $i^*M \to \psi_f(M) \to \Psi_f[K](1) \to \dots$, the 
latter is the \textbf{vanishing cycle}.

An application to geometric representation theory: \todo{paper}.

More generally, can take parabolic $P \subset G$. We have the projection $q: G[[t]] \to G$, and take $\bar P = q^{-1}(P)$.
Then $\bar P$-equivariant constructible sheaves on $G((t))/\bar P$ should be equivalent to representations of some other group.
Now take a family over $X$ a Riemann surface; this is related to the vanishing cycle, but none of us understand this yet.

