\pdfoutput=1
%The other issue is that some packages, such as microtype, produce different output under pdflatex. By default the arXiv goes from dvi to ps to pdf, so if you need pdflatex you have to set the \pdfoutput flag in the TeX file.
\newif\ifpersonal
\newif\ifarxiv
\personaltrue % comment to remove personal notes
\arxivtrue % comment to display shortened version for journal submissions
\RequirePackage[l2tabu, orthodox]{nag} %detect whether obsolete packages are used
\documentclass[10pt,a4paper,reqno,oneside]{book} %reqno places equation numbers on the right
\linespread{1.2}
%\allowdisplaybreaks[1]
\usepackage{calligra}
\usepackage{amsmath,amsthm,amssymb,mathrsfs,mathtools,bm,eucal,tensor} % math related
\usepackage{microtype,fixltx2e} % latex technical issues
\usepackage[scaled]{beramono,berasans}
\usepackage{enumerate,comment,braket,xspace,tikz-cd} %utilities
\usepackage[all,cmtip]{xy} % utilities
\usepackage[utf8]{inputenc} % input encoding
\usepackage[T1]{fontenc} % font encoding
\usepackage{lmodern}
\definecolor{linkcolor}{HTML}{005050}
\usepackage[centering,vscale=0.7,hscale=0.7]{geometry}
\usepackage{hyperref}
\usepackage[capitalize]{cleveref}
\usepackage{graphicx}
\usepackage{xparse}
\usepackage{url}

%\makeevenhead{headings}{\thepage}{}{\leftmark}
%\setlrmarginsandblock{3cm}{3.5cm}{*}
%\setlength\marginparwidth{2.5cm}
%\checkandfixthelayout
%
%\setlength\headheight{24pt}

\usepackage{vmargin}
\setpapersize{A4}
\setmarginsrb{25mm}{10mm}{25mm}{10mm}%
{12mm}{10mm}{5mm}{10mm}

\usepackage{fancyhdr}
\pagestyle{fancy}
%%%Settings
\renewcommand{\chaptermark}[1]{\markboth{#1}{}}
\renewcommand{\sectionmark}[1]{\markright{\thesection\ #1}}
\fancyhf{}
\fancyhead[LE,RO]{\bfseries\thepage}
\fancyhead[RE]{\bfseries\footnotesize\nouppercase{\leftmark}}
\fancyhead[LO]{\bfseries\footnotesize\nouppercase{\rightmark}}

\theoremstyle{plain}
\newtheorem{thm-intro}{Theorem}
\newtheorem{thm}{Theorem}[section]
\newtheorem*{thm*}{Theorem}
\newtheorem{lem}[thm]{Lemma}
\newtheorem{prop}[thm]{Proposition}
\newtheorem{conj}[thm]{Conjecture}
\newtheorem{cor}[thm]{Corollary}
\newtheorem{cor-intro}[thm-intro]{Corollary}
\newtheorem{assumption}[thm]{Assumption}
\theoremstyle{definition}
\newtheorem{defin}[thm]{Definition}
\newtheorem{exercise}[thm]{Exercise}
\newtheorem{defin-intro}[thm-intro]{Definition}
\newtheorem{notation}[thm]{Notation}
\theoremstyle{remark}
\newtheorem*{rem*}{Remark}
\newtheorem{eg}[thm]{Example}
\newtheorem{eg-intro}[thm-intro]{Example}
\newtheorem{rem}[thm]{Remark}
\newtheorem{rem-intro}[thm-intro]{Remark}
\numberwithin{equation}{section}
\newtheorem{construction}[thm]{Construction}

% personal remarks

\ifpersonal
\newcommand*{\personal}[1]{\textcolor[rgb]{0.6,0.6,1}{(Personal: #1)}}
\newcommand*{\todo}[1]{\textcolor{red}{(Todo: #1)}}
\else
\newcommand*{\personal}[1]{\ignorespaces}
\newcommand*{\todo}[1]{\ignorespaces}
\fi

% Fonts
\newcommand{\C}{\mathbb C}
\newcommand{\CP}{\mathbb{CP}}
\newcommand{\F}{\mathbb F}
\newcommand{\Q}{\mathbb Q}
\newcommand{\R}{\mathbb R}
\newcommand{\Z}{\mathbb Z}
\newcommand{\N}{\mathbb N}

\newcommand{\rB}{\mathrm B}
\newcommand{\rD}{\mathrm D}
\newcommand{\rH}{\mathrm H}
\newcommand{\rI}{\mathrm I}
\newcommand{\rL}{\mathrm L}
\newcommand{\rP}{\mathrm P}
\newcommand{\rQ}{\mathrm Q}
\newcommand{\rR}{\mathrm R}
\newcommand{\rb}{\mathrm b}
\newcommand{\rd}{\mathrm d}
\newcommand{\rh}{\mathrm h}
\newcommand{\rs}{\mathrm s}
\newcommand{\rt}{\mathrm t}


\newcommand{\fA}{\mathfrak A}
\newcommand{\fB}{\mathfrak B}
\newcommand{\fC}{\mathfrak C}
\newcommand{\fD}{\mathfrak D}
\newcommand{\fH}{\mathfrak H}
\newcommand{\fS}{\mathfrak S}
\newcommand{\fT}{\mathfrak T}
\newcommand{\fU}{\mathfrak U}
\newcommand{\fV}{\mathfrak V}
\newcommand{\fX}{\mathfrak X}
\newcommand{\fY}{\mathfrak Y}
\newcommand{\fZ}{\mathfrak Z}
\newcommand{\ff}{\mathfrak f}
\newcommand{\fm}{\mathfrak m}
\newcommand{\fn}{\mathfrak n}
\newcommand{\fs}{\mathfrak s}
\newcommand{\ft}{\mathfrak t}

\newcommand{\cA}{\mathcal A}
\newcommand{\cB}{\mathcal B}
\newcommand{\cC}{\mathcal C}
\newcommand{\cD}{\mathcal D}
\newcommand{\cE}{\mathcal E}
\newcommand{\cF}{\mathcal F}
\newcommand{\cH}{\mathcal H}
\newcommand{\cG}{\mathcal G}
\newcommand{\cI}{\mathcal I}
\newcommand{\cJ}{\mathcal J}
\newcommand{\cK}{\mathcal K}
\newcommand{\cL}{\mathcal L}
\newcommand{\cM}{\mathcal M}
\newcommand{\cN}{\mathcal N}
\newcommand{\cO}{\mathcal O}
\newcommand{\cP}{\mathcal P}
\newcommand{\cR}{\mathcal R}
\newcommand{\cS}{\mathcal S}
\newcommand{\cT}{\mathcal T}
\newcommand{\cU}{\mathcal U}
\newcommand{\cV}{\mathcal V}
\newcommand{\cW}{\mathcal W}
\newcommand{\cX}{\mathcal X}
\newcommand{\cY}{\mathcal Y}
\newcommand{\cZ}{\mathcal Z}
\DeclareFontFamily{U}{BOONDOX-calo}{\skewchar\font=45 }
\DeclareFontShape{U}{BOONDOX-calo}{m}{n}{<-> s*[1.05] BOONDOX-r-calo}{}
\DeclareFontShape{U}{BOONDOX-calo}{b}{n}{<-> s*[1.05] BOONDOX-b-calo}{}
\DeclareMathAlphabet{\mathcalboondox}{U}{BOONDOX-calo}{m}{n}
%\DeclareMathAlphabet{\mathcalligra}{T1}{calligra}{m}{n}
\newcommand{\cf}{\mathcalboondox f}

\newcommand{\bbA}{\mathbb A}
\newcommand{\bbD}{\mathbb D}
\newcommand{\bbG}{\mathbb G}
\newcommand{\bbL}{\mathbb L}
\newcommand{\bbP}{\mathbb P}
\newcommand{\bbT}{\mathbb T}
\newcommand{\bbV}{\mathbb V}

\newcommand{\bA}{\mathbf A}
\newcommand{\bD}{\mathbf D}
\newcommand{\bP}{\mathbf P}
\newcommand{\bQ}{\mathbf Q}
\newcommand{\bT}{\mathbf T}
\newcommand{\bX}{\mathbf X}
\newcommand{\bY}{\mathbf Y}
\newcommand{\be}{\mathbf e}
\newcommand{\br}{\mathbf r}
\newcommand{\bu}{\mathbf u}
\newcommand{\balpha}{\bm{\alpha}}
\newcommand{\bDelta}{\bm{\Delta}}
\newcommand{\brho}{\bm{\rho}}

\newcommand{\sC}{\mathscr C}
\newcommand{\sX}{\mathscr X}
\newcommand{\sD}{\mathscr D}
\newcommand{\sU}{\mathscr U}


% Decorations

% Definition of \widebar from http://tex.stackexchange.com/questions/16337/can-i-get-a-widebar-without-using-the-mathabx-package/60253#60253
\makeatletter
\let\save@mathaccent\mathaccent
\newcommand*\if@single[3]{%
	\setbox0\hbox{${\mathaccent"0362{#1}}^H$}%
	\setbox2\hbox{${\mathaccent"0362{\kern0pt#1}}^H$}%
	\ifdim\ht0=\ht2 #3\else #2\fi
}
%The bar will be moved to the right by a half of \macc@kerna, which is computed by amsmath:
\newcommand*\rel@kern[1]{\kern#1\dimexpr\macc@kerna}
%If there's a superscript following the bar, then no negative kern may follow the bar;
%an additional {} makes sure that the superscript is high enough in this case:
\newcommand*\widebar[1]{\@ifnextchar^{{\wide@bar{#1}{0}}}{\wide@bar{#1}{1}}}
%Use a separate algorithm for single symbols:
\newcommand*\wide@bar[2]{\if@single{#1}{\wide@bar@{#1}{#2}{1}}{\wide@bar@{#1}{#2}{2}}}
\newcommand*\wide@bar@[3]{%
	\begingroup
	\def\mathaccent##1##2{%
		%Enable nesting of accents:
		\let\mathaccent\save@mathaccent
		%If there's more than a single symbol, use the first character instead (see below):
		\if#32 \let\macc@nucleus\first@char \fi
		%Determine the italic correction:
		\setbox\z@\hbox{$\macc@style{\macc@nucleus}_{}$}%
		\setbox\tw@\hbox{$\macc@style{\macc@nucleus}{}_{}$}%
		\dimen@\wd\tw@
		\advance\dimen@-\wd\z@
		%Now \dimen@ is the italic correction of the symbol.
		\divide\dimen@ 3
		\@tempdima\wd\tw@
		\advance\@tempdima-\scriptspace
		%Now \@tempdima is the width of the symbol.
		\divide\@tempdima 10
		\advance\dimen@-\@tempdima
		%Now \dimen@ = (italic correction / 3) - (Breite / 10)
		\ifdim\dimen@>\z@ \dimen@0pt\fi
		%The bar will be shortened in the case \dimen@<0 !
		\rel@kern{0.6}\kern-\dimen@
		\if#31
		\overline{\rel@kern{-0.6}\kern\dimen@\macc@nucleus\rel@kern{0.4}\kern\dimen@}%
		\advance\dimen@0.4\dimexpr\macc@kerna
		%Place the combined final kern (-\dimen@) if it is >0 or if a superscript follows:
		\let\final@kern#2%
		\ifdim\dimen@<\z@ \let\final@kern1\fi
		\if\final@kern1 \kern-\dimen@\fi
		\else
		\overline{\rel@kern{-0.6}\kern\dimen@#1}%
		\fi
	}%
	\macc@depth\@ne
	\let\math@bgroup\@empty \let\math@egroup\macc@set@skewchar
	\mathsurround\z@ \frozen@everymath{\mathgroup\macc@group\relax}%
	\macc@set@skewchar\relax
	\let\mathaccentV\macc@nested@a
	%The following initialises \macc@kerna and calls \mathaccent:
	\if#31
	\macc@nested@a\relax111{#1}%
	\else
	%If the argument consists of more than one symbol, and if the first token is
	%a letter, use that letter for the computations:
	\def\gobble@till@marker##1\endmarker{}%
	\futurelet\first@char\gobble@till@marker#1\endmarker
	\ifcat\noexpand\first@char A\else
	\def\first@char{}%
	\fi
	\macc@nested@a\relax111{\first@char}%
	\fi
	\endgroup
}
\makeatother


\newcommand{\oDelta}{\widebar\Delta}
\newcommand{\oGamma}{\widebar\Gamma}
\newcommand{\oSigma}{\widebar\Sigma}
\newcommand{\oalpha}{\widebar\alpha}
\newcommand{\obeta}{\widebar\beta}
\newcommand{\otau}{\widebar\tau}
\newcommand{\oC}{\widebar C}
\newcommand{\oD}{\widebar D}
\newcommand{\oE}{\widebar E}
\newcommand{\oG}{\widebar G}
\newcommand{\oM}{\widebar M}
\newcommand{\oR}{\widebar R}
\newcommand{\oS}{\widebar S}
\newcommand{\oU}{\widebar U}
\newcommand{\oW}{\widebar W}
\newcommand{\oX}{\widebar X}
\newcommand{\oY}{\widebar Y}
\newcommand{\oPhi}{\overline{\Phi}}


\newcommand{\ok}{\widebar k}
\newcommand{\ov}{\widebar v}
\newcommand{\ox}{\widebar x}
\newcommand{\oy}{\widebar y}
\newcommand{\oz}{\widebar z}

\newcommand{\hh}{\widehat h}
\newcommand{\hf}{\widehat f}
\newcommand{\hA}{\widehat A}
\newcommand{\hB}{\widehat B}
\newcommand{\hC}{\widehat C}
\newcommand{\hE}{\widehat E}
\newcommand{\hF}{\widehat F}
\newcommand{\hI}{\widehat I}
\newcommand{\hL}{\widehat L}
\newcommand{\hU}{\widehat U}
\newcommand{\hZ}{\hat Z}
\newcommand{\hbeta}{\widehat\beta}
\newcommand{\hGamma}{\widehat\Gamma}
\newcommand{\hPhi}{\widehat{\Phi}}
\newcommand{\hPsi}{\widehat{\Psi}}

\newcommand{\hrP}{\widehat \rP}

\newcommand{\tw}{\widetilde w}
\newcommand{\tW}{\widetilde W}
\newcommand{\tk}{\tilde k}
\newcommand{\tv}{\tilde v}
\newcommand{\tB}{\widetilde B}
\newcommand{\tD}{\widetilde D}
\newcommand{\tI}{\widetilde I}
\newcommand{\tM}{\widetilde M}
\newcommand{\tN}{\widetilde N}
\newcommand{\tP}{\widetilde P}
\newcommand{\tR}{\widetilde R}
\newcommand{\tX}{\widetilde X}
\newcommand{\tfX}{\widetilde{\fX}}
\newcommand{\tfB}{\widetilde{\fB}}
\newcommand{\tsX}{\widetilde{\sX}}
\newcommand{\tH}{\widetilde H}
\newcommand{\tY}{\widetilde Y}
\newcommand{\tbeta}{\widetilde{\beta}}
\newcommand{\tphi}{\widetilde{\phi}}
\newcommand{\ttau}{\widetilde{\tau}}

% Global tropicalization
\newcommand{\Ih}{I^\mathrm{h}}
\newcommand{\Iv}{I^\mathrm{v}}
\newcommand{\IX}{I_\fX}
\newcommand{\IY}{I_\fY}
\newcommand{\SD}{S_\fD}
\newcommand{\SX}{S_\fX}
\newcommand{\SsXH}{S_{(\sX,H)}}
\newcommand{\SY}{S_\fY}
\newcommand{\CsXH}{C_{(\sX,H)}}
\newcommand{\oF}{\overline{F}}
\newcommand{\oP}{\overline{P}}
\newcommand{\oSX}{\overline{\SX}}
\newcommand{\oIX}{\overline{\IX}}


% Vanishing cycles
\newcommand{\fXe}{\fX_\eta}
\newcommand{\fXs}{\fX_s}
\newcommand{\ofX}{\widebar{\fX}}
\newcommand{\ofXs}{\widebar{\fX}_s}
\newcommand{\fYe}{\fY_\eta}
\newcommand{\fYs}{\fY_s}
\newcommand{\fXbs}{\fX_{\bar s}}
\newcommand{\fXbe}{\fX_{\bar\eta}}
\newcommand{\fDe}{\fD_\eta}
\newcommand{\LX}{\Lambda_{\fX}}
\newcommand{\LXe}{\Lambda_{\fX_\eta}}
\newcommand{\LXs}{\Lambda_{\fXbs}}
\newcommand{\QXe}{\Q_{\ell,\fX_\eta}}
\newcommand{\QXbs}{\Q_{\ell,\fXbs}}
\newcommand{\sXe}{\sX_\eta}
\newcommand{\sXs}{\sX_s}
\newcommand{\LUe}{\Lambda_{\fU_\eta}}
\newcommand{\fCbs}{\fC_{\bar s}}
\newcommand{\QUe}{\Q_{\ell,\fU_\eta}}
\newcommand{\QCe}{\Q_{\ell,\fC_\eta}}
\newcommand{\QCs}{\Q_{\ell,\fCbs}}

% stacks

\newcommand{\hcC}{\mathrm h\cC}
\newcommand{\hcD}{\mathrm h\cD}
\newcommand{\PSh}{\mathrm{PSh}}
\newcommand{\Sh}{\mathrm{Sh}}
\newcommand{\Shv}{\mathrm{Shv}}
\newcommand{\Tuupperp}{\tensor*[^\cT]{u}{^p}}
\newcommand{\Tulowerp}{\tensor*[^\cT]{u}{_p}}
\newcommand{\Tuuppers}{\tensor*[^\cT]{u}{^s}}
\newcommand{\Tulowers}{\tensor*[^\cT]{u}{_s}}
\newcommand{\pu}{\tensor*[_p]{u}{}}
\newcommand{\su}{\tensor*[_s]{u}{}}
\newcommand{\Tpu}{\tensor*[^\cT_p]{u}{}}
\newcommand{\Tsu}{\tensor*[^\cT_s]{u}{}}
\newcommand{\Dfpull}{\tensor*[^\cD]{f}{^{-1}}}
\newcommand{\Dfpush}{\tensor*[^\cD]{f}{_*}}
\newcommand{\Duuppers}{\tensor*[^\cD]{u}{^s}}
\newcommand{\Dulowers}{\tensor*[^\cD]{u}{_s}}
\newcommand{\Geom}{\mathrm{Geom}}
\newcommand{\LPr}{\mathcal{P}\mathrm{r}^\rL}
\newcommand{\RPr}{\mathcal{P}\mathrm{r}^\rR}
\newcommand{\LPromega}{\mathcal{P}\mathrm{r}^{\rL, \omega}}
\newcommand{\LPromegast}{\mathcal{P}\mathrm{r}^{\rL, \omega}_{\mathrm{Ex}}}
\newcommand{\CX}{\cC_{/X}}
\newcommand{\CY}{\cC_{/Y}}
\newcommand{\CXP}{(\cC_{/X})_{\bP}}
\newcommand{\GeomXP}{(\mathrm{Geom}_{/X})_\bP}
\newcommand{\GeomYP}{(\mathrm{Geom}_{/Y})_\bP}
\newcommand{\infcat}{$\infty$-category\xspace}
\newcommand{\infcats}{$\infty$-categories\xspace}
\newcommand{\infsite}{$\infty$-site\xspace}
\newcommand{\infsites}{$\infty$-sites\xspace}
\newcommand{\inftopos}{$\infty$-topos\xspace}
\newcommand{\inftopoi}{$\infty$-topoi\xspace}
\newcommand{\pres}{{}^{\mathrm L} \mathcal P \mathrm{res}}
\newcommand{\Grpd}{\mathrm{Grpd}}
\newcommand{\sSet}{\mathrm{sSet}}
\newcommand{\rSet}{\mathrm{Set}}
\newcommand{\Ab}{\mathrm{Ab}}
\newcommand{\DAb}{\cD(\Ab)}
\newcommand{\tauan}{\tau_\mathrm{an}}
\newcommand{\qet}{\mathrm{q\acute{e}t}}
\newcommand{\tauet}{\tau_\mathrm{\acute{e}t}}
\newcommand{\tauqet}{\tau_\mathrm{q\acute{e}t}}
\newcommand{\bPsm}{\bP_\mathrm{sm}}
\newcommand{\bPqsm}{\bP_\mathrm{qsm}}
\newcommand{\Modh}{\textrm{-}\mathrm{Mod}^\heartsuit}
\newcommand{\Mod}{\textrm{-}\mathrm{Mod}}
\newcommand{\Coh}{\mathrm{Coh}}
\newcommand{\Cohb}{\mathrm{Coh}^\mathrm{b}}
\newcommand{\Cohh}{\mathrm{Coh}^\heartsuit}
\newcommand{\QCohh}{\mathrm{QCoh}^\heartsuit}
\newcommand{\RcHom}{\rR\!\mathcal H\!\mathit{om}}
\newcommand{\kfiltered}{$\kappa$-filtered\xspace}
\newcommand{\Stn}{\mathrm{Stn}}
\newcommand{\Sch}{\mathrm{Sch}}
\newcommand{\FSch}{\mathrm{FSch}}
\newcommand{\Aff}{\mathrm{Aff}}
\newcommand{\Afflfp}{\mathrm{Aff}^{\mathrm{lfp}}}
\newcommand{\An}{\mathrm{An}}
\newcommand{\Afd}{\mathrm{Afd}}
\newcommand{\Top}{\mathcal T\mathrm{op}}
\newcommand{\bfMap}{\mathbf{Map}}



% DAnG

\newcommand{\dAnk}{\mathrm{dAn}_k}
\newcommand{\Ank}{\mathrm{An}_k}
\newcommand{\cTan}{\cT_{\mathrm{an}}}
\newcommand{\cTannc}{\cT_{\mathrm{an}}^{\mathrm{nc}}}
\newcommand{\cTank}{\cT_{\mathrm{an}}(k)}
\newcommand{\cTdisck}{\cT_{\mathrm{disc}}(k)}
\newcommand{\cTet}{\cT_{\mathrm{\acute{e}t}}}
\newcommand{\cTetnc}{\cTet^{\mathrm{nc}}}
\newcommand{\cTetk}{\cT_{\mathrm{\acute{e}t}}(k)}
\newcommand{\Strloc}{\mathrm{Str}^\mathrm{loc}}
\newcommand{\RTop}{\tensor*[^\rR]{\Top}{}}
\newcommand{\LTop}{\tensor*[^\rL]{\Top}{}}
\newcommand{\RHTop}{\tensor*[^\rR]{\mathcal{H}\Top}{}}
\newcommand{\LRT}{\mathrm{LRT}}
\newcommand{\Tor}{\mathrm{Tor}}
\newcommand{\dAfd}{\mathrm{dAfd}}
\newcommand{\dAfdk}{\mathrm{dAfd}_k}
\newcommand{\biget}{\mathrm{big,\acute{e}t}}
\newcommand{\trunc}{\mathrm{t}_0}
\newcommand{\Hyp}{\mathrm{Hyp}}
\newcommand{\HSpec}{\mathrm{HSpec}}
\newcommand{\CAlg}{\mathrm{CAlg}}
\newcommand{\trunctopoi}{\Spec^{\cG_{\mathrm{an}}^{\le 0}(k)}_{\cG_{\mathrm{an}(k)}}}

% Formal Gluing

\newcommand{\IndPro}[1]{\mathrm{Ind}(\mathrm{Pro}(#1))}
\newcommand{\GFRings}{\mathrm{GFRings}}
\newcommand{\Pro}{\mathrm{Pro}}
\newcommand{\Ind}{\mathrm{Ind}}
\newcommand{\preNbd}{\mathrm{PNbd}}
\newcommand{\Nbd}{\mathrm{Nbd}^{\circ}}
\newcommand{\cHom}{\cH \mathrm{om}}
\newcommand {\D} {\mathsf{L}}
\newcommand{\St}{\mathbf{St}}
\newcommand{\dSt}{\mathbf{dSt}}
\newcommand{\Tw}{\mathrm{Tw}}
\newcommand{\Lan}{\mathrm{Lan}}
\newcommand{\IndCoh}{\mathrm{IndCoh}}
\newcommand{\QCoh}{\mathrm{QCoh}}
\newcommand{\Perf}{\mathrm{Perf}}
\newcommand{\lex}{\mathrm{lex}}
\newcommand{\Dsing}{\rD_\mathrm{sing}}
\newcommand{\fib}{\mathrm{fib}}
\newcommand{\cofib}{\mathrm{cofib}}
\newcommand{\stMap}{\mathrm{Map}^{\mathrm{st}}}
\newcommand{\Zar}{\mathrm{Zar}}
\newcommand{\Cat}{\mathrm{Cat}}
\newcommand{\AbCat}{\mathrm{AbCat}}
\newcommand{\bfCoh}{\mathbf{Coh}}
\newcommand{\bfPerf}{\mathbf{Perf}}
\newcommand{\bfQCoh}{\mathbf{QCoh}}
\newcommand{\Catst}{\Cat_\infty^{\mathrm{Ex}}}
\newcommand{\Catstidem}{\Cat_\infty^{\mathrm{Ex}, \mathrm{idem}}}
\newcommand{\Catstlc}{\Cat_\infty^{\mathrm{Ex}, \mathrm{l.c.}}}
\newcommand{\Catstlb}{\Cat_\infty^{\mathrm{Ex}, \mathrm{l.b.}}}

\newcommand{\bfBun}{\operatorname{\mathbf{Bun}}}
\newcommand{\Bun}{\operatorname{\mathrm{Bun}}}
\newcommand{\Bunhat}{\operatorname{\mathbf{B\widehat{un}}}}

% Special symbols
\newcommand{\bcM}{\widebar{\mathcal M}}
\newcommand{\bcC}{\widebar{\mathcal C}}
\newcommand{\bcMgn}{\widebar{\mathcal M}_{g,n}}
\newcommand{\bcMol}{\widebar{\mathcal M}_{0,1}}
\newcommand{\bcMot}{\widebar{\mathcal M}_{0,3}}
\newcommand{\bcMof}{\widebar{\mathcal M}_{0,4}}
\newcommand{\bcMon}{\widebar{\mathcal M}_{0,n}}
\newcommand{\bcMgnprime}{\widebar{\mathcal M}_{g,n'}}
\newcommand{\bcMgnijprime}{\widebar{\mathcal M}_{g_{ij},n'_{ij}}}
\newcommand{\bMgnt}{\widebar{M}^\mathrm{trop}_{g,n}}
\newcommand{\Mmdisc}{M_{m\textrm{-disc}}}
\newcommand{\Gm}{\mathbb G_{\mathrm m}}
\newcommand{\Gmk}{\mathbb G_{\mathrm m/k}}
\newcommand{\Gmkprime}{\mathbb G_{\mathrm m/k'}}
\newcommand{\Gmnan}{(\Gm^n)\an}
\newcommand{\Gmknan}{(\Gmk^n)\an}
\newcommand{\Lin}{\mathit{Lin}}
\newcommand{\Simp}{\mathit{Simp}}
\newcommand{\vol}{\mathit{vol}}
\newcommand{\LanD}{\mathcal L_{an}^D}

% Categories


% Shorthands
\newcommand{\kc}{k^\circ}
\newcommand{\llb}{[\![}
\newcommand{\rrb}{]\!]}
\newcommand{\llp}{(\!(}
\newcommand{\rrp}{)\!)}
\newcommand{\an}{^\mathrm{an}}
\newcommand{\alg}{^\mathrm{alg}}
\newcommand{\loweralg}{_\mathrm{alg}}
\newcommand{\bad}{^\mathrm{bad}}
\newcommand{\ess}{^\mathrm{ess}}
\newcommand{\ness}{^\mathrm{ness}}
\newcommand{\et}{_\mathrm{\acute{e}t}}
\newcommand{\Et}{_\mathrm{\acute{E}t}}
\newcommand{\ev}{\mathrm{ev}}
%\newcommand{\eistar}{\mathbf e_i^*}
%\newcommand{\ejstar}{\mathbf e_j^*}
%\newcommand{\ekstar}{\mathbf e_k^*}
\newcommand{\mult}{\mathit{mult}}
\newcommand{\inv}{^{-1}}
\newcommand{\id}{\mathrm{id}}
\newcommand{\gn}{$n$-pointed genus $g$ }
\newcommand{\gnprime}{$n'$-pointed genus $g$ }
\newcommand{\GW}{\mathrm{GW}}
\newcommand{\GWon}{\GW_{0,n}}
\newcommand{\canal}{$\mathbb C$-analytic\xspace}
\newcommand{\nanal}{non-archimedean analytic\xspace}
\newcommand{\kanal}{$k$-analytic\xspace}
\newcommand{\ddim}{$d$-dimensional\xspace}
\newcommand{\ndim}{$n$-dimensional\xspace}
\newcommand{\narch}{non-archimedean\xspace}
\newcommand{\nminusone}{$(n\!-\!1)$}
\newcommand{\nminustwo}{$(n\!-\!2)$}
\newcommand{\red}{^\mathrm{red}}
\renewcommand{\th}{^\mathrm{\tiny th}}
\newcommand{\Wall}{\mathit{Wall}}
\newcommand{\vlb}{virtual line bundle\xspace}
\newcommand{\mvlb}{metrized \vlb}
\newcommand{\wrt}{with respect to\xspace}
\newcommand{\Zaffine}{$\mathbb Z$-affine\xspace}
\newcommand{\sw}{^\mathrm{sw}}
\newcommand{\Trop}{\mathrm{Trop}}
\newcommand{\trop}{^\mathrm{trop}}
\newcommand{\op}{^\mathrm{op}}
\newcommand{\Cech}{\check{\mathcal C}}
\newcommand{\DM}{Deligne-Mumford\xspace}
\providecommand{\abs}[1]{\lvert#1\rvert}
\providecommand{\norm}[1]{\lVert#1\rVert}
\newcommand{\fr}{\mathfrak}
\newcommand{\p}{\partial}



% Arrows
\newcommand*{\longhookrightarrow}{\ensuremath{\lhook\joinrel\relbar\joinrel\rightarrow}}
\newcommand*{\DashedArrow}[1][]{\mathbin{\tikz [baseline=-0.25ex,-latex, dashed,#1] \draw [#1] (0pt,0.5ex) -- (1.3em,0.5ex);}}

\usetikzlibrary{decorations.markings} %arrows for open immersions and closed immersions
\tikzset{
  closed/.style = {decoration = {markings, mark = at position 0.5 with { \node[transform shape, xscale = .8, yscale=.4] {/}; } }, postaction = {decorate} },
  open/.style = {decoration = {markings, mark = at position 0.5 with { \node[transform shape, scale = .7] {$\circ$}; } }, postaction = {decorate} }
}


%Operators
\DeclareMathOperator{\Alg}{Alg}
\DeclareMathOperator{\Anc}{Anc}
\DeclareMathOperator{\Area}{Area}
\DeclareMathOperator{\at}{at}
\DeclareMathOperator{\Aut}{Aut}
\DeclareMathOperator{\Bl}{Bl}
\DeclareMathOperator{\cdga}{cdga}
\DeclareMathOperator{\CH}{CH}
\DeclareMathOperator{\Ch}{Ch}
\DeclareMathOperator{\Chow}{Chow}
\DeclareMathOperator{\Coker}{Coker}
\DeclareMathOperator{\codim}{codim}
\DeclareMathOperator{\cosk}{cosk}
\DeclareMathOperator{\Der}{Der}
\DeclareMathOperator{\dgVect}{dgVect}
\DeclareMathOperator{\Div}{Div}
\DeclareMathOperator{\dist}{dist}
\DeclareMathOperator{\dMan}{dMan}
\DeclareMathOperator{\End}{End}
\DeclareMathOperator{\Ext}{Ext}
\DeclareMathOperator{\Fun}{Fun}
\DeclareMathOperator{\FunR}{Fun^R}
\DeclareMathOperator{\FunL}{Fun^L}
\DeclareMathOperator{\Gal}{Gal}
\DeclareMathOperator{\Hom}{Hom}
\DeclareMathOperator{\Image}{Im}
\DeclareMathOperator{\Int}{Int}
\DeclareMathOperator{\Isom}{Isom}
\DeclareMathOperator{\Ker}{Ker}
\DeclareMathOperator{\KurNbd}{KurNbd}
\DeclareMathOperator{\loc}{loc}
\DeclareMathOperator{\LocTopInf}{LocTopInf}
\DeclareMathOperator{\Map}{Map}
\DeclareMathOperator{\Mor}{Mor}
\DeclareMathOperator{\NE}{NE}
\DeclareMathOperator{\oStar}{\widebar{\Star}}
\DeclareMathOperator{\pt}{pt}
\DeclareMathOperator{\Pic}{Pic}
\DeclareMathOperator{\Proj}{Proj}
\DeclareMathOperator{\rank}{rank}
\DeclareMathOperator{\Res}{Res}
\DeclareMathOperator{\RHom}{RHom}
\DeclareMathOperator{\Sp}{Sp}
\DeclareMathOperator{\Spa}{Spa}
\DeclareMathOperator{\SpB}{Sp_\mathrm{B}}
\DeclareMathOperator{\Spec}{Spec}
\DeclareMathOperator{\Spf}{Spf}
\DeclareMathOperator{\Star}{Star}
\DeclareMathOperator{\supp}{supp}
\DeclareMathOperator{\Sym}{Sym}
\DeclareMathOperator{\Symp}{Symp}
\DeclareMathOperator{\Td}{Td}
\DeclareMathOperator{\Tdisc}{T_{\text{disc}}}
\DeclareMathOperator{\Tr}{Tr}
\DeclareMathOperator{\tr}{tr}
\DeclareMathOperator{\val}{val}
\DeclareMathOperator{\vdim}{vdim}
\DeclareMathOperator{\Vect}{Vect}
\DeclareMathOperator{\vir}{vir}

\DeclareMathOperator*{\hofib}{hofib}
\DeclareMathOperator*{\hocofib}{hocofib}
\DeclareMathOperator*{\colim}{colim}
\DeclareMathOperator*{\holim}{holim}
\DeclareMathOperator*{\hocolim}{hocolim}
\DeclareMathOperator*{\cotimes}{\widehat{\otimes}}


\title{Derived Algebraic Geometry Seminar: Upenn 2016}

\begin{document}

\maketitle
\tableofcontents

\chapter*{Introduction}
\addcontentsline{toc}{chapter}{Introduction} \markboth{INTRODUCTION}{}

This contains notes livetexed for the Derived Algebraic Geometry Seminar currently being held at the University of Pennsylvania math department in the 2016-17 academic year.  We are following Mauro Porta's plan to introduce derived algebraic geometry, and then to look at derived geometric objects with extra structure (initially we will be looking at the case of symplectic and Poisson structures).

This is a draft and errors should be expected.

\chapter{$\infty$-category theory}
Talk by Mauro Porta.

\section{Why $\infty$-categories?}

Our main reason for studying $\infty$-categories in this seminar is that derived schemes form an $\infty$-category.
Some other applications of $\infty$-categories are the following.

\begin{enumerate}
\item \label{item:formal_moduli}
Formal moduli problems over a field $k$ of characteristic 0 are equivalent to $\text{dgLie}_{k}$, but this is
an equivalence of $\infty$-categories. We can see explicitly why this equivalence is plausible. For $F$ a formal moduli problem, 
$T_xF[1]$ is a dgLie algebra. Conversely, Maurer-Cartan elements on the RHS determine $F(k[\epsilon])$, 
i.e. infinitesimal formal moduli problems. Brackets then allow the complete recovery of $F$. 

\item 
The $\infty$-category of rational homotopy types is equivalent to that of dgLie algebras over $\Q$, concentrated in
positive degrees:
\[	S_*^{\text{rat}} \cong \text{dgLie}_{\Q}^{\geq 1}	\] 
This statement is related to item \ref{item:formal_moduli}: Lurie gives nice proof using formal moduli problems, see 
\cite{DAG-XIII}.

\item To $X \in \Sch_k$, we associate its derived category of quasi-coherent sheaves, $D(X) = D(QCoh(X))$. 
It's a powerful invariant of $X$, especially when $X$ is not
smooth. For example, it contains the cotangent complex and dualizing complex, $\bbL_X , \omega_X \in D(X)$, which are not 
necessarily bounded if $X$ is not smooth.

The problem is that we cannot reconstruct $D(X)$, the derived category in the classical sense, by patching: $D(X) \not \simeq 
\lim_{\{U\} \text{ Zariski cover }} D(U)$. For example, take $X = \bbP^1_k$, and its standard cover by 2 open affines $U_0, U_1$.
We show that the functor:
\[	D(\bbP^1) \to D(U_0) \times_{D(U_{01})} D(U_1)	\]
is not faithful, by exhibiting a morphism in $D(\bbP^1)$ which gets mapped to 0. Start from the observation that morphisms
from the structure sheaf $\mathcal{O}_{\bbP^1}$ are the same as sections of the target sheaf, which implies:
\[	\R(\Hom) \big(\mathcal{O}_{\bbP^1},\mathcal{O}_{\bbP^1}(-2)[1]\big) \cong \R \Gamma\big(\mathcal{O}_{\bbP^1}(-2)[1]\big) .	\]
This complex has nontrivial cohomology in degree 0:
\begin{equation}
\label{eq:derived_counterex}
	H^0 \R \Gamma\big(\mathcal{O}_{\bbP^1}(-2)[1]\big) \cong H^1\big(\mathcal{O}_{\bbP^1}(-2)\big) \cong k.
\end{equation}
However, when passing to the affine patches, $D(U_i) \simeq D(k[T]-Mod)$, and the complexes corresponding to the
restrictions of $\mathcal{O}_{\bbP^1}$ and $\mathcal{O}_{\bbP^1}(-2)[1]$ are the following.
\[
\begin{tikzcd}
0\arrow{r} & 0\arrow{r} & k[T]\arrow{r} & 0 \\
0\arrow{r} & k[T]\arrow{r} & 0\arrow{r} & 0
\end{tikzcd}
\]
As such, there are no non-zero morphisms between the restrictions. Equivalently, when restricting to affine opens,
the first cohomology in equation \ref{eq:derived_counterex} is 0.

On the other hand, we will see that the $\infty$-derived category of $X$ (which we temporarily 
denote by $L_{\text{qcoh}}(X)$) can be patched using the homotopy fiber product:
\[	L_{qcoh}(\bbP^1_k) \simeq  L_{qcoh}(U_0) \times_{L_{qcoh}(U_{01})} L_{qcoh}(U_1) .\]

\item Let $\mathcal{M}_{ell}$ be the moduli stack of elliptic curves, i.e. the functor $F$ sending
$\Spec(A)$ to the classes of elliptic curves over $\Spec(A)$. It is not a sheaf, because two elliptic curves can become
isomorphic after a base extension. The problem here is that we were trying to take $F : \cA ff^{\op} \to \cS et$, and we can't talk
about isomorphisms in $\cS et$. Classically one solves this problem by replacing sets by groupoids, which are equivalent to
1-homotopy types.
\[
\begin{tikzcd}
\; & & \cG pd \cong \cS^{\leq 1}\arrow{d} \\
\cA ff^{\op}\arrow{urr}{\text{stacks}}\arrow[swap]{rr}{\begin{array}{c}\text{naive moduli}\\
\text{problems}\end{array}} & & \cS et \cong \cS^{\leq 0}
\end{tikzcd}
\]
We can define higher stacks by extending the tower to higher homotopy types, and ultimately to the category of spaces.
\[
\begin{tikzcd}
\; & &  \cS\arrow{d} \\
\; & &  \vdots\arrow{d} \\
\; & &  \cS^{\leq 1}\arrow{d} \\
\cA ff^{\op}\arrow{uuurr}{\begin{array}{c}\text{higher}\\\text{stacks}\end{array}}
\arrow{urr}{\text{stacks}}\arrow[swap]{rr}{\begin{array}{c}\text{naive moduli}\\\text{problems}\end{array}} & & 
\cS et \cong \cS^{\leq 0}
\end{tikzcd}
\]
In later talks, we'll
see that the perfect complexes $\cP erf$ form an $\infty$-stack which doesn't factor through finite homotopy types.
\end{enumerate}


\section{Three ways of working with $\infty$-categories}
To be attempted in order of desperation:
\begin{enumerate}
\item \label{model_indep} 
Reason model-independently to get a clean proof. The trick is that there are key statements (not proven model
independently; some are proven by Lurie and can be found in \cite{HTT}) which behave like a ``non-minimal set of axioms''.
One should learn a roadmap to \cite{HTT}, in order to know where to find these statements.

\item Internal rectification. Cut the number of homotopies necessary to define the object. Example: an $\infty$-category with
products, see it as a symmetric monoidal category with products. $Mon_{E_1}(\cC) \simeq Fun^{\times}(\Delta^{\op}, \cC)
\to Fun(\Delta^{\op}, \cC)$.
The reference is \cite{Lurie_Higher_algebra}, 4.1.2.6. 

\item \label{item:real_rectification}
Try a ``real rectification'' result, i.e. work with a model-categorical presentation. For example, take $\cS$,
the $\infty$-category
of spaces, fix $x,y \in X$, want diagram. $\cS = \infty(sSet_{Kan})$. Theorem 4.2.4.1 in \cite{HTT} says that, in this situation, 
$\infty$-categorical limits correspond to homotopy limits. We use this theorem to show that $Path_X(x,y)$ is the $\infty$-limit
of the diagram.

Rectification: something is defined up to homotopy, and we try to reduce the necessary homotopies. Suppose we have
$\cM$, compare $Fun(\Delta^2,\cM)$ to $\Fun(\Delta^2, \infty \cM)$. The first one gives 3 objects and 3 morphisms, while
the other gives a homotopy between composition and the other map. The theorem is that you can forget the homotopy, and
just remember the data on the LHS. (HTT 4.2.4.4)
\end{enumerate}

In what follows we give examples where we can get by with procedure \ref{model_indep}. 

\begin{defin}
An $\infty$-cateogry is a simplicial set $\cC$ such that all inner horns have fillers. In other words, for all $0<i<n$,
the dotted arrow in the following diagram exists.
\[
\begin{tikzcd}
\Lambda^{n}_{i}\arrow[hook]{d}\arrow{r} & \cC \\
\Delta^n\arrow[dotted]{ur} & 
\end{tikzcd}
\]
\end{defin}

Note that this achieves what we want: inner horn fillings act as composition of morphisms, but this composition is not unique. 
``Higher Topos Theory is the book where all of category theory is carried out without ever talking about composition.'' 
A few problems arise
from here:

\begin{enumerate}
\item How do we define Yoneda? A morphism $X \to Y$ is supposed to determine a morphism $h_X \to h_Y$ by composition, which
is not well-defined.
\item Let $\cC$ be an $\infty$-category. We want $f: x \to y$ in $\cC$ to determine a functor $f_*: \cC_{/X} \to \cC_{/Y}$
between over-categories, where, morally speaking, $g: Z \to X$ is sent to the composition $f \circ g$.
Again, this composition is not well-defined.
\end{enumerate}

To the rescue comes Corollary 2.4.7.12 in \cite{HTT}. 
\begin{thm}
\label{thm:cartesian_fib}
Let $f: \cC \to \cD$ be an $\infty$-functor between $\infty$-categories. Then the projection
\[	\mathcal{P} : Fun(\Delta^1, \cD) \times_{Fun(\{1\},\cD)} \cC \to Fun(\{0\}, \cD) 	\]
is a \textbf{cartesian fibration}. Moreover, a morphism in the source is $\mathcal{P}$-\textbf{cartesian} iff its image in $\cC$ 
is an \textbf{equivalence}.
\end{thm}
Note that the $\infty$-functors $Fun(\cC,\cD)$ are nothing but the internal $\Hom$ in $s\cS et$. 
\[	Fun(\cC,\cD)_n = s\cS et(\cC \times \Delta^n, \cD)	\]
It's standard to prove that, if $\cC, \cD$ are $\infty$-categories, then so is $\Hom(\cC, \cD)$.

We will spend much of section \ref{sect:cartesian} defining the terms in bold in Theorem \ref{thm:cartesian_fib}. In Example
\ref{eg:composition}, we will use Theorem \ref{thm:cartesian_fib} to obtain the desired pushforward map between overcategories.



\section{Equivalences and Cartesian fibrations}
\label{sect:cartesian}

\begin{defin}
$g : x \to y$ in $\cC$ is an \textbf{equivalence} if any of the following equivalent conditions hold.
\begin{enumerate}[(a)]
\item \label{item:inverse_fillers}
The outer horn which maps $01\mapsto g$ and $02 \mapsto 1_x$ has a filler.
\[
\begin{tikzcd}
\Lambda^2_0 \arrow{r}{g,1_x}\arrow[hook]{d} & \cC \\
\Delta^2 \arrow[dotted]{ur} & 
\end{tikzcd}
\]
Morally speaking, the restriction of the dotted arrow to the $12$ face is the right inverse of $g$.

Moreover, the outer horn which maps $12\mapsto g$ and $02 \mapsto 1_y$ has a filler.
\[
\begin{tikzcd}
\Lambda^2_2 \arrow{r}{1_y,g}\arrow[hook]{d} & \cC \\
\Delta^2 \arrow[dotted]{ur} & 
\end{tikzcd}
\]
Morally speaking, the restriction of the dotted arrow to the $01$ face is the left inverse of $g$.

\item \label{item:infinity_sphere}
The same as variant \ref{item:inverse_fillers}, but with higher homotopies included. Formally, we introduce the
Kan complex $S^{\infty}$, defined as 0-coskeleton of the discrete simplicial set with 2 vertices. (For more details see
the exercises \cite{Mauro_Exercises}.)
We say $g$ is equivalence if there is a lift in the following diagram.
\[
\begin{tikzcd}
\Delta^1 \arrow{r}{g}\arrow[hook]{d} & \cC \\
S^{\infty}\arrow[dotted]{ur} & 
\end{tikzcd}
\]

\item \label{item:iso_homcat}
We say that $g$ is an equivalence if its image in the homotopy category $h(\cC)$ is an isomorphism.
\footnote{Recall that this is a 1-category with objects Ob$(\cC)$ and morphisms $\Hom(x,y) = \pi_0(\cC(x,y))$.}
\end{enumerate}
\end{defin}

In the definition, going from version \ref{item:infinity_sphere} to version \ref{item:inverse_fillers} of is a rectification result,
in the sense of procedure \ref{item:real_rectification} described above.

Next, we recall the notions of cartesian morphism and cartesian fibration in the context of 1-categories.
\begin{defin}
Let $\mathcal{P} : C \to D$ be a functor between 1-categories. If $x \in Ob(C)$ and $f\in \Hom(x,y)$, we use the notation
$\bar x := \mathcal{P}(x)$, $\bar f = \mathcal{P}(f)$. In the following diagram, the first 2 rows are in $C$, while the
third one is in $D$. However, we would like to think about the ``square'' as a pullback square.
\[
\begin{tikzcd}
z\arrow[dashed]{dd}{\mathcal{P}} & & \\
& x\arrow{r}{f}\arrow[dashed]{d}{\mathcal{P}} & y\arrow[dashed]{d}{\mathcal{P}} \\
\bar z & \bar x \arrow{r}{\bar f} & \bar y
\end{tikzcd}
\]
We say that $f$ is a $\mathcal{P}$-\textbf{cartesian morphism} if the data of a morphism $z \to y$ in $C$ and a 
morphism $\bar z \to \bar x$
in $D$ uniquely determine a morphism $z \to x$ in $C$, such that the ``diagram'' commutes.

We say that $\mathcal{P}$ is a \textbf{cartesian fibration} if for all $y \in C$ and all $\bar x \overset{\bar f}{\to} \bar y$ 
morphism in $D$, $\exists f : x \to y \in \mathcal{C}$ such that $\mathcal{P}(f) = \bar f$ and $f$ is $\mathcal{P}$-cartesian.
\end{defin}

The analogous definitions for $\infty$-categories are the following.

\begin{defin}
Let $\mathcal{P} : \cC \to \cD$ be an $\infty$-functor. A 1-morphism in $\cC$, which is the same as an edge $f: \Delta^1
\to \cC$, is $\mathcal{P}$\textbf{-cartesian} if for all $n\geq 2$, the following outer horn has a filler.
\[
\begin{tikzcd}
\Delta^1 = \Delta^{\{n-1,n\}}\arrow{d}\arrow{dr}{f} & \\
\Lambda^n_n\arrow{d}\arrow{r} & \cC \arrow{d}{\mathcal{P}} \\
\Delta^n \arrow{r}\arrow[dashed]{ur} & \cD
\end{tikzcd}
\]
Morally speaking, when $n=2$, this says that for any edge $g: z \to f(1)$ and edge $\bar h : \bar z \to \overline{f(0)}$,
there exist an edge $h : z \to f(0)$ and a homotopy $g \simeq f \circ h$, such that $\mathcal{P}(h) = \bar h$.

We say that $\mathcal{P}$ is a \textbf{cartesian fibration} if for every edge $a : \bar x \to \bar y$ of $\cD$, and every
object $y$ of $\cC$ such that $\mathcal{P}(y) = \bar y$, there exists a $\mathcal{P}$-cartesian edge $f : x \to y$ such that
$\mathcal{P}(f) = a$.
\end{defin}

Recall that, in the study of fibered 1-categories, one proves that cartesian fibrations with base $D$ are the same as
lax 2-functors from $D$ to the 2-category of 1-categories. (This is known as the ``Grothendieck construction'', see
for example, Proposition I.3.26 in \cite{FGAex}.) Explicitly, given a cartesian fibration $\mathcal{P} : C \to D$,
the corresponding lax 2-functor maps an object $d \in D$ to the fiber $\mathcal{P}^{-1}(d)$.
Theorem 3.2.0.1, the main theorem of Chapter 3 in \cite{HTT}, is the analog of this result for the setting of $\infty$-categories.

\begin{thm}
\label{thm:cartesian_equiv}
For any $\infty$-category $\cC$, there is an equivalence of $\infty$-categories:
\begin{equation}		
\text{CartesianFibr}/\cC \simeq \Fun(\cC^{op},\Cat_{\infty}).
\end{equation}
\end{thm}

\begin{eg}
\label{eg:composition}
Recall that we started out by trying to construct an $\infty$-functor $f_* : \cC_{/x} \to \cC_{/y}$ between overcategories,
given an 1-morphism $f: x \to y$ in $\cC$. Taking $F : \cC \to \cC$ as the identity, Theorem \ref{thm:cartesian_fib} gives
a Cartesian fibration over $\cC$:
\[	  \big\{ (f : x \to y, a) | \{f :x \to y\} \in \cC, F(a) \cong y \big\} \to \cC,	\]
where a pair $(f : x \to y, a)$ maps to $x$. We recognize the fiber over $x$ as the undercategory $\cC_{x/}$:
\[	\Hom_{sSet}(\Delta^n, \cC_{x/}) = \big\{\alpha: \Delta^{n+1} \to \cC | \alpha_{\Delta^{[0,\dots,n}} = x \big\}.	\]

Theorem \ref{thm:cartesian_equiv} then produces an $\infty$-functor:
\begin{align*}
\cC^{\op} &\to \infty-\cC at \\
x & \mapsto \cC_{x/} \\
f:x\to y & \mapsto f^* : \cC_{y/} \to \cC_{x/}.
\end{align*}
We have obtained a pullback map on undercategories. To obtain the pushforward on overcategories, start with $F : \cC^{\op} \to \cC$
as the contravariant identity functor instead.
\end{eg}

Next, we discuss a simpler example. Let $\cC$ be an $\infty$-category, and let $x \in \cC$ be an initial object. 
We want to construct a functor $\cC \to \cC_{x/}$. Note that this is silly in 1-category theory, since there's a unique 
morphism $x \to y$. To aid us in the context of $\infty$-categories, we start
by giving a good definition. 

\begin{defin}
$x \in \cC$ \textbf{initial} if $\forall y \in \cC$, $Map_{\cC}(x,y)$ is contractible.
\end{defin}

The key result, proved, for example, in \cite{Groth}, is the following.
\begin{prop}
If $\cC$ is an $\infty$ category, then $x \in \cC$ is initial iff the canonical projection $\cC_{x/} \to \cC$ is
a trivial Kan fibration.
\end{prop}

To solve our problem, note that $\cC$ is cofibrant in the Kan model structure, so there exists a lift in the diagram:
\[
\begin{tikzcd}
\emptyset\arrow{d}\arrow{r} & \cC_{x/}\arrow{d} \\
\cC \arrow{r}{\id}\arrow[dashed]{ur} & \cC .
\end{tikzcd}
\]

In the exercises, we also encounter the following problem. Suppose $\cC$ has pushouts and a zero object. Construct 
an $\infty$-functor $\cC \to \cC$ sending $x$ to the pushout of 0 and 0 over $x$. \todo{write this up, either here or in the
exercises}






\chapter{Derived Affines}
Talk by Benedict Morrissey.

\section{3 perspectives on derived affines}

First recall the notion of affines in classical AG: $\Aff_k^{Cl} \simeq (\cC Ring)^{\op}$. We get schemes by gluing these together.
There's also the functor of points viewpoint: $X \in \Aff_k^{Cl}$ defines a sheaf by sending $\Spec R \mapsto \Hom
(\Spec R, X)$. The schemes are then precisely the sheaves in the Zariski topology.  Already in classical AG, there exist 
constructions which move us out of this category: 
both Serre's intersection theorem and Illusie's notion of the cotangent complex use derived functors.
So by introducing DAG, we will understand better these structures in classical AG.

We will talk about 3 approaches to derived affines --- all of these consist of embedding the classical category $CRing_{k}$ into a larger category in which we have a derived tensor product.  In this section we assume that we are working over a ring $k$ of characteristic zero\footnote{We note here that we neglect to mention the important generalization of rings given by E-$\infty$ ring spectra, as described in chapter 7 of \cite{Lurie_Higher_algebra}.  In the case where we are working over $k$ a $\mathbb{Q}$-algebra this infinity category is equivalent to those described in this section as shown in  \cite{Lurie_DAG_V} proposition 4.1.11.  When we remove the characteristic zero assumption the statements about the Model structure on CDGA's no longer hold.  One can still use simplicial commutative rings or E-$\infty$ algebras, though these give different $\infty$ categories.}.
\begin{enumerate}
\item \label{item:scr} Simplicial commutative rings;
\item \label{item:cdga} Commutative differential graded algebras. (CDGA's)
\item \label{item:lawvere} Lawvere theory;
\end{enumerate}

\begin{rem}
Classically gluing is easy. For example, fiber products are computed by reducing to the affine case, where it's just the
tensor product of rings. In DAG, the derived tensor product is only defined up to quasi-isomorphism, so gluing can only be
defined in a category which allows homotopy, such as an $\infty$-category. For today's talk we mostly use the model category description;
an application of Dwyer-Kan localization produces an $\infty$-category.
\end{rem}

\subsection{Simplicial Commutative Rings}

For approach \ref{item:scr}, recall that the simplicial category $\Delta$ is:
\[ 	Ob(\Delta) = \{n \in \N \cup \{0\} \}		\]
where morphisms are compositions of face maps $\delta_{i}^{n}:[n-1]\rightarrow [n]$ for $0 \leq i < n$ and degeneracy maps $s_{i}^{n}: [n+1]\rightarrow [n]$, subject to the simplicial identities as can be found in e.g. \cite{gillam2013simplicial}. 
\begin{defin}
The \textbf{category of simplicial commutative rings} is the category of contravariant functors:
\[	 SCR_k = \Hom(\Delta^{\op}, CRing_k).	\] 
\end{defin}

\begin{rem}
There's a model category structure on this: fibrations are Kan fibrations on the underlying simplicial sets, i.e. morphisms
$f: A \to B$ of simplicial commutative rings, such that all horns have fillers:
\[
\begin{tikzcd}
\Lambda^n_j\arrow[hook]{d}\arrow{r} & A\arrow{d}{f} \\
\Delta^n\arrow{r}\arrow[dashed]{ur} & B.
\end{tikzcd}
\]

Weak equivalences
are weak homotopy equivalences on the underlying simplicial sets. Cofibrations are then determined from the axioms of a
model category; note that they are \textit{not} the same as cofibrations of the underlying simplicial sets.
\end{rem}

\begin{rem}
\label{rem:transfer_model_structure}
We're using transfer to put the model structure on $SCR_k$. To explain what that means, under suitable conditions, there's 
a general procedure for defining a model structure on a category $\cB$, given a model category $\cA$ and an adjoint functor
pair:
\[
\begin{tikzcd}
 \cA \arrow[shift left]{r}{F} & \cB \arrow[shift left]{l}{G} .
\end{tikzcd}
\]
The procedure forces the adjoint functor pair to be a Quillen adjunction. In our case, we use the free-forgetful adjunction:
\[
\begin{tikzcd}
 s\cS et \arrow[shift left]{r}{F} & SCR_k \arrow[shift left]{l}{U}
\end{tikzcd}
\]
to transfer the Kan model structure to $SCR_k$. The key point which allows this to work is that all objects are fibrant. 
Cofibrations are more difficult to characterize, but the cofibrant objects are precisely the quasi-free ones. (That is,
the ones isomorphic to a free object.)
\end{rem}

\subsection{CDGA's}

Next, we introduce CDGA's and the Dold-Kan equivalence --- which shows that this category is the same as that of simplicial commutative rings under our assumption that we are working over a characteristic zero field. Recall that we have a Quillen equivalence:
\[
\begin{tikzcd}
	s\cV ect \arrow[shift left]{r} & dg-\cV ect^{\leq 0} \arrow[shift left]{l}
\end{tikzcd}
\]
between simplicial vector spaces and differential graded vector spaces, concentrated in nonpositive degrees. We want to
talk about commutative monoids in these categories, $sc\cA lg_k$ and $cdg-\cA lg^{\leq 0}$, respectively.
The model structure on $cdg-\cA lg^{\leq 0}$ can also be obtained by transfer from the free-forgetful adjunction; we obtain
that the weak equivalences are quasi-isomorphisms, and the fibrations are degree-wise surjections.

\begin{thm}[Symmetric monoidal Dold-Kan (A proof can be found in \cite{schwede2003equivalences})]
There is a Quillen equivalence:\footnote{Note that, in general, a Quillen equivalence is not an equivalence of categories.
It does, however, induce an equivalence of Dwyer-Kan localizations (and hence also of homotopy categories).}
\[
\begin{tikzcd}
	sc\cA lg_k \arrow[shift left]{r}{N} & cdg-\cA lg_k^{\leq 0} \arrow[shift left]{l}{\Gamma}.	
\end{tikzcd}
\]
Moreover, if the simplicial commutative algebra $A_*$ corresponds to the commutative dg-algebra $B_{\bullet}$, then
$\pi_i(A_*) \cong H^i(B_{\bullet})$.
\end{thm}

\begin{rem}
We describe $N$:
$A_* \in sc\cA lg_k$ maps in the first stage to $\tilde A_{\bullet}$, where $\tilde A_{-n} = A_n$, and the differential 
is the alternating sum of the face maps.
$N(A_*)$ is then the quotient of $\tilde A_{\bullet}$ by the images of the degeneracy maps.
For $C$ a CDGA we can describe $(\Gamma C)_{n}:=Hom_{ch^{-}}(N(\Delta^{n}), C)$ where we are in fact using the above definition of $N$ to give a functor from simplicial abelian groups to $ch^{-}$ --- the category of non positively graded chain complexes, and $\Delta^{n}$ is the simplicial abelian group freely generated by an $n$-simplex.  
\end{rem}

We have a similar result for the category of simplicial modules for a given simplicial ring, and the category of dg-modules for its image in CDGA's.  Note that the categories of simplicial modules and of non-positively graded modules for a given CDGA both have model structures.

\begin{thm}[\cite{schwede2003equivalences}]
If $A$ is a simplicial ring the categories of simplicial $A$-modules and of negatively graded $N(A)$-modules are Quillen equivalent.

If $A$ is a CDGA, the categories of negatively graded $A$-modules and of simplicial $\Gamma(A)$-modules are equivalent.
\end{thm}


We now define the truncation functor of a CDGA.  We can use the above Quillen equivalence to also define truncation functors on the category of Simplicial Commutative Rings.

Let $CDGA_{k}^{\leq n} \hookrightarrow CDGA_{k}$ denote the subcategory of $CDGA_{k}$ consisting of objects $A$ such that $H^{i}(A) =0$ for all $i>n$.  Note that $CDGA_{k}^{\leq 0} \cong CRing_{k}$.  The inclusion has a right adjoing $\tau^{\leq n}: CDGA_{k}\rightarrow CDGA^{\leq n}$.  For $A=(A_{n})$, \[(\tau^{\leq n}(A))_{m}= \begin{cases} 
      A_{m} & 0 > m > -n \\
      A/im(d^{m+1}) & m=n \\
      0 & m < n.
   \end{cases}
\]

%Note: Bousfield localization to put a model cateogry structure on n-homotopy type simplicial sets. Just declare the map
%from the $n$-sphere to the point to be a weak equivalence.

\subsection{Lawvere Theories}

We move on to approach \ref{item:lawvere} to derived affines, the Lawvere Theory description. 
This is important because
it's the only one of the 3 procedures which carries through in the analytic setting (see e.g. \cite{Porta_Yu_Higher_analytic_stacks_2014}). You get holomorphic rings, smooth rings,
and much of the theory of DAG can be carried in this setting.\todo{clear this part up}

The idea of Lawvere theory is to describe all objects with some type of algebraic structure as functors between the free objects and the category $Set$.
For example,
\[	AbGps \cong \Fun^{\times}(FAb^{\op}, Set).	\]
There's a map $\Z \to \Z \times \Z$, which sends $1\mapsto 1\times 1$. Since $F$ preserves products, $F(\Z) \times F(\Z)
\cong F(\Z \times \Z) \to F(\Z)$.  In fact $F(\Z)$ has the structure of an abelian group, and the equivalence of categories above is realized on objects by $F \mapsto F(Z)$.  We denote by $T_{disc}$ the opposite category of free commutative rings.  Free commutative rings are the rings $k[x_1, \dots, x_n]$. Hence $T_{disc}$ is the subcategory of the category of affine schemes with objects the planes $\{\bbA^n\}$.
$\Fun^{\times}(T_{disc}, Set) \cong CRing$. On objects we map a functor to it's value on the group ring $\bbA^{1}$, $F\mapsto F(\bbA^{1})$.  The addition and multiplication on $\bbA^{1}$ give $F(\bbA^{1})$ the structure of a (commutative) ring.  

Now pass to $SCR_k \cong s\Fun^{\times}(T_{disc}, Set) \cong \Fun^{\times}(T_{disc}, sSet) \cong \Fun^{\times}
(T_{disc}, S)$, where $S$ is the infinity category of spaces. The last step is a very hard rectification theorem, proved
by Lurie-Bergner.\footnote{HTT Propositions 5.5.9.2} 

 %$\Fun^{\times}(T_{disc}, sSet)_{projective}$. 

\section{Derived Affines as Ringed Spaces}

Finally, we take the viewpoint of seeing a scheme as a locally ringed space. For $A \in cdg-\cA lg_k$, we look at
the truncation $\Spec H^0(A)$, which is an affine scheme in the classical sense. We can regard $A$ 
as a sheaf of cdg-algebras on the truncation, as long as we can understand how localization works for cdg-algebras. 
We claim that it suffices to localize the commutative algebra $A_0$. Indeed, we have the multiplication map:
\[	\mu:	A_0 \times A_i \to A_i,		\]
so given a multiplicative subset $S \subset A_0$, we define the localization $S^{-1}A_i$ as $\mu(S^{-1}A_0 \times A_i)$.
If this makes sense, we get a sheaf $\mathcal{O}_A$ of cdg-algebras.

We would like to define derived affines as pairs $\big(\Spec H^0(A), \mathcal{O}_A\big)$. There is a subtlety: a priori 
this only gives a 2-category, and we need $\infty$-categories.  The key to resolving this is to define the notion of a sheaf valued in an $(\infty,1)$-category.



\section{Our favorite classes of morphisms}
\begin{defin}
Given $f: A \to B$ in $SCR_k$, we get maps:
\[	\pi_*(A) \otimes_{\pi_0(A)} \pi_0(B) \to \pi_*(B)	\]
of graded modules. We say that $f$ is \textbf{strong} if this is an isomorphism of graded modules.
\end{defin}

\begin{defin}
\label{def:fav_morphisms}
We define $f: A \to B$ to be \textbf{\'{e}tale} (resp. \textbf{smooth}, \textbf{Zariski open immersion}, \textbf{flat})
if $f$ is strong and $\pi_0(A) \to \pi_0(B)$ is \'{e}tale (resp. smooth, Zariski open immersion, flat) in the classical sense.
\end{defin}

\begin{rem}
The strength condition on $f$ is quite restrictive: for example, a strong map from a non-derived domain must have a non-derived
target.
\end{rem}

\begin{defin}
Let $X = \Spec(A)$ be a derived affine over $k$. Then the \textbf{small \'{e}tale site} of $X$ is:
\[	X_{\text{\'{e}t}} = \{\text{\'{e}tale maps }\Spec(B) \to \Spec(A)\}.	\]
\end{defin}

In order to obtain the small \'{e}tale site in the sense of classical AG, one needs to pass to the truncated version
of the \'etale maps: $\pi_0(f) : \Spec(\pi_0(B)) \to \Spec(\pi_0(A))$. Then one can prove there's an equivalence of 
$\infty$-categories between the derived and classical \'etale sites. In particular, this shows that 
$X_{\text{\'{e}t}}$ is a 1-category. This is one of the ingredients in the proof of the easy version
of Lurie representability. Moreover, the same holds for the small smooth site and the small Zariski site.

After introducing the cotangent complex $\bbL_f$ of a morphism $f$, we will see that $f$ is \'{e}tale iff $\pi_0(f)$ 
is of finite presentation and $\bbL_f \simeq 0$.

\begin{defin}
$f:A \to B$ is \textbf{of finite presentation} if the functor $\Map_A(B, - ) : sc\cR ing_k \to \cS$ commutes with
filtered colimits. 
\end{defin}

Unlike in the underived case, being of finite presentation is very strong, because it has a hidden regularity condition.
In particular, we have the proposition due to Lurie:

\begin{prop}
$f: A \to B$ is of finite presentation in the derived sense iff $\pi_0(f)$ is of finite presentation in the classical sense
(also called to order 0) and the cotangent complex $\bbL_f$ is perfect.
\end{prop}

\begin{eg}
Let $X = \bbA^3$, and $Y$ a closed subscheme of $X$ which is not a local complete intersection. 
Then the inclusion $\iota :Y \to X$ is not of finite presentation in the derived sense.
Indeed, by a conjecture of Quillen, which is now a theorem of Abramov, for maps between classical schemes, the cotangent 
complex is either concentrated in degrees 0 and -1, or it's unbounded. Since $Y$ is not lci, the first case is ruled out,
and $\bbL_{\iota}$ is unbounded.
\end{eg}





\chapter{Stable $\infty$-categories}
Talk by Michael Gerapetritis.

\section{Motivation}
\label{sect:motivation}

In the 1-categorical setting, if $\cC$ is a category, we may require that $\cC(A,B)$ be a set. 
To get particularly well-behaved categories, namely the additive categories, we require that $\cC(A,B)$ is actually
an abelian group. 

We try to replicate this in the $\infty$-category setting.
Let $\cC$ be an $\infty$-category, then $\cC(X,Y)$ is a space. We want to discover what is the good extra structure to have on 
this space; we will call the corresponding $\infty$-categories stable.


\section{Stable $\infty$-categories and triangulated 1-categories}

\begin{defin}
An $\infty$-category $\cC$ is \textbf{stable} if:
\begin{itemize}
\item $\cC$ is pointed, i.e. it has a zero object;
\item every morphism $f : X \to Y$ admits fibers and cofibers;
\item a triangle is a fiber iff it is a cofiber.
\end{itemize}
\end{defin}

Recall that a \textbf{triangle} in $\cC$ is a map of simplicial sets $\Delta^1 \times \Delta^1 \to \cC$,
i.e. a homotopy commutative diagram with the zero object in the bottom-left corner:
\[
\begin{tikzcd}
X\arrow{r}{f}\arrow{d} & Y\arrow{d}{g} \\
0\arrow{r} & Z
\end{tikzcd}
\]
The triangle is a \textbf{fiber} if it is a pullback square, and a \textbf{cofiber} if it is a pushout square. We say
that $f: X \to Y$ admits a fiber (resp. cofiber) when $\exists W$ (resp $Z$) such that:
\[
\begin{tikzcd}
W\arrow{r}\arrow{d} & X\arrow{d}{f} \\
0\arrow{r} & Y
\end{tikzcd}
\]
is a pullback square (or, respectively:
\[
\begin{tikzcd}
X\arrow{r}{f}\arrow{d} & Y\arrow{d} \\
0\arrow{r} & Z
\end{tikzcd}
\]
is a pushout square).

\begin{rem}
Note that the data of a triangle consists not only of homotopy commutative diagrams as above, but also of choices of 
homotopies between the branches. This is crucial, since it ensures that cones are
functorial at the level of the homotopy category. This functoriality does not hold in a general triangulated category. 
(See Theorem \ref{thm:stable_triang} for the relation between stable
$\infty$-categories and triangulated 1-categories.)
\end{rem}

\begin{eg}
Our two main examples are $\infty$-categories of spectra (see Section \ref{sect:spectra}) and of modules over a CDGA or SCR
(see Section \ref{sect:modules}).
\end{eg}

Recall the data for a triangulated category.

\begin{defin}
A category $\cD$ is triangulated if:
\begin{enumerate}
\item $\cD$ is additive;
\item \label{item:translation_equiv}
 $\cD$ admits a translation functor $T : \cD \overset{\simeq}{\to} \cD$;
\item $\cD$ has a collection of distinguished triangles:
\[
\begin{tikzcd}
X\arrow{r} & Y \arrow{r} & Z \arrow{r} & X[1]
\end{tikzcd}
\]
\end{enumerate}
This data is required to satisfy some axioms, but we won't go into details here.
\end{defin}

\begin{thm}
\label{thm:stable_triang}
If $\cC$ is a stable $\infty$-category, then $h\cC$ is triangulated.
\end{thm}

For a proof see \cite{Lurie_Higher_algebra}. We won't go over it, let's just say that translation is given by $\Sigma$,
and distinguished triangles are precisely the images of fiber sequences (or equivalently, cofiber sequences), as resulting
from the following diagram.
\[
\begin{tikzcd}
X\arrow{r}\arrow{d} & Y \arrow{r}\arrow{d} & 0\arrow{d} \\
0\arrow{r} & Z \arrow{r} & X[1]
\end{tikzcd}
\]

\begin{prop}
$\cC$ is stable iff the following hold:
\begin{enumerate}
\item $\cC$ admits finite limits and colimits;
\item any square is a pushout iff it is a pullback.
\end{enumerate}
\end{prop}
\begin{proof}
Again, we don't give a full proof. Let's just see why products and coproducts must exist in a stable $\infty$-category. 
Note first that $\Sigma$ is an equivalence of
$\infty$-categories. Indeed, $\Sigma$ is a left adjoint functor; moreover, the unit and counit of the adjunction become 
isomorphisms in the homotopy category, due to condition \ref{item:translation_equiv} in the definition of a triangulated
category. Then we use the following diagram.
\[
\begin{tikzcd}
\Omega(X)\arrow{d}\arrow{r} & 0\arrow{d}\arrow{r} & Y\arrow{d} \\
0\arrow{r} & X\arrow{r} & X \oplus Y
\end{tikzcd}
\]
We have defined $X \oplus Y$ as the cofiber of $\Omega(X) \overset{0}{\to} Y$, which is postulated to exist in a stable $\infty$
-category. This turns the outer rectangle into a pushout square, and it follows that the square on the right is also a pushout
square. Thus $X\oplus Y$ is the coproduct of $X$ and $Y$. We reason dually to obtain products.
\end{proof}

\begin{defin}
Let $\cC, \cC'$ be stable $\infty$-categories, and $F: \cC \to \cC'$ an $\infty$-functor which maps 0 objects to 0 objects. 
Equivalently, $F$ maps triangles to triangles.
If $F$ maps fiber sequences to fiber sequences, we say that $F$ is \textbf{exact}.
\end{defin}

\begin{lem}
\label{lem:exact}
TFAE:
\begin{enumerate}
\item $F$ is exact;
\item $F$ is right-exact, i.e. commutes with finite colimits;
\item $F$ is left-exact, i.e. commutes with finite limits.
\end{enumerate}
\end{lem}

This is very useful: sometimes it's really easy to check that a functor is right or left exact, e.g. if it's a left or right
adjoint, respectively.


\section{Modules}
\label{sect:modules}

For a useful example of the result in Lemma \ref{lem:exact}, we look at $\cC = A-\cM od$, where $A$ is a CDGA or SCR over $k$.
(By $A-\cM od$ we mean the unbounded derived category.) The easiest way
to see $A-\cM od$ as an $\infty$-category is to put a model structure on chain complexes, say the projective one, and then
take the underlying $\infty$-category.
We claim that $A-\cM od$ is a stable $\infty$-category. Using the theorem
Mauro talked about in Lecture 1, limits and colimits exists in the $\infty$-category iff they exist in the model category.
\todo{reference theorem} It remains to prove the following.

\begin{lem}
A triangle in $A-\cM od$ is a fiber iff it is a cofiber.
\end{lem}
\begin{proof}
We prove one direction; the other argument is dual to this one. Assume that $f:M^{\bullet} \to N^{\bullet}$ 
is the fiber of a map $g$. Take a cofibrant replacement of $f$, get $\tilde M, \tilde N$ cofibrant and a homotopy
pullback square:
\todo{figure out how to do the cartesian symbol in tikz}
\[
\begin{tikzcd}
\tilde M^{\bullet}\arrow[hook]{r}{\tilde f}\arrow{d} & \tilde N^{\bullet}\arrow{d}{\tilde g} \\
0\arrow{r}  & P^{\bullet} .
\end{tikzcd}
\]
$\tilde f$ is cofibrant, so it's a degree-wise injection. Then $g$ is a degreewise surjection, and it follows that the 
square is a strict pushout. \todo{wait, how did this work again?}
\end{proof}

Now suppose we have $f: A \to B$ a morphism of $CDGA_k^{\leq 0}$. It induces the adjunction of model categories:
\[
\begin{tikzcd}
A-Mod\arrow[shift left]{r}{f^*} & B-Mod\arrow[shift left]{l}{f_*},
\end{tikzcd}
\]
where $f_*$ is the forgetful functor, and $f^*(M) = M \otimes_A B$. So this gives an adjunction of $\infty$-categories:
\footnote{Here we use $L$ and $R$ to indicate that the functors are derived.
In later talks derived functors will be the default, and we will omit the symbols $L$ and $R$.}
\[
\begin{tikzcd}
\label{tikz:adjunction}
A-\cM od\arrow[shift left]{r}{Lf^*} & B-\cM od\arrow[shift left]{l}{Rf_*}.
\end{tikzcd}
\]
Explicitly, $Lf^*$ is constructed by first choosing a cofibrant replacement $\tilde M$ for $M$, 
and then taking $\tilde M \otimes_A B$. The
answer doesn't depend on cofibrant replacement, up to coherent isomorphism. Then $Lf^*$ is a left adjoint
functor, so it follows from general nonsense that it's right exact. Lemma \ref{lem:exact} then implies that $Lf^*$
is also left exact and exact.

\begin{rem}
\label{rem:exact_t}
If $f$ is not flat in the sense of Definition \ref{def:fav_morphisms}, then the exactness of
$Lf^*$ comes at the price of losing t-exactness. To explain what we mean, pick $M \in A-Mod$, such that
$H^i(M) = 0$ unless $i = 0$. But then $Lf^*(M) = M \otimes_A^{\bbL} B$, and $H^{-i}(M\otimes_A^{\bbL}B) = \Tor_i^A(M,B)$, which
is $\neq 0$ in general, because $f$ is not flat. So even though $M$ was homologically concentrated in degree 0, $Lf^*(M)$ may
not be. In other words, the failure of a functor of (Grothendieck) abelian categories to preserve limits translates into a lack 
of t-exactness of the derived functor. In the following section we define t-structures and t-exactness for $\infty$-categories.
\end{rem}


\section{t-structures}
\label{sect:t-struct}

\begin{defin}
If $\cC$ is a stable $\infty$-category, a \textbf{t-structure}\footnote{$t$ stands for truncation} on 
$\cC$ is the data of two full subcategories of $\cC$, $\cC^{\leq 0}$ and
$\cC^{\geq 0}$, \footnote{Note that we use cohomological notation, while Lurie in \cite{Lurie_Higher_algebra} uses homological
notation. Therefore gradings have opposite signs in this seminar and in \cite{Lurie_Higher_algebra}.} such that:

\begin{enumerate}
\item \label{item:no_morph_right}
$\pi_0 \Map_{\cC}(X,Y[-1]) = 0$ if $X \in \cC^{\leq 0}$ and $Y \in \cC^{\geq 0}$.
\footnote{In a stable $\infty$-category, we sometimes use the shift notation $[n]$ to denote the $|n|$-fold iterated application
of the $\Sigma$ functor (if $n$ is positive) or the $\Omega$ functor (if $n$ is negative). This notation is justified by
Proposition \ref{prop:stable_shift}.}

\item $X \in \cC^{\leq 0}, X[1]  \in \cC^{\leq 0}$;

\item \label{item:fiber_seq}
$\forall X$, $\exists$ fiber sequence $X' \to X \to X''$, where $X' \in \cC^{\leq 0}$, $X'' \in \cC^{\geq 1}$.
\end{enumerate}
\end{defin}

\begin{rem}
Condition \ref{item:no_morph_right} has the following intuitive meaning in the case $\cC = A-\cM od$. $0$-morphisms in 
$\cC$ are chain maps which preserve degree, while
higher morphisms are homotopies which shift the degree to the left; morphisms that shift degree to the right are not allowed.
Then, if $X \in \cC^{\leq 0}$ and $Y \in \cC^{\geq 0}$, no nonzero morphisms should be allowed between $X$ and $Y[-1]$:
\[
\begin{tikzcd}
\dots \arrow{r} & X_{-2} \arrow{r} & X_{-1} \arrow{r} & X_0 \arrow{r} & 0\arrow{r} & 0\arrow{r} & 0\arrow{r} & \dots \\
\dots \arrow{r} & 0 \arrow{r} & 0 \arrow{r} & 0 \arrow{r} & Y_0\arrow{r} & Y_1\arrow{r} & Y_2\arrow{r}  & \dots
\end{tikzcd}
\]
\end{rem}

\begin{rem}
$X'$ and $X''$ are uniquely determined by $X$.
\end{rem}

\begin{thm}
The inclusion $\cC^{\leq 0} \to \cC$ has a right adjoint, which we denote $\tau_{\leq 0} : \cC \to \cC^{\leq 0}$. Similarly
we get $\tau_{\geq 0} : \cC \to \cC^{\geq 0}$.
\end{thm}

\begin{cor}
For all $X \in \cC$, the fiber sequence of \ref{item:fiber_seq} is just:
\[	\tau_{\leq 0} X \to X \to \tau_{\geq 1} X.	\]
\end{cor}

\begin{prop}
Denote by $\cC^{\heartsuit} := \cC^{\leq 0} \cap \cC^{\geq 0}$, the \textbf{heart} or \textbf{core} of the t-structure. 
It is an abelian 1-category.
\end{prop}

\begin{prop}
\label{prop:stable_les_homology}
Let $\cC$ be stable. Then if:
\[	X \to Y \to Z	\]
is a fiber sequence, then we have a long exact sequence of $H^i$, where $H^i(X) := \tau_{\geq i} \circ \tau_{\leq i} (X)$.
\end{prop}

Putting the last few results together, from $\cC$ a presentable stable $\infty$-category with $t$-structure, the heart 
is Grothendieck abelian. Write $A = \cC^{\heartsuit}$. Then we can form $\cD(A)$, the $\infty$-derived category of $A$.
The next theorem describes the relationship between $\cC$ and $\cD(A)$.

\begin{thm}[Lurie]
$\cD(A)$ has a universal property which produces an $\infty$-functor:
\[	\cD(A) \to \cC.	\]
In general this is very far from being an equivalence.
\end{thm}

\begin{eg}
Let $A \in CDGA^{\leq 0}_k$. The theorem gives a map:
\begin{equation}
\label{eq:map_heart}
	(A-\cM od)^{\heartsuit} \to  (H^0(A)-Mod)^{\heartsuit}.
\end{equation}
This is one of the most important facts
in DAG, because it reduces problems about the $\infty$-category of $A$-modules to problems in classical categories of modules,
where one can work with generators and relations. The map in \ref{eq:map_heart} is an equivalence iff $A \simeq H^0(A)$ 
are quasi-isomorphic.\todo{figure out what's the precise relationship here}
\end{eg}

\begin{defin}
Let $\cC, \cD$ be stable $\infty$-categories with $t$-structures. Then an exact functor $F : \cC \to \cD$ is:
\begin{enumerate}
\item \textbf{left t-exact} if $F(\cC^{\leq 0}) \subset \cD^{\leq 0}$;
\item \textbf{right t-exact} if $F(\cC^{\geq 0}) \subset \cD^{\geq 0}$;
\item \textbf{t-exact} if both.
\end{enumerate}
\end{defin}

\begin{eg}
\label{eg:modules_t_exact}
For $A,B \in CDGA_k^{\leq 0}$, $f:A \to B$, we have the adjunction:
\[
\begin{tikzcd}
A-\cM od\arrow[shift left]{r}{Lf^*} & B-\cM od\arrow[shift left]{l}{Rf_*}.
\end{tikzcd}
\] 
Every object is fibrant, so we don't need to
derive the functors. $Rf_*$ is both left and right t-exact. $Lf^*$ is not right t-exact, because of nontrivial $\Tor^i$
terms; see \ref{rem:exact_t}. However, $Lf^*$ is right t-exact: 
morally speaking, Projective resolution only puts stuff in negative degrees. We give an $\infty$-categorical proof.

Pick $M \in A-Mod^{\geq 0}$. We want $Lf^*(M) \in B-Mod^{\leq 0}$. To check this is the same as checking that
$\forall N \in B-Mod^{\geq 1}$, $\Map_{B-Mod}(Lf^*M,N) \cong 0$. But this is $Map_{A-Mod}(M,Rf_*N) \cong 0$, which
follows since $Rf_*$ was t-exact.
\end{eg}


\section{Spectra}
\label{sect:spectra}

Going back to the question left unanswered in Section \ref{sect:motivation}, the extra structure we want on
morphism spaces of stable $\infty$-categories is $\Map_{\cC}(X,Y) \in \Sp^{\leq 0}$.

\begin{defin}
\textbf{Spectra} are sequences $\{F_i\}$ of objects in $\cC$ such that $F_n \simeq \Omega F_{n+1}$. Alternatively,
we identify them with objects of the homotopy limit:
\[	\dots \overset{\Omega}{\to} \cC \overset{\Omega}{\to} \cC \overset{\Omega}{\to} \dots	\]
\end{defin}

\begin{rem}
We must be careful with defining morphisms between spectra: we want squares to commute up to coherent homotopy.
Moreover, it's hard to get a monoidal model structure on the category of spectra: this was done only in the 2000s, after Hovey 
introduced symmetric spectra. Lurie has a very categorical and very nice way of putting a monoidal structure 
at the level of the $\infty$-category
directly. See the last chapter of \cite{Groth}, and also 4.8.2 of \cite{Lurie_Higher_algebra}.
\end{rem}

\begin{thm}
$Sp(\cC)$ is stable.
\end{thm}

This gives a canonical stabilization for every $\infty$-category. The proof of the theorem follows from the following characterization
of stable $\infty$-categories,
and the fact that $\Omega:Sp(\cC) \to Sp(\cC)$ is an equivalence.

\begin{prop}
\label{prop:stable_shift}
$\cC$ is a pointed $\infty$-category. TFAE:
\begin{enumerate}
\item $\cC$ is stable;
\item $\cC$ admits colimits and $\Sigma : \cC \to \cC$ is an equivalence;
\item $\cC$ admits limits and $\Omega : \cC \to \cC$ is an equivalence;
\end{enumerate}
\end{prop}

%\begin{thm}
%Let $K$ be a simplicial set, and $F : K \to \cC at_{\infty}$. These correspond to cartesian fibrations, by HTT chapter 3,
%as explained by Mauro in the first lecture. We find a cartesian fibration $\mathscr{P} : \mathscr{X} \to K$.
%\todo{insert diagram}
%We have $\N \to \Cat_{\infty}$, we associate $\mathscr{X} \overset{\mathscr{P}}{\to} \N$. Look at section $s
%: \N \to \mathscr{X}$. $n \to m$ goes to:\todo{complete the diagram}
%\[
%\begin{tikzcd}
%s(n)\arrow{d}\arrow{r} & s(m)\arrow{d} \\
%n \arrow{r} & m
%\end{tikzcd}
%\]
%Thus we get a cocartesian fibration. Conclusion: let $\cD$ be the full subcategory of $\Map_{/K}(K,\mathscr{X})$ 
%spanned by cocartesian sections. Then
%$\cD$ is equivalent to $invlim_{\cC at_{\infty}} K$. In Theorem 3.3.3.1 of \cite{HTT}, Lurie has $\cD = \Map^{\flat}_{K^{\sharp}}
%(K^{\sharp}, \mathscr{X}^{fillthisinhere})$.
%\end{thm}

%Upshot: we need $sSet^+_{/K}$, where some collection of arrows is marked. We think of the marked ones as the ones which
%will become cocartesian.
%\[	sSet^+ = \{ (L,S) | L \in sSet, S\subset \Fun(\Delta^1, L), S \text{ contains the identities }\}	\]
%We have two functors $sSet \to sSet^+$. One is $K \mapsto K^{\flat} = (K, \text{identities})$. The other is
%$K \mapsto (K,\text{all morphisms}) = K^{\sharp}$. $\mathscr{P} : \mathscr{X} \to K$ cartesian, so
%$\mathscr{X}^{fillthisin} = (\mathscr{X}, \text{p-cartesian edges})$.

%We have an adjunction $Forget: sSet^+ \to sSet_{Joyal} : (-)^{\sharp}$. What is $L(Forget)$? Cocartesian morphisms
%over a point are just equivalences. So $L(Forget)$ is $\infty$-categorical localization.






\chapter{The Cotangent Complex}
Talk by Sukjoo Lee.

\section{Motivation}
\label{sect:cc_motivation}

We recall from classical AG: if $f:A \to B$ is a homomorphism between commutative rings and $M$ is a $B$-module, 
an A-\textbf{derivation} of $B$ into $M$ is a map $d : B \to M$ such that:
\begin{itemize}
\item $d\big( f(a)\big) = 0$, for all $a\in A$;
\item $d(bb') = db\; b' + b\; db'$ (Leibniz rule).
\end{itemize}
We denote by $Der_A(B,M)$ the set of all derivations of $B$ into $M$. There is also an absolute version, where
we take $f:0 \to A$, and the first condition is automatic.

\begin{defin}
The \textbf{module of relative K\"ahler differentials} of $B$ over $A$ is a derivation $(\Omega^{1}_{B/A}, d_A)$ 
over $A$ satisfying the universal property:
\[
\begin{tikzcd}
B \arrow{r}{d_A}\arrow[swap]{rd}{d'} & \Omega^1_{B/A}\arrow[dotted]{d}{\exists !} \\
 & M.
\end{tikzcd}
\]
Equivalently, $\Hom_{B-Mod}(\Omega^1_{B/A}, M) \simeq Der_A(B,M)$. (+ absolute version).
\end{defin}

\begin{prop}
If $A \to B \to C$ is a sequence of maps of commutative rings, then the following sequence of $C$-modules is exact:
\begin{equation}
\label{eq:ses_omegas}
	\Omega^1_{B/A} \otimes_B C \to \Omega^1_{C/A} \to \Omega^1_{C/B} \to 0.
\end{equation}
\end{prop}

One of the goals for this talk is to extend the sequence to the left. If $CRing$ was an Abelian category, we would
attempt to derive the functor $\Omega^1$; however, this is not the case. Instead, what we do is generalize the notion of
K\"ahler differential to the $\infty$-categorical setting, and show that this gives an extension to the left of the sequence
\ref{eq:ses_omegas}. Slogan: ``$\infty$-category theory allows us to do derived functors in a non-linear setting''. 



\section{Generalization and definition}
\label{sect:cc_def}

Note that generalizing the Leibniz rule to the $\infty$-category setting is hard, because we'd have to replace the 
equality with a homotopy.
Instead, consider the following idea. For a ring homomorphism $\phi : A \to B$, we want a new homomorphism $\phi': A \to B$ 
``sufficiently close'' to $\phi$. For example, take $I \subset B$ an ideal
with $I^2 = 0$. Then ``sufficiently close'' means that $\phi' : A \to B$ is congruent to $\phi$ modulo $I$, i.e.:
\[	\forall\; a \in A,\; \phi(a) - \phi'(a) \in I.	\]
For a fixed $\phi$, we have a bijective correspondence:
\begin{equation}
\label{eq:corresp_derivations}
\left\{ \begin{array} {c} \phi': A \to B \text{ such that} \\ \phi' \equiv \phi \text{ mod } I \end{array} \right\}
\longleftrightarrow
\left\{ \begin{array} {c} d: A \to I \text{ satisfying} \\ \text{the Leibniz rule} \end{array} \right\} .
\end{equation}


If $M$ an $A$-module, take $B := A \oplus M$ equipped with 
the ring structure such that $M^2 = 0$:
\[	(a_1, m_1) (a_2,m_2) = (a_1 a_2, a_1 \dot m_2 + a_2\dot m_1).	\]
We fix $\phi : A \to B$ the natural inclusion of $A$ into the coproduct (as abelian groups); with the given ring structure,
$\phi$ is also a ring homomorphism. Take the ideal $I = M$. Then the correspondence \ref{eq:corresp_derivations} becomes:
\begin{equation}
\label{eq:derivations_classical}
\Map_{/A}(A, A\oplus M) \cong Der(A,M).
\end{equation}


This is something we can generalize. We work with $A \in SCR_k$, and the category of $A-\cM od$; all categories in the rest of
the talk are $\infty$-categories. Take $M \in A-\cM od$ and construct $A \oplus M$, whose underlying simplicial set is the
coproduct, and whose ring structure is defined levelwise (see \cite{HAG-II}, 1.2.1.1 for details).

We adapt equation \ref{eq:derivations_classical} to this setting, by defining:
\[	\cD er(A,M) = \Map_{/A}(A,A\oplus M) \in \cS.	\]
Moving from the absolute version of derivations to the relative one, for $f : A \to B$ in $SCR_k$ and $M \in B-\cM od$, we 
define: 
\[	\cD er_A (B,M) = \Map_{A-Alg/B}(B, B\oplus M)\in \cS.	\]


We obtain functors $\cD er(A,-) : A-Mod \to \cS$ and $\cD er_A(B,-) : B-Mod \to \cS$. We claim that these functors are corepresentable,
and call the corepresenting objects the \textbf{absolute cotangent complex} $\bbL_A$ and \textbf{relative cotangent complex}
$\bbL_{B/A}$, respectively. Equivalently, this means:
\begin{align*}
\Map_{A-Mod}(\bbL_{A}, M) &\cong \Map_{/A}(A, A\oplus M), \\
\Map_{B-Mod}(\bbL_{B/A}, M) &\cong \Map_{A-Alg/B}(B, B\oplus M).
\end{align*}

The proof of corepresentability relies on the following lemma.
\begin{lem}
\item $Der_A(B,-)$ and $Der(A,-)$ commute with limits and $K$-filtered colimits.\footnote{
$\omega$-filtered would mean that the representing guy can be given by presentation with gen and relation. Otherwise we just
mean for everything strictly less than cardinality $K$, which could be continuum or more.} \todo{explain the footnote better}
\end{lem}

Then the result follows by Theorem 5.5.2.7 in \cite{HTT}, which we reproduce here.\footnote{Related to this is the Adjoint
Functor Theorem 5.5.2.9, which is one of the most important theorems in $\infty$-category theory. It's also the reason
we love presentable $\infty$-categories.}

\begin{thm}
Let $\cC$ be a presentable $\infty$-category and $F : \cC \to \cS$ a functor. Then $F$ is corepresentable by an object of
$\cC$ if and only if $F$ preserves $K$-filtered colimits and all small limits.
\end{thm}

\begin{rem}
Note that preserving small limits is obviously necessary in order to be corepresentable, since $\Hom$ is a right adjoint,
and thus preserves small limits.\todo{think more about the small hypothesis} Let's explain this in more detail
for 1-category theory. We have the diagram:
\[
\begin{tikzcd}
A-Mod \arrow{r}{\simeq} & Ab(CRing_{/A})\arrow{d}{\text{forget}} & \\
 & CRing_{/A} \arrow{r}{\Hom(A, -)} & Set .
\end{tikzcd}
\]
The equivalence on the first line works by sending an $A$-module $M$ to $A \oplus M$, and a ring $B$ over $A$ to the kernel
of $B \to A$. \todo{finish this}

For $\infty$-category theory, the relevant diagram is:
\[
\begin{tikzcd}
A-\cM od \arrow{r}{\simeq} & \cS p(SCR_{/A})\arrow{d}{\text{forget}} & \\
 & SCR_{/A} \arrow{r}{\Map_{/A}(A, -)} & \cS .
\end{tikzcd}
\]
The reference for this is \cite{Lurie_Higher_algebra}, section 7.4.\todo{wait, where exactly? can't find it}
\end{rem}


In \cite{HAG-II}, Chapter 1, we find an explicit (although not useful in practice, according to Mauro) model 
for $\bbL_A$. Take a simplicial resolution $\tilde A \to A$, which is also a cofibrant replacement. Then we have:
\[	 \Omega_{\tilde A}^1 \otimes^{\bbL}_{\tilde A} A \in A-Mod	\]
is a model for $\bbL$, where the complex $\Omega_{\tilde A}^1$ is build by taking K\"ahler differentials degree-wise:
\[	(\Omega_{\tilde A}^1)_{\bullet} := \Omega^1_{\tilde A_{\bullet}}.	\]





\section{Examples and Properties}
\label{sect:cc_eg}

In this section we compute $\bbL_{k[x]/k}$ and $\bbL_{k/k[x]}$, where $x$ is in degree -1, as opposed to 1, by our 
cohomological convention. (See remark \ref{rem:degree_convention} for what this implies; $k[x]$ is not what it seems.)
In the process we go over some of the properties of cotangent complexes.

\begin{lem}
Let $A \in SCR_k$ and $M \in A-\cM od$. The cotangent complex of $f: A \to \Sym_A M$ is:
\[	\bbL_{\Sym_A M /A}\cong M \otimes^{\bbL}_A \Sym_A M .	\]
\end{lem}
\begin{proof}
For all $\Sym_A M$-module $N$,
\begin{align*}
	\Map_{\Sym_A M -\cM od}(\bbL_{\Sym_A M/A}, N) &\cong \Map_{A-\cM od}(\Sym, \Sym\oplus N ) \cong \Map_{A-\cM od}(M, f_* N) \\
&\cong \Map_{\Sym_A M - \cM od}(M \otimes^{\bbL}_A \Sym_A M ,N).
\end{align*}
Here the first equivalence is definitional, the second follows from the universal property of $\Sym_A M$, and the third is
the adjunction \ref{tikz:adjunction}.
\end{proof}

For our first example, note that $k[x] \cong \Sym_k(k[1])$. Then the answer is $k[1] \otimes^{\bbL}_k k[x]$, which is just 
$k[x]$ concentrated in degree $-1$.

\begin{rem}
\label{rem:degree_convention}
Note that, since $k[1]$ is concentrated in degree $-1$, so is $\Sym_k(k[1])$; it does not have information in all 
nonnegative degrees, as the notation may mislead one into thinking. We just get a copy of $k$ in degree 0 and one
in degree -1,
and this is what we call $k[x]$. This is because multiplication in the symmetric algebra is graded commutative:
\[	x y = (-1)^{|x||y|} yx,	\]
so in particular for $x$ of degree 1 we get $x^2 = -x^2 = 0$. If we started with $k[2]$ instead,
$\Sym_k(k[2])$ would be nontrivial in all negative even degrees and commutative in the classical sense. More generally, 
$\Sym(k[n])$ gives what we would classically
call a symmetric algebra if $n$ is even, or an alternating algebra if $n$ is odd.
\end{rem}

Some properties of cotangent complex:
\begin{prop}
\;
\begin{enumerate}
\item For $A \to B \to C$ in $SCR_k$, there is a homotpy cofiber sequence in $C-Mod$:
\[	\bbL_{B/A} \otimes^{\bbL}_B C \to \bbL_{C/A} \to \bbL_{C/B}.	\]
\item Base change: given a homotopy pullback square
\[
\begin{tikzcd}
A\arrow{r}\arrow{d} & B\arrow{d} \\
A'\arrow{r} & B',
\end{tikzcd}
\]
there is an equivalence $\bbL_{B/A} \otimes^{\bbL}_B B' \simeq \bbL_{B'/A'}$.
\end{enumerate}
\end{prop}

To compute $\bbL_{k/k[x]}$, use the cofiber sequence associated to the sequence of maps $k \to k[x] \to k$. We get the
cofiber sequence in $k-\cM od$:
\begin{equation}
\label{eq:cofiber_poly}
	\bbL_{k[x]/k} \otimes^{\bbL}_{k[x]} k \to \bbL_{k/k} \to \bbL_{k/k[x]}.
\end{equation}
By our previous computation, the first term is:
\[ \bbL_{k[x]/k} \otimes^{\bbL}_{k[x]} k \simeq k[1] \otimes_k^{\bbL} k[x] \otimes^{\bbL}_{k[x]} k \simeq k[1].	\]
(Using associativity for derived tensor product.) The second term in \ref{eq:cofiber_poly} is 0, so the cofiber sequence
is actually a suspension diagram. 
\[
\begin{tikzcd}
k[1]\arrow{r}\arrow{d} & 0\arrow{d} \\
0\arrow{r} & \bbL_{k/k[x]}
\end{tikzcd}
\]
Then $\bbL_{k/k[x]}\simeq k[2]$. 
%That this is consistent with our intuition from the classical case: 
%if, to begin with, $k$ is in degree 0 to begin with, and
%$\bbL_{k/k[x]} = k[1]$, concentrated in degree -1. It's great that it doesn't have anything in degree 0, since Kahler differentials
%of a closed immersion are 0.

Going back to Section \ref{sect:cc_motivation}, we complete the exact sequence \ref{eq:ses_omegas} on the left. 
Using stability of $C-\cM od$, the cofiber sequence gives a long exact sequence on homology (recall proposition
\ref{prop:stable_les_homology}; in particular, $H^i(X) = \tau^{\geq i} \circ \tau^{\leq i}(X)$).
\[
\begin{tikzpicture}[descr/.style={fill=white,inner sep=1.5pt}]
\label{tikz:les}
        \matrix (m) [
            matrix of math nodes,
            row sep=1em,
            column sep=2.5em,
            text height=1.5ex, text depth=0.25ex
        ]
        { \dots & H^{-1}(\bbL_{B/A} \otimes^{\bbL}_{B} C) & H^{-1}(\bbL_{C/A}) & H^{-1}(\bbL_{C/B}) & \\
            & H^0(\bbL_{B/A} \otimes^{\bbL}_{B} C) & H^0(\bbL_{C/A}) & H^0(\bbL_{C/B}) & \\
            & H^1(\bbL_{B/A} \otimes^{\bbL}_{B} C) & H^1(\bbL_{C/A}) & H^1(\bbL_{C/B}) & \dots \\
        };

        \path[overlay,->, font=\scriptsize,>=latex]
        (m-1-1) edge (m-1-2)
        (m-1-2) edge (m-1-3)
        (m-1-3) edge (m-1-4)
        (m-1-4) edge[out=350,in=170]  (m-2-2)
        (m-2-2) edge (m-2-3)
        (m-2-3) edge (m-2-4)
        (m-2-4) edge[out=350,in=170]  (m-3-2)
        (m-3-2) edge (m-3-3)
        (m-3-3) edge (m-3-4)
				(m-3-4) edge (m-3-5);
\end{tikzpicture}
\]
We claim that:
\begin{enumerate}
\item \label{item:h0_underived}
\textit{For underived rings}, $H^i(\bbL_{B/A}) = 0$ if $i>0$;
\item \textit{For underived rings}, $H^0(\bbL_{B/A}) \simeq \Omega^1_{B/A}$;
\item \textit{In general}, $H^0(\bbL_{B/A}) \simeq \Omega^1_{\pi_0(B)/\pi_0(A)}$.
\end{enumerate}
An application of these facts is the desired extension
to the left of the sequence \ref{eq:ses_omegas}. The facts are proved in \cite{Lurie_Higher_algebra}, section 7.4.3. 
We will say more about claim \ref{item:h0_underived}, but first we need to talk about connectivity. 



\section{Connectivity}
\label{sect:cc_connectivity}

\begin{defin}
A space $X$ is \textbf{n-connective} if $\pi_i(X,x) = 0$ for all $x \in X$ and $i<n$. We say $X$ is \textbf{connective}
if it's 0-connective, \textbf{connected} if it's 1-connective. $f:X \to Y$ is \textbf{n-connective} if fiber$(f)$ is n-connective.
\end{defin}

The following is in \cite{Lurie_Higher_algebra}, 7.4.3.2, and it's VERY important.
\begin{thm}[Connectivity estimate]
\label{thm:connectivity_estimate}
Assume $f : A \to B$ is a map in $SCR_k$ and cofib$(f)$ is $n$-connective. Then there exists a map:
\[	\mathcal{E}_f: B \otimes_A^{\bbL} Cofib(f) \to \bbL_{B/A}	\]
in $B-Mod$, which is $2n$-connective.
\end{thm}

\begin{rem}
The proof is not hard; the only difficulty is constructing the map, which we can do after we learn Postnikov towers.
\todo{reference once we have the postnikov notes}
\end{rem}

\begin{cor}
The hypothesis of Theorem \ref{thm:connectivity_estimate} implies $\bbL_{B/A}$ is $n$-connective.
\end{cor}
\begin{proof}
We look at the fiber sequence:
\[	fib(\mathcal{E}_f) \to B\otimes^{\bbL}_A cofib(f) \to \bbL_{B/A},	\]
and get a long exact sequence of homotopy groups. So it suffices to show that:
\begin{enumerate}
\item \label{item:cofiber_connective}
 $B\otimes_A^{\bbL} Cofib(f)$ is $n$-connective;
\item \label{item:epsilon_connective}
$fib(\mathcal{E}_f)$ is $n-1$-connective.
\end{enumerate}
\ref{item:epsilon_connective} is implied by Theorem \ref{thm:connectivity_estimate}; note that theorem is actually considerably
stronger. Property \ref{item:cofiber_connective} is proved in \cite{Sq0}. The proof there uses a spectral sequence due to 
Quillen: for $M,N \in A-Mod$, $A \in SCR_k$,
\[	\Tor_p^{\pi_q(A)}(\pi_qM, \pi_qN) \Longrightarrow \pi_{p+q}(M \otimes^{\bbL}_A N).	\]
\end{proof}

\begin{rem}
In particular, cotangent complexes are 0-connective for commutative rings. This gives a proof of fact
\ref{item:h0_underived} at the end of the previous section.
\end{rem}

\begin{cor}
For $A \in SCR_k$, $\bbL_A$ is 1-connective. Moreover, $f:A \to \pi_0(A)$ is 1-connective, so $\bbL_{\pi_0(A)/A}$ is
1-connective.
\end{cor}

The most important corollary:
\begin{cor}
$f: A \to B$ is an equivalence iff $\pi_0(f) : \pi_0(A) \to \pi_0(B)$ is and $\bbL_{B/A} \simeq 0$. One direction obvious, the
other comes from the fact that $\bbL_{B/A}$ is $n$-connected for all $n$.
\end{cor}

\begin{rem}
Slogan: ``DAG = classical AG + DDT''.
Lurie's representability theorm is a great example of the philosophy: it says that a derived stack is representable iff
its truncation is representable and its cotangent complex is nice enough. We won't get to see this in the seminar, since we'll
change course towards structured DAG instead.
\end{rem}


\begin{rem}
Cotangent complexes we glue for free, which was not possible before $\infty$-categories. This allows to reduce many questions
to the affine setting, where we may have to do actual computations if things go wrong.
\end{rem}

We have one talk on Postnikov tower, and one on perfect complexes, then we leave the affine setting forever.

\todo{look at last 2 exercises from stable $\infty$-category}







\chapter{Square Zero Extensions}
Talk by Matei Ionita.

\section{Square Zero Extensions}
\label{sect:sq0_sq0}

Recall that, given $A \in cdga^{\leq 0}_k$ and $M \in A-\cM od$, we defined derivations from $A$ into $M$ as:
\[	\cD er_k(A,M) = \Map_{A-\cA lg / k} (A, A \oplus M) .	\]
Alternatively, these are the same as sections of the projection map $A \oplus M \to A$. Morally speaking, we'd
like to define square-zero extensions as homotopy fibers of derivations, i.e. $f: A^{\eta} \to A$ is a square-zero
extension of $A$ by $M$ if there is a homotopy pullback square:
\[
\begin{tikzcd}
A^{\eta}\arrow{r}{f}\arrow{d} & A\arrow{d}{d_{\eta}} \\
0\arrow{r} & M[1].
\end{tikzcd}
\]
The problem is that the above diagram doesn't make sense, because a derivation is not a morphism in $cdga^{\leq 0}_k$.
In section 7.4.1 of \cite{Lurie_Higher_algebra}, Lurie addresses this by using the category of tangent 
correspondences, which acts
like a ``tangent bundle'' of the category $cdga^{\leq 0}_k$, with $A-\cM od$ acting as the tangent space 
$T_A cdga^{\leq 0}_k$. In this new category
the diagram makes sense. However, we don't introduce all this technology here, and instead translate Lurie's (more general)
definition of square zero extensions into a more accessible version.

\begin{defin}
\label{defin:sq0}
A map $\tilde f: \tilde A \to A$ is a \textbf{square-zero extension} of $A$ by $M$ if it's equivalent in the category
$cdga^{\leq 0}_{/A}$ to a map $f: A^{\eta} \to A$ such that there is a homotopy pullback diagram in $cdga^{\leq 0}_k$:
\[
\begin{tikzcd}
A^{\eta}\arrow{r}{f} \arrow{d} & A\arrow{d}{d_{\eta}} \\
A\arrow{r}{d_0} & A \oplus M[1].
\end{tikzcd}
\]
Here $d_0$ is the zero derivation.
\end{defin}

\begin{rem}
We explain why the shift by 1 is necessary in definition \ref{defin:sq0}, by studying the split square-zero extension.
We claim that, with the shift in place, the following diagram is a homotopy pullback.\todo{replace with better explanation}
\[
\begin{tikzcd}
A \oplus M \arrow{r}\arrow{d} & A\arrow{d} \\
A\arrow{r} & A\oplus M[1] 
\end{tikzcd}
\]
To see this, extend the diagram by considering the map $0 \to A$, and the resulting pullback square in the category
$A-\cM od$:
\[
\begin{tikzcd}
M\arrow{r}\arrow{d} & A \oplus M \arrow{r}\arrow{d} & A\arrow{d} \\
0\arrow{r} & A\arrow{r} & A\oplus M[1] .
\end{tikzcd}
\]
Indeed, the vertical map $A\oplus M \to A$ is surjective, hence a fibration in $A-\cM od$, and then the naive pullback $M$ 
is a homotopy pullback. Moreover, the outer square is also a homotopy pullback in $A-\cM od$, because it's equivalent to:
\[
\begin{tikzcd}
M\arrow{r}\arrow{d} & 0\arrow{d} \\
0\arrow{r} & M[1].
\end{tikzcd}
\]
It follows that the square on the right is a homotopy pullback in $A-\cM od$. But all maps in this square are maps of
$A$-algebras, so we claim that the square is actually a homotopy pullback in $A-\cA lg$.
\end{rem}

\begin{rem}
Definition \ref{defin:sq0} is easy and clean, but it is hard to see whether a given map satisfies it. For example, if
$A \to B$ is a square zero extension of commutative rings by a $B$-module $M$, in the classical sense, the shift
$M[1]$ makes us leave the classical category of modules. Moreover, it's hard to prove that the given map $A \to B$
comes from the fiber product structure of $A$.
\end{rem}

We would like to construct a functor $\Phi : \cD er (A,M) \to \Fun(\Delta^1, cdga^{\leq 0}_k)$ whose essential image are the square-zero
extensions. Morally speaking, $\Phi$ sends $d_{\eta} : A \to A \oplus M$ to its homotopy fiber. The rest of this section makes
this construction precise.

\begin{defin}
The $\infty$-category $\cD er_A$ of \textbf{derivations of A} has objects derivations $d : A \to M$ and spaces of morphisms
$\cD er_A(M_1, M_2) = A-\cM od_{/A}(M_1,M_2)$. The $\infty$-category $ \tilde \cD er_A$ of \textbf{extended derivations of A}
has objects consisting of homotopy pullback squares:
\[
\begin{tikzcd}
A^{\eta}\arrow{r}{f} \arrow{d} & A\arrow{d}{d_{\eta}} \\
A\arrow{r}{d_0} & A \oplus M,
\end{tikzcd}
\]
and spaces of morphisms consisting of morphisms of squares.
\end{defin}

Note that $\tilde \cD er_A$ can be described as the full $\infty$-subcategory of $\Fun(\Delta^1 \times \Delta^1, cdga^{\leq 0}_{/A})$
whose objects are homotopy pullback squares and have prescribed restrictions: $F(\{0,0\}) = A$ and $F(\{1\}\times \Delta_1) = 
d_0 : A \to A\oplus M$.

There are two functors $F_1, F_2 : \tilde \cD er_A : \Fun(\Delta^1, cdga^{\leq 0}_{/A})$ obtained by restricting to $\Delta_1 \times \{1\}$
and $\{0\} \times \Delta^1$, respectively. Note that their essential images are A-derivations and square-zero extensions of $A$,
respectively, so that we have:
\[
\begin{tikzcd}
\; & \tilde \cD er_A \arrow[swap]{ld}{F_1}\arrow{rd}{F_2} & \\
\cD er_A & & \Fun(\Delta_1, cdga_{/A}).
\end{tikzcd}
\]
We prove that $F_1$ is a trivial Kan fibration, which implies that it has a section $s$. This will allow us to define
$\Phi = F_2 \circ s$.

\begin{lem}
$F_1$ is a trivial Kan fibration.
\end{lem}
\begin{proof}
Consider the decomposition:
\[
\begin{tikzcd}
\Fun(\Delta^1 \times \Delta^1, cdga_{/A})\arrow{rr}{R}\arrow{rd}{R_1} & & \Fun(\Delta^1 \times \{1\}, cdga_{/A}) \\
& \Fun(\Lambda^2_2, cdga_{/A})\arrow{ur}{R_2} & 
\end{tikzcd}
\]
$F_1$ is the restriction of $R$ to $\tilde \cD er_A$. Then we have:
\begin{enumerate}
\item Using Proposition 4.3.2.15 in \cite{HTT}, a restriction functor $\Fun(\cC, \cD) \to \Fun(\cC_0, \cD)$ is a trivial
Kan fibration as long as all functors in $\Fun(\cC, \cD)$ are Kan extensions of those in $\Fun(\cC_0, \cD)$. We apply this twice.
\item The pullback squares in $\Fun(\Delta^1 \times \Delta^1, cdga^{\leq 0}_{/A})$ are Kan extensions, because all limits are Kan
extensions. It follows that $R_1|_{\tilde \cD er_A}$ is a trivial Kan fibration.
\item $R_1(\tilde \cD er_A)$, the images of extended derivations in $\Fun(\Lambda^2_2, cdga^{\leq 0}_{/A})$, are left Kan extensions.
It follows that $R_2$ restricted to the images of extended derivations is a trivial Kan fibration.
\end{enumerate}
\end{proof}

Then we invoke the theorem saying that every trivial Kan fibration has a section \todo{reference this}, and define
$\Phi = F_2 \circ s$.




\section{n-small extensions}
\label{sect:sq0_nsmall}

Let $f: A \to B$ be a map in $cdga^{\leq 0}_k$, and let $I = \textbf{hofib}(f)$. In other words, $I$ is the homotopy pullback of
the following diagram of non-unital commutative monoid objects in $A-\cM od$:
\[
\begin{tikzcd}
I\arrow{r}\arrow{d} & A\arrow{d}{f} \\
0 \arrow{r} & B.
\end{tikzcd}
\]
This induces a non-unital commutative monoid structure on $I$; in particular, $I$ is an $A$-module, and
there is a multiplication map $I \otimes_A I \to I$. The following is proposition 7.4.1.14. in \cite{Lurie_Higher_algebra}.

\begin{prop}
\label{prop:mult_nullhomotopic}
The multiplication map $I \otimes_{A^{\eta}} I \to I$ is nullhomotopic.
\end{prop}


This motivates our definition of n-small extensions.
The following definition and remarks are 7.4.1.18-7.4.1.21 in
\cite{Lurie_Higher_algebra}.

\begin{defin}
Let $f : A \to B$ be a map in $cdga_k^{\leq 0}$, and let $n\geq 0$. We say that $f$ is an \textbf{n-connective extension}
if $hofib(f) \in cdga_k^{\leq -n}$. We say that $f$ is an \textbf{n-small extension} if it is an n-connective extension and,
moreover:
\begin{enumerate}
\item \label{item:mult_nullhomotopic}
$hofib(f) \in cdga_k^{\geq -2n}$;
\item the multiplication map $hofib(f) \otimes hofib(f) \to hofib(f)$ is nullhomotopic.
\end{enumerate}
\end{defin}

\begin{rem}
If $f:A \to B$ is an n-connective extension, from the long exact sequence on homotopy groups we see that 
$\pi_0(A) \to \pi_0(B)$ is surjective.
\end{rem}

\begin{rem}
Suppose that $f:A \to B$ is an n-connective extension with $hofib(f) \in cdga_k^{\geq -2n}$. Since $hofib(f) \in cdga_k^{\leq -n}$,
we also have that $hofib(f) \otimes hofib(f) \in cdga_k^{\leq -2n}$. It follows that, at the level of homotopy groups,
the only potentially nonzero map is:
\begin{equation}
\label{eq:pi2n_vanish}
	\pi_{2n}\big(hofib(f) \otimes hofib(f)\big) \to \pi_{2n}\big(hofib(f)\big).
\end{equation}
Therefore condition \ref{item:mult_nullhomotopic} in the definition of an n-small extension simply requires that the map
\ref{eq:pi2n_vanish} is 0.
\end{rem}

\begin{eg}
Let $A$ be a commutative ring, which we regard as a discrete commutative dga. A map $\tilde A \to A$ in $cdga_k^{\leq 0}$
is a 0-small extension if and only if:
\begin{enumerate}
\item $\tilde A$ is also discrete;
\item $f: \tilde A \to A$ is a surjective commutative ring homomorphism;
\item if $I$ is the kernel of $f$, then $I^2 = 0$, as a consequence of \ref{eq:pi2n_vanish}.
\end{enumerate}
So we recover the square-zero extensions in classical AG.
\end{eg}

We want to prove that all n-small extensions are square-zero extensions. (But not vice-versa!) First, we identify
what n-smallness should correspond to in terms of derivations. It's what you'd expect.

\begin{defin}
Let $\cD er_{\text{n-con}}(A)$ denote the full subcategory of derivations $\eta : A \to M[1]$ such that $M \in A-\cM od^{\leq -n}$.
Let $\cD er_{\text{n-sm}}(A)$ denote the full subcategory of derivations $\eta : A \to M[1]$ such that 
$M \in A-\cM od^{\leq -n} \cap A-\cM od^{\geq -2n}$.
\end{defin}

The following is Theorem 7.4.1.23 in \cite{Lurie_Higher_algebra}, and is the main result of this talk.

\begin{thm}
Let $\Phi : \cD er(A) \to \Fun(\Delta^1, cdga_k^{\leq 0})$ be the functor constructed in section \ref{sect:sq0_fib}. For
each $n\geq 0$, it induces an equivalence of categories:
\[	\Phi_{\text{n-sm}} : \cD er_{\text{n-sm}} \to \Fun_{\text{n-sm}}(\Delta^1, cdga_k^{\leq 0}).	\]
\end{thm}

\begin{proof}
We just give a sketch. First, note that for a derivation $d_\eta : A \to A \oplus M[1]$, there is an equivalence
between the homotopy fiber of the square-zero extension $A^{\eta} \to A$ and $M$. Moreover, multiplication on
the fiber of a square-zero extension is nullhomotopic, by Proposition \ref{prop:mult_nullhomotopic}.
 It follows that the functor $\Phi$ restricts indeed to a functor
$\Phi_{\text{n-sm}} : \cD er_{\text{n-sm}} \to \Fun_{\text{n-sm}}(\Delta^1, cdga_k^{\leq 0})$. 

$\Phi$ admits a left adjoint $\Psi$, which sends a square-zero extension $A^{\eta} \to A$ to the derivation classified
by $\bbL_A \to \bbL_{A/A^{\eta}}$. This restricts to a left adjoint of $\Phi_{\text{n-conn}}$, but we need to truncate in
order to get an adjoint $\tau \circ \Psi_{\text{n-conn}}$ for $\Phi_{\text{n-sm}}$. Then we prove that this adjoint pair
is an equivalence.
\end{proof}


\begin{cor}
Every n-small extension is a square-zero extension.
\end{cor}

\begin{cor}
Let $A \in cdga_k^{\leq 0}$, then every map in the Postnikov tower:
\[	\dots \to \tau^{\geq -2}(A) \to \tau^{\geq -1}(A) \to \tau^{\geq 0} (A) 	\]
is a square-zero extension.
\end{cor}

This is because the n-th stage is obviously an n-small extension, the homotopy fiber being equal to $\pi_{n}(A)[n]$ 
concentrated in degree $n$.  This corollary is highly important, as it allows statements about derived affines $A$ to be proved
by induction on the Postnikov tower. The base step, for $\pi_0(A)$, is a classical AG statement, which is proved by classical
methods. The inductive step reduces to a linear problem involving the derivation
associated to the square-zero extension $\tau^{\geq i}(A) \to \tau^{\geq i-1}(A)$. The next section exemplifies this
philosophy.




\section{Induction on Postnikov tower}
\label{sect:sq0_induction}

\begin{prop}
Let $A\in sCA_{k}$, Assume we are given $j:\Spec(\pi_{0}(A))\to \mathbb{A}^{n}$, then there exists a lift 
of the map $j$ to a map $\Spec(A)\rightarrow \mathbb{A}^{n}$.
\end{prop}

\begin{proof}
We use induction on the Postnikov tower. Suppose that there is a map $j_{n}: Spec(\tau^{\leq n}A)\rightarrow 
\mathbb{A}^{n}$, we show that there is a lifting $j_{n+1}$ from $Spec(\tau^{\leq n+1}A)$ to $\mathbb{A}^{n}$.  
If we can prove this then as $A=lim(\tau^{\leq n}A)$ and there thus exists a lifting of the map $j$ to a map 
from $Spec(A)$. 

We show how this works for the first stage, the construction of $j_1 : \Spec(\tau^{\leq 1}(A)) \to \bbA^n$.
\[
\begin{tikzcd}
\label{tikz:postnikov_pullback}
k[x_1, \dots, k_n]\arrow{drr}{j}\arrow[swap]{ddr}{j}\arrow[dashed]{dr} & & & \\
 & \tau^{1}(A)\arrow{r}\arrow{d} & \pi_0(A)\arrow{d}{d_{\eta}}\arrow[bend left]{ddr}{=} & \\
 & \pi_0(A)\arrow[swap, bend right]{drr}{=} \arrow{r}{d_0} & \pi_0(A) \oplus \pi_1(A)[2]\arrow{dr}{p} & \\
 & & & \pi_0(A) .
\end{tikzcd}
\]

In order to invoke the universal property of the homotopy pullback, we need a homotopy between two maps
$k[x_1, \dots, k_n] \to \pi_0(A) \oplus \pi_1(A)[2]$:
\[	d_{\eta} \circ j \cong d_{0} \circ j.	\]
It suffices to show that this homotopy exists, i.e. that the space $X$ of such homotopies is nonempty. However,
we will accomplish more: we find the homotopy type of $X$, which allows us to comment on the (non-)uniqueness
of the lift $j_1$. Note that, in an $\infty$-category, the derivations $d_0$ and $d_{\eta}$ come with the data which expresses
them as sections of the projection $p:\pi_0(A) \oplus \pi_1(A)[2] \to \pi_0(A)$, i.e. homotopies $p\circ d_0
\Rightarrow \id_A$ and $p\circ d_{\eta} \Rightarrow id_A$; this is the bottom-right part of diagram \ref{tikz:postnikov_pullback}.
Moreover, giving a map $k[x_1, \dots, k_n] \to \pi_0(A) \oplus \pi_1(A)[2]$ (such as $d_{\eta} \circ j$ or $\cong d_{0} \circ j$)
is, using the universal property of pullbacks, equivalent to giving a section of the pullback map $B \to k[x_1, \dots, k_n]$:
\[
\begin{tikzcd}
k[x_1, \dots, k_n]\arrow{dr} & & k[x_1, \dots, k_n]\arrow{dl} & \pi_0(A)\arrow{dr} & & \pi_0(A)\arrow{dl} \\
& B\arrow{d} & & & \pi_0(A) \oplus \pi_1(A)[2]\arrow{d} & \\
& k[x_1, \dots, k_n] \arrow{rrr}{j} & & & \pi_0(A) & 
\end{tikzcd}
\]
Note that $\pi_0(A) \oplus \pi_1(A)[2] \to \pi_0(A)$ is a degreewise surjection, hence a fibration in the model structure
of $cdga_k^{\leq 0}$. (See \ref{rem:transfer_model_structure} for how this model structure is obtained, via the free-forgetful
adjunction.) It follows that the homotopy pullback by $j$ is the same as the naive pullback, so $B = k[x_1, \dots, k_n]
\oplus j_*\pi_1(A)[2]$.

Putting everything together, a lift $j_1 : \Spec \tau^{\leq 1}(A) \to \bbA^1$ is the same as a homotopy between $d'_0$
and $ d'_{\eta}$, the images of $d_0$ and $d_{\eta}$ under pullback. 
But $d'_{0},d'_{\eta} \in Map_{cdga_k^{\leq 0} / k[x_{1},...,x_{n}]}(k[x_{1},...,x_{n}]), k[x_{1},...,x_{n}]\oplus 
j_{*}\pi_{1}(A)[2])\cong Map_{k[x_{1},...,x_{n}]-Mod}(\bbL_{k[x_{1},...,x_{n}]}, j_{*}\pi_{1}(A)[2])$; denote this space
by $X$. We compute its homotopy type. 

Since $k[x_1, \dots, x_n]$ is discrete and smooth over $k$, $\bbL_{k[x_1, \dots, x_n]} \cong \Omega^1_{k[x_1, \dots, x_n]}[0]$.
On the other hand, the functor $j_*$ is t-exact (see \ref{eg:modules_t_exact}), so $j_* \pi_1(A)[2]$ is concentrated in degree 2.
It follows that we have:
\[	\pi_i(X) = \left\{ \begin{array} {l} 0 \text{ if } i \neq 2, \\ 
\Hom\big(\Omega^1_{k[x_1, \dots, x_n]}, \pi_1(A)\big) \cong \pi_1(A)^{\oplus n} \text{ if } i = 2 \end{array}\right. .	\]
So $X \simeq K(\pi_1(A)^{\oplus n}, 2)$. In particular, $X$ is path connected, so the required path between $d'_0$ and $d'_{\eta}$
exists. We can say more:
\[	\{ \text{homotopies between } d'_0 \text{ and } d'_{\eta} \} \simeq \Omega X \simeq K(\pi_1(A)^{\oplus n}, 1).	\]
We conclude that any two lifts $j_1, j_1' : k[x_1, \dots, x_n] \to \tau^{\leq 1}(A)$ are homotopic, but not coherently
homotopic.
\end{proof}



\chapter{Perfect Complexes}
Talk by Benedict Morrissey.

Half of this is about perfect complexes in classical AG, and the second half about what we do in the derived setting. 
In 2-3 weeks we will see that perfect complexes actually form a stack.

\section{Classical}
\label{sect:perf_classical}

Let $X$ be a scheme, then we look at $Ch^{\bullet}(QCoh(X))$.

\begin{defin}
$E^{\bullet} \in Ch^{\bullet}(QCoh(X))$ is \textbf{perfect} if it's Zariski locally quasi-isomorphic to an object
of $Ch^b(Vect_X)$. 
\end{defin}

\begin{rem}
This is not the same as requiring cohomology to be finitely supported.
\end{rem}

\begin{defin}
$E^{\bullet} \in Ch^{\bullet}(QCoh(X))$ has \textbf{Tor amplitude} in $[a,b]$ if for all $\mathcal{F} \in \mathcal{O}_X -Mod$,
\[	H^k(E^{\bullet} \otimes^{\bbL}_{\mathcal{O}_X} \mathcal{F}) = 0	\]
for $k \not \in [a,b]$. In particular, if $E$ is in the heart, this is just saying that the Tor with any given sheaf is bounded.
\end{defin}

\begin{rem}
For $\mathcal{F} = \mathcal{O}_X$, we just get the cohomology of $E^{\bullet}$.
\end{rem}

This is sometimes difficult to work with, so we have:

\begin{defin}
$E^{\bullet} \in Ch^{\bullet}(QCoh(X))$ is \textbf{almost perfect} if Zariski locally there is a $n$-quasi-isomorphic to
something in $Ch^b(Vect_X)$. n-quasi-isomorphism means isomorphism on cohomologies for $k\geq n+1$, and surjection
for degree $n$.\todo{Figure out cohomological convention}
\end{defin}

Perfect obviously implies almost perfect; this descends to the derived category of quasi-coherent sheaves.

\begin{thm}
$E^{\bullet}$ is perfect iff $E^{\bullet}$ is almost perfect (for some $n$) and has finite Tor amplitude.
\end{thm}
\begin{proof}
Locally free means flat, so tensoring it with anything preserves the tor amplitude. The other direction is in
TT, Higher algebraic K-theory of schemes, 2.2.12.
\end{proof}

Alternative definition: $E^{\bullet} \in D(Coh(X))$. Locally on some $U$ we have an $\mathcal{O}_X(U)$-module
$E|_{U}^{\bullet}$. We require that this is bounded above and has coherent cohomologies. Equivalently, $\tau_{\leq n}(E^{\bullet}|_n)$
is compact in $\tau_{\leq n} Mod_{\mathcal{O}_X(U)}$.

\begin{thm}
For $X$ affine, $E^{\bullet}$ is perfect if and only if it's globally quasi-isomorphic to an object of $Ch^b(Vect_X)$.
\end{thm}

\begin{thm}
If $X$ is smooth and Noetherian, then $D(Coh^b(X)) \simeq D(Perf(X))$.
\end{thm}

We'll prove this by Serre regularity.

\begin{defin}
A is \textbf{regular} if $\dim_k(m/m^2) = \dim_{Krull} A$. The \textbf{global dimension} of $A$ is:
\[	gldim(A) = \sup_{M \in A-Mod} \text{projdim}(M),	\]
where the latter is the minimum length of a projective resolution of $M$.
\end{defin}

\begin{thm}[Serre regularity]
If $A$ is a Noetherian local ring, TFAE:
\begin{enumerate}
\item $A$ is regular;
\item $gldim(A) < \infty$;
\item $gldim(A) = \dim_{Krull}A$.
\end{enumerate}
\end{thm}

Going back to $X$ smooth and Noetherian, we know that all local rings are regular. $E^{\bullet} \in Coh^b(X)$, then
$E^{\bullet}_p$ is a $\mathcal{O}_{X,p}$ module. 

\begin{proof}
We actually just do the case of $E$ in the heart, because it's easier. (For the other one, we probably resoluve to a double
complex, and take the total complex.)

Take a projective resolution of $E^{\bullet}_p$ as a $\mathcal{O}_{X,p}$ module; we know it must be finite:
\[	0 \to P_n \to \dots \to P_2 \to P_1 \to E^{\bullet}_p \to 0.	\]
We can do this in an open set around $p$:\todo{draw this from paper}

This takes care of one direction. We then use Tor dimension to show that, if $E$ is perfect, it's in $D(Coh^b(X))$.
\end{proof}

For the non-smooth case, we look at the ind-completion $Ind(Perf(X)) \simeq QCoh(X)$. (Always true for $X$ quasi-compact,
quasi-separated.) On the other side, $Ind(Coh^b(X)) = IndCoh(X)$. The quotient $IndCoh(X)/QCoh(X) = D_{sing}(X)$, which
really sees the singularities of $X$.


\section{derived}
\label{sect:perf_derived}

Let $A \in SCR_k$; recall that $\cM od_A$ is a stable $\infty$-category.

\begin{defin}
$\cM od^{\text{perf}}_A \subset \cM od_A$ is the smallest stable subcategory containing $A$ and closed under retracts.
Recall that $A$ is a retract of $B$ if there exist maps $i:A \to B, r:B \to A$ such that $r\circ i = \id_A$.
\end{defin}

\begin{defin}
$N \in \cC$ is \textbf{compact} if $\cH om_{\cC}(N, -)$ commutes with filtered colimits. The latter means that
the index category is nonempty, and for all $i,j \in I$, there exists $k$ such that $i \to k \leftarrow j$, and
coequalizers exist.
\end{defin}

\begin{eg}
Consider $(\cM od_A)^{\heartsuit}$. The compact objects are the finitely presented ones. We have a map $A^n \to M$,
so:
\[	\Hom(M, \varinjlim_I B_i) \simeq \varinjlim_I \Hom(M,B_i),	\]
because if we have a map $M \to B$, we can fully describe it by the composition $A^n \to B$. Each of the $n$ generators goes
to some $B_{i_k}$, so by the definition of filtered index category, there exists some $B_j$ such that $A^n \to B_j$. 

Conversely, starting with compact $M$, we look at finitely generated submodules $M_i$, and we have:
\[	\Hom(M, \varinjlim_I M_i) \simeq \varinjlim_I(M,M_i).	\]
In particular, the identity map $M \to M$ factors through some $M_j$, so $M = M_j$.
\end{eg}

\begin{thm}
$M \in \cM od_A$ is perfect iff it's compact.
\end{thm}
\begin{proof}
$\cM od_A^{\text{perf}} \subset \cM od_A^{\text{cpct}}$. \todo{add diagram from paper}
\[
\begin{tikzcd}
\;
\end{tikzcd}
\]

Since DK is an equivalence, we only need to argue that truncation and the forgetful functor preserve filtered colimits.
For the first one: filtered colimits are t-exact. For the second one: it does.

For the other direction, we have the inclusion $\cM od_A^{\text{perf}} \to \cM od_A$, we factor this though Ind, which
is just the completion with respect to filtered colimits.
\[
\begin{tikzcd}
\cM od_A^{\text{perf}} \arrow{r} & \cM od_A \\
 & Ind(\cM od_A^{\text{perf}}) \arrow{u}{\phi}
\end{tikzcd}
\]
$f$ is obviously fully faithful, because $Ind(\cM od_A^{\text{perf}})^{\omega} = \Mod_A^{perf}$.\footnote{We'll talk more about
this equality later, it follows because Perf is idempotent complete.} Mapping spaces
in $Ind$ are computed by:
\[ \cM ap_{Ind(\cC)}(\colim_{i \in I} \mathcal{F}_i, \colim_{j \in J} \mathcal{G}_j)	
= \lim_{i \in I} \colim_{j \in J}	\cM ap_{\cC}(\mathcal{F}_i, \mathcal{G}_j) .\]
This means we're computing mapping spaces by the same formula, so $\phi$ is fully faitfhul.
\end{proof}

The following is 7.2.4.5 in \cite{Lurie_Higher_algebra}:
\begin{thm}
$M \in \Mod_A^{perf}$, we have:
\begin{enumerate}
\item $\pi_n M = 0$ for $n >>0$;
\item If $\pi_m M \cong 0$ for all $m>k$, then $\pi_k M$ is finitely presented as a $\pi_0(M)$-module.
\end{enumerate}
\end{thm}


\begin{eg}
Think about $\Sym(k[2])$ as a module over itself. It is perfect by definition, but it's not bounded below.
\end{eg}



\bibliographystyle{plain}
\bibliography{dahema}

\end{document}








