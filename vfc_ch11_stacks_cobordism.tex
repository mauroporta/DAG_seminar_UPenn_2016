\chapter{Derived $C^{\infty}$ DM Stacks and Derived Cobordism}
\label{chap:cobordism}


Sukjoo Lee talked about:
\begin{enumerate}
\item Right category of DM stacks.
\item Derived cobordism and virtual fundamental class.
\item Follow MT paper to define $\mathcal{O}_X$ modules and Postnikov towers.
\end{enumerate}

\section{DM stacks}
In the algebraic case, we defined $\mathcal{T}_{Zar} = N(CRing_k^{Zar})^{\op}$, where the admissible maps are open immersions.
The local models are roughly hypersurfaces of affine space.
For DM stacks we introduced $\mathcal{T}_{\'et} = N(CRing_k^{sm})^{\op}$, where objects are commutative rings with \'etale maps.
Admissible maps are the \'etale ones. In the smooth case \'etale and open immersion are more or less the same. \todo{explain this}

In the analytic framework, following DAG IX and Mauro's thesis, we introduce $\mathcal{T}_{an} = N(\cC)$, where $\cC$ is category
of complex manifolds. The topology is \'etale, which means we're considering local homeomorphisms. We obtain derived analytic
geometry. In DAG V, 3.2.5, 3.2.8, we have criteria for when two pre-geometries are Morita equivalent (i.e. have the same sheaf
theory). So we can also consider $\mathcal{T}_0$, same but with open immersions. Also $\mathcal{T}_1$ has Stein manifolds,
i.e. complex manifolds with data of embedding $U \to \C^n$. Using each of the criteria mentioned, we can check that these 3
pregeometries are Morita equivalent. It follows that we can use whichever one we please.

In DAG V, Lurie introduced $T_{disc} = N(Diff)$, nerve of category of smooth manifolds together with embedding in $\R^n$. The
admissible morphisms are the open immersions. This is the analog of $\mathcal{T}_1$ above. We could also change this to
smooth manifolds with local diffeomorphisms. This pre-geometry gives Derived $C^{\infty}$ geometry. Then we have:
\[	\Sch(\mathcal{T}_{diff}) \subset ^L Top(T_{diff}).	\]
\todo{write table}
Note that here we get $dMan$ in the sense of Macpherson, not Spivak. We need to impose quasi-smoothness somewhere if we want Spivak's
definition.


\section{Derived Cobordism}
Following Spivak's two papers, which only consider quasi-smooth manifolds, we give:
\begin{defin}
Let $\mathcal{X}$ be a derived manifold, and $f:\mathcal{X} \to \R$ a morphism. Given a point $i\in \R$, we say $f$ is $i$-collared
if there exists $\epsilon>0$ such that $\mathcal{X}_{|f-i|<\epsilon} \cong \mathcal{X}_{f=i} \times (-\epsilon, \epsilon)$.

$\mathcal{Z}_i, \mathcal{Z}_1$ are derived cobordant if there exists $f: \mathcal{X} \to \R$ proper map and $f$ is $i$-collared
for $i=0,1$, $\mathcal{X}_{f=i} = \mathcal{Z}_i$.
\end{defin}

\begin{rem}
If $T$ is a CW complex, then derived cobordism over $T$ is a pair $(a,f)$ where $a: |\mathcal{X}| \to T$ is a continuous map,
and $f$ is as before. For us we will take $T = K(\Z_2, n)$.
\end{rem}

\begin{prop}
We have a fully faithful functor $Man \to \dMan$. This induces $i_* : \Omega \to \Omega^{der}$, which is
\begin{enumerate}
\item well defined;
\item an isomorphism.
\end{enumerate}
\end{prop}

The idea is to take a Whitney embedding $X \to \R^N$ for some large $N$, take the projection $\R^n \to \R$ on the last factor.
The composition of these 2 is the $f$ we need.

Next, we define the fundamental class of a derived manifold. We start with:
\begin{defin}
Cohomology and homology:
\[ H^n(X;R) = [X,K(R,n)] 	\]
\[	H_n(X;R) = \Hom(H^n(X;R), R).	\]
\end{defin}

There are two problems with this definition: only works for $R$ field, and we don't know how to give a chain level definition.

Then we define $[\mathcal{X}]$ by taking $[M]$ for the (unique up to cobordism) smooth manifold $M$ in the derived cobordism class
of $[\mathcal{X}]$.

If $M$ is a manifold and $E$ is a vector bundle, consider:
\[
\begin{tikzcd}
\mathcal{X}\arrow{d}\arrow{r} & M\arrow{d}{s} \\ M\arrow{r}{z} & E.
\end{tikzcd}
\]
\begin{prop}
$[\mathcal{X}]$ is the pullback of the Euler class of $E$.
\end{prop}

In this situation, $X$ admits an implicit atlas with one chart. Pardon
\[
\begin{tikzcd}
\check{C}^{\dim M - \dim E}(X)\arrow{r}{\simeq}\arrow{ddr} & 	C_{\dim E} (M, M \setminus X)\arrow{d}{s_*} \\
 & C_{\dim E}(E , E \setminus \{0\})\arrow{d}{\simeq} \\
 & \Z .	
\end{tikzcd}
\]

Questions:
\begin{enumerate}
\item How to do the oriented case, i.e. $R = \Z$. It's talked about in Chapter 3 of Spivak, also in Pardon.
\item When we work with DM stacks, for example 1-localic, we only have an underlying 1-topos, or a groupoid presentation. How
do we define cohomology? Perhaps we can just take $\pi_0\big(\Map(\mathcal{X}, K(R,n))\big)$.
\item What is cobordism for DM stacks?
\end{enumerate}

Note that there's a paper by Behrend, called ``Cohomology of stacks''. He defines cohomology from a groupoid presentation. Note that
Pardon defines the fundamental class for orbifolds by just using a groupoid presentation, and taking invariants under the group
action.



\section{Modules}
We can define derived $C^{\infty}$ spaces, the analogs of Mauro and Tony's derived $k$-analytic spaces. These are 
$\Sch(\mathcal{T}_{diff})$. We still have some algebraisation functor, to the pre-geometry where we only take polynomials.
\todo{state this better}

We want to define $C^{\infty}$ square-zero extensions. We want to prove that $t_n(X) \to t_{n+1}(X)$ is a $C^{\infty}$ square
zero extension.

We also want to investigate the behavior under closed immersions. In the analytic case, if $A \to B$ is a closed immersion, then
$(\bbL_{B/A}^{an})^{alg} = \bbL_{B^{alg}/A^{alg}}$. This is very important, since otherwise we can't do computations. We don't know
if any of this is true in the $C^{\infty}$ world.